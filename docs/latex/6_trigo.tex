This chapter is contributed by C. Daramy-Loirat
and F. de~Dinechin.  

\section*{Introduction}
This chapter describes the implementations of sine, cosine and
tangent, as they share much of their code.


%%%%%%%%%%%%%%%%%%%%%%%%

\section{Argument reduction}
We only need to compute sine and cosine over the interval
$[0;\frac{\Pi}{2}]$ thanks to their periodicity and symmetry properties.

The argument reduction is obtained by an additive Cody \& Waite
algorithm, according to the following scheme:

$x = y + k*\frac{\Pi}{256}$, such that $ |y| \leq \frac{\Pi}{512}$ and
$k$ is an integer in $[0,255]$.


For every $k$ from 0 to 63, $\sin (k*\frac{\Pi}{256})$ and $\cos
(k*\frac{\Pi}{256})$ are tabulated in double-double precision. This is
enough to have all possible values of sine and cosine for every $k$ in
$[0;255]$, by using the following properties:
        
\begin{equation}        \sin(x) = \cos(x - \frac{\Pi}{2})\end{equation}

\begin{equation}        \sin(x) = -\sin(-x)\end{equation}

\begin{equation}        \cos(x) = \cos(-x)\end{equation}

Finally, we compute a table method by the way:

\begin{equation}        
  \sin(x) = \sin(k*\frac{\Pi}{256} + y) =  \cos(k*\frac{\Pi}{256})* \sin(y) +  \sin(k*\frac{\Pi}{256}) *\cos(y) 
\end{equation}

\begin{equation}
        \cos(x) = \cos(k*\frac{\Pi}{256} + y) = \cos(k*\frac{\Pi}{256})* \cos(y) -  \sin(k*\frac{\Pi}{256}) *\sin(y) \end{equation}\\\begin{equation} tan(x) = \frac{\sin(x)}{\cos(x)} = \frac{\cos(k*\frac{\Pi}{256})* \sin(y) +  \sin(k*\frac{\Pi}{256}) *\cos(y)}{\cos(k*\frac{\Pi}{256})* \cos(y) -  \sin(k*\frac{\Pi}{256}) *\sin(y)}
\end{equation}
$y$ is small enough so we compute $\sin(y)$ and $\cos(y)$ through a
polynomial evaluation. (See file trigo\_fast.c, code $do\_cos$ and
$do\_sin$).

\section{polynomial evaluation}
Assume $\epsilon \leq 1$ and $y < \frac{\Pi}{512}$ \\
$\cos(y)$ and $\sin(y)$ are computed in double precision.

\subsection{Sine}
If $y \leq 2^{-26}$, then $\sin(y)\sim y + \epsilon*2^{-54.5}$ 
        
If $y > 2^{-26}$, then $\sin(y) \sim y *(1+ ts) + \epsilon *
2^{-63.8}$, where $ts = y^2*(s3 + y^2*(s5 + y^2*s7)))$ with $s3$, $s5$ and
$s7$ the Taylor coefficients.

\subsection{Cosine}

If $y \leq 2^{-27}$, then $\cos(y)\sim y + \epsilon*2^{-55.0}$

If $y > 2^{-27}$, then $\cos(y)\sim 1+ tc + \epsilon * 2^{-71.8}$,
where $tc = y^2*(c2 + y^2*(c4 + y^2*c6))$ with $c2$, $c4$ and $c6$ the
Taylor coefficients.
        


\section{Reconstruction}
Assume $a = k*\frac{\Pi}{256}$, $sa = \sin(a)$, $ca = \cos(a)$ and
$"_h"$ and $"_l" $ denote "hi" and 'low" parts of a double-double
floating point number.

\subsection{Sine}
According to equation (4), we have to compute: 
 \begin{eqnarray*}
  \sin(a+y) &=& \sin(a) \cos(y)  + \cos(a)\sin(y)  \\
  & \approx& (sa_h+sa_l)(1+t_c) + (ca_h+ca_l)(y_h+y_l)(1+t_s)
\end{eqnarray*}


As shown on the following graph, we don't need to take into account the smallest argument $ca_l * y_l * (1+ts)$\\
And we compute a $2^{-64}$ double-double precision result with only
one Mul12 and two Add12.
\begin{center}
 \small
 \setlength{\unitlength}{3ex}
      \framebox{
        \begin{picture}(22,9)(-3,-4.2)
        \put(10,4){\line(0,-1){8}}  \put(9.5,4){$\epsilon$}
  
          \put(4,3.2){$sa_h$} \put(0.05,3){\framebox(7.9,0.7){}}
          \put(12,3.2){$sa_l$}  \put(8.05,3){\framebox(7.9,0.7){}}

          \put(6,2.2){$sa_ht_c$} \put(2.05,2){\framebox(7.9,0.7){}}
          \put(14,2.2){$sa_lt_c$}  \put(10.05,2){\framebox(7.9,0.7){}}

          \put(5,1.2){$ca_hy_h$} \put(1.05,1){\framebox(7.9,0.7){}}
          \put(13,1.2){$ca_hy_h$}  \put(9.05,1){\framebox(7.9,0.7){}}

          \put(12,0.2){$ca_hy_l $}  \put(8.05,0){\framebox(7.9,0.7){}}
          \put(12,-0.8){$ca_ly_h $}  \put(8.05,-1){\framebox(7.9,0.7){}}

         \put(7,-1.8){$ca_hy_ht_s$} \put(3.05,-2){\framebox(7.9,0.7){}}
         \put(14,-2.8){$ca_hy_lt_s $}  \put(10.05,-3){\framebox(7.9,0.7){}}
         \put(14,-3.8){$ca_ly_ht_s $}  \put(10.05,-4){\framebox(7.9,0.7){}}
 
        \end{picture}
      }
  \end{center}

\paragraph{Cancellation}


There may be cancellation when $k = 0$, that is to say when $\sin(a) =
0$, then the computation is simplified automatically. (See in
trigo\_fast.c, code do\_cos and do\_sin).  We just need to compute
$\cos(a) * \sin(y)$, wich is reached with the polynomial precision. 

 

\subsection{Cosine}
According to equation (5), we have to compute in double-double precision:
 \begin{eqnarray*}
  \cos(a+y) &=& \cos(a) \cos(y)  - \sin(a)\sin(y)  \\
  & \approx& (ca_h+ca_l)(1+t_c) - (sa_h+sa_l)(y_h+y_l)(1+t_s)
\end{eqnarray*}

As shown on the following graph, we don't need to take into account
the smallest argument $sa_l*y_l*t_s$.  We compute a $2^{-63}$
precision result with only one Mul12 and two Add12.
\begin{center}
 \small
 \setlength{\unitlength}{3ex}
      \framebox{
        \begin{picture}(22,9)(-3,-4.2)
        \put(10,4){\line(0,-1){8}}  \put(9.5,4){$\epsilon$}
  
          \put(4,3.2){$ca_h$} \put(0.05,3){\framebox(7.9,0.7){}}
          \put(12,3.2){$ca_l$}  \put(8.05,3){\framebox(7.9,0.7){}}

          \put(6,2.2){$ca_ht_c$} \put(2.05,2){\framebox(7.9,0.7){}}
          \put(14,2.2){$ca_lt_c$}  \put(10.05,2){\framebox(7.9,0.7){}}

          \put(5,1.2){$-sa_hy_h$} \put(1.05,1){\framebox(7.9,0.7){}}
          \put(13,1.2){$-sa_hy_h$}  \put(9.05,1){\framebox(7.9,0.7){}}

          \put(12,0.2){$-sa_hy_l $}  \put(8.05,0){\framebox(7.9,0.7){}}
          \put(12,-0.8){$-sa_ly_h $}  \put(8.05,-1){\framebox(7.9,0.7){}}

         \put(7,-1.8){$-sa_hy_ht_s$} \put(3.05,-2){\framebox(7.9,0.7){}}
         \put(14,-2.8){$-sa_hy_lt_s $}  \put(10.05,-3){\framebox(7.9,0.7){}}
         \put(14,-3.8){$-sa_ly_ht_s $}  \put(10.05,-4){\framebox(7.9,0.7){}}
 
        \end{picture}
      }
  \end{center}

\paragraph{Cancellation}

There may be cancellation when $k = 0$, that is to say when $\sin(a) =
0$, then the computation is simplified automatically. (See in
trigo\_fast.c, code do\_cos and do\_sin)

We just need to compute $\cos(a) * \cos(y)$, wich is reached with the
polynomial precision.


\subsection{Tangent}

The tangent is obtained by the division of the sine by the cosine.
The procedure "Div22" guarantees a $2^{-104}$ precision result, wich is surely enough to have a precise final result.

If $x < 2^{-26}$, x is returned.
In other cases, we will compute a $2^{-63}$ precision sine and a $2^{-71}$ precision cosine.


So we have a final relative error smaller than $2^{-9}$. This is the
reason why our rounding constant is equal to $1+2^{-9}$.

