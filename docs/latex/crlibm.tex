\documentclass[a4paper]{book} 

\usepackage{fullpage}
\usepackage{isolatin1} 
\usepackage{graphicx}
\usepackage{palatino}
\usepackage{listings}
\usepackage{longtable} %pour les preuves des programmes

\usepackage{lscape}    % pour le mode landscape
\usepackage{makeidx}
\usepackage{color}
\usepackage{amstext,url,latexsym,amsfonts,amssymb,amsmath,amsthm}


\newtheorem{example}{Example}
\newtheorem{theorem}{Theorem}
\newtheorem{algorithm}{Algorithm}
\newtheorem{propriete}{Property}



\newcommand{\pref}[1]{$<$\ref{#1}$>$}
\newcommand{\quick}{{quick}}
\newcommand{\accurate}{{accurate}}
\newcommand{\ulp}{\mbox{ulp}}
\newcommand{\crlibm}{\texttt{crlibm}}
\newcommand{\scslib}{\texttt{scslib}}

\renewcommand{\epsilon}{\varepsilon}

\newcommand{\round}{\circ}
\newcommand{\roundup}{\bigtriangleup}
\newcommand{\rounddown}{\bigtriangledown}
\newcommand{\intpart}[1]{\left\lceil #1 \right\rfloor}
\newcommand{\maxv}[1]{{\overline{#1}}}
\newcommand{\maxx}{\maxv{x}}
\newcommand{\maxeps}{\maxv{\epsilon}}
\newcommand{\maxdelta}{\maxv{\delta}}
\newcommand{\maxz}{{\overline{z}}}
\newcommand{\maxs}[1]{\overline{s}_{#1}}
\newcommand{\mins}[1]{\underline{s}_{#1}}
\newcommand{\maxp}{\overline{|P|}}
\newcommand{\minp}{\underline{|P|}}
\newcommand{\infnorm}[1]{||{#1}||^{\infty}}

\newcommand{\abserr}[1]{\delta_{\mathrm{#1}}}
\newcommand{\relerr}[1]{\epsilon_{\mathrm{#1}}}
\newcommand{\maxabserr}[1]{\maxv{\delta}_{\mathrm{#1}}}
\newcommand{\maxrelerr}[1]{\maxv{\epsilon}_{\mathrm{#1}}}


%\newcommand{\deltapprox}{\delta_{\mathrm{approx}}}
%\newcommand{\deltaround}{\delta_{\mathrm{round}}}




% Environnement pour disposer de propri�t� ``� la leslie lamport''
\newcounter{propexp} % reset propexp � chaque chapitre
\newenvironment{prop}
{\begin{trivlist}\refstepcounter{propexp}\item[\hbox to 0pt{\spring {\bf $<$\arabic{propexp}$>$}}]}
{\ifvmode\smallskip\fi$\bullet$\end{trivlist}}

% D�finition de l'environnement proof
\def\spring{\hskip 0pt minus 1fil}
\def\@yproof[#1]{\@proof{ #1}}
\def\@proof#1{\begin{trivlist}\item[\hbox to 0pt{\spring $\diamond$}]
\emph{Proof#1. }}
\newenvironment{preuve}
{\@proof{}}{\hfill\null\hfill$\square$\end{trivlist}}



% Definition des options pour le packages listings
\lstset{numbers=left,
  numberstyle=\tiny,
  stepnumber=1,
  numbersep=5pt,
  lineskip=-1pt,
  extendedchars=true,
  basicstyle=\footnotesize,
  breaklines,
  showstringspaces=false,
  frame=single,
  language={[ANSI]C},
  firstnumber=last,
  escapeinside={(*@}{@*)}
}




\title{CR-LIBM\\A library of correctly rounded elementary functions in double-precision}
\author{\mbox{Catherine Daramy-Loirat}, \mbox{David Defour}, \mbox{Florent~de~Dinechin},\\ \mbox{Matthieu Gallet}, \mbox{Nicolas Gast},  \mbox{Jean-Michel Muller}}

%% \abstract{
%%   CRLibm is a mathematical library with correct rounding in
%%   double-precision in the four IEEE-754 rounding modes. This report
%%   explains the code and proves the correct rounding property.
%% }

%% \RRIresume{%
%%   CRLibm est une biblioth�que math�matique offrant l'arrondi correct
%%   en double pr�cision selon les quatre modes d'arrondi. Ce rapport
%%   d�taille les algorithmes utilis�s et prouve fonction par fonction la
%%   propri�t� d'arrondi correct.
%% }%

%% \RRIkeywords{elementary functions, correct rounding, IEEE-754, double precision}

%% \RRImotscles{fonctions �l�mentaires, arrondi correct, IEEE-754, double pr�cision}

%% \RRItheme{2}

%% \RRIprojet{Ar�naire}

%% \RRIinria{\RRnote{%
%%     This  text  is  also  available   as  a  research  report  of  the
%%     Laboratoire    de     l'Informatique    du    Parall�lisme    {\tt
%%       http://www.ens-lyon.fr/LIP}.
%% }}%




\begin{document}
\maketitle


\section*{Important warning}

This report describes and proves version \input{../../VERSION} of the
\texttt{crlibm} library. It may therefore not correspond to the latest
version. An up-to-date version will always be distributed along with
the code.

\vfill
\section*{Many thanks to...}
\begin{itemize}
\item Vincent Lef\`evre, from \textsc{Inria}-Lorraine, without whom this
  project would't have started, and who has since then lended a helpful hand in
  more than 17 occasions;
\item The Ar\'enaire project at \'ENS-Lyon, especially Marc Daumas,
  Sylvie Boldo, Guillaume Melquiond, Nathalie Revol and Arnaud
  Tisserand;
\item The MPFR developers, and especially the \textsc{Inria}
  \textsc{Spaces} project;
\item The Intel Nizhniy-Novgorod Lab, especially Andrey Naraikin,
  Sergei Maidanov and Evgeny Gvozdev;
\item The contributors of bits and pieces, Phil Defert and Eric
  McIntosh from CERN, Patrick Pelissier from \textsc{Inria}.
\item All the people who have reported bugs, hoping that they continue
  until they have to stop: Evgeny Gvozdev from Intel, Christoph Lauter
  from TUM\"unchen, Eric McIntosh from CERN, Patrick Pelissier and
  Paul Zimmermann from \textsc{Inria} Lorraine.
\item Peter Markstein at HP, Neil Toda at Sun, Shane Story at Intel,
  and many others for interesting and sometimes heated discussions.
\item Serge Torres for LIPForge 
\end{itemize}

This work was partly funded by the \textsc{Inria}, the \'ENS-Lyon and the Universit\'e de Perpignan.

\tableofcontents

\addtocounter{chapter}{-1}
\chapter{Getting started with crlibm}
\section{What is \crlibm?}

The \crlibm\ project aims at developing a portable, proven, correctly rounded,
and efficient mathematical library (\texttt{libm}) for double precision. 

\begin{description}
\item[correctly rounded] Current \texttt{libm} implementation do not
  always return the floating-point number that is closest to the exact
  mathematical result. As a consequence, different \texttt{libm}
  implementation will return different results for the same input,
  which prevents full portability of floating-point applications. In
  addition, few libraries support but the round-to-nearest mode of the
  IEEE754/IEC 60559 standard for floating-point arithmetic (hereafter
  usually referred to as the IEEE-754 standard). \crlibm\ provides the
  four rounding modes: To nearest, to $+\infty$, to $-\infty$ and to
  zero.

\item[portable] \crlibm\ is written in C and will be compiled by any
  compiler fulfilling basic requirements of the ISO/IEC 9899:1999
  (hereafter referred to as C99) standard.  This is the case of
  \texttt{gcc} version 3 and higher which is available on most
  computer systems. It also requires a floating-point implementation
  respecting the IEEE-754 standard, which is also available on
  most modern systems. \crlibm\ has been tested on a large range of
  systems.

\item[proven] Other libraries attempt to provide correctly-rounded
  result. For theoretical and practical reasons, this behaviour is
  difficult to prove, and in extreme cases termination is not even
  guaranteed. \crlibm\ intends to provide a comprehensive proof of the
  theoretical possibility of correct rounding, the algorithms used,
  and the implementation, assuming C99 and IEEE-754 compliance.

\item[efficient] performance and resource usage of \crlibm\ should be
  comparable to existing \texttt{libm} implementations, both in
  average and in the worst case. In contrast, other correctly-rounded
  libraries have worst case performance and memory consumption several
  order of magnitude larger than standard \texttt{libm}s.

\end{description}

The ultimate goal of the \crlibm\ project is to push towards the
standardization of correctly-rounded elementary functions.

\section{Compilation and installation}
See the \texttt{INSTALL} file in the main directory. This library is
developed using the GNU autotools, and can therefore be compiled on
most Unix-like systems by \texttt{./configure; make}. 


The command \texttt{make check} will launch the selftest.
For more advanced testing you will need to have MPFR installed (see
\url{www.mpfr.org}) and to pass the \texttt{--enable-mpfr} flag to
\texttt{configure}. For other flags, see \texttt{./configure --help} .

\section{Using \texttt{crlibm} functions in your program}

Currently \texttt{crlibm} functions have different names from the
standard \texttt{math.h} functions. For example, for the sine function
(\texttt{double sin(double)} in the standard \texttt{math.h}), you
have four different functions in \texttt{crlibm} for the four
different rounding modes. These functions are named \texttt{sin\_rn},
\texttt{sin\_ru}, \texttt{sin\_rd} and \texttt{sin\_rz} for round to the
nearest, round up, round down and round to zero respectively. These
functions are declared in the C header file \texttt{crlibm.h}.

The \texttt{crlibm} library relies on double-precision IEEE-754
compliant floating-point operations.  For some processors and some
operating systems (most notably IA32 and IA64 processors under
GNU/Linux), the default precision is set to double-extended.  On such
systems you will need to call the \texttt{crlibm\_init()} function
before using any \texttt{crlibm} function to ensure such compliance.
This has the effect of setting the processor flags to IEEE-754
double-precision with rounding to the nearest mode.  This function
returns the previous processor status, so that previous mode can be
restored using the function \texttt{crlibm\_exit()}. Note that you
probably only need one call to \texttt{crlibm\_init()} at the beginning
of your program, not one call before each call to a mathematical
function.

Here is a non-exhaustive list of systems on which
\texttt{crlibm\_init()} is NOT needed, and which can therefore use
\crlibm\ as a transparent replacement of the standard \texttt{libm}:

\begin{itemize}
\item Most Power/PowerPC based systems, including those from Apple or from IBM;
\item All the 64-bit Linux versions: the reason is that all
  x86-compatible processors (by AMD and Intel) supporting 64-bit
  addressing also feature SSE2 FP instructions, which are cleaner and
  more efficient than the legacy x87 FPU. On such systems, SSE2 is
  therefore used by default by \texttt{gcc} for double-precision FP
  computing.
\item On recent 32-bit x86 processors also featuring SSE2 extensions
  (including pentium 4 and later, and generally most processors
  produced after 2005), you can try to force the use of SSE2
  instructions using \texttt{configure --enable-sse2}. Beware, the
  code produced will not run on older hardware.
\end{itemize}

Here's an example function named \texttt{compare.c} using the cosine
function from \texttt{crlibm} library.

\begin{lstlisting}[label={chap0:lst:prog_example},caption={compare.c},firstnumber=1]
#include<stdio.h>
#include<math.h>
#include<crlibm.h>

int main(void){
  double x, res_libm, res_crlibm;

  printf("Enter a floating point number: ");
  scanf("%lf", &x);
  res_libm = cos(x);
  crlibm_init(); /* no need here to save the old processor state returned by crlibm_init() */ 
  res_crlibm = cos_rn(x);
  printf("\n x=%.25e \n", x);
  printf("\n cos(x) with the system : %.25e \n", res_libm);
  printf("\n cos(x) with crlibm     : %.25e \n", res_crlibm);
  return 0;
}
\end{lstlisting}

This example will be compiled with \texttt{gcc compare.c -lm -lcrlibm -o compare}


\section{Currently available functions}

The currently available functions are summarized in
Table~\ref{tab:currentstate}.
\begin{table}[t]
  \begin{center}
\renewcommand{\arraystretch}{1.2}
\begin{tabular}{|c|c|c|c|c||c|c|}    \hline
 & \multicolumn{4}{c||}{\crlibm\ name} &\multicolumn{2}{c|}{State of the proof} \\ \cline{2-7}
 \raisebox{5pt}{C99} & to nearest & to $+ \infty$ & to $- \infty$ & to zero
 & Worst cases & Proof of the code \\ \hline\hline
    exp & exp\_rn & exp\_ru & exp\_rd & exp\_rz & complete& complete (formal)\\ \hline
    expm1 & expm1\_rn & expm1\_ru & expm1\_rd & expm1\_rz & complete & partial\\ \hline
    log & log\_rn & log\_ru & log\_rd & log\_rz & complete& complete\\ \hline
    log1p & log1p\_rn & log1p\_ru & log1p\_rd & log1p\_rz & complete& partial \\ \hline
    log2 & log2\_rn & log2\_ru & log2\_rd & log2\_rz & complete& partial\\ \hline
    log10 & log10\_rn & log10\_ru & log10\_rd & log10\_rz& complete& partial \\ \hline
    sin & sin\_rn & sin\_ru & sin\_rd & sin\_rz & $[-\pi, \pi]$& complete (paper+formal)\\ \hline
    cos & cos\_rn & cos\_ru & cos\_rd & cos\_rz & $[-\pi/2, \pi/2]$& complete (paper+formal)\\ \hline
    tan & tan\_rn & tan\_ru & tan\_rd & tan\_rz & $[-\pi/2, \pi/2]$& complete (paper+formal)\\ \hline
    asin & asin\_rn & asin\_ru & asin\_rd & asin\_rz & complete & partial \\ \hline
    acos & acos\_rn & acos\_ru & acos\_rd & acos\_rz & complete & partial\\ \hline
    atan & atan\_rn & atan\_ru & atan\_rd & atan\_rz & complete & complete (paper)\\ \hline
    sinh & sinh\_rn & sinh\_ru & sinh\_rd & sinh\_rz & complete & complete (paper)\\ \hline
    cosh & cosh\_rn & cosh\_ru & cosh\_rd & cosh\_rz & complete & complete (paper)\\ \hline
    sinpi & sinpi\_rn & sinpi\_ru & sinpi\_rd & sinpi\_rz & complete & complete (formal)\\ \hline
    cospi & cospi\_rn & cospi\_ru & cospi\_rd & cospi\_rz & complete & complete (formal)\\ \hline
    tanpi & tanpi\_rn & tanpi\_ru & tanpi\_rd & tanpi\_rz & $[2^{-25},2^{-5}]$& complete (formal)\\ \hline
%    asinpi & asinpi\_rn & asinpi\_ru & asinpi\_rd & asinpi\_rz & complete & partial \\ \hline
%    acospi & acospi\_rn & acospi\_ru & acospi\_rd & acospi\_rz & complete & partial\\ \hline
    atanpi & atanpi\_rn & atanpi\_ru & atanpi\_rd & atanpi\_rz & $[\tan(2^{-25}\pi),\tan(2^{-5}\pi)]$ & complete (paper)\\ \hline
    pow & pow\_rn &  &  &  & see chapter \ref{chap:pow} & see chapter \ref{chap:pow}\\ \hline
\end{tabular}
\end{center}
  
\caption{Current state of \crlibm.}
\label{tab:currentstate}
\end{table}

Here are some comments on this table:
\begin{itemize}
\item Every function takes a double-precision number and returns a
  double-precision number.
\item For trigonometric functions the angles are
  expressed in radian.
\item The two last columns describe the state of the
  proof:
  \begin{itemize}
  \item The first indicates the state of the search for worst cases
    for correct rounding \cite{LMT98,Lef2000}. If it indicates
    ``complete'', it means that the function is guaranteed to return
    correct rounding on its whole floating-point input range.
    Otherwise, it mentions the interval on which the function is
    guaranteed to return correct rounding. Note that \crlibm\ is
    designed in such a way that there is a very high probability that it
    is correctly rounded everywhere, however this is not yet proven
    formally. This question is explained in details in section
    \ref{section:crlibm-presentation}.

  \item The second indicates the state of the proof of the code
    itself. Some (older) functions have a lengthy paper proof in this
    document, some other have a partial or complete formal proof using
    the Gappa proof assistant \cite{Melqu05,DinLauMel2005}.
  \end{itemize}
\end{itemize}


\section{Writing portable floating-point programs}

Here are some rules to help you design programs which have to
produce exactly the same results on different architectures and
different operating systems.
\begin{itemize}
\item Try to use the same compiler on all the systems.
\item Demand C99 compliance (pass the \texttt{-C99},
  \texttt{-std=c99}, or similar flag to the compiler). For Fortran,
  demand F90 compliance.
\item Call \texttt{crlibm\_init()} before you begin floating-point
  computation. This ensures that the computations will all be done in
  IEEE-754 double-precision with round to nearest mode, which is the
  largest precision well supported by most systems. On IA32
  processors, problems may still occur for extremely large or
  extremely small values.
\item Do not hesitate to rely heavily on parentheses (the compiler
  should respect them according to the standards, although of course some
  won't). Many times, wondering where the parentheses should go in an
  expression like \texttt{a+b+c+d} will even help you improve the
  accuracy of your code.
\item Use \texttt{crlibm} functions in place of \texttt{math.h} functions.
\end{itemize}



%%% Local Variables: 
%%% mode: latex
%%% TeX-master: "crlibm"
%%% End: 



\chapter{Introduction: Goals and methods \label{chap:intro}}




\section{Correct rounding and elementary functions}
\label{sect:intro}

The need for accurate elementary functions is important in many
critical programs.  Methods for computing these functions include
table-based methods\cite{Far81,Tan91}, polynomial approximations and
mixed methods\cite{DauMor2k}. See the books by Muller\cite{Muller97} or
Markstein\cite{Markstein2000} for recent surveys on the subject.

The IEEE-754 standard for floating-point arithmetic\cite{IEEE754}
defines the usual floating-point formats (single and double
precision). It also specifies the behavior of the four basic operators
($+,-,\times,\div$) and the square root in four rounding modes (to the
nearest, towards $+\infty$, towards $-\infty$ and towards $0$). Its
adoption and widespread use have increased the numerical quality of,
and confidence in floating-point code. In particular, it has improved
\emph{portability} of such code and allowed construction of
\emph{proofs} on its numerical behavior. Directed rounding modes
(towards $+\infty$, $-\infty$ and $0$) also enabled efficient
\emph{interval arithmetic}\cite{Moore66,KKLRW93}.

However, the IEEE-754 standard specifies nothing about elementary
functions, which limits these advances to code excluding such
functions.  Currently, several options exist: on one hand, one can use
today's mathematical libraries that are efficient but without any
warranty on the correctness of the results. To be fair, most modern
libraries are \emph{accurate-faithful}: trying to round to nearest,
they return a number that is one of the two FP numbers surrounding the
exact mathematical result, and indeed return the correctly rounded
result most of the time. This behaviour is sometimes described using
phrases like \emph{99\% correct rounding} or \emph{0.501 ulp accuracy}.

When stricter guarantees are needed, some multiple-precision packages
like MPFR \cite{MPFRweb} offer correct rounding in all rounding modes,
but are several orders of magnitude slower than the usual mathematical
libraries for the same precision. Finally, there are are currently
three attempts to develop a correctly-rounded \texttt{libm}. The first
was IBM's \texttt{libultim}\cite{IBMlibultimweb} which is both
portable and fast, if bulky, but lacks directed rounding modes needed
for interval arithmetic. The second was Ar\'enaire's \texttt{crlibm},
which was first distributed in 2003. The third is Sun
correctly-rounded mathematical library called \texttt{libmcr}, whose
first beta version appeared in 2004.  These libraries are reviewed in
\ref{section:lib-overview}.

The goal of the \crlibm\ project is to build on a combination of
several recent theoretical and algorithmic advances to design a proven
correctly rounded mathematical library, with an overhead  in terms of
performance and resources acceptable enough to replace existing
libraries transparently. 

More generally, the \crlibm\ project serves as an open-source
framework for research on software elementary functions. As a side
effect, it may be used as a tutorial on elementary function
development.




%%%%%%%%%%%%%%%%%%%%%%%%%%%%%%%%%%%%%%%%%%%%%%%%%%%%%%%%%%%%%
\section{A methodology for efficient correctly-rounded functions}
\label{section:methodology}


\subsection{The Table Maker's Dilemma}

With a few exceptions, the image $\hat{y}$ of a floating-point number $x$ by
a transcendental function $f$ is a transcendental number, and can
therefore not be represented exactly in standard numeration systems.
The only hope is to compute the floating-point number that is closest
to (resp.  immediately above or immediately below) the mathematical
value, which we call the result \emph{correctly rounded} to the
nearest (resp.  towards $+\infty$ or towards $-\infty$).

It is only possible to compute an approximation ${y}$ to the real
number $\hat{y}$ with precision $\maxeps{}$. This ensures that the real value
$\hat{y}$ belongs to the interval $[{y}(1-\maxeps{}) , {y}(1+\maxeps{})]$.
Sometimes however, this information is not enough to decide correct
rounding. For example, if $[{y}(1-\maxeps{}) , {y}(1+\maxeps{})]$
contains the middle of two consecutive floating-point numbers, it is
impossible to decide which of these two numbers is the correctly
rounded to the nearest of $\hat{y}$. This is known as the Table Maker's
Dilemma (TMD). For example, if we consider a numeration system in radix $2$ with $n$-bit mantissa floating point number and $m$ the number of significant bit in $y$ such that $\maxeps{} \leq 2^m$, then the TMD occurs:

\begin{itemize}
\item for rounding toward $+\infty$, $-\infty$, $0$, when the result is of the form:

$$\overbrace{\underbrace{1.xxx...xx}_{n~bits}111111...11}^{m~bits}xxx...$$
or:
$$\overbrace{\underbrace{1.xxx...xx}_{n~bits}000000...00}^{m~bits}xxx...$$
\item for rounding to nearest, when the result is of the form:
$$\overbrace{\underbrace{1.xxx...xx}_{n~bits}011111...11}^{m~bits}xxx...$$
or :
$$\overbrace{\underbrace{1.xxx...xx}_{n~bits}100000...00}^{m~bits}xxx...$$
\end{itemize}


\subsection{The onion peeling strategy}

A method described by Ziv \cite{Ziv91} is to increase the precision
$\maxeps$ of the approximation until the correctly rounded value can
be decided.  Given a function $f$ and an argument $x$, the value of
$f(x)$ is first evaluated using a quick approximation of precision
$\maxeps_1$.  Knowing $\maxeps_1$, it is possible to decide if
rounding is possible, or if more precision is required, in which case
the computation is restarted using a slower approximation of precision
$\maxeps_2$ greater than $\maxeps_1$, and so on. This approach makes
sense even in terms of average performance, as the slower steps are
rarely taken.

However there was until recently no practical bound on the termination
time of such an algorithm. This iteration has been proven to
terminate, but the actual maximal precision required in the worst case
is unknown.  This might prevent using this method in critical
application.




\section{The Correctly Rounded Mathematical Library}
\label{section:crlibm-presentation}

Our own library, called \crlibm\ for \emph{correctly rounded
  mathematical library}, is based on the work of
Lef\`evre and Muller \cite{LMT98,Lef2000} who computed the worst-case $\maxeps$
required for correctly rounding several functions in double-precision
over selected intervals in the four IEEE-754 rounding modes. For
example, they proved that 157 bits are enough to ensure correct rounding
of the exponential function on all of its domain for the four IEEE-754
rounding modes.

\subsection{Two steps are enough}
Thanks to such results, we are able to guarantee correct rounding in
two iterations only, which we may then optimize separately. The first
of these iterations is relatively fast and provides between 60 and 80
bits of accuracy (depending on the function), which is sufficient in
most cases. It will be referred throughout this document as the \quick\ 
phase of the algorithm. The second phase, referred to as the
\accurate\ phase, is dedicated to challenging cases. It is slower but
has a reasonably bounded execution time, tightly targeted at
Lef\`evre's worst cases.

Having a proven worst-case execution time lifts the last obstacle to a
generalization of correctly rounded transcendentals. Besides, having
only two steps allows us to publish, along with each function, a proof
of its correctly rounding behavior.


\subsection{Portable IEEE-754 FP for a fast first step}
The computation of a tight bound on the approximation error of the
first step ($\maxeps_1$) is crucial for the efficiency of the onion
peeling strategy: overestimating $\maxeps_1$ means going more often
than needed through the second step, as will be detailed below in
\ref{sec:error-accuracy-perf}. As we want the proof to be portable as
well as the code, our first steps are written in strict IEEE-754
arithmetic. On some systems, this means preventing the
compiler/processor combination to use advanced floating-point features
such as fused multiply-and-add or extended double precision. It also
means that the performance of our portable library will be lower than
optimized libraries using these features (see \cite{DinErshGast2005} for
recent research on processor-specific correct-rounding).

To ease these proofs, our first steps make wide use of classical, well
proven results like Sterbenz' lemma or other floating-point theorems.
When a result is needed in a precision higher than double precision
(as is the case of $y_1$, the result of the first step), it is
represented as as the sum of two floating-point numbers, also called a
\emph{double-double} number.  There are well-known algorithms for
computing on double-doubles, and they are presented in the next
chapter. An advantage of properly encapsulating double-double
arithmetic is that we can actually exploit fused multiply-and-add
operators in a transparent manner (this experimental feature is
currently available for the Itanium and PowerPC platforms, when using
the \texttt{gcc} compiler).

At the end of the \quick\ phase, a sequence of simple tests on
$y_1$ knowing $\maxeps_1$ allows to decide whether to go for
the second step. The sequence corresponding to each rounding mode is
shared by most functions and is also carefully proven in the next
chapter.


\subsection{Ad-hoc, fast multiple precision for  accurate second step}
For the second step, we may use two specific multiple-precision libraries:

\begin{itemize}
\item We first designed an ad-hoc multiple-precision library called Software
  Carry-Save library \emph{(scslib)} which is lighter and faster than
  other available libraries for this specific application
  \cite{DefDin2002,DinDef2003}. This choice is motivated by
  considerations of code size and performance, but also by the need to
  be independent of other libraries: Again, we need a library on which
  we may rely at the proof level. This library is included in \crlibm,
  but also distributed separately \cite{SCSweb}. This library is
  described in more details in \ref{sec:SCSLib}.
\item More recently, we have been using redundant triple-double
  arithmetic for the second step. This approach is lighter, about ten
  times faster, and has the advantage of making it easier to reuse
  information from the fast step in the accurate one. The drawback is
  that it is more difficult to master. The basic triple-double
  procedures, and associated usage theorems, are described in a
  separate document (\texttt{tripledoubleprocedures.pdf}) also
  available in this distribution.
\end{itemize}


\subsection{Relaxing portability}

The \crlibm\ framework has been used to study the performance
advantage of using double-extended (DE) arithmetic when available.
More specifically, the first case may be implemented fully in DE
precision, and the second step may be implemented fully in double-DE
arithmetic. Experiments have been performed on the logarithm and
arctangent functions \cite{DinErshGast2005}. On some systems (mostly
Linux on an IA32 processor) the logarithm will by default use this
technology.

Another useful, non-portable hardware feature is the fused
multiply-and-add available on Power/PowerPC and Itanium. The \crlibm\
code does its best to use it when available.


\subsection{Proving the correct rounding property}

Throughout this document, what we call ``proving'' a function mostly
means proving a tight bound on the total relative error of our
evaluation scheme. The actual proof of the correct rounding property
is then dependent on the availability of an actual worst-case accuracy
for correctly rounding this function, as computed by Lef\`evre and
Muller. Three cases may happen:
\begin{itemize}
\item The worst case have been computed over the whole domain of the
  function. In this case the correct rounding property for this
  function is fully proven. The state of this search for worst cases
  is described in Table~\ref{tab:currentstate}
  page~\pageref{tab:currentstate}.

\item The worst cases have been computed only over a subdomain of the
  function. Then the correct rounding property is proven on this
  subdomain. Outside of this domain \texttt{crlibm} offers
  ``astronomical confidence'' that the function is correctly rounded:
  to the best of current knowledge \cite{Gal86, DinErshGast2005}, the
  probability of the existence of a misrounded value in the function's
  domain is lower than $2^{-40}$. This is
  the case of the trigonometric functions, for instance. The
  actual domain on which the proof is complete is mentionned in the
  respective chapter of this document, and summed up in Table~\ref{tab:currentstate}.
\item The search for worst cases hasn't begun yet.
\end{itemize}

We acknowledge that the notion of astronomical confidence breaks the
objective of guaranteed correct rounding, and we sidestep this problem
by publishing along with the library (in this document) the domain of
full confidence, which will only expand as time goes.  Such behaviour
has been proposed as a standard in \cite{DefHanLefMulRevZim2004}.  The
main advantage of this approach is that it ensures bounded and
consistent worst-case execution time (within a factor 100 of that of
the best available faithful \texttt{libm}s), which we believe is
crucial to the generalization of correctly rounded functions.

The alternative to our approach would be to implement a
multi-layer onion-peeling strategy, as do GNU MPFR and Sun's
\texttt{libmcr}. There are however many drawbacks to this approach, too:

\begin{itemize}

\item One may argue that, due to the finite nature of computers, it
  only pushes the bounds of astronomy a little bit further.

\item The multilayer approach is only proven to terminate on
  elementary functions: the termination proof needs a theorem stating
  for example that the image of a rational by the function (with some
  well-known exceptions) will not be a rational. For other library
  functions like  special functions, we have no such theorem.
  For these functions, we prefer take the risk of a misrounded
  value than the risk of an infinite loop.

\item Similarly, the multilayer approach has potentially unbounded
  execution time and memory consumption which make it unsuitable for
  real-time or safety-critical applications, whereas crlibm will only
  be unsuitable if the safety depends on correct rounding, which is
  much less likely.

\item Multilayer code is probably much more complex and error prone.
  One important problem is that it contains code that, according all
  probabilities, will never be run. Therefore, testing this code can
  not be done on the final production executable, but on a different
  executable in which previous layers have been disabled. This
  introduces the possibility of undetected bugs in the final
  production executable.

\end{itemize}

In the future, we will add, to those \texttt{crlibm} functions for
which the required worst-case accuracy is unknown, a misround
detection test at the end of the second step. This test will either
print out on standard error a lengthy warning inviting to report this
case, or launch MPFR computation, depending on a compilation switch.

% TODO



\subsection{Error analysis and the accuracy/performance tradeoff
  \label{sec:error-accuracy-perf}}

As there are two steps on the evaluation, the proof also usually
consists of two parts. The code of the second, accurate step is
usually very simple and straightforward:
\begin{itemize}
\item Performance is not that much of an issue, since this step is rarely taken.
\item All the special cases have already been filtered by the first step.
\item The \texttt{scslib} library provides an overkill of precision.
\end{itemize}

Therefore, the error analysis of the second step, which ultimately
proves the correct rounding property, is not very difficult.

For the first step, however, things are more complicated:
\begin{itemize}
\item We have to handle special cases (infinities, NaNs, signed
  zeroes, over- and underflows).
\item Performance is a primary concern, sometimes leading to ``dirty
  tricks'' obfuscating the code.
\item We have to compute a \emph{tight} error bound, as explained below.
\end{itemize}

Why do we need a tight error bound? Because the decision to launch the
second step is taken by a \emph{rounding test}  depending on
\begin{itemize}
\item the approximation $y_h+y_l$ computed in the first step, and
\item this error bound $\maxeps_1$, which is computed statically.
\end{itemize}

The various rounding tests are detailed and proven in
\ref{section:testrounding}.  The important notion here is that
\emph{the probability of launching the second, slower step will be
  proportional to the error bound $\maxeps_1$ computed for the first step}.

This defines the main performance tradeoff one has to manage when
designing a correctly-rounded function: The average evaluation time
will be
\begin{equation}
  T_{\mbox{\small avg}} = T_1 + p_2T_2 \label{eq:Tavg}
\end{equation}
where $T_1$ and $T_2$ are the execution time of the first and second
phase respectively (with $T_2\approx 100T_1$ in \crlibm), and $p_2$ is
the probability of launching the second phase (typically we aim at
$p_2=1/1000$ so that the average cost of the second step is less than
$10\%$ of the total.  

As $p_2$ is almost proportional to $\maxeps_1$, to minimise the average
time, we have to
\begin{itemize}
\item balance $T_1$ and $p_2$: this is a performance/precision
  tradeoff (the faster the first step, the less accurate)
\item and compute a tight bound on the overall error  $\maxeps_1$.
\end{itemize}

Computing this tight bound is the most time-consuming part in the
design of a correctly-rounded elementary function. The proof of the
correct rounding property only needs a proven bound, but a loose bound
will mean a larger $p_2$ than strictly required, which directly
impacts average performance. Compare $p_2=1/1000$ and $p_2=1/500$ for
$T_2=100T_1$, for instance. As a consequence, when there are multiple
computation paths in the algorithm, it makes sense to precompute
different values of $\maxeps_1$ on these different paths (see for
instance the arctangent and the logarithm).






Apart from these considerations, computing the errors is mostly
textbook science. Care must be taken that only \emph{absolute} error
terms (noted $\delta$) can be added, although some error terms (like
the rounding error of an IEEE operation) are best expressed as
\emph{relative} (noted $\epsilon$). Remark also that the error needed
for the theorems in \ref{section:testrounding} is a \emph{relative}
error.  Managing the relative and absolute error terms is very
dependent on the function, and usually involves keeping upper and
lower bounds on the values manipulated along with the error terms.

Error terms to consider are the following:
\begin{itemize}
\item approximation errors  (minimax or Taylor),
\item rounding error, which fall into two categories:
  \begin{itemize}
  \item roundoff errors in values tabulated as doubles or
    double-doubles (with the exception of roundoff errors on the coefficient
    of a polynomial, which are counted in the appproximation error),
  \item roundoff errors in IEEE-compliant operations.
  \end{itemize}
\end{itemize}




\section{An overview of other  available mathematical libraries\label{section:lib-overview}}

Many high-quality mathematical libraries are freely available and have
been a source of inspiration for this work.

Most mathematical libraries do not offer correct rounding. They can be classified as 
\begin{itemize}
\item portable libraries  assuming IEEE-754
  arithmetic, like \emph{fdlibm}, written by Sun\cite{FDLIBMweb};
\item  Processor-specific libraries, by
  Intel\cite{HarKubStoTan99,IntelOpenSource} and
  HP\cite{Markstein2000,Markstein2001} among other.
\end{itemize}

Operating systems often include several mathematical libraries, some of which are derivatives of one
of the previous.

To our knowledge, three libraries currently offer correct rounding:
\begin{itemize}
\item The \emph{libultim} library, also called MathLib, is developed
  at IBM by Ziv and others \cite{IBMlibultimweb}. It provides correct
  rounding, under the assumption that 800 bits are enough in all case.
  This approach suffers two weaknesses. The first is the absence of
  proof that 800 bits are enough: all there is is a very high
  probability.  The second is that, as we will see in the sequel, for
  challenging cases, 800 bits are much of an overkill, which can
  increase the execution time up to 20,000 times a normal execution.
  This will prevent such a library from being used in real-time
  applications.  Besides, to prevent this worst case from degrading
  average performance, there is usually some intermediate levels of
  precision in MathLib's elementary functions, which makes the code
  larger, more complex, and more difficult to prove (and indeed this
  library is scarcely documented).
  
  In addition this library provides correct rounding only to nearest.
  This is the most used rounding mode, but it might not be the most
  important as far as correct rounding is concerned: correct rounding
  provides a precision improvement over current mathematical libraries
  of only a fraction of a {unit in the last place} \emph{(ulp)}.
  Conversely, the three other rounding modes are needed to guarantee
  intervals in interval arithmetic.  Without correct rounding in these
  directed rounding modes, interval arithmetic looses up to one
  \emph{ulp} of precision in each computation.
  
\item \emph{MPFR} is a multiprecision package safer than
  \emph{libultilm} as it uses arbitrary multiprecision. It provides
  most of elementary functions for the four rounding modes defined by
  the IEEE-754 standard. However this library is not optimized for
  double precision arithmetic. In addition, as its exponent range is
  much wider than that of IEEE-754, the subtleties of subnormal numbers
  are difficult to handle properly using such a multiprecision
  package.

\item The \texttt{libmcr} library, by K.C. Ng, Neil Toda and others at
  Sun Microsystems, had its first beta version published in december
  2004. Its purpose is to be a reference implementation for correctly
  rounded functions in double precision. It has very clean code,
  offers arbitrary multiple precision unlike \texttt{libultim}, at the
  expense of slow performance (due to, for example dynamic allocation
  of memory). It offers the directed rounding modes, and rounds in the
  mode read from the processor status flag.
\end{itemize}


\section{Various policies in \crlibm}

\subsection{Naming the functions}
Current \crlibm\ doesn't by default replace your existing \texttt{libm}: the
functions in \crlibm\ have the C99 name, suffixed with \texttt{\_rn},
\texttt{\_ru}, \texttt{\_rd}, and \texttt{\_rz} for rounding to the
nearest, up, down and to zero respectively. They require the processor
to be in round to nearest mode. Starting with version 0.9 we should
provide a compile-time flag which will overload the default
\texttt{libm} functions with the crlibm ones with rounding to nearest.

It is interesting to compare this to the behaviour of Sun's library:
First, Sun's \texttt{libmcr} provides only one function for each C99
function instead of four in \crlibm, and rounds according to the
processor's current mode. This is probably closer to the expected
long-term behaviour of a correctly-rounded mathematical library, but
with current processors it may have a tremendous impact on
performance. Besides, the notion of ``current processor rounding
mode'' is no longer relevant on recent processors like the Itanium
family, which have up to four different modes at the same time.  A
second feature of \texttt{libmcr} is that it overloads by default the
system \texttt{libm}.

The policy implemented in current \crlibm\ intends to provide best
performance to the two classes of users who will be requiring correct
rounding: Those who want predictible, portable behaviour of
floating-point code, and those who implement interval arithmetic. Of course, we
appreciate any feedback on this subject.

\subsection{Policy concerning IEEE-754 flags}

Currently, the \crlibm\ functions try to raise the Overflow and
Underflow flags properly. Raising the other flags (especially the
Inexact flag) is possible but considered too costly for the expected
use, and will usually not be implemented. We also appreciate feedback
on this subject.

\subsection{Policy concerning conflicts between correct rounding and
  expected mathematical properties}
As remarked in \cite{DefHanLefMulRevZim2004}, it may happen that the
requirement of correct rounding conflicts with a basic mathematical
property of the function, such as its domain and range. A typical
example is the arctangent of a very large number which, rounded up,
will be a number larger than $\pi/2$ (fortunately, $\round(\pi/2) <
\pi/2$). The policy that will be implemented in \crlibm\ will be
\begin{itemize}
\item to give priority to the mathematical property in round to
  nearest mode (so as not to hurt the innocent user who may expect
  such a property to be respected), and 
\item to give priority to correct rounding in the directed rounding
  modes, in order to provide trustful bounds to interval arithmetic.
\end{itemize}

Again, this policy is open to discussion.

\section{Organization of the source code}

For recent functions implemented using triple-double arithmetic, both
quick and accurate phase are provided in a single source file,
\emph{e.g.} \texttt{exp-td.c}.

For older functions using the SCS library, each function is
implemented as two files, one with the \texttt{\_accurate} suffix (for
instance \texttt{trigo\_accurate.c}), the other named with the
\texttt{\_fast} suffix (for instance \texttt{trigo\_fast.c}).

The \emph{software carry-save} multiple-precision library is contained
in a subdirectory called \texttt{scs\_lib}.

The common C routines that are detailed in Chapter~\ref{chap:common} of
this document are defined in \texttt{crlibm\_private.c} and
\texttt{crlibm\_private.h}.

Many of the constants used in the C code have been computed thanks to
Maple procedures which are contained in the \texttt{maple}
subdirectory. Some of these procedures are explained in
Chapter~\ref{chap:common}. For some functions, a Maple procedure
mimicking the C code, and used for debugging or optimization purpose,
is also available.


The code also includes programs to test the \texttt{crlibm} functions
against MPFR, \texttt{libultim} or \texttt{libmcr}, in terms of correctness and
performance. They are located in the \texttt{tests} directory.

Gappa proof scripts are located in the \texttt{gappa} directory.

%%% Local Variables: 
%%% mode: latex
%%% TeX-master: "crlibm"
%%% End: 


\chapter{Common notations, theorems and procedures \label{chap:common}}
% Common notations, theorems and procedures

\section{Notations\label{section:notations}}

Throughout the paper, we will note, and  .





The following notations will be used throughout this document:
\begin{itemize}

\item  $+$, $-$ and  $\times$ denote the usual
mathematical operations.

\item $\oplus$, $\ominus$ and $\otimes$ denote the
corresponding floating-point operations in IEEE-754 double precision,
in the IEEE-754 \emph{round to nearest} mode.

\item $\epsilon$ (usually with some index) denotes an error and
  $\delta$ (with the same index) denotes a bound on this error.

\item $\epsilon_{-k}$ -- with a negative index -- denotes an error bounded by $2^{-k}$.
  
\item For a floating-point number $x$, the value of the least
  significant bit of its mantissa is classically denoted $\ulp(x)$.

\end{itemize}




\section{Common C procedures \label{section:commonC}}

\subsection{Sterbenz Lemma}

\begin{theorem}[Sterbenz Lemma~\cite{Ste74,Gol91}]
\label{sterbenz}
If $x$ and $y$ are floating-point numbers, and if ${y}/{2} \leq x \leq 2y$ then $x\ominus y$ is computed exactly, without any rounding error.
\end{theorem}


\subsection{Conversions}

\subsubsection{Double-precision numbers in memory\label{section:memory}}

A double precision floating-point number uses $64$ bits that is two
times the size of an integer usually represented with $32$ bits. The
order in which are stored in memory the two $32$ bits words depends on
the architecture. An architecture is said \emph{Little Endian} if the
lower part of the number is stored in memory at the smallest address;
x86 processor use this representation. Conversely, an architecture
with the high part of the number stored in memory at the smallest
address is said \emph{Big Endian}; PowerPC processor use this
representation.


The following code extract from a double precision number $x$ its
upper, and lower part.

\begin{lstlisting}[label={chap0:lst:endian},
  caption={Extract upper and lower part of a double precision number $x$},firstnumber=1]
/* LITTLE_ENDIAN/BIG_ENDIAN are defined by the user or  */
/* automatically by tools such as autoconf/automake.   */

#ifdef LITTLE_ENDIAN
#define HI(x) *(1+(int*)&x)
#define LO(x) *(int*)&x
#elif BIG_ENDIAN
#define HI(x) *(int*)&x
#define LO(x) *(1+(int*)&x)
#endif
\end{lstlisting}



\subsubsection{Conversion from floating-point to integer}

\begin{theorem}[Conversion floating-point to integer \cite{AMD01}]
  The following algorithm convert a floating-point number $d$ into an
  integer $i$ with rounding to nearest mode.

\begin{lstlisting}[label={chap0:lst:conversion2},caption={Solution 2},firstnumber=1]
#define DOUBLE2INT(i, d) \
  {double t=(d+6755399441055744.0); i=LO(t);}
\end{lstlisting}
\end{theorem}
This algorithm adds the constant $2^{52}+2^{51}$ to the floating-point
number to put the integer part of $x$, in the lower part of the
floating-point number.  We use $2^{52}+2^{51}$ and not $2^{52}$,
because the value $2^{51}$ is used to contain possible carry
propagations with negative numbers.



\subsection{Algorithms for double-double arithmetic}

In this section, we give the basic algorithms for computing with
numbers represented the sum of two floating-point numbers (or
\emph{double-double} numbers). The algorithms are given as plain C
functions, but it may be preferable, for performance issue, to
implement them as macros or inlined functions. This implementation
decision depends on the ability of the compiler.


\subsubsection{Exact sum algorithm {Add12}}

This algorithm is also known as the Fast2Sum algorithm in the litterature.
\begin{theorem}[Exact sum~\cite{Knu73, Boldo2001}]
  Let $a$ and $b$ be floating-point numbers, then the following method
  computes two floating-point numbers $s$ and $r$, such that $s+r =
  a+b$ exactly, and $s$ is the floating-point number which is closest
  to $a+b$.

\begin{lstlisting}[label={lst:Add12Cond},caption={Add12Cond},firstnumber=1]
void Add12Cond(double *s, double *r, a, b) 
{
  double z;
  s = a + b;            
  if ((HI(a)&0x7FF00000)> (HI(b)&0x7FF00000)){  
    z = s - a;           
    r = b - z;           
  }else {                 
    z = s - b;           
    r = a - z;           
  } 
}                         
\end{lstlisting}
This algorithm requires $3$ floating-point additions, $2$ masks and $1$ test over integer.
\end{theorem}


If we are able to prove that  the exponent of $a$ is always greater than that
of $b$, then the previous algorithm to perform an exact addition of 2
floating-point numbers becomes :
\begin{lstlisting}[label={lst:Add12},caption={Add12},firstnumber=1]
void Add12Cond(double *s, double *r, a, b) 
{
  double z;
  s = a + b;            
  z = s - a;  
  r = b - z; 
}            
\end{lstlisting}
The cost of this algorithm is $3$ floating-point additions.






\subsubsection{Exact product algorithm {Mul12}}

This algorithm is also known as the Dekker algorithm.

\begin{theorem}[Exact product \cite{Dek71,Knu73}]
  Let $a$ and $b$ be two floating-point numbers, with $p \geq 2$ the
  size of their mantissa. Let $c=2^{\frac{\lceil p \rceil}{2}}+1$. The
  following method computes the two floating-point numbers $r1$ and
  $r2$ such that $r1 + r2 = a + b$ with $r1 = a \otimes b$ in the case
  $p=53$ (double precision):


\begin{lstlisting}[label={lst:Mul12Cond},caption={Mul12Cond},firstnumber=1]
void Mul12Cond(double *r1, double *r2, double a, double b){
  double two_m53 = 1.1102230246251565404e-16;  /* 0x3CA00000, 0x00000000 */
  double two_53  = 9007199254740992.;          /* 0x43400000, 0x00000000 */
  double c        = 134217729.;                /* 0x41A00000, 0x02000000 */ 
  double u, up, u1, u2, v, vp, v1, v2, r1, r2;

  if (HI(a)>0x7C900000) u = a*two_m53; 
  else            u = a;
  if (HI(b)>0x7C900000) v = b*two_m53; 
  else            v = b;

  up = u*c;        vp = v*c;
  u1 = (u-up)+up;  v1 = (v-vp)+vp;
  u2 = u-u1;       v2 = v-v1;
  
  *r1 = u*v;
  *r2 = (((u1*v1-*r1)+(u1*v2))+(u2*v1))+(u2*v2)

  if (HI(a)>0x7C900000) {*r1 *= two_e53; *r2 *= two_53;} 
  if (HI(b)>0x7C900000) {*r1 *= two_e53; *r2 *= two_53;} 
} 
\end{lstlisting}

We have to test $a$ and $b$ before and after the core of the
algorithms in order to avoid overflow by multiplying by $c$. The
global cost in the worst case is $4$ tests over integers, $10$
floating-point additions and $13$ floating-point multiplications.
\end{theorem}

If we are able to prove that $a$ and $b$ are less then $2^{970}$, we can skip
this test, and get the following algorithm:
\begin{lstlisting}[label={lst:Mul12},caption={Mul12},firstnumber=1]
void inline Dekker(double *r1, double *r2, double u, double v){
  double c        = 134217729.;                /* 0x41A00000, 0x02000000 */ 
  double up, u1, u2, vp, v1, v2;

  up = u*c;        vp = v*c;
  u1 = (u-up)+up;  v1 = (v-vp)+vp;
  u2 = u-u1;       v2 = v-v1;
  
  *r1 = u*v;
  *r2 = (((u1*v1-*r1)+(u1*v2))+(u2*v1))+(u2*v2)
}
\end{lstlisting}
which reduces the cost of this algorithm to $10$ floating-point
additions and $7$ floating-point multiplications.







\subsubsection{Double-double addition {Add22}}
  
This algorithm computes the sum of two double-double numbers as a double-double.

\begin{lstlisting}[label={Add22},caption={Add22},firstnumber=1]
void Add22( double * z, double * zz, double x, double xx, double y, double yy){
double r,s;

r = x+y;
s = (ABS(x) > ABS(y))? (x-r+y+yy+xx) : (y-r+x+xx+yy);
*z = r+s;
*zz = r - (*z) + s;
}
\end{lstlisting}

Here ABS is a macro that returns the absolute value of a floating-point number.


\subsubsection{Double-double multiplication {Mul22}}

\begin{lstlisting}[label={Mul22},caption={Mul22},firstnumber=1]
void Mul22(double x, double xx, double y, double yy, double * z, double * zz){
  double c, cc;
  
  Mul12(&c, &cc, x, y);
  cc = x*yy + xx*y + cc;
  *z = c+cc;
  *zz = c - (*z) + cc;
}  
\end{lstlisting}





\subsection{Test if rounding is possible\label{section:testrounding}}

We assume here that an evaluation of $y=f(x)$ has been computed with a
total relative error smaller than $\delta$, and that the result is
available as the sum of two non-overlapping floating-point numbers
$y_h$ and $y_l$ (as is the case if computed by the previous
algorithms). This section gives and proves algorithms for testing if
$y_h$ is the correctly rounded value of $y$.


%% \subsubsection{Rounding to the nearest}
%% % All this commented out, because of the  improved test below
%% \begin{theorem}[Correct rounding of a double-double to a double]
%%   Let $y$ and $\delta$ be real numbers, and $e$, $y_h$ and $y_l$ be
%%   floating-point numbers such that 
%%   \begin{itemize}
%%   \item $y_h=y_h\oplus y_l$,
%%   \item none of $y_h$ and $y_l$ is a  NaN.
%%   \item $|y_h+y_l - y| < \delta.|y_h|$
%%   \item $0< \delta \le 2^{-54}$
%%   \item $e\ge 1+ 2^{55}\delta$ 
%% \end{itemize}

%% The following test determines whether $y_h$ is the
%%   correctly rounded value of $y$ in  round to nearest mode.

%% \begin{lstlisting}[label={roundingtonearest},
%%   caption={Test for correct rounding to nearest},
%%   firstnumber=1]
%% if( (*@$y_h$@*) == ((*@$y_h$@*) + ((*@$y_l$@*)*e)) )
%%   return (*@$y_h$@*);
%% else /* more accuracy is needed, lauch accurate phase */
%% \end{lstlisting}
%% \end{theorem}

%% \begin{proof}
%% %%   The intuition here is the following: the exact value $y=f(x)$ is in
%% %%   an interval $](y_h+y_l)(1-\delta),(y_h+y_l)(1+\delta)[$.  The
%% %%   floating-point number $e$ is precomputed such that we have a
%% %%   guarantee that $y$ is in the interval $[y_h+y_l, y_h+y_l\otimes e]$
%% %%   (or $[y_h+y_l\otimes e, y_h+y_l]$ if $y_l$ is negative).  By
%% %%   hypothesis we know that $\round(y_h + y_l) = y_h$.  The test tries
%% %%   to round the other bound of the interval.  If the rounding of this
%% %%   bound is also $y_h$, then thanks to the monotonicity of the rounding
%% %%   function, it is also the rounding of all the points in the interval,
%% %%   including $y$.
  
%% %%   We now give a more formal proof.
  
%%   The property holds if $y_h$ is $\pm \infty$.
  
%%   In the following we assume that $y_h\ge0$, as the other
%%   case is symmetric.
  
%%   Let us note  $u=\ulp(y_h)$. By definition of the \ulp,
%%   we have $y_h \in [2^{52}u, (2^{53}-1)u]$, which implies $y_h < 2^{53}u$. 


%% %% The error bound
%% %%   $|y_h+y_l - y| < \delta |y_h|$ therefore implies $|y_h+y_l - y| <
%% %%   \delta |y_h|$.

  
%%   What we want to prove is that if the test is true, then $y_h =
%%   \round(y)$. We will prove that $|y_l-\frac{1}{2}u| \ge 2^{53}\delta
%%   u$, which implies $|y_l-\frac{1}{2}u| \ge \delta y_h$, which implies
%%   that we are not in a difficult case for correct rounding to the
%%   nearest.

%%   Suppose that the test is true ($y_h \oplus y_l \otimes e = y_h$). 
%%   With IEEE-54 compliant rounding to
%%   nearest, this implies $|y_l \otimes e|
%%   \le \frac{1}{2}u$, which in turn implies $|y_l \times e|
%%   (1-2^{-53}) \le \frac{1}{2}u$. 
  
%%   Consider the case when $y_l$ is positive:  $0\le y_l \le \frac{1}{2}u$.
%%   We get $y_l \times (1+e-1)(1-2^{-53}) \le \frac{1}{2}u$, or
%%   \begin{equation}
%%   \frac{1}{2}u - y_l \ge y_l (e-1)(1-2^{-53})\label{eq:prooftestRN}
%%   \end{equation}  
%%   Now if $0 \le y_l < \frac{1}{4}u$, then since $|\delta|<2^{-54}$ we
%%   are in an easy case for rounding to the nearest, and $y_h =
%%   \round(y)$. If $y_l \ge \frac{1}{4}u$, then (\ref{eq:prooftestRN})
%%   implies $\frac{1}{2}u - y_l \ge \frac{1}{4}u (e-1)(1-2^{-53}) >
%%   2^{53}\delta u$ (since  $e-1 \ge 2^{55}\delta$).
  
%%   The case when $y_l$ is negative is similar.
%% \end{proof}

%% Note that a fused multiply-and-add may be used safely for the
%% computation of $y_h+y_l\times e$.


\subsubsection{Rounding to the nearest}

\begin{theorem}[Correct rounding of a double-double to a double]
  Let $y$ and $\delta$ be real numbers, and $e$, $y_h$ and $y_l$ be
  floating-point numbers such that 
  \begin{itemize}
  \item $y_h=y_h\oplus y_l$,
  \item none of $y_h$ and $y_l$ is a  NaN.
  \item $|y_h+y_l - y| < \delta.|y|$
  \item $0< \delta \le 2^{-53-k}$ with $k\ge 2$ integer
  \item $e\ge 1+  \dfrac{2^{53+k+1}\delta}{(2^{k}-1)(1-2^{-53})}$
\end{itemize}

The following test determines whether $y_h$ is the
  correctly rounded value of $y$ in  round to nearest mode.

\begin{lstlisting}[label={roundingtonearest},
  caption={Test for correct rounding to nearest},
  firstnumber=1]
if( (*@$y_h$@*) == ((*@$y_h$@*) + ((*@$y_l$@*)*e)) )
  return (*@$y_h$@*);
else /* more accuracy is needed, lauch accurate phase */
\end{lstlisting}
\end{theorem}

\begin{proof}
%%   The intuition here is the following: the exact value $y=f(x)$ is in
%%   an interval $](y_h+y_l)(1-\delta),(y_h+y_l)(1+\delta)[$.  The
%%   floating-point number $e$ is precomputed such that we have a
%%   guarantee that $y$ is in the interval $[y_h+y_l, y_h+y_l\otimes e]$
%%   (or $[y_h+y_l\otimes e, y_h+y_l]$ if $y_l$ is negative).  By
%%   hypothesis we know that $\round(y_h + y_l) = y_h$.  The test tries
%%   to round the other bound of the interval.  If the rounding of this
%%   bound is also $y_h$, then thanks to the monotonicity of the rounding
%%   function, it is also the rounding of all the points in the interval,
%%   including $y$.
  
%%   We now give a more formal proof.
  
  The property holds if $y_h$ is $\pm \infty$.
  
  In the following we assume that $y\ge0$ (so $y_h\ge0$), as the other
  case is symmetric.
  
  Let us note $u=\ulp(y_h)$. By definition of the \ulp, we have $y_h
  \in [2^{52}u, (2^{53}-1)u]$, which implies $y<2^{53}u$ as $y < y_h + y_l + \delta y
  < (2^{53}-1)u +\frac{1}{2}u +\frac{1}{2}u$.


%% The error bound
%%   $|y_h+y_l - y| < \delta |y_h|$ therefore implies $|y_h+y_l - y| <
%%   \delta |y_h|$.

  
  What we want to prove is that if the test is true, then $y_h =
  \round(y)$. We will prove that  if the test is true, then $|y_l-\frac{1}{2}u| > 2^{53}\delta
  u$, which implies $|y_l-\frac{1}{2}u| > \delta y$, which implies
  that we are not in a difficult case for correct rounding to the
  nearest.

  Suppose that the test is true ($y_h \oplus y_l \otimes e = y_h$). 
  With IEEE-54 compliant rounding to
  nearest, this implies $|y_l \otimes e|
  \le \frac{1}{2}u$, which in turn implies $|y_l \times e|
  (1-2^{-53}) \le \frac{1}{2}u$. 
  
  Consider the case when $y_l$ is positive:  $0\le y_l \le \frac{1}{2}u$.
  We get $y_l \times (1+e-1)(1-2^{-53}) \le \frac{1}{2}u$, or
  \begin{equation}
  \frac{1}{2}u - y_l \ge y_l (e-1)(1-2^{-53})\label{eq:prooftestRN2}
  \end{equation}  
  Now
  \begin{itemize}
  \item if $0 \le y_l < (\frac{1}{2} - \frac{1}{2^{k+1}})u$, then since
    $|\delta|<2^{-53+k}$ we are in an easy case for rounding to the
    nearest, and $y_h = \round(y)$.
  \item If $y_l \ge (\frac{1}{2} - \frac{1}{2^{k+1}})u =
    \frac{2^{k}-1}{2^{k+1}}u$, then (\ref{eq:prooftestRN2}) implies
    $\frac{1}{2}u - y_l \ge \frac{2^{k}-1}{2^{k+1}} (e-1)(1-2^{-53}) >
    2^{53}\delta u$, from $e\ge 1+
    \frac{2^{53+k+1}\delta}{(2^{k}-1)(1-2^{-53})}$.
\end{itemize}  
  The case when $y_l$ is negative is similar.
\end{proof}



\subsubsection{Rounding up}

TODO

\subsubsection{Rounding down}

TODO







\section{The Software Carry Save library}







\section{Common Maple procedures \label{section:commonMaple}}


\subsection{Conversions}

Procedure \texttt{nearest} provides the closest IEEE-double number
from input value \texttt{u}.

\begin{lstlisting}[caption={nearest},firstnumber=1]

nearest := proc(u)
local arrondi, signe, x, exposant, mantisseinfinie, mantisse;
Digits:=200:
if u = 0 then arrondi := 0
else 
  if (u < 0) then signe := -1; x := -u else signe := 1; x := u fi;
  exposant := floor(evalf(log(x)/log(2.0)));
  mantisseinfinie := x*2^(52-exposant);
  if frac(mantisseinfinie) < 0.5 then mantisse := round(mantisseinfinie)
   else
     mantisse := floor(mantisseinfinie);
     if type(mantisse,odd) then mantisse := mantisse+1 fi;
   fi;
   arrondi := signe*mantisse*2^(exposant-52)
fi;
arrondi;
end:
\end{lstlisting}



Procedure \texttt{IEEEdouble} returns the exponent, the mantissa and the
binary conversion of the closest IEEE-double number of input value \texttt{x}.

\begin{lstlisting}[caption={IEEEdouble},firstnumber=1]
IEEEdouble:=proc(x) local signe, logabsx, exposant, mantisse, mantisseinfinie, resultat; 
 Digits:=200;
if (x=0) then [0,0,0]; 
else 
 if (x<0) then signe:=-1:
 else signe:=1:
 fi:
 exposant := floor(evalf(log(signe*x)/log(2.0)));
 if (exposant>1023) then mantisse:=infinity:
 elif (exposant<-1022) then mantisse:=0:
 else 
  mantisseinfinie := signe*x*2^(52-exposant);
  if frac(mantisseinfinie) <
0.5 then mantisse := round(mantisseinfinie)
   else
     mantisse := floor(mantisseinfinie);
     if type(mantisse,odd) then mantisse := mantisse+1 fi;
   fi;
 mantisse:= signe*mantisse*2**(-52);
 Digits := 53;
 resultat:=convert(mantisse*2.0^(exposant),binary);
 fi;
[exposant, mantisse, resultat];
fi;
end:
\end{lstlisting}

Procedure \texttt{hi\_lo} returns two IEEE-double numbers $x\_hi$ and $x\_lo$ so that $x = x\_hi + x\_lo + \epsilon_{-103}$.

\begin{lstlisting}[caption={hi\_lo},firstnumber=1]

hi_lo:= proc(x)
global x_hi, x_lo, res:
local tamp:
x_hi:= nearest(evalf(x)):
res:=x-x_hi:
if (res = 0) then
  x_lo:=0:
else
  x_lo:=nearest(evalf(res)):
end if;
(x_hi,x_lo);
end:
\end{lstlisting}
\vspace{0.5cm}



Procedure "ieee2Hexa" returns the closest IEEE-double number from x, in hexadecimal.

\begin{lstlisting}[caption={ieee2Hexa},firstnumber=1]

ieee2Hexa:= proc(x)
  local signe, hex1, hex2, ma, manti, expo, expos, bina, bin1, bin2, dec1, dec2;
  if(x=0) then resultat:=["00000000","00000000"];
  elif(x=-0) then resultat:=["80000000","00000000"];
  elif(x=2) then resultat:=["40000000","00000000"];
  elif(x=-2) then resultat:=["C0000000","00000000"];
  else
   ma:=IEEEdouble(x);
   expo:=ma[1]:
   if (ma[2]<0) then 
    manti:=2**64 + 2**63+(-ma[2]-1)*2**52+(expo+1023)*2**52;
   else 
     manti:=2**64 + (ma[2]-1)*2**52+(expo+1023)*2**52;
   fi:
   hex2:=convert(manti, hex); 
   hex2:=convert(hex2, string):  
  
   resultat:=[substring(hex2,2..9), substring(hex2,10..18)];
  end if;
  resultat;
end proc:
\end{lstlisting}
\vspace{0.5cm}

Procedure "Hexa2ieee" returns the decimal IEEE-double number associated with the hexadecimal enter value "hexa".

\begin{lstlisting}[caption={Hexa2ieee},firstnumber=1]

Hexa2ieee:= proc(hexa)
local dec, bin, expobin, expo, mantis, sign, hex1, hex2, hexcat;
global res;

hex1:= op(1, hexa):
hex2:= op(2, hexa):
hexcat:= cat(hex1, hex2);
dec:= convert(hexcat, decimal, hex):

if(dec >= 2**63) then
  dec = dec - 2**63:
  sign:= -1:
else
  sign:= 1:
fi;  
expo:= trunc(dec/(2**52))-1023:
mantis:= 1+frac(dec/(2**52));
res:= evalf(sign*2**(expo)*mantis);
end proc:
\end{lstlisting}


\subsection{Procedures for polynomial approximation}


Procedure \texttt{Poly\_exact2} takes in arguments a polynomial \texttt{P} and a
integer \texttt{n}. It returns a truncated polynomial, of wich coefficients
are exactly IEEE-double numbers. The \texttt{n} first coefficients are written
over 2 IEEE-double numbers.
 

\begin{lstlisting}[caption={poly\_exact2},firstnumber=1]

poly_exact2:=proc(P,n)
local i, coef, coef_t:
global deg, Q, psup, pinf, pfull:
psup:=0: pinf:=0:
Q:=[];
convert(psup, polynom): convert(pinf, polynom):
deg:=degree(P,x):
  for i from 0 to deg do
    coef:=coeff(P,x,i):
    if (coef=0) then
      Q:= (Q,[0,0]):
    elif(coef=1) then
        Q:= (Q,[1,0]):
        psup:=psup+x^i:
    else        
      coef_t:=hi_lo(coef):
      Q:= (Q,[coef_t]):
      psup:=psup + coef_t[1]*x^i:
        if(i<n) then
        pinf:=pinf + coef_t[2]*x^i:
        fi;
    end if;
  od:
Q:=Q[2..deg+2];
pfull:=expand(psup+pinf):
end:
\end{lstlisting}
\vspace{0.5cm}



The following Maple procedure gives a bound on the accumulated
rounding errors in the evaluation of a polynomial returned by the
previous procedure, assuming all the operations are performed in
IEEE-754 round to the nearest mode, and assuming $x$ is exact.

Here is an explanation of this computation, for a polynomial of degree 4.




%\newcommand{\maxx}{{\mathrm{max}_{|x|}}}
%\newcommand{\maxs}[1]{{\mathrm{max}_{|s_{#1}|}}}

 We note 
\begin{itemize}
\item $P$ the polynomial, $d$ its degree
\item $c_j$ the coefficient of $P$ of degree $j$
\item $S_j(x)$ the intermediate polynomial in the Horner evaluation:
  $S_d=c_d$ and $S_j(x) = c_j+xS_{j+1}$ for $1\le j <d$
\item $\maxx$ the maximum value of $|x|$ over the considered interval
\item $p_j = x \otimes s_j $ 
\item $s_j =   c_j \oplus p_{j+1}$ with $s_d = c_d$ 
\item $\maxs{j}$ the  maximum value that $s_j$ may take for $|x|
  \le \maxx$.
\item $\infnorm{S_j}$ the infinite norm of $S_j$ for $-\maxx \le
  x\le \maxx$.
\end{itemize}

Given $ |x| \leq \maxx$, we have

\begin{align*}
  s_4 & \ = \ c_4\\
  \maxs{4} & \ = \  |c_4|\\
  p_4 & \ = \  x \otimes s_4 \ = \  x c_4 + \epsilon_{-53} x c_4 \\
      & \ = \  xc_4 + \epsilon_4 \quad \mathrm{with} \quad |\epsilon_4| \le \delta_4 = 2^{-53}\maxx \maxs{4}\\
  s_3 & \ = \  c_3 \oplus p_4 \ = \  c_3 + p_4 + \epsilon_{-53}(c_3 + p_4) \\
      & \ = \  c_3 +  xc_4 + \epsilon_4 + \epsilon_{-53}(c_3 + xc_4 + \epsilon_4) \\
      & \ = \  c_3 +  xc_4 + \epsilon'_3  
      \quad \mathrm{with} |\epsilon'_3| \le \delta'_3\\
  \delta'_3 & \ =\  \delta_4 + 2^{-53}(\infnorm{c_3 + xc_4} + \delta_4)\  =\  \delta_4 + 2^{-53}(\infnorm{S_3} + \delta_4)\\
  \maxs{3} & \ = \  \infnorm{c_3 + xc_4} + \delta'_3 \ = \  \infnorm{S_3} + \delta'_3\\ 
  p_3 & \ = \  x \otimes s_3 \ = \  x s_3 + \epsilon_{-53} x s_3 \\
      & \ = \  x(c_3 +  xc_4 + \epsilon'_3) + \epsilon_{-53} x s_3\\
      & \ = \  x(c_3 +  xc_4) + x\epsilon'_3 + \epsilon_{-53} x s_3\\
      & \ = \  x(c_3 +  xc_4) +  \epsilon_3 \quad \mathrm{with} \quad 
      |\epsilon_3| \le \delta_3\\
  \delta_3 & \ =\ \maxx\delta'_3 + 2^{-53} \maxx \maxs{3}\\
...
\end{align*}




 \chapter{The natural logarithm \label{chap:log}}
 
\newcommand{\middlei}{\mathrm{middle}[i]}


\section{Overview}

The goal is to compute logarithm function so that it provides
correctly rounded result in double precision. The worst-case accuracy
required for this purpose is $118$ bits according to Lef�vre and
Muller \cite{LefMul2004}.

We therefore proceed in two phases \cite{Ziv91}.  The first, quick phase
(program \texttt{log\_fast.c}) is  accurate only to $59-63$
bits.  If this is not enough to decide correct rounding, a second step
accurate to $120$ bits using the SCS library is lauched (program
\texttt{log.c}).


\subsubsection*{Definition interval and exceptional cases}

The natural logarithm is defined over positive floating point numbers.  

\begin{itemize}
\item If $x \le 0$ , then $\log(x)$ should return $NaN$
\item If $x = +\infty$ , then $\log(x)$ should return $+\infty$. 
\end{itemize}

This is true in all rounding modes.

\subsubsection*{Avoiding denormals} 

If $x < 2^{-1022}$ , ie if x is a subnormal number, then we use the equation
$$\log(x) = -52 * \log(2) + \log\left(\frac{x}{2^{-52}}\right)$$ where
$\displaystyle \frac{x}{2^{-52}}$ is now a normalized number.

As $\log(1+\epsilon) \approx \epsilon$ when $\epsilon\rightarrow 0$,
the smallest exponent of a logarithm for a double-precision input
number will be for the input values $\log(1+2^{-52})$ and
$\log(1-2^{-52})$. This ensures that the output will never be a
denormal. This will allow us to ensure that no denormal ever appears
in the computation of the logarithm of a double-precision input
number.




\section{Quick phase}

\subsection{Description of the algorithm}

The algorithm consists of an argument reduction using the well-known
property of the logarithm, and a polynomial evaluation using a degree
12 polynomial.

\subsubsection{Argument reduction and reconsruction}

It is based around the equation 
\begin{equation}
x = 2^{E} * y \label{eq:argred}
\end{equation}
where $E$ is an integer, and $y$ satisfies
\begin{equation}
\frac{11}{16}<\frac{\sqrt{2}}{2} < y < \sqrt{2}<\frac{23}{16} \quad.
\end{equation}

The final reconstruction will then use the equation
 \begin{equation}
\log(x) = E * \log(2) + \log(y) \quad.
\end{equation}

The interval $[\frac{11}{16},\frac{23}{16}]$ being too large for a
polynomial approximation of acceptable degree, it is broken down into
8 intervals given in Table~\ref{table:TablePolysLog1}.  Note that the
first four intervals are of size $2^{-4}$, while the last four are of
size $2^{-3}$.  The value of $i$, the index of the interval $X[i]$ to
which $y$ belongs, will be computed out of a few bits of $y$.

Noting $\middlei$ the middle of the $i$-th interval, the final range
reduction consists in computing $z = y - \middlei$. On each interval, a
polynomial $P[i](z)$ approximates $\log(y)$. In the following $P[i]$
will be noted $P$ when no ambiguity arises.



\begin{table}[htdp]\caption{polynomial precision\label{table:TablePolysLog1}}
\renewcommand{\arraystretch}{1.3}
\begin{center}
\begin{tabular}{|c|c|c|c|c|}
\hline
polynomial &   definition interval     &   $\middlei$   & max value of $|z|$\\
\hline  
P[0] & $[\frac{11}{16},\frac{12}{16}]$ &  $\frac{23}{32}$ & $2^{-5}$  \\ 
\hline 
P[1] & $[\frac{12}{16},\frac{13}{16}]$ &  $\frac{25}{32}$ & $2^{-5}$  \\
\hline 
P[2] & $[\frac{13}{16},\frac{14}{16}]$ &  $\frac{27}{32}$ & $2^{-5}$  \\ 
\hline 
P[3] & $[\frac{14}{16},\frac{15}{16}]$ &  $\frac{29}{32}$ & $2^{-5}$  \\ 
\hline
P[4] & $[\frac{15}{16},\frac{17}{16}]$ &  $1$             & $2^{-4}$ \\ 
\hline 
P[5] & $[\frac{17}{16},\frac{19}{16}]$ &  $\frac{18}{16}$ & $2^{-4}$  \\ 
\hline 
P[6] & $[\frac{19}{16},\frac{21}{16}]$ &  $\frac{20}{16}$ & $2^{-4}$  \\ 
\hline 
P[7] & $[\frac{21}{16},\frac{23}{16}]$ &  $\frac{22}{16}$ & $2^{-4}$  \\ 
\hline
\end{tabular}\end{center}\end{table}

\subsubsection{Polynomial approximation}

On each interval, we have a polynomial $P(z)$ of degree 12 which
approximates $\log(y)$ with an error less than $2^{-60}$.  Each
polynomial has coefficients which are exactly representable as IEEE
doubles, with the two first coefficients being exactly representable
as the sum of two doubles: $c_0 = c_0^{hi} + c_0^{lo}$ and $c1 =
c_1^{hi} + c_1^{lo}$. 


The polynomials are produced by a program in
\texttt{maple/coef\_log.mw}, which directly produces the file
\texttt{log\_fast.h}.


\subsubsection{Reconstruction}

The reconstruction computes: 
$$\log(x) \approx E\times \log(2) + P(z)$$
where $P(z)$ has been
computed by the previous step as the sum of two double-precision
numbers.  The constant $\log(2)$ is also stored as the sum of two
double-precision numbers, and $E$ is a relatively small integer. This
computation uses double-double arithmetic.


\subsubsection{Error analysis}


The polynomials are evaluated thanks to a Horner scheme:

$$P(z) = c_0^{hi}+c_0^{lo} + z .(c_1^{hi} +c_1^{lo} + z .(c_2 + z
  .(c_3 + ...
+ z .(c_{11} + z . (c_{11} + (c_{12} . z))))))))))))
$$
where the two last iterations may use double-double arithmetic if
required by the overall target accuracy, as detailed below.


The reconstruction adds another small error.  As $|E|<1024+52$, we have
$|E|\log(2)<746$, and the maximum absolute error of storing $\log(2)$
as two doubles and multiplying by $E$ is smaller than $2^{-90}$.


The program in \texttt{maple/coef\_log.mw} computes, on each interval,
the maximum approximation error $\deltapprox$ (this is a relative
error), the accumulated rounding error of the Horner scheme
$\deltaround$ (this is an absolute error) and the maximum value of the
polynomial on the interval $\maxp$. This Maple script directly outputs
the values of the rounding constants required by theorems of section
\ref{section:testrounding}.


Error analysis distinguishes four cases, corresponding to four
execution paths in the program.
\begin{enumerate}
\item {If $|E\log(2)| > \mathtt{MIN\_FASTPATH}=128.5$} then the whole
  polynomial evaluation may be performed in double-precision, with a
  total (approximation and rounding) absolute error smaller than
  $2^{-55.7}$. As the absolute value of $P(x)$ in this scheme is
  always smaller than $0.38$, this ensures a relative error on
  $E\log(2)+P(x)$ smaller than $2^{-62.8}$. This case represents a
  trade-off between accuracy (which impacts the percentage of calls to
  the \accurate\ phase), and speed.

  
\item {If $16.5<|E\log(2)|\le 128.5$} then the last two additions and
  the last multiplication of the Horner polynomial evaluation are
  performed in double-double-precision. The total relative error in
  this case is smaller than $2^{-64.8}$.
  
\item {If $0<|E\log(2)| < 16.5$} then we use double-double arithmetic
  as in the previous case, and the bound on the relative error is
  $2^{-60.2}$.
  
\item {If E=0} then the last two additions and the last two
  multiplications of the Horner polynomial evaluation are performed in
  double-double-precision. The total relative error in this case is
  smaller than $2^{-57.7}$. This is the error for the fifth
  polynomial, which has $\middlei=1$ and therefore approximates
  $\log(1+z)$. Therefore its first coefficient is exactly zero. In the
  case when $E\ne0$ it only means a smaller absolute error, and
  therefore a smaller relative error of $E\log(2)+P(x)$. However, for
  $E=0$, the overall relative error is the relative error in
  evaluating $P(x)$, and has to be computed specifically as shown in
  \ref{sec:error_maple}.
\end{enumerate}

 




\subsection{Details of computer program}

A procedure \texttt{log\_quick} contains the computation shared by the
three functions \texttt{log\_rn}, \texttt{log\_ru} and
\texttt{log\_rd} (the function \texttt{log\_rz} calls either
\texttt{log\_ru} or \texttt{log\_rd}).  This procedures returns an
approximation to the log as two double-precision numbers, and an index
in an array of constants for testing if correct rounding is possible.
This array contains the relative error for directed rounding modes
(see~\ref{th:roundingDirected} p.~\pageref{th:roundingDirected}) , and
the rounding constant computed as per Theorem~\ref{th:roundingRN1}
p.~\pageref{th:roundingRN1} for round-to nearest.


\subsubsection{Exceptional cases and argument reduction}

This part  is shown for \texttt{log\_rn}, but it is identical for the three functions.

\newpage
\begin{lstlisting}[caption={Exceptional cases},firstnumber=1]
 double log_rn(double x) { 
   db_number y;
   double res_hi,res_lo,roundcst;
   int E,rndcstindex;

   E=0;
   y.d=x;

   /* Filter cases */
   if (y.i[HI_ENDIAN] < 0x00100000){        /* x < 2^(-1022)    */
     if (((y.i[HI_ENDIAN] & 0x7fffffff)|y.i[LO_ENDIAN])==0){
       return -1.0/0.0;     
     }                                     /* log(+/-0) = -Inf */
     if (y.i[HI_ENDIAN] < 0){ 
       return (x-x)/0;                      /* log(-x) = Nan    */
     }
     /* Subnormal number */
     E = -52;           
     y.d *= two52.d;      /* make x a normal number    */ 
   }
    
   if (y.i[HI_ENDIAN] >= 0x7ff00000){
     return  x+x;                                /* Inf or Nan       */
   }
   
   /* reduce to  y.d such that sqrt(2)/2 < y.d < sqrt(2) */
   E += (y.i[HI_ENDIAN]>>20)-1023;                              /* extract the exponent */
   y.i[HI_ENDIAN] =  (y.i[HI_ENDIAN] & 0x000fffff) | 0x3ff00000;        /* do exponent = 0 */
   if (y.d > SQRT_2){
     y.d *= 0.5;
     E++;
   }

   /* Call the actual computation */
   log_quick(&res_hi, &res_lo, &rndcstindex, &y, E);
\end{lstlisting}


\begin{tabular}{ll}
Lines  6,7 &  Initialize E and y\\
Line 10 & Test if x is null, negative or a subnormal number.\\
Line 11,12 & Test if x is $\pm 0$ and return  $-\infty$ \\
Line 14,15 & If x is negative,  return NaN and raise an exception.\\
Line 18,19 & else $x$ is subnormal, then compute:\\
& $log(x) = -52\times log(2) + log(\frac{x}{2^{-52}})$ \\
& (this computation is exact) \\ 
Line 22,23 & If $x$ is $\infty$ or NaN, return $\infty$ or NaN.\\
Line 27 & E contains x 's exponent. \\
Line 28 & y.d is reduced to $\frac{x}{2^E}$. Correct because $y$ was a normal number.\\
Line 29-31 & Now, we have: $ 1 \leq y.d < 2$ and we want $ \frac{1}{\sqrt2} \leq y.d < \sqrt2$.\\
& So, if it's not the case, we update E and y.\\
\end{tabular}


\newpage
\begin{lstlisting}[caption={Procedure \texttt{log\_quick}},firstnumber=1]
static void log_quick(double *pres_hi, double *pres_lo, int* prndcstindex, db_number * py, int E) {
   double ln2_times_E_HI, ln2_times_E_LO, res_hi, res_lo;
   double z, res, P_hi, P_lo;
   int k, i;
   
    /* find the interval including y.d */
    i = ((((*py).i[HI_ENDIAN] & 0x001F0000)>>16)-6) ;
    if (i < 10)
      i = i>>1;
    else
      i = ((i-1)>>1);
    
    z = (*py).d - (middle[i]).d;  /* (exact thanks to Sterbenz Lemma) */
    

    /* Compute ln2_times_E = E*log(2)   in double-double */
    Mul22(&ln2_times_E_HI, &ln2_times_E_LO, ln2hi.d, ln2lo.d, (double)E, 0.);


    /* Now begin the polynomial evaluation of log(1 + z)      */

    res = (Poly_h[i][DEGREE]).d;

    for(k=DEGREE-1; k>1; k--){
      res *= z;
      res += (Poly_h[i][k]).d;
    }

    if((ln2_times_E_HI*ln2_times_E_HI < MIN_FASTPATH*MIN_FASTPATH)) {
      /* Slow path */
      if(E==0) {
        *prndcstindex = 0 ;
        /* In this case we start with a double-double multiplication to get enough relative accuracy */ 
        Mul12(&P_hi, &P_lo, res, z); 
        Add22(&res_hi, &res_lo, (Poly_h[i][1]).d,  (Poly_l[i][1]).d, P_hi, P_lo);
        Mul22(&P_hi, &P_lo, res_hi, res_lo, z, 0.); 
        Add22(pres_hi, pres_lo, (Poly_h[i][0]).d, (Poly_l[i][0]).d, P_hi, P_lo);
      } 
      else
        {
          if((ln2_times_E_HI*ln2_times_E_HI > 16.5*16.5))
            *prndcstindex = 2; 
          else 
            *prndcstindex =1;
          P_hi=res*z;  P_lo=0.; 
          Add22(&res_hi, &res_lo, (Poly_h[i][1]).d,  (Poly_l[i][1]).d, P_hi, P_lo);
          Mul22(&P_hi, &P_lo, res_hi, res_lo, z, 0.); 
          Add22(&res_hi, &res_lo, (Poly_h[i][0]).d, (Poly_l[i][0]).d, P_hi, P_lo);
      
        /* Add E*log(2)  */
          Add22(pres_hi, pres_lo, ln2_times_E_HI, ln2_times_E_LO, res_hi, res_lo);
        }
    }
    else { /* Fast path */
      
      *prndcstindex = 3 ;
      res =   z*((Poly_h[i][1]).d + z*res);
      Add22(&res_hi, &res_lo, (Poly_h[i][0]).d , (Poly_l[i][0]).d, res, 0.);

        /* Add E*log(2)  */
      Add22(pres_hi, pres_lo, ln2_times_E_HI, ln2_times_E_LO, res_hi, res_lo);
    }
}
\end{lstlisting}

\begin{tabular}{ll}
Line 7  & To find the interval $X_i$ containing $y$, we  look upon\\ 
        &   the last bit of the exponent and the 4 first bits of the mantissa.\\
Lines 8-11  & Reduction over $i$ in order to have an index $i$ between 0 and 7\\
        & corresponding to the 8 intervals of Table~\ref{table:TablePolysLog1}.\\
Line 13 & Let us prove that z.d is computed exactly, without rounding error: \\
& \vspace{1ex}To use Sterbenz Lemma, we need to prove that  ${\middlei}/{2} <  y.d < \middlei \times 2$.\\ 
& \vspace{1ex}For every i in $[0;7]$, we have $\middlei-\frac{1}{16} \leq y.d \leq \middlei+\frac{1}{16}$.\\
& \vspace{1ex}Table~\ref{table:TablePolysLog1} gives: $\frac{25}{32} \leq {\middlei} \leq \frac{22}{16}$\\ 
& \vspace{1ex}Therefore ${\middlei}/{2} \leq \frac{22}{32} < \frac{23}{32} \leq \middlei - \frac{1}{16} \leq y.d$\\
& and $y.d \leq \middlei + \frac{1}{16} \leq \frac{23}{16} < \frac{25}{16} \leq \middlei\times 2$\\
Line 17 & The computation of $E\log(2)$ is done as early as possible. \\
        & Neither $E$ nor  $E\log(2)$ is greater than $2^{970}$ so we may use the unconditional Mul22. \\
Lines 24-27 & Begin of Horner evaluation in double arithmetic\\
Line 29 & This test is equivalent to $|ln2\_times\_E_HI| < \mathtt{MIN\_FASTPATH}$\\
        & Currently $\mathtt{MIN\_FASTPATH}=128.5$ but the Maple programs in \texttt{coef\_log.mw}\\
        & allow to change this value.\\
Lines 31-53 & Slow path, divided two cases:\\
Lines 31-38 & ~Case $E=0$: in this case, there is no addition of $E\log(2)$ in the end, \\
            & ~~~to reach the required relative accuracy we do two $\times$ and two $+$ in double double\\
Lines 40-52 & ~Case $E\ne 0$ Thanks to the final addition to $E\log(2)$,\\
            & ~~we have to bound only the absolute accuracy of the polynomial evaluation\\
            & ~~which allows to save the first Mult22.\\
Lines 54-62 & Fast path : the polynomial evaluation is entirely in double-precision, and the \\
            & ~ final addition with a large  $E\log(2)$ scales the resulting bad absolute accuracy\\
            & ~ to acceptable relative accuracy\\ 
Lines 34,36,47 & None of the intermediate values reaches  $2^{970}$ so we may use the unconditional Mul22.\\
Lines 35,37,46,48 & The Maple procedure checks that the first argument is always greater \\
                  & than the second, so we may use the unconditional Add22.\\
Lines 51,61 & $E\log(2)$ is greater than  $\log(2)\approx 0.69$, and $\infnorm{P_i(x)}<0.4$\\
                  & so we may use the unconditional Add22.\\

\end{tabular}



\subsection{Rounding}

\subsubsection{Rounding to the nearest}

The code for rounding is strictly identical to that of
Theorem~\ref{th:roundingRN1}.  The condition to this theorem that
$\mathtt{res\_hi}\ge 2^{-1022+53}$ is ensured by the image domain of
the $\log$ over the floating-point numbers: as already mentionned, the
smallest possible value of the logarithm of a floating-point number is
larger than $2^{-53}$. As $\mathtt{res\_hi}$ approximates this value
with a relative accuracy much smaller than $1$, the condition is
ensured.


\subsection{Directed rounding}

Here again, the code is strictly identical to that of
Theorem~\ref{th:roundingDirected}, and the conditions to this theorem
are ensured by the image domain of the log.




%%%%%%%%%%%%%%%%%%%%%%%%%%%%%%%%%%%%%%%%%%%%%%%%%%%%%%%%%%%%%%%%%%%%%%%%%%%%%%%%%%%%%
\section{Accurate phase}


The function called is \texttt{scs\_log\_rn} for a result rounded to
nearest, \texttt{scs\_log\_rd} for rounding down, or
\texttt{scs\_log\_ru} for rounding up.


\subsection{Argument reduction}

Argument reduction is the same as in the first step: 
$x = 2^E \times y$,
with $y = 1 + f$, $f \leq 2^{-1}$
and  $\frac{1}{\sqrt(2)} \leq y < sqrt(2)$

Therefore the \texttt{scs\_log\_*} functions take as arguments $y$ and $E$, computed
in the first step (similarly, exceptional cases are not considered
again).

As in the first step we will compute the log using $$\log(x) = E \times log(2) + log(1+f)\quad .$$

Now we define $w_i = 1 + i\times2^{-4}$, for $i = -6 ... 6$, and we select the $w_i$ closest to $1+f$, in order to have: 

$$log(1+f) = log(w_i) + log(1+\frac{1+f-w_i}{w_i})$$

where $r=\frac{1+f-w_i}{w_i} \leq 2^{-5}$.

$log(w_i)$ and $log(2)$ are tabulated. 

\begin{lstlisting}[caption={Argument reduction},firstnumber=1]
void scs_log(scs_ptr res, db_number y, int E){ 
  scs_t R, sc_ln2_times_E, res1, addi;
  scs_ptr ti, inv_wi;
  db_number z, wi;
  int i;

 /* to normalize y.d and round to nearest      */
  /* + (1-trunc(sqrt(2.)/2 * 2^(4))*2^(-4) )+2.^(-(4+1))*/ 
  z.d = y.d + norm_number.d; 
  i = (z.i[HI_ENDIAN] & 0x000fffff);
  i = i >> 16; /* 0<= i <=11 */
  

  wi.d = (11+i)*(double)0.6250e-1;

  /* (1+f-w_i) */
  y.d -= wi.d; 
  
  /* Table reduction */
  ti     = table_ti_ptr[i]; 
  inv_wi = table_inv_wi_ptr[i];
   
  /* R = (1+f-w_i)/w_i */
  scs_set_d(R, y.d);
  scs_mul(R, R, inv_wi);

\end{lstlisting}




\subsection{Polynomial approximation}

$log(1+\frac{1+f-w_i}{w_i})$ is approximated by a polynomial
$Q(\frac{1+f-w_i}{w_i})$ with an overall relative error less than $2^{-130}$.

The polynomials are available in \texttt{log.h}.


\begin{lstlisting}[caption={Polynomial approximation},firstnumber=1]
  scs_mul(res1, constant_poly_ptr[0], R);
  for(i=1; i<20; i++){
    scs_add(addi, constant_poly_ptr[i], res1);
    scs_mul(res1, addi, R);
  }
\end{lstlisting}

TODO:
Add a Maple script that computes the relative error of a polynomial given an error on the input. It is required here.




\subsection{Reconstruction}
At the end, we compute:

\begin{equation}result = E\times log(2) + log(w_i) + Q(\frac{1+f-w_i}{w_i})\end{equation}

\begin{lstlisting}[caption={Reconstruction},firstnumber=1]
  if(E==0){
    scs_add(res, res1, ti);
  }else{
    /* sc_ln2_times_E = E*log(2)  */
    scs_set(sc_ln2_times_E, sc_ln2_ptr);

    if (E >= 0){
      scs_mul_ui(sc_ln2_times_E, (unsigned int) E);
    }else{
      scs_mul_ui(sc_ln2_times_E, (unsigned int) -E);
      sc_ln2_times_E->sign = -1;
    }
    scs_add(addi, res1, ti);
    scs_add(res, addi, sc_ln2_times_E); 
  }
}
\end{lstlisting}



\subsection{Rounding}

The procedure \texttt{scs\_log} returns an SCS number. It is rounded
in the required mode by one of \texttt{scs\_get\_d}, \texttt{scs\_get\_d\_pinf} and
\texttt{scs\_get\_d\_minf} in the corresponding functions in \texttt{log\_fast.c}. 



%%%%%%%%%%%%%%%%%%%%%%%%%%%%%%%%%%%%%%%%%%%%%%%%%%%%%%%%%%%%%
\section{Analysis of the logarithm performance}
\label{section:log_results}

\subsection{Speed}
Table \ref{tbl:log_abstime} (produced by the \texttt{crlibm\_testperf}
executable) gives absolute timings for a variety of processors and
operating systems. Contributions to this table for new
processors/OS/compiler combinations are welcome.

\begin{table}[!htb]
\begin{center}
\renewcommand{\arraystretch}{1.2}
\begin{tabular}{|l|r|r|r||r|}
\hline\hline
 \multicolumn{4}{|c|}{Pentium 4 Xeon / Linux Debian sarge / gcc 3.3}   \\ 
 \hline
                         & min time      & max time      & avg time \\ 
 \hline
 \texttt{libm}           & 724          & 7088          &        732 \\ 
 \hline
  \texttt{mpfr}          & 568          & 290924        &      83827 \\ 
 \hline
  \texttt{libultim}      & 784          & 485320        &       1062 \\ 
 \hline\hline
 \texttt{crlibm}         & 968          & 51392         &       1147 \\ 
 \hline
 \hline
 \multicolumn{4}{|c|}{Pentium III / Linux Debian woody / gcc 3.2}   \\
 \hline
                         & min time      & max time      & avg time \\
 \hline
 \texttt{libm}           & 387          & 1138          &        390 \\
 \hline
  \texttt{mpfr}          & 431          & 315982        &      92250 \\
 \hline
  \texttt{libultim}      & 376          & 217005        &        534 \\
 \hline\hline
 \texttt{crlibm}         & 383          & 22861         &        654 \\
 \hline
\end{tabular}
\end{center}
\caption{Absolute timings for the logarithm
  \label{tbl:log_abstime}}
\end{table}


In average, the second step is taken in 0.23\% of the calls, which
seems a rather good balance considering the respective costs of the
first and second steps (seen in the table as the min and max times,
respectively).


\subsection{Memory requirements}

Table size is
\begin{itemize}
\item for the \quick\ phase,
  $8\times 15\times8=960$ bytes for the eight polynomials, plus
  another $64$ bytes for the rounding constants, or a total of exactly
  1kB.
\item for the \accurate\ phase, $1+13+13+20$ SCS constants ($\log(2)$,
  the 13 $log(w_i)$ and the 13 $\frac{1}{w_i}$, and the polynomial of
  degree 20 with a null first coefficient), or a little more than 2kB.
\end{itemize}





%%%%%%%%%%%%%%%%%%%%%%%%%%%%%%%%%%%%%%%%%%%%%%%%%%%%%%%%%%%%%
\section{Conclusion and perspectives}


In the log we have a fairly good balance between both evaluation
phases, which was obtained thanks to automated testing of various
sceniarii. Here, designing a rigorous proof along with the code helped
tuning performance.

Our log is a few percent slower in average than Ziv/IBM's, but if we
inline the code of \texttt{log\_quick} then the speed becomes similar
or better: Inlining saves a function call, but also allows other minor
improvements, like removing the need for the constant index, and
loading the constants for rounding test in advance to hide its loading
time. However, as it increases the size of the code and degrades its
maintainability we chose to keep our current approach for the initial
release.


To improve performance further we should derive specific code for 
the cases when $x$ is close to $1$, as does the Ziv/IBM Library. This
could bring a further 50\% performance improvement in average, at the
cost of a more complicated proof.






%% \section{Second step polynomial coefficients.}

%% \begin{table}
%% \caption{Polynomial P 0} 
%% \begin{tabular}{|c|c|c|c|}
%% \hline &&& \\
%% coeff n� & exponent & mantissa & binary number \\ 
%% &&&\\ \hline &&& \\ 
%% 0 HI &$ -2 $ & $\frac{-743638168966267}{562949953421312}$ & $ -.10101001000101010111000000111001110001010001111011000e-1 $ \\ 
%% &&&\\ \hline &&& \\ 
%% 0 LO &$ -57 $ & $\frac{7027957893218633}{4503599627370496}$ & $ .11000111101111110001111110101101110001100000101001000e-56 $ \\ 
%% &&&\\ \hline &&& \\ 
%% 1 HI &$ 0 $ & $\frac{1566469435607129}{1125899906842624}$ & $ 1.0110010000101100100001011001000010110010000101100100 $ \\ 
%% &&&\\ \hline &&& \\ 
%% 1 LO &$ -55 $ & $\frac{3075156854129589}{2251799813685248}$ & $ .10101110110011010110101111111110101101000011101101001e-54 $ \\ 
%% &&&\\ \hline &&& \\ 
%% 2 &$ -1 $ & $\frac{-8717742945987501}{4503599627370496}$ & $ -.11110111110001011110110110011100010011110101110101101 $ \\ 
%% &&&\\ \hline &&& \\ 
%% 3 &$ -1 $ & $\frac{8086022442655257}{4503599627370496}$ & $ .11100101110100011000111001111010101110001111000011001 $ \\ 
%% &&&\\ \hline &&& \\ 
%% 4 &$ -1 $ & $\frac{-1054698579476787}{562949953421312}$ & $ -.11101111110011111000100110001011001100000100110011000 $ \\ 
%% &&&\\ \hline &&& \\ 
%% 5 &$ 0 $ & $\frac{4695701500917009}{4503599627370496}$ & $ 1.0000101011101011011100110011111011101111010100010001 $ \\ 
%% &&&\\ \hline &&& \\ 
%% 6 &$ 0 $ & $\frac{-5444291595459813}{4503599627370496}$ & $ -1.0011010101111000110111101010100110011101110011100101 $ \\ 
%% &&&\\ \hline &&& \\ 
%% 7 &$ 0 $ & $\frac{3246286994737197}{2251799813685248}$ & $ 1.0111000100001111011000101110010100100001000001011010 $ \\ 
%% &&&\\ \hline &&& \\ 
%% 8 &$ 0 $ & $\frac{-1976000318188175}{1125899906842624}$ & $ -1.1100000101001010010110110100001010101010101000111100 $ \\ 
%% &&&\\ \hline &&& \\ 
%% 9 &$ 1 $ & $\frac{4885240473419053}{4503599627370496}$ & $ 10.001010110110001100110101111000011110111110100101101 $ \\ 
%% &&&\\ \hline &&& \\ 
%% 10 &$ 1 $ & $\frac{-382458832789537}{281474976710656}$ & $ -10.101101111011000001000101101111010100000001000010000 $ \\ 
%% &&&\\ \hline &&& \\ 
%% 11 &$ 2 $ & $\frac{299676142110095}{281474976710656}$ & $ 100.01000010001101110010011111111110000101100011110000 $ \\ 
%% &&&\\ \hline &&& \\ 
%% 12 &$ 2 $ & $\frac{-5063189123827543}{4503599627370496}$ & $ -100.01111111001111000110010101001010000000011101010111 $ \\ 
%% &&&\\ \hline &&& \\ 
%% 13 &$ 7 $ & $\frac{-8903658134248167}{4503599627370496}$ & $ -11111101.000011101010001110100010010110001101011100111 $ \\ 
%% &&&\\ \hline
%% \end{tabular}
%% \end{table}

%% \begin{table}
%% \caption{Polynomial P 1} 
%% \begin{tabular}{|c|c|c|c|}
%% \hline &&&\\
%% coeff n�& exponent & mantissa & binary number \\ 
%% &&&\\ \hline &&& \\ 
%% 0 HI &$ -3 $ & $\frac{-8894071639880569}{4503599627370496}$ & $ -.11111100110010001110001101100101100111011001101111000e-2 $ \\ 
%% &&&\\ \hline &&& \\ 
%% 0 LO &$ -57 $ & $\frac{-552389204741253}{281474976710656}$ & $ -.11111011001100101001000100101001110001100100001001111e-56 $ \\ 
%% &&&\\ \hline &&& \\ 
%% 1 HI &$ 0 $ & $\frac{5764607523034235}{4503599627370496}$ & $ 1.0100011110101110000101000111101011100001010001111011 $ \\ 
%% &&&\\ \hline &&& \\ 
%% 1 LO &$ -56 $ & $\frac{-4361576903664429}{2251799813685248}$ & $ -.11110111111011010100110000100000000011100011001011001e-55 $ \\ 
%% &&&\\ \hline &&& \\ 
%% 2 &$ -1 $ & $\frac{-7378697629483821}{4503599627370496}$ & $ -.11010001101101110001011101011000111000100001100101101 $ \\ 
%% &&&\\ \hline &&& \\ 
%% 3 &$ -1 $ & $\frac{3148244321913127}{2251799813685248}$ & $ .10110010111101001111110000000111100101001001001001110 $ \\ 
%% &&&\\ \hline &&& \\ 
%% 4 &$ -1 $ & $\frac{-6044629098073239}{4503599627370496}$ & $ -.10101011110011000111011100010001100001000110010010111 $ \\ 
%% &&&\\ \hline &&& \\ 
%% 5 &$ -1 $ & $\frac{1547425048982465}{1125899906842624}$ & $ .10101111111010111111111100001011100011100111100000100 $ \\ 
%% &&&\\ \hline &&& \\ 
%% 6 &$ -1 $ & $\frac{-3301173437887807}{2251799813685248}$ & $ -.10111011101001100110010101100001101101110101001111110 $ \\ 
%% &&&\\ \hline &&& \\ 
%% 7 &$ -1 $ & $\frac{7243719460619263}{4503599627370496}$ & $ .11001101111000001111111011101010011011000001111111111 $ \\ 
%% &&&\\ \hline &&& \\ 
%% 8 &$ -1 $ & $\frac{-4056482338782501}{2251799813685248}$ & $ -.11100110100101011001011001001101011100111101001001010 $ \\ 
%% &&&\\ \hline &&& \\ 
%% 9 &$ 0 $ & $\frac{144183934662417}{140737488355328}$ & $ 1.0000011001000100111000000110101001101110001000100000 $ \\ 
%% &&&\\ \hline &&& \\ 
%% 10 &$ 0 $ & $\frac{-664564121389669}{562949953421312}$ & $ -1.0010111000110101011100100011000101100101001100101000 $ \\ 
%% &&&\\ \hline &&& \\ 
%% 11 &$ 0 $ & $\frac{7406777826722191}{4503599627370496}$ & $ 1.1010010100000110110011011001010010101100000110001111 $ \\ 
%% &&&\\ \hline &&& \\ 
%% 12 &$ 0 $ & $\frac{-1853617081185135}{1125899906842624}$ & $ -1.1010010101110110101101101110101100011100110110111100 $ \\ 
%% &&&\\ \hline &&& \\ 
%% 13 &$ 6 $ & $\frac{-1467725186095279}{1125899906842624}$ & $ -1010011.0110111000110110011001011000010110001010111100 $ \\ 
%% &&&\\ \hline &&& \\ 
%% \end{tabular}
%% \end{table}

%% \begin{table}
%% \caption{Polynomial P 2} 
%% \begin{tabular}{|c|c|c|c|}
%% \hline &&& \\
%% coeff n�& exponent & mantissa & binary number \\ 
%% &&&\\ \hline &&& \\ 
%% 0 HI &$ -3 $ & $\frac{-3060628955209433}{2251799813685248}$ & $ -.10101101111110100000001101011010101000011110110110001e-2 $ \\ 
%% &&&\\ \hline &&& \\ 
%% 0 LO &$ -61 $ & $\frac{2527594736042079}{2251799813685248}$ & $ .10001111101011010101100000100110000011000000010111110e-60 $ \\ 
%% &&&\\ \hline &&& \\ 
%% 1 HI &$ 0 $ & $\frac{667199944795629}{562949953421312}$ & $ 1.0010111101101000010010111101101000010010111101101000 $ \\ 
%% &&&\\ \hline &&& \\ 
%% 1 LO &$ -54 $ & $\frac{1332688516023975}{1125899906842624}$ & $ .10010111100000100101011100110011000000001101010011100e-53 $ \\ 
%% &&&\\ \hline &&& \\ 
%% 2 &$ -1 $ & $\frac{-6326043921025223}{4503599627370496}$ & $ -.10110011110011000000011100000101111110000100011000111 $ \\ 
%% &&&\\ \hline &&& \\ 
%% 3 &$ -1 $ & $\frac{2499177845343309}{2251799813685248}$ & $ .10001110000011111101001011111011001111000100010011010 $ \\ 
%% &&&\\ \hline &&& \\ 
%% 4 &$ -2 $ & $\frac{-8885965672331789}{4503599627370496}$ & $ -.11111100100011011110100011011011000101011101000001100e-1 $ \\ 
%% &&&\\ \hline &&& \\ 
%% 5 &$ -2 $ & $\frac{8425211896372323}{4503599627370496}$ & $ .11101111011101010111110000010001111011001110001100011e-1 $ \\ 
%% &&&\\ \hline &&& \\ 
%% 6 &$ -2 $ & $\frac{-8321196934766279}{4503599627370496}$ & $ -.11101100100000001010110100011011001110001111011000111e-1 $ \\ 
%% &&&\\ \hline &&& \\ 
%% 7 &$ -2 $ & $\frac{8453280679506371}{4503599627370496}$ & $ .11110000010000011011011001000000101101011000111000011e-1 $ \\ 
%% &&&\\ \hline &&& \\ 
%% 8 &$ -2 $ & $\frac{-4383182203135729}{2251799813685248}$ & $ -.11111001001001111011001000100011010000100110111100010e-1 $ \\ 
%% &&&\\ \hline &&& \\ 
%% 9 &$ -1 $ & $\frac{4616603552660769}{4503599627370496}$ & $ .10000011001101100011011000110110100110011110100100001 $ \\ 
%% &&&\\ \hline &&& \\ 
%% 10 &$ -1 $ & $\frac{-4925251927700411}{4503599627370496}$ & $ -.10001011111110111110110010011100101000111001110111011 $ \\ 
%% &&&\\ \hline &&& \\ 
%% 11 &$ -1 $ & $\frac{6183144258300187}{4503599627370496}$ & $ .10101111101111000100101110101000111000110100100011011 $ \\ 
%% &&&\\ \hline &&& \\ 
%% 12 &$ -1 $ & $\frac{-366830079310573}{281474976710656}$ & $ -.10100110110100001010010001001000111111010111011010000 $ \\ 
%% &&&\\ \hline &&& \\ 
%% 13 &$ 4 $ & $\frac{-2107375484601665}{1125899906842624}$ & $ -11101.111100101001011001011100101011000001010100000100 $ \\ 
%% &&&\\ \hline 
%% \end{tabular}
%% \end{table}


%% \begin{table}
%% \caption{Polynomial P 3} 
%% \begin{tabular}{|c|c|c|c|}
%% \hline &&& \\
%% coeff n�& exponent & mantissa & binary number \\ 
%% &&&\\ \hline &&& \\ 
%% 0 HI &$ -4 $ & $\frac{-7093354803841417}{4503599627370496}$ & $ -.11001001100110101111001011101010110010100100110001000e-3 $ \\ 
%% &&&\\ \hline &&& \\ 
%% 0 LO &$ -58 $ & $\frac{180242050465785}{140737488355328}$ & $ .10100011111011011101111101100100001111111111100011111e-57 $ \\ 
%% &&&\\ \hline &&& \\ 
%% 1 HI &$ 0 $ & $\frac{2484744621997515}{2251799813685248}$ & $ 1.0001101001111011100101100001000110100111101110010110 $ \\ 
%% &&&\\ \hline &&& \\ 
%% 1 LO &$ -56 $ & $\frac{4960769888924131}{4503599627370496}$ & $ .10001100111111100101100111111000010101011000111100011e-55 $ \\ 
%% &&&\\ \hline &&& \\ 
%% 2 &$ -1 $ & $\frac{-5483574338201415}{4503599627370496}$ & $ -.10011011110110100100000100100100001110010011101000111 $ \\ 
%% &&&\\ \hline &&& \\ 
%% 3 &$ -2 $ & $\frac{1008473441508307}{562949953421312}$ & $ .11100101010011001110000101001111110001101111010011000e-1 $ \\ 
%% &&&\\ \hline &&& \\ 
%% 4 &$ -2 $ & $\frac{-3338394840820967}{2251799813685248}$ & $ -.10111101110001000000100111101001101101101100111001110e-1 $ \\ 
%% &&&\\ \hline &&& \\ 
%% 5 &$ -2 $ & $\frac{5893993650002925}{4503599627370496}$ & $ .10100111100001000111001010101110100010001101111101100e-1 $ \\ 
%% &&&\\ \hline &&& \\ 
%% 6 &$ -2 $ & $\frac{-2709882287004295}{2251799813685248}$ & $ -.10011010000010011111101000110001100110011010100001110e-1 $ \\ 
%% &&&\\ \hline &&& \\ 
%% 7 &$ -2 $ & $\frac{5126082608473109}{4503599627370496}$ & $ .10010001101100010010100011100111101111010010000010100e-1 $ \\ 
%% &&&\\ \hline &&& \\ 
%% 8 &$ -2 $ & $\frac{-2474366161612937}{2251799813685248}$ & $ -.10001100101001101100010001000001101101110000100010010e-1 $ \\ 
%% &&&\\ \hline &&& \\ 
%% 9 &$ -2 $ & $\frac{606794337740459}{562949953421312}$ & $ .10001001111110000001010000111100111110011010101011000e-1 $ \\ 
%% &&&\\ \hline &&& \\ 
%% 10 &$ -2 $ & $\frac{-2678423821842551}{2251799813685248}$ & $ -.10011000010000000011001001000111010101111000011101110e-1 $ \\ 
%% &&&\\ \hline &&& \\ 
%% 11 &$ -2 $ & $\frac{4901274782692199}{4503599627370496}$ & $ .10001011010011010111011110110010111100111001101100111e-1 $ \\ 
%% &&&\\ \hline &&& \\ 
%% 12 &$ 3 $ & $\frac{5562257110400349}{4503599627370496}$ & $ 1001.1110000101101011111100111001101101111110101011101 $ \\ 
%% &&&\\ \hline &&& \\ 
%% 13 & $ 0 $ & $ 0 $ & $ 0 $ \\ 
%% &&&\\ \hline 
%% \end{tabular}
%% \end{table}



%% \begin{table}
%% \caption{Polynomial P 4} 
%% \begin{tabular}{|c|c|c|c|}
%% \hline &&& \\
%% coeff n�& exponent & mantissa & binary number \\ 
%% &&&\\ \hline &&& \\ 
%% 0 HI & $ 0 $ & $ 0 $ & $ 0 $ \\ 
%% &&&\\ \hline &&& \\ 
%% 0 LO & $ 0 $ & $ 0 $ & $ 0 $ \\ 
%% &&&\\ \hline &&& \\ 
%% 1 HI & $ 0 $ & $ 1 $ & $ 1. $ \\ 
%% &&&\\ \hline &&& \\ 
%% 1 LO &$ -73 $ & $\frac{4866424671436317}{4503599627370496}$ & $ .10001010010011111110011001001110111011111111000011100e-72 $ \\ 
%% &&&\\ \hline &&& \\ 
%% 2 & $ -1 $ & $ -1 $ & $ -.10000000000000000000000000000000000000000000000000000 $ \\ 
%% &&&\\ \hline &&& \\ 
%% 3 &$ -2 $ & $\frac{6004799503160661}{4503599627370496}$ & $ .10101010101010101010101010101010101010101010101010100e-1 $ \\ 
%% &&&\\ \hline &&& \\ 
%% 4 &$ -2 $ & $\frac{-4503599627370601}{4503599627370496}$ & $ -.10000000000000000000000000000000000000000000001101000e-1 $ \\ 
%% &&&\\ \hline &&& \\ 
%% 5 &$ -3 $ & $\frac{7205759403793217}{4503599627370496}$ & $ .11001100110011001100110011001100110011001101101000000e-2 $ \\ 
%% &&&\\ \hline &&& \\ 
%% 6 &$ -3 $ & $\frac{-1501199875682485}{1125899906842624}$ & $ -.10101010101010101010101010101010011101100001011010100e-2 $ \\ 
%% &&&\\ \hline &&& \\ 
%% 7 &$ -3 $ & $\frac{5146971002059239}{4503599627370496}$ & $ .10010010010010010010010010010001111110011100111100111e-2 $ \\ 
%% &&&\\ \hline &&& \\ 
%% 8 &$ -3 $ & $\frac{-1125900001289929}{1125899906842624}$ & $ -.10000000000000000000000010110100001001001101100100100e-2 $ \\ 
%% &&&\\ \hline &&& \\ 
%% 9 &$ -4 $ & $\frac{8006400287554951}{4503599627370496}$ & $ .11100011100011100011101010101000100011011110110000110e-3 $ \\ 
%% &&&\\ \hline &&& \\ 
%% 10 &$ -4 $ & $\frac{-3602718697199167}{2251799813685248}$ & $ -.11001100110010100111010100000010110000010110001111110e-3 $ \\ 
%% &&&\\ \hline &&& \\ 
%% 11 &$ -4 $ & $\frac{818791763735267}{562949953421312}$ & $ .10111010001010111111010000001110001111001011100011000e-3 $ \\ 
%% &&&\\ \hline &&& \\ 
%% 12 &$ -4 $ & $\frac{-6070595428018661}{4503599627370496}$ & $ -.10101100100010010110010100100000011111110100111100100e-3 $ \\ 
%% &&&\\ \hline &&& \\ 
%% 13 &$ -4 $ & $\frac{2804528659819663}{2251799813685248}$ & $ .10011111011010110100001101101000011111101100100011110e-3 $ \\ 
%% &&&\\ \hline
%% \end{tabular}
%% \end{table}


%% \begin{table}
%% \caption{Polynomial P 5} 
%% \begin{tabular}{|c|c|c|c|}
%% \hline &&& \\
%% coeff n�& exponent & mantissa & binary number \\ 
%% &&&\\ \hline &&& \\ 
%% 0 HI &$ -4 $ & $\frac{4243581083941235}{2251799813685248}$ & $ .11110001001110000011101101110001010101111001011100110e-3 $ \\ 
%% &&&\\ \hline &&& \\ 
%% 0 LO &$ -60 $ & $\frac{-194419322244045}{140737488355328}$ & $ -.10110000110100101100011010100011110010111100110100000e-59 $ \\ 
%% &&&\\ \hline &&& \\ 
%% 1 HI &$ -1 $ & $\frac{2001599834386887}{1125899906842624}$ & $ .11100011100011100011100011100011100011100011100011100 $ \\ 
%% &&&\\ \hline &&& \\ 
%% 1 LO &$ -55 $ & $\frac{6780843324223619}{4503599627370496}$ & $ .11000000101110010010000011101100111100011000010000011e-54 $ \\ 
%% &&&\\ \hline &&& \\ 
%% 2 &$ -2 $ & $\frac{-222399981598543}{140737488355328}$ & $ -.11001010010001011000011111100110101101110100111100000e-1 $ \\ 
%% &&&\\ \hline &&& \\ 
%% 3 &$ -3 $ & $\frac{4217362614017791}{2251799813685248}$ & $ .11101111101110101011010000000111111101011110111111110e-2 $ \\ 
%% &&&\\ \hline &&&\\ 
%% 4 &$ -3 $ & $\frac{-2811575076012255}{2251799813685248}$ & $ -.10011111110100011100110101011010101001000000110111110e-2 $ \\ 
%% &&&\\ \hline &&&\\ 
%% 5 &$ -4 $ & $\frac{3998684548554481}{2251799813685248}$ & $ .11100011010011001000010011000001010111011110111100001e-3 $ \\ 
%% &&&\\ \hline &&&\\ 
%% 6 &$ -4 $ & $\frac{-2961988555123393}{2251799813685248}$ & $ -.10101000010111101001101100111010100000101010110000001e-3 $ \\ 
%% &&&\\ \hline &&&\\ 
%% 7 &$ -4 $ & $\frac{4513513385732155}{4503599627370496}$ & $ .10000000010010000010000111010000010111010100000111011e-3 $ \\ 
%% &&&\\ \hline &&&\\ 
%% 8 &$ -5 $ & $\frac{-7021014623365341}{4503599627370496}$ & $ -.11000111100011001001101011010100101101010100011011101e-4 $ \\ 
%% &&&\\ \hline &&&\\ 
%% 9 &$ -5 $ & $\frac{5541480771625565}{4503599627370496}$ & $ .10011101011111111001010001000111011000010001001011101e-4 $ \\ 
%% &&&\\ \hline &&&\\ 
%% 10 &$ -6 $ & $\frac{-138651326119215}{70368744177664}$ & $ -.11111100001101001000111000010100111110100101111000000e-5 $ \\ 
%% &&&\\ \hline &&&\\ 
%% 11 &$ -5 $ & $\frac{2405838712720693}{2251799813685248}$ & $ .10001000110000011000111110011101101111001101001101001e-4 $ \\ 
%% &&&\\ \hline &&&\\ 
%% 12 &$ -6 $ & $\frac{-3033201869659221}{2251799813685248}$ & $ -.10101100011010101110010101111111100100011100010101010e-5 $ \\ 
%% &&&\\ \hline &&&\\ 
%% 13 &$ -1 $ & $\frac{-5882258564419063}{4503599627370496}$ & $ -.10100111001011110001000001100001101011000110111110111 $ \\ 
%% &&&\\ \hline 
%% \end{tabular}
%% \end{table}


%% \begin{table}
%% \caption{Polynomial P 6} 
%% \begin{tabular}{|c|c|c|c|}
%% \hline &&& \\
%% coeff n�& exponent & mantissa & binary number \\ 
%% &&&\\ \hline &&& \\ 
%% 0 HI &$ -3 $ & $\frac{4019796858195217}{2251799813685248}$ & $ .11100100011111111011111000111100110101001101000100001e-2 $ \\ 
%% &&&\\ \hline &&& \\ 
%% 0 LO &$ -57 $ & $\frac{-2950359927797621}{2251799813685248}$ & $ -.10100111101101010110001100110111010010110111011101001e-56 $ \\ 
%% &&&\\ \hline &&& \\ 
%% 1 HI &$ -1 $ & $\frac{3602879701896397}{2251799813685248}$ & $ .11001100110011001100110011001100110011001100110011010 $ \\ 
%% &&&\\ \hline &&& \\ 
%% 1 LO &$ -55 $ & $\frac{-3752959905019767}{2251799813685248}$ & $ -.11010101010101001100000100011111111010110111011101110e-54 $ \\ 
%% &&&\\ \hline &&& \\ 
%% 2 &$ -2 $ & $\frac{-5764607523034235}{4503599627370496}$ & $ -.10100011110101110000101000111101011100001010001111011e-1 $ \\ 
%% &&&\\ \hline &&& \\ 
%% 3 &$ -3 $ & $\frac{6148914691236995}{4503599627370496}$ & $ .10101110110000110011111000011111011001110010010000011e-2 $ \\ 
%% &&&\\ \hline &&& \\ 
%% 4 &$ -4 $ & $\frac{-7378697629484743}{4503599627370496}$ & $ -.11010001101101110001011101011000111000100011011000110e-3 $ \\ 
%% &&&\\ \hline &&& \\ 
%% 5 &$ -4 $ & $\frac{4722366480912847}{4503599627370496}$ & $ .10000110001101111011110100000100110000001000111001111e-3 $ \\ 
%% &&&\\ \hline &&& \\ 
%% 6 &$ -5 $ & $\frac{-393530540111015}{281474976710656}$ & $ -.10110010111101001111110000000110100110100101001110000e-4 $ \\ 
%% &&&\\ \hline &&& \\ 
%% 7 &$ -6 $ & $\frac{4317595647888997}{2251799813685248}$ & $ .11110101011011010100100100101110000001001110011001010e-5 $ \\ 
%% &&&\\ \hline &&& \\ 
%% 8 &$ -6 $ & $\frac{-6044631180159527}{4503599627370496}$ & $ -.10101011110011000111101011110010010101010101000100111e-5 $ \\ 
%% &&&\\ \hline &&& \\ 
%% 9 &$ -7 $ & $\frac{8590943574074681}{4503599627370496}$ & $ .11110100001010110101011101100000001001010010100111000e-6 $ \\ 
%% &&&\\ \hline &&& \\ 
%% 10 &$ -7 $ & $\frac{-6188699963576117}{4503599627370496}$ & $ -.10101111111001001011011111110111100110100111100110101e-6 $ \\ 
%% &&&\\ \hline &&& \\ 
%% 11 &$ -7 $ & $\frac{5702144446736261}{4503599627370496}$ & $ .10100010000100001000111110110001101001010011110000100e-6 $ \\ 
%% &&&\\ \hline &&& \\ 
%% 12 &$ -8 $ & $\frac{-424997866135913}{281474976710656}$ & $ -.11000001010001000100001101101011101100001011010001111e-7 $ \\ 
%% &&&\\ \hline &&& \\ 
%% 13 &$ -3 $ & $\frac{-5757341135212823}{4503599627370496}$ & $ -.10100011101000100010101110001001101110001000100010111e-2 $ \\ 
%% &&&\\ \hline 
%% \end{tabular}
%% \end{table}


%% \begin{table}
%% \caption{Polynomial P 7} 
%% \begin{tabular}{|c|c|c|c|}
%% \hline &&& \\
%% coeff n�& exponent & mantissa & binary number \\ 
%% &&&\\ \hline &&& \\
%% 0 HI &$ -2 $ & $\frac{2868376209600353}{2251799813685248}$ & $ .10100011000011000101111000010000111000101111011000010e-1 $ \\ 
%% &&&\\ \hline &&& \\ 
%% 0 LO &$ -56 $ & $\frac{8799248525173221}{4503599627370496}$ & $ .11111010000101101111010110010101000100010000111100100e-55 $ \\ 
%% &&&\\ \hline &&& \\ 
%% 1 HI &$ -1 $ & $\frac{3275345183542179}{2251799813685248}$ & $ .10111010001011101000101110100010111010001011101000110 $ \\ 
%% &&&\\ \hline &&& \\ 
%% 1 LO &$ -56 $ & $\frac{-6555200597316529}{4503599627370496}$ & $ -.10111010010011110101110010011000010001101001110110001e-55 $ \\ 
%% &&&\\ \hline &&& \\ 
%% 2 &$ -2 $ & $\frac{-297758653049289}{281474976710656}$ & $ -.10000111011001111010101101011111001101001110010010000e-1 $ \\ 
%% &&&\\ \hline &&& \\ 
%% 3 &$ -3 $ & $\frac{2309885308503577}{2251799813685248}$ & $ .10000011010011010100000101010100100100000110000110001e-2 $ \\ 
%% &&&\\ \hline &&& \\ 
%% 4 &$ -4 $ & $\frac{-5039749764007811}{4503599627370496}$ & $ -.10001111001111010000000101110011100001100011110000011e-3 $ \\ 
%% &&&\\ \hline &&& \\ 
%% 5 &$ -5 $ & $\frac{1466109022249467}{1125899906842624}$ & $ .10100110101011010110001101101111001001000011111101100e-4 $ \\ 
%% &&&\\ \hline &&& \\ 
%% 6 &$ -6 $ & $\frac{-7108407380574949}{4503599627370496}$ & $ -.11001010000010000111100010000110101101000001011100101e-5 $ \\ 
%% &&&\\ \hline &&& \\ 
%% 7 &$ -7 $ & $\frac{8862430084207537}{4503599627370496}$ & $ .11111011111000101010101001101000001000001001110110001e-6 $ \\ 
%% &&&\\ \hline &&& \\ 
%% 8 &$ -7 $ & $\frac{-5639728263573257}{4503599627370496}$ & $ -.10100000010010100110110001111110001111010111100001000e-6 $ \\ 
%% &&&\\ \hline &&& \\ 
%% 9 &$ -8 $ & $\frac{3645840892850449}{2251799813685248}$ & $ .11001111001111011111011110110101001101110001000100010e-7 $ \\ 
%% &&&\\ \hline &&& \\ 
%% 10 &$ -8 $ & $\frac{-2386362174521563}{2251799813685248}$ & $ -.10000111101001100010001111011011011111000000110110110e-7 $ \\ 
%% &&&\\ \hline &&& \\ 
%% 11 &$ -9 $ & $\frac{3173561151701001}{2251799813685248}$ & $ .10110100011001010110010010010001111011000000000010010e-8 $ \\ 
%% &&&\\ \hline &&& \\ 
%% 12 &$ -10 $ & $\frac{-8467205555863361}{4503599627370496}$ & $ -.11110000101001110000011101011011001011011111101000000e-9 $ \\ 
%% &&&\\ \hline &&& \\ 
%% 13 & $ 0 $ & $ 0 $ & $ 0 $ \\ 
%% &&&\\ \hline 
%% \end{tabular}
%% \end{table}

%% Erreurs approx

%% P0:
%% -63.0335561543503153519561060236275821115193176282828598936000091969791396082118533396$

%% P1:
%% -62.1442297174139549747120885846224119378071897891851912972827117256112491790838501306$

%% P2:
%% -61.2333262205549290003996733087260577169790459154067176452546458667687388357147177354$

%% P3:
%% -60.4190986335205152072260108229699448188586300049015401437646252096816380253547811958$

%% P4:
%% -62.9461911106917443178043861037065674873169791834709346433334048487967658174065656673$

%% P5:
%% -60.2165945189274221232763599838279416602051132104701245665550547127176437549725373247$

%% P6:
%% -61.6071069727527318902661883152800182616307327795335555694099125712238629376830307421$

%% P7:
%% -63.0374421600999012862137048917495991411594406510710431721052393396795383773111066083$



 \chapter{The exponential \label{chap:exp}}
 This chapter is contributed by David Defour, with assistance from F.
de~Dinechin and J.M. Muller.

Warning: This chapter was the first to be written, while the library
was in prototype stage. It may therefore present some inconsistencies
in notations with the other chapters. 

\section{Overview of the algorithm}
\label{section:overview}
We are now going to present and proof the correct rounding of the evaluation scheme chosen for the exponential within \emph{crlibm}.

We have done the evaluation of the exponential in two steps. First, we use the \emph{quick} phase of the algorithm to get an approximation good to $68$ bits of the result. Then we perform a test to check whether we need to use the \emph{accurate} phase, based on multiprecision operators from SCSlib~\cite{SCSweb} .

To increase the trust of the code, with have included constants in hexadecimal format, and for concision reason only the one in big endian format are present. However to help the reader we are giving the corresponding decimal values.

\section{Quick phase}

Here is the general scheme chosen for the first step of the evaluation of the exponential : 

\begin{enumerate}
\item 
{\bf ``Mathematical'' range reduction} \\
We want to compute $\exp(x)$. We evaluate the reduced argument $$(r\_hi + r\_lo)\in [-\ln(2)/2, +\ln(2)/2]$$ such that :

$$x=k. \ln(2) + (r\_hi + r\_lo)$$

then $\exp(x) = \exp(r\_hi + r\_lo) . 2^{k}$ with $(r\_hi + r\_lo) \in [-\ln(2)/2, +\ln(2)/2]$
\item
{\bf Tabular range reduction} \\

Let $index\_flt$ be the first 8 bits of $(r\_hi + r\_lo)$, we want

$$\exp(r\_hi + r\_lo) = \exp(index\_flt) \times \exp(rp\_hi + rp\_lo)$$

where $(rp\_hi + rp\_lo) = (r\_hi + r\_lo) - index\_flt$ such that $(rp\_hi + rp\_lo) \in [-2^{-9}, +2^{9}]$ and $\exp(index\_flt)=(ex\_hi + ex\_lo)$ is obtain with a table lookup.


\item
{\bf Polynomial evaluation} \\
We evaluate the polynom $P\_r$ of degree 3 such that : 

$$exp(rp\_hi + rp\_lo) \approx 1 + (rp\_hi + rp\_lo) + \frac{1}{2}.(rp\_hi+rp\_lo)^2 +(rp\_hi+rp\_lo)^3 . (P\_r)$$

with
$P\_r = c_0 + c_1 .rp\_hi + c_2 .rp\_hi^2 + c_3 .rp\_hi^3$
and
$rp\_hi \in [-2^{-9}, +2^{9}]$

\item
{\bf Reconstruction} \\

$$
\begin{array}{rcl}
\exp(x) &=& 2^k . (ex\_hi+ex\_lo). \\
 &&(1 + (rp\_hi+rp\_lo) + \frac{1}{2}.(rp\_hi+rp\_lo)^2 + (rp\_hi+rp\_lo)^3.P\_r)
. (1+\epsilon_{-68})\\
\end{array}
$$

\end{enumerate}




\subsection{Handling special cases}

\subsubsection{Select the evaluation range to avoid overflows and underflows}
\label{chap3:exp:overflows}
In the sequel of this paper, we will consider input numbers in the range $[u\_bound, o\_bound]$, where $u\_bound$ and $o\_bound$ are :


$$u\_bound = \bigtriangleup \left(\ln\left(\left(1-2^{-53}\right).2^{-1075}\right)\right) = -745.1332\ldots$$
$$o\_bound = \bigtriangledown \left( \ln \left(\left(1-2^{-53}\right).2^{1024}\right)\right) = 709.7827\ldots$$

In the rounding mode to nearest, the exponential of a number greater than $o\_bound$ is an overflow, whereas the exponential of a number less than $u\_bound$ is rounded to $0$, and raise an inexact flag.

However, subtler under/overflow situations may arise in two cases, which we should avoid :
\begin{itemize}
\item An intermediate computation may raise an overflow although the final result is representable as an IEEE-754 floating-point number.

\item In IEEE-754 arithmetic, when a result is between $2^{-1023}$ and $2^{-1074}$, a gradual underflow exception arises to signal that the precision of the result is reduced in a drastic way.
\end{itemize}

In both cases, as we will show in the following, it is possible to avoid the exception by predicting that it will occur, and appropriately scaling the input number in the range reduction phase.



\subsubsection{Rounding to nearest}

\begin{lstlisting}[caption={Handling special cases in rounding to nearest},firstnumber=1]
static const union{int i[2]; double d;}
#ifdef BIG_ENDIAN
 _largest    = {0x7fefffff, 0xffffffff},
 _smallest   = {0x00000000, 0x00000001},
 _u_bound    = {0xC0874910, 0xD52D3052}, /* -7.45133219101941222107e+02 */
 _o_bound    = {0x40862E42, 0xFEFA39F0}; /*  7.09782712893384086783e+02 */
#else
 ...
#endif
#define largest    _largest.d
#define smallest   _smallest.d
#define u_bound    _u_bound.d
#define o_bound    _o_bound.d

unsigned int hx;

hx  = HI(x);                                         (*@ \label{exp:code:1} @*)
hx &= 0x7fffffff;                                    (*@ \label{exp:code:3} @*)

/* Filter special cases */
if (hx >= 0x40862E42){                               (*@ \label{exp:code:4} @*)
  if (hx >= 0x7ff00000){                             (*@ \label{exp:code:5} @*)
    if (((hx&0x000fffff)|LO(x))!=0)                  (*@ \label{exp:code:6} @*)
      return x+x;                            /* NaN */  (*@ \label{exp:code:7} @*)
    else return ((hx&0x80000000)==0)? x:0.0; /* exp(+/-inf) = inf,0 */(*@ \label{exp:code:8} @*)
  }
  if (x > o_bound) return largest *largest ; /* overflow  */  (*@ \label{exp:code:9} @*)
  if (x < u_bound) return smallest*smallest; /* underflow */  (*@ \label{exp:code:10} @*)
}


if (hx <= 0x3C900000) return 1.;             /* if (hx <= 2^(-54)) */ (*@ \label{exp:code:10bis} @*)

\end{lstlisting}


\begin{preuve}
\begin{longtable}[c]{@{line }p{0.08\textwidth}p{0.81\textwidth}}
% en cas de cesure c'est ce que l'on place 
% en debut de page
\endhead
% en fin de page
\endfoot 
% en fin de derniere page
\endlastfoot 
\ref{exp:code:1} & Put the high part of $x$ in $hx$.
\\
\ref{exp:code:3} & Remove the sign information within $hx$. It will make tests on special cases simpler.
\\
\ref{exp:code:4} & Test equivalent to $if (|x|>=709.7822265625)$. This test is true if $x>u\_bound$, $x<o\_bound$, $x=\pm inf$ or $x=NaN$. This test is performed with integers to make it faster.
\\
(\ref{exp:code:5}-\ref{exp:code:7}) & Test if $x=\pm inf$ or $x=NaN$ and give the corresponding results (exact $+\infty$ or $0$).
\\
\ref{exp:code:9} & Under the assumption that the compiler correctly translates the floating-point number we have $o\_bound = 390207173010335/549755813888$. If $x>o\_bound$ then $\exp(x)=+\inf$. The multiplication $largest*largest$ leaves to the compiler the generation of an overflow and the corresponding flags. 
\\
\ref{exp:code:10} & Under the assumption that the compiler correctly translates the floating-point number we have $u\_bound = -3277130554578985/4398046511104$. If $x<u\_bound$ then $\exp(x)=+0$.  The multiplication $largest*largest$ leaves to the compiler the generation of an underflow and the corresponding flags.
\\
\ref{exp:code:10bis} & Test equivalent to $if (|x| \leq 2^{-54})$. This test is performed with integers to make it faster and is valid because $x \notin \{NaN, \infty\}$. In addition, this test allows to handle cases when $x$ is a denormalized number. We have the following property : 
\begin{prop}
  \label{chap3:exp:prop10bis}
  $|x| > 2^{-54}$~~and~~~~$x\notin\{NaN, \infty\}$
\end{prop}
Indeed, in rounding to nearest, if $|x| \leq 2^{-54}$ then $\exp(x) = 1.0$. This test prevents to encounter a denormalized number in the rest of the program.

\\
\end{longtable}
\end{preuve}




\subsubsection{Rounding toward $+ \infty$}

\begin{lstlisting}[caption={Handling special cases in rounding toward $+\infty$},firstnumber=1]
static const union{int i[2]; double d;}
#ifdef BIG_ENDIAN
 _largest    = {0x7fefffff, 0xffffffff},
 _smallest   = {0x00000000, 0x00000001},
 _u_bound    = {0xC0874910, 0xD52D3052}, /* -7.45133219101941222107e+02 */
 _o_bound    = {0x40862E42, 0xFEFA39F0}, /*  7.09782712893384086783e+02 */
 _two_m52_56 = {0x3CB10000, 0x00000000}; /*  2.35922392732845764840e-16 */
#else
 ...
#endif
#define largest    _largest.d
#define smallest   _smallest.d
#define u_bound    _u_bound.d
#define o_bound    _o_bound.d
#define two_m52_56 _two_m52_56.d


unsigned int hx;

hx  = HI(x);    
hx &= 0x7fffffff;

/* Filter special cases */
if (hx >= 0x40862E42){
  if (hx >= 0x7ff00000){
    if (((hx&0x000fffff)|LO(x))!=0)
      return x+x;                            /* NaN */
    else return ((hx&0x80000000)==0)? x:0.0; /* exp(+/-inf) = inf,0 */
  }
  if (x > o_bound) return largest*largest;   /* overflow  */
  if (x < u_bound) return smallest*1.0;      /* 2^(-1074) */
}

if (hx < 0x3CA00000){                        /* if (hx <= 2^(-53)) */
  if (HI(x) < 0)
    return 1. + smallest;                    /* 1 and inexact */
  else
    return 1. + two_m52_56;                  /* 1 + 2^(-52) and inexact */ 
}

\end{lstlisting}


\begin{preuve}
This program is similar to the one used in rounding to nearest mode with the exception of :
\begin{itemize}
\item
When $(x < u\_bound)$, in rounding toward $+\infty$, we have to give as result the smallest representable number ($2^{-1074}$) with the inexact flag raised.
\item
When $(|x| < 2^{-53})$, in rounding toward $+\infty$, we have to give as result $1.0$ if $x<0$ with the inexact flag raised or $1+2^{-52}$ with the inexact flag raised if $x>0$.
\end{itemize}
\end{preuve}



\subsubsection{Rounding toward $- \infty$}

\begin{lstlisting}[caption={Handling special cases in rounding toward $- \infty$},firstnumber=1]
static const union{int i[2]; double d;}
#ifdef BIG_ENDIAN
 _largest    = {0x7fefffff, 0xffffffff},
 _smallest   = {0x00000000, 0x00000001},
 _u_bound    = {0xC0874910, 0xD52D3052}, /* -7.45133219101941222107e+02 */
 _o_bound    = {0x40862E42, 0xFEFA39F0}, /*  7.09782712893384086783e+02 */
 _two_m52_56 = {0x3CB10000, 0x00000000}; /*  2.35922392732845764840e-16 */
#else
 ...
#endif
#define largest    _largest.d
#define smallest   _smallest.d
#define u_bound    _u_bound.d
#define o_bound    _o_bound.d
#define two_m52_56 _two_m52_56.d

unsigned int hx;

hx  = HI(x);
hx &= 0x7fffffff;

/* Filter special cases */
if (hx >= 0x40862E42){
  if (hx >= 0x7ff00000){
    if (((hx&0x000fffff)|LO(x))!=0)
      return x+x;                            /* NaN */
    else return ((hx&0x80000000)==0)? x:0.0; /* exp(+/-inf) = inf,0 */
  }
  if (x > o_bound) return largest*1.0;       /* (1-2^(-53))*2^1024  */ 
  if (x < u_bound) return smallest*smallest; /* underflow */
}

if (hx < 0x3CA00000){                        /* if (hx <= 2^(-53))   */
  if (HI(x) < 0)
    return 1. - two_m52_56;                  /* 1-2^(-52) and inexact */
  else
    return 1. + smallest;                    /* 1 and inexact */
}

\end{lstlisting}



\begin{preuve}
This program is similar to the one used in rounding to nearest mode with the exception of :
\begin{itemize}
\item
When  $(x > o\_bound)$, in rounding toward $-\infty$, we have to give as result the largest representable number  ($(1-2^{-53}).2^{1024}$) with the inexact flag raised.
\item
When $(|x| < 2^{-53})$, in rounding toward $-\infty$, we have to give as result  $1.0 - 2^{-52}$ if $x<0$ with the inexact flag raised or $1.0$ with the inexact flag raised if $x>0$.
\end{itemize}
\end{preuve}



\subsubsection{Rounding toward $0$}

The exponential function is continue and positive, therefore the rounding toward $0$ mode is equivalent to the rounding toward $- \infty$ mode.


\subsection{Range reduction}
\subsubsection{Presentation}
The typical range reduction used for the exponential use the following property :
$$
e^{a+b}=e^a e^b
$$

\subsubsection{First reduction step}
The purpose of this first range reduction is to replace the input number $x\in[u\_bound, o\_bound]$ with two floating-point numbers $r\_hi$, $r\_lo$ and an integer $k$ such that :

$$
x = k.\ln(2) + (r\_hi+r\_lo).(1+ \epsilon)
$$
with  $|r\_hi+r\_lo| < \frac{1}{2}  \ln(2)$

This ``additive'' range reduction may generate a cancellation if $x$ is close from a multiple of $\ln(2)$. Using a method from Kahan based on continuous fractions (see Muller \cite{Muller97} pp 154) we compute the worst cases for the range reduction and give the results in Table \ref{chap3:exp:table:reduction}.

\begin{table}[h]
\begin{center}
\begin{tabular}{|c|c|c|}
\hline  Interval &  Worst cases & Number of bits lost  \\ \hline
\hline  $]2^{1024}, 2^{1024}[$& $5261692873635770 \times 2^{499}$ &  $66,8$\\
\hline  $[-1024, 1024]$&  $7804143460206699 \times 2^{-51}$ & $57,5$\\
\hline
\end{tabular}
\caption{Worst cases corresponding to the closest number multiple of  $\ln(2)$, for the additive range reduction of the exponential. The maximum number of bits lost by cancellation is also indicated \label{chap3:exp:table:reduction}.}
\end{center}
\end{table}

The interval $[u\_bound, o\_bound]$ on which we are evaluating the exponential is included within $[-1024, 1024]$. Then  at most $58$ bits can be  canceled during the subtraction of the closest multiple of  $\ln(2)$ to the input number $x$.


\begin{theorem}
The sequence of instructions of the program \ref{exp:lst:reduction1} computes two floating-point numbers in double precision $r\_hi$, $r\_lo$ and an integer $k$ such that
$$
r\_hi + r\_lo = (x - k\times \ln2) + \epsilon_{-69}
$$
with $k$ the closest integer to $x / \ln 2$.
\end{theorem}

\begin{lstlisting}[label={exp:lst:reduction1},caption={First range reduction}]
static const union{int i[2]; double d;}
#ifdef BIG_ENDIAN
 _ln2_hi     = {0x3FE62E42, 0xFEFA3800}, /*  6.93147180559890330187e-01 */ (*@ \label{exp:code:001} @*)
 _ln2_me     = {0x3D2EF357, 0x93C76000}, /*  5.49792301870720995198e-14 */
 _ln2_lo     = {0x3A8CC01F, 0x97B57A08}, /*  1.16122272293625324218e-26 */ (*@ \label{exp:code:002} @*)
 _inv_ln2    = {0x3FF71547, 0x6533245F}; /*  1.44269504088896338700e+00 */
#else
 ...
#endif
#define ln2_hi     _ln2_hi.d
#define ln2_me     _ln2_me.d
#define ln2_lo     _ln2_lo.d
#define inv_ln2    _inv_ln2.d

double r_hi, r_lo, rp_hi, rp_lo;
double u, tmp;
int k;

DOUBLE2INT(k, x * inv_ln2)                     (*@ \label{exp:code:11} @*)

if (k != 0){ 
  /* r_hi+r_lo =  x - (ln2_hi + ln2_me + ln2_lo)*k */
  rp_hi = x-ln2_hi*k;
  rp_lo =  -ln2_me*k;
  Add12Cond(r_hi, u, rp_hi, rp_lo);         (*@ \label{exp:code:12} @*)
  r_lo = u - ln2_lo*k;                         (*@ \label{exp:code:13} @*)
}else {
  r_hi = x;  r_lo = 0.;                        (*@ \label{exp:code:14} @*)
}

\end{lstlisting}


\begin{preuve}
\begin{longtable}[c]{@{line }p{0.08\textwidth}p{0.81\textwidth}}
% en cas de cesure c'est ce que l'on place 
% en debut de page
\endhead
% en fin de page
\endfoot 
% en fin de derniere page
\endlastfoot 
(\ref{exp:code:001}-\ref{exp:code:002})& 
\begin{prop}
By construction:
\label{chap3:exp:prop0}
  $ln2\_hi +ln2\_me +ln2\_lo = \ln(2) (1+ \epsilon_{-140})$
\end{prop}
\begin{prop}
\label{chap3:exp:prop0b}
  $|ln2\_hi| \leq 2^{0}$ ~~~ $|ln2\_me| \leq 2^{-44}$ ~~~   $|ln2\_hi| \leq 2^{-86}$
\end{prop}

\begin{prop}
\label{chap3:exp:prop1}
  $ln2\_hi$ and $ln2\_me$ hold at most $42$ bits of precision
\end{prop}
\\
\ref{exp:code:11} & Put in $k$ the closest integer of $x*inv\_ln2$. We use the property of DOUBLE2INT that convert a floating-point number in rounding to nearest mode (program \ref{sec:double2int}, page \pageref{sec:double2int}).

In addition $k$ satisfy the following property :
\begin{prop}
  \label{chap3:exp:prop2}
  $\lfloor x \times inv\_ln2 \rfloor \leq k \leq \lceil x \times inv\_ln2
  \rceil ~~~ \mbox{et} ~~~  |k| \leq \frac{x}{2} \times inv\_ln2$
\end{prop}

We have seen in Section \ref{chap3:exp:overflows} :
$-745.1332\ldots <x<709.7827\ldots $, then : 
\begin{prop}
  \label{chap3:exp:prop3}
  $-1075 \leq k \leq 1025$ and $|k|$ is an integer on at most $11$ bits
\end{prop}
\\
\ref{exp:code:12}& 
Properties \pref{chap3:exp:prop1} and \pref{chap3:exp:prop3} give us : 
\begin{prop}
  \label{chap3:exp:prop5}
  $ln2\_hi \otimes k = ln2\_hi \times k$~~ and ~~$ln2\_me \otimes k = ln2\_me \times k$ exactly
\end{prop}


By property \pref{chap3:exp:prop2} we have :
$$
 (x \times inv\_ln2 - 1) \times ln2\_hi \leq k \times ln2\_hi \leq (x \times inv\_ln2 + 1) \times ln2\_hi
$$

$$
x/2 \leq k \times ln2\_hi \leq 2.x
$$

By the Sterbenz theorem (theorem \ref{sec:sterbenz}, page \pageref{sec:sterbenz}), we have 
$$x \ominus (ln2\_hi \otimes k) = x - (ln2\_hi \otimes k)$$ 
Combined with property \pref{chap3:exp:prop5} we have :
\begin{prop}
  \label{chap3:exp:prop6}
  $x \ominus (ln2\_hi \otimes k) = x - (ln2\_hi \times k)$ exactly
\end{prop}

We use conditional Add12 algorithm (with tests on entries), because $x-ln2\_hi \times k$ can be equal to zero (due to the $58$ bits of cancellation). The conditional Add12 algorithm (program \ref{lst:Add12Cond}, page \pageref{lst:Add12Cond}) leads to 
$$r\_hi + u = (x \ominus ln2\_hi \otimes k) + (- ln2\_me \otimes k)$$
 
With properties  \pref{chap3:exp:prop5} and \pref{chap3:exp:prop6} we have : 
\begin{prop}
  \label{chap3:exp:prop4}
  $r\_hi + u = (x - ln2\_hi \times k) + (- ln2\_me \times k)$ exactly
\end{prop}
\\
\ref{exp:code:13}&
By the property \pref{chap3:exp:prop3} we have :
\begin{prop}
  \label{chap3:exp:prop4b}
$|ln2\_lo \times k| \leq 2^{-75}$,

$ln2\_lo \otimes k = (ln2\_lo \times k).(1 + \epsilon_{-54})$
\end{prop}

$$
\begin{array}{rclr}
r\_lo &=& u \ominus (ln2\_lo \otimes k)          &\\
      &=& (u - (ln2\_lo \otimes k)).(1 + \epsilon_{-54}) & \\
      &=& (u - (ln2\_lo \times k).(1 + \epsilon_{-54})).(1 + \epsilon_{-54}) & \mbox{\pref{chap3:exp:prop4b}}\\
      &=& (u - (ln2\_lo \times k)).(1 + \epsilon_{-54}) + \epsilon_{-129} + \epsilon_{-183} & \\
\end{array}
$$

That gives us :

\begin{prop}
  \label{chap3:exp:prop7}
$ r\_lo = (u - (ln2\_lo \times k)).(1 + \epsilon_{-54}) + \epsilon_{-129} + \epsilon_{-183}$
\end{prop}

We have :

$$
\begin{array}{rclr}
r\_hi + r\_lo 
&=& r\_hi + (u - (ln2\_lo \times k)).(1 + \epsilon_{-54}) + \epsilon_{-129} + \epsilon_{-183} 
&\mbox{\pref{chap3:exp:prop7}}\\

&=& (x - ln2\_hi \times k) + (- ln2\_me \times k) - (ln2\_lo \times k)&\\
& & + (u - (ln2\_lo \times k)).\epsilon_{-54} + \epsilon_{-129} + \epsilon_{-183} 
&\mbox{\pref{chap3:exp:prop4}}\\

&=& (x - k.\ln(2)) + k.\epsilon_{-140} + (u - (ln2\_lo \times k)).\epsilon_{-54} + &\\
& & \epsilon_{-129} + \epsilon_{-183}
&\mbox{\pref{chap3:exp:prop0}}\\

&=& (x - k.\ln(2)) + (u - (ln2\_lo \times k)).\epsilon_{-54} + \epsilon_{-128} + \epsilon_{-183} &\\

\end{array}
$$


In the worst case, we are losing at most $58$ bits by cancellation (Table \ref{chap3:exp:table:reduction}). By property \pref{chap3:exp:prop4}, we deduce that $u=0$ in that case, which combine with property \pref{chap3:exp:prop4b} ($|ln2\_lo \times k| \leq 2^{-75}$) gives us :

\begin{prop}
  \label{chap3:exp:prop8}
  $r\_hi + r\_lo = (x - k\times \ln2) +\epsilon_{-127+58=-69}$ 
\end{prop}
In addition  $69$ bits is a precision that can be represented as the sum of 2 floating-point numbers in double precision.
\\
\ref{exp:code:14} &
If $k=0$ then no subtraction is necessary, then
 $r\_hi + r\_lo = x$ exactly.
\\
\end{longtable}
\end{preuve}

At the end of this first rang reduction we have :
$$
\exp(x) = 2^{k}. \exp(r\_hi + r\_lo + \epsilon_{-69}) = 2^{k}. \exp(r\_hi + r\_lo).(1+ \epsilon_{-69}) 
$$



\subsubsection{Second range reduction}
The number $(r\_hi+r\_lo)$ is still too big to be used in a polynomial evaluation. Then, a second range reduction needs to be done. This second range reduction is based on the additive property of the exponential $e^{a+b}=e^a e^b$, and on the tabulation of some values of the exponential.

Let  $index\_flt$ be the $\ell$ first bits of $(r\_hi+r\_lo)$, then we have :

$
\begin{array}{rcl}
\exp(r\_hi+r\_lo) 
&=& \exp(index\_flt) . \exp(r\_hi + r\_lo-index\_flt) \\
&\approx& (ex\_hi + ex\_lo) . \exp(rp\_hi + rp\_lo) \\
\end{array}
$

\noindent where $ex\_hi$ and $ex\_lo$ are double precision floating-point numbers extractes from table address by $index\_flt$, such that $ex\_hi + ex\_lo \approx \exp(index\_flt)$. The input argument after this reduction step will be represented as the sum of two double precision floating-point numbers $rp\_hi$ and $rp\_lo$ such that $$rp\_hi + rp\_lo = r\_hi + r\_lo-index\_flt$$


Tests on memory show that the optimal table size for the range reduction is $4$KBytes~\cite{Defour02}. If we want to store these vales and keep enough precision (at least $69$bits), we need 2 floating-point numbers ($16$ bytes) per value.

Let $\ell$ be the parameter such that $[-2^{-\ell-1},2^{-\ell-1}]$ is the range after reduction, then we want :
$$
\lceil \ln 2.2^{\ell} \rceil 16 \leq (2^{12} = 4096)
$$

With $\ell=8$ we have $\lceil \ln (2) . 2^8 \rceil 16 \mbox{~bytes} = 2848$ bytes, and the evaluation range is reduced to $[-2^{-9},2^{-9}]$. After this reduction step, we have $|rp\_hi + rp\_lo| \leq 2^{-9}$.


\noindent The corresponding sequence of instructions performing this second range reduction is :

\begin{lstlisting}[label={exp:lst:reduction2},caption={Second range reduction}]
/* Constants definition */
static const union{int i[2]; double d;}
#ifdef BIG_ENDIAN
 _two_44_43  = {0x42B80000, 0x00000000}; /*  26388279066624.  */ (*@ \label{exp:code:21} @*)
#else
 ...
#endif
#define two_44_43  _two_44_43.d
#define bias       89;               (*@ \label{exp:code:22} @*)

double ex_hi, ex_lo, index_flt;
int index;

index_flt = (r_hi + two_44_43);      (*@ \label{exp:code:23} @*)
index     = LO(index_flt); 
index_flt-= two_44_43;
index    += bias;
r_hi     -= index_flt;               (*@ \label{exp:code:24} @*)

/* Results normalization */
Add12(rp_hi, rp_lo, r_hi, r_lo)   (*@ \label{exp:code:25} @*)

/* Table lookup */
ex_hi = tab_exp[index][0];           (*@ \label{exp:code:26} @*)
ex_lo = tab_exp[index][1];           (*@ \label{exp:code:27} @*)

\end{lstlisting}

\begin{preuve}
\begin{longtable}[c]{@{line }p{0.08\textwidth}p{0.81\textwidth}}
% en cas de cesure c'est ce que l'on place 
% en debut de page
\endhead
% en fin de page
\endfoot 
% en fin de derniere page
\endlastfoot 
\ref{exp:code:21}&
The constant $two\_44\_43 = 2^{44} + 2^{43}$ is used in rounding to nearest mode to extract the  $\ell=8$ leading bits of $r\_hi + r\_lo$. 
\\
\ref{exp:code:22}&
In C language, tables are address with positive index. We consider positive values as well as negative one for $index$, therefore we need to use a bias equal to $178/2 =89$.
\\
(\ref{exp:code:23}, \ref{exp:code:24})&
This sequence of instructions is similar to the one used within DOUBLE2INT (program \ref{sec:double2int}, page \pageref{sec:double2int}). It puts in $index$ variable, bits of weight $2^0$ to $2^{-8}$, minus the value of the bias. Meanwhile, it puts in $index\_flt$ the floating-point value corresponding to the first $8$ bits of $r\_hi$.

In line \ref{exp:code:24} we have :
\begin{prop}
  \label{chap3:exp:index_flt}
  $r\_hi =  r\_hi - index\_flt$ exactly
\end{prop}

\\
\ref{exp:code:25}&
The Add12 algorithm guarantee :
\begin{prop}
  \label{chap3:exp:rphi_rplo}
$|rp\_hi| \leq 2^{-9}$ ~~and~~ $|rp\_lo| \leq 2^{-63}$, 

$rp\_hi + rp\_lo = r\_hi + r\_lo$ exactly
\end{prop}

\\
\ref{exp:code:26}, \ref{exp:code:27}&
We perform table lookup of the 2 values $ex\_hi$ and $ex\_lo$. The table is build such that only one cache miss can be encounter for these 2 table lookups.

By construction of the table $tab\_exp$ we have :
\begin{prop}
  \label{chap3:exp:ex_hilo}
  $|ex\_hi| \leq 2^{-1}$ and $|ex\_lo| \leq 2^{-55}$ 

  $ex\_hi + ex\_lo = \exp(index\_flt).(1+ \epsilon_{-109})$
\end{prop}

\\
\end{longtable}
\end{preuve}


At the end of this second range reduction we have :
$$
\exp(x) = 2^{k}. (ex\_hi + ex\_lo) . \exp(rp\_hi + rp\_lo) . (1+ \epsilon_{-69}) . (1+ \epsilon_{-109})
$$



\subsection{Polynomial evaluation}
Let $r=(rp\_hi + rp\_lo)$, we need to evaluate $\exp(r)$ with $r \in [-2^{-9},2^{-9}]$. We will evaluate $f(r)=(\exp(r)-1-r-\frac{r^2}{2})/r^3$ with the following polynom of degree 3 :
$$
P(r)= c_0 + c_1 r + c_2 r^2 + c_3 r^3
$$ 
where
\begin{itemize}
\item $c_0 = 6004799503160629/36028797018963968 \leq 2^{-2}$
\item $c_1 = 750599937895079/18014398509481984 \leq 2^{-4}$
\item $c_2 = 300240009245077/36028797018963968 \leq 2^{-6}$
\item $c_3 = 3202560062254639/2305843009213693952 \leq 2^{-9}$
\end{itemize}
with $c_0, c_1, c_2, c_3$ exactly representable with double precision floating-point number.

\noindent By using  \texttt{infnorm} function from Maple we get the following error :
\begin{prop}
  \label{chap3:exp:evalpoly}
  $\exp(r) = (1 + r + \frac{1}{2}r^2 + r^3. P(r)).(1+\epsilon_{-78})$
  with $r \in [-2^{-9},2^{-9}]$
\end{prop}

For efficiency reason, we will evaluate $P(rp\_hi)$ instead of $P(rp\_hi + rp\_lo)$. The corresponding error from this approximation is :
$$
\begin{array}{rcll}
P(rp\_hi + rp\_lo) - P(rp\_hi) &=& 
c_1 .  rp\_lo + &\\&&
c_2 . (rp\_lo^2 + 2.rp\_hi.rp\_lo) +  &\\&&
c_3 . (rp\_lo^3 + 3.rp\_hi^2.rp\_lo + 3.rp\_hi.rp\_lo^2) & \mbox{\pref{chap3:exp:rphi_rplo}} \\
 &\leq& \epsilon_{-67} +  \epsilon_{-75} &\\
\end{array}
$$
The, the property \pref{chap3:exp:evalpoly} become :

\begin{prop}
  \label{chap3:exp:evalpoly2}
  $\exp(r) = (1 + r + \frac{1}{2}r^2 + r^3. P(rp\_hi)) + \epsilon_{-78} + \epsilon_{-86}$
  with $r \in [-2^{-9},2^{-9}]$
\end{prop}



\noindent $P\_r=P(rp\_hi)$ is evaluated by the following sequences of instructions : 

\begin{lstlisting}[label={exp:lst:evalpoly},caption={\'Polynomial evaluation}]
static const union{int i[2]; double d;}
#ifdef BIG_ENDIAN
 _c0         = {0x3FC55555, 0x55555535}, /*  1.66666666666665769236e-01 */
 _c1         = {0x3FA55555, 0x55555538}, /*  4.16666666666664631257e-02 */
 _c2         = {0x3F811111, 0x31931950}, /*  8.33333427943885873823e-03 */
 _c3         = {0x3F56C16C, 0x3DC3DC5E}; /*  1.38888903080471677251e-03 */
#else
 ...
#endif
#define c0         _c0.d
#define c1         _c1.d
#define c2         _c2.d
#define c3         _c3.d
double P_r;

P_r = (c_0 + rp_hi * (c_1 + rp_hi * (c_2 + (rp_hi * c_3)))); 

\end{lstlisting}


\begin{preuve}

We have :

\begin{tabular}{@{$\bullet$}l@{~~~then~~~}l@{~~~and~~~}l} 
$P_0 = c\_3 \otimes rp\_hi $ & $|P_0| \leq 2^{-18}$ & 
$P_0 = (c\_3 \times rp\_hi)  + \epsilon_{-72}$\\

$P_1 = c\_2 \oplus P_0$ & $|P_1| \leq 2^{-6}$ &
$P_1 = (c\_2 + P_0) + \epsilon_{-60}$ \\

$P_2 = P_1 \otimes rp\_hi$ & $|P_2| \leq 2^{-15}$ &
$P_2 = (P_1 \times rp\_hi) + \epsilon_{-69}$\\

$P_3 = c\_1 \oplus P_2$ & $|P_3| \leq 2^{-4}$ &
$P_3 = (c\_1 + P_2)  + \epsilon_{-58}$ \\

$P_4 = P_3 \otimes rp\_hi$ & $|P_4| \leq 2^{-13}$ &
$P_4 = (P_3 \times rp\_hi) + \epsilon_{-67}$\\

$P_5 = c\_0 \oplus P_4$ & $|P_5| \leq 2^{-2}$ &
$P_5 = (c\_0 + P_4) + \epsilon_{-56}$ \\
\end{tabular}

By combining all these errors we get :
\begin{prop}
  \label{chap3:exp:prop27bis}
$|P\_r| \leq 2^{-2}$

$P\_r = (c\_0 + rp\_hi \times (c\_1 + rp\_hi \times (c\_2 + (rp\_hi \times c\_3)))) +  \epsilon_{-55} + \epsilon_{-65}$

$P\_r = P(rp\_hi) + \epsilon_{-55} + \epsilon_{-65}$
\end{prop}

By the properties \pref{chap3:exp:evalpoly2} and \pref{chap3:exp:prop27bis} :

\begin{prop}
  \label{chap3:exp:evalpoly3}
  $\exp(r) = (1 + r + \frac{1}{2}r^2 + r^3. P\_r)  + \epsilon_{-78} + \epsilon_{-81}$
  with $r \in [-2^{-9},2^{-9}]$
\end{prop}

\end{preuve}

At the end of the polynomial evaluation scheme we have :
$$
\exp(x) = 2^{k}. (ex\_hi + ex\_lo) . ( 1 + r + \frac{1}{2}r^2 + r^3. P\_r  + \epsilon_{-78} + \epsilon_{-81}) . (1+ \epsilon_{-69}) . (1+ \epsilon_{-109})
$$


\subsection{Reconstruction}
Along previous step of the algorithm we get the following results :

\begin{itemize}
\item $k$, $r\_hi$ and $r\_lo$ during the additive range reduction,
\item $ex\_hi$, $ex\_lo$, $rp\_hi$ and $rp\_lo$ during the table range reduction,
\item $P\_r$ with the polynomial range reduction.
\end{itemize}




The reconstruction step consist in merging all these results in order to get $\exp(x)$.
This step is based on the following mathematical formulae :
$$
\exp(x) = 2^k . (ex\_hi+ex\_lo).
(1 + (rp\_hi+rp\_lo) + \frac{1}{2}.(rp\_hi+rp\_lo)^2 + (rp\_hi+rp\_lo)^3.P\_r)
$$

However, some terms in this equation are too small compared to dominants terms, and should not be taken into account. Then we approximate :
$$
Rec = (ex\_hi + ex\_lo) . (1 + (rp\_hi +rp\_lo) + \frac{1}{2}.(rp\_hi +rp\_lo)^2 + (rp\_hi +rp\_lo)^3.P\_r)
$$

 by

$$
Rec^* = ex\_hi \times (1 + rp\_hi + rp\_lo + \frac{1}{2}(rp\_hi)^2 + P\_r \times (rp\_hi)^3) + ex\_lo \times (1 + rp\_hi + \frac{1}{2}(rp\_hi)^2)
$$
The corresponding error is given by :
$$
\begin{array}{l}
Rec - Rec^* = \\
(ex\_hi + ex\_lo).(rp\_hi.rp\_lo + \frac{1}{2} rp\_lo^2) + \\
ex\_hi.rp\_lo.(3.rp\_hi^2 + 3 . rp\_hi.rp\_lo + rp\_lo^2) + \\
ex\_lo.rp\_hi.(3.rp\_lo^2 + 3 . rp\_hi.rp\_lo + rp\_hi^2) + ex\_lo.rp\_lo^3 + \\
ex\_lo . rp\_lo  \\
\leq 2^{-74} + 2^{-82} + 2^{-88}\\
\end{array}
$$

That gives us the following property :
\begin{prop}
  \label{chap3:exp:reconstruction}
The error done when approximating $Rec$ by $Rec^*$ is :
$$  Rec  = Rec^* + \epsilon_{-74} + \epsilon_{-81}$$
\end{prop}


The order in which are executed the instructions is choosen in order to minimize the error. These terms and the intermediate computations with their magnitude order are given in Figure \ref{chap4:fig:reconstruction}, page \pageref{chap4:fig:reconstruction}.
 

\begin{lstlisting}[label={exp:lst:reconstruction1},caption={Reconstruction}]
double R1, R2, R3_hi, R3_lo, R4, R5_hi, R5_lo, R6, R7, R8, R9, R10, R11, crp_hi;


R1 = rp_hi * rp_hi;                        (*@ \label{exp:code:28} @*)

crp_hi = R1 * rp_hi;                       (*@ \label{exp:code:29} @*)
/* Correspond to R1 /= 2; */
HI(R1) = HI(R1)-0x00100000;                (*@ \label{exp:code:30} @*)

R2 =  P_r * crp_hi;                        (*@ \label{exp:code:31} @*)

Mul12(R3_hi, R3_lo, ex_hi, rp_hi);        (*@ \label{exp:code:32} @*)
R4 = ex_hi * rp_lo;                        (*@ \label{exp:code:33} @*)

Mul12(R5_hi, R5_lo, ex_hi, R1);           (*@ \label{exp:code:34} @*)
R6 = R4 + (ex_lo * (R1 + rp_hi));          (*@ \label{exp:code:35} @*)

R7  = ex_hi * R2;                          (*@ \label{exp:code:36} @*)
R7 += (R6 + R5_lo) + (R3_lo + ex_lo);      (*@ \label{exp:code:37} @*)

Add12(R9, R8, R7, R5_hi);               (*@ \label{exp:code:38} @*)

Add12(R10, tmp, R3_hi, R9);             (*@ \label{exp:code:39} @*)
R8 += tmp;                                 (*@ \label{exp:code:40} @*)

Add12(R11, tmp, ex_hi, R10);            (*@ \label{exp:code:41} @*)
R8 += tmp;                                 (*@ \label{exp:code:42} @*)

Add12(R11, R8, R11, R8);                (*@ \label{exp:code:43} @*)

\end{lstlisting}



\begin{preuve}
\begin{longtable}[c]{@{line }p{0.08\textwidth}p{0.81\textwidth}}
% en cas de cesure c'est ce que l'on place 
% en debut de page
\endhead
% en fin de page
\endfoot 
% en fin de derniere page
\endlastfoot 
\ref{exp:code:28}&
$|rp\_hi| \leq 2^{-9}$ then :
\begin{prop}
  \label{chap3:exp:prop28}
  $|R1| \leq 2^{-18}$,

  $R1 = (rp\_hi)^2 . (1 + \epsilon_{-54})$
\end{prop}\\
\ref{exp:code:29}&
By using property \pref{chap3:exp:prop28} and $|rp\_hi| \leq 2^{-9}$ we get :
\begin{prop}
  \label{chap3:exp:prop29}
  $|crp\_hi| \leq 2^{-27}$,

  $crp\_hi = (rp\_hi)^3    . (1 + \epsilon_{-53})$
\end{prop}\\
\ref{exp:code:30}&
This operation is a division by 2, done by subtracting $1$ to the exposant. This oepration is valid and exact if $R1$ is not a denormalized number, which is the case (property \pref{chap3:exp:prop10bis} $|x| \geq 2^{-54}$), and the table  \ref{chap3:exp:table:reduction} showing that we have at most $58$ bits of cancellation).

\begin{prop}
  \label{chap3:exp:prop30}
  $|R1| \leq 2^{-19}$,

  $R1 =  \frac{1}{2}(rp\_hi)^2 . (1 + \epsilon_{-54})$
\end{prop}\\
\ref{exp:code:31}&
By using properties  \pref{chap3:exp:prop27bis} and \pref{chap3:exp:prop29}
\begin{prop}
  \label{chap3:exp:prop31}
  $|R2| \leq 2^{-29}$,

  $R2 = P\_r \times (crp\_hi) .(1 + \epsilon_{-54})$ ,

  $R2 = P\_r \times (rp\_hi)^3 .(1 + \epsilon_{-52})$
\end{prop}\\
\ref{exp:code:32}&
By using Mul12 algorithm (programme \ref{lst:Mul12}, page
\pageref{lst:Mul12}) and properties  \pref{chap3:exp:rphi_rplo} and \pref{chap3:exp:ex_hilo} we have :
\begin{prop}
  \label{chap3:exp:prop32}
  $|R3\_hi| \leq 2^{-10}$ and $|R3\_lo| \leq 2^{-64}$,

  $R3\_hi + R3\_lo = ex\_hi \times rp\_hi$ exactly
\end{prop}\\
\ref{exp:code:33}&
By using properties \pref{chap3:exp:rphi_rplo} and \pref{chap3:exp:ex_hilo} :
\begin{prop}
  \label{chap3:exp:prop33}
  $|R4| \leq 2^{-64}$,

  $R4 = ex\_hi \times rp\_lo . (1 + \epsilon_{-54})$
\end{prop}\\
\ref{exp:code:34}&
By using Mul12 algorithm and properties \pref{chap3:exp:ex_hilo} and \pref{chap3:exp:prop30} we have :
\begin{prop}
  \label{chap3:exp:prop34}
  $|R5\_hi| \leq 2^{-20}$ and $|R5\_lo| \leq 2^{-74} $,

  $(R5\_hi + R5\_lo) = ex\_hi \times R1$ exactly, 

  $(R5\_hi + R5\_lo) = (ex\_hi \times \frac{1}{2}(rp\_hi)^2).(1 + \epsilon_{-54})$
\end{prop}\\
\ref{exp:code:35}&
By using properties  \pref{chap3:exp:ex_hilo} and \pref{chap3:exp:prop30} :

$|R1 + rp\_hi| \leq 2^{-8}$,

$R1 \oplus rp\_hi = R1 + rp\_hi + \epsilon_{-62}$,

$|ex\_lo \times (R1 + rp\_hi)| \leq 2^{-63}$,

$ex\_lo \otimes (R1 \oplus rp\_hi) = ex\_lo \times ( \frac{1}{2}(rp\_hi)^2 + rp\_hi) + \epsilon_{-116} $.


which combined with property \pref{chap3:exp:prop33} gives us : 
\begin{prop}
  \label{chap3:exp:prop35}
  $|R6| \leq 2^{-62}$,

  $R6 = R4 + \left(ex\_lo \times \left(\frac{1}{2}(rp\_hi)^2 + rp\_hi \right) + \epsilon_{-116}\right) + \epsilon_{-116}$

  $R6 = (ex\_hi \times rp\_lo) + \left(ex\_lo \times \left(\frac{1}{2}(rp\_hi)^2 + rp\_hi\right)\right) + \epsilon_{-115} +\epsilon_{-118}$
\end{prop}\\

\ref{exp:code:36}&
By using properties \pref{chap3:exp:ex_hilo} and \pref{chap3:exp:prop31} we get : 
\begin{prop}
  \label{chap3:exp:prop36}
  $|R7| \leq 2^{-30}$,
 
  $R7 = (ex\_hi \times R2) . (1 + \epsilon_{-54})$,

  $R7 = (ex\_hi \times P\_r \times (rp\_hi)^3).(1+ \epsilon_{-52}+\epsilon_{-53})$
\end{prop}\\
\ref{exp:code:37}&
By using properties \pref{chap3:exp:prop34} and \pref{chap3:exp:prop35} we get :
\begin{prop}
\label{chap3:exp:prop37a}
$|R6 + R5\_lo| \leq 2^{-61}$,

$R6 \oplus R5\_lo = R6 + R5\_lo + \epsilon_{-115} $ ou

$R6 \oplus R5\_lo = (ex\_hi \times rp\_lo) + (ex\_lo \times (\frac{1}{2}(rp\_hi)^2 + rp\_hi)) + R5\_lo + \epsilon_{-113}$
\end{prop}

By using properties \pref{chap3:exp:ex_hilo} and \pref{chap3:exp:prop32} we get :
\begin{prop}
\label{chap3:exp:prop37b}
$|R3\_lo + ex\_lo| \leq 2^{-54}$,

$R3\_lo \oplus ex\_lo = R3\_lo + ex\_lo + \epsilon_{-108} $
\end{prop}

By using properties \pref{chap3:exp:prop37a} and \pref{chap3:exp:prop37b} we get :

$|R6 + R5\_lo + R3\_lo + ex\_lo| \leq 2^{-53}$ et

$(R6 \oplus R5\_lo) \oplus (R3\_lo \oplus ex\_lo) = (R6 \oplus R5\_lo) + (R3\_lo \oplus ex\_lo) + \epsilon_{-107}$.


which combine with property \pref{chap3:exp:prop36} gives us :
\begin{prop}
  \label{chap3:exp:prop37}
  $|R7| \leq 2^{-29}$,

$
\begin{array}{rcl}
R7 &=& 
(ex\_hi \times  P\_r \times (rp\_hi)^3).(1+\epsilon_{-52}+\epsilon_{-53}) + \\ &&
((ex\_hi \times rp\_lo) + (ex\_lo \times (\frac{1}{2}(rp\_hi)^2 + rp\_hi)) + R5\_lo + \epsilon_{-113}) + \\&&
(R3\_lo + ex\_lo + \epsilon_{-108}) \\

&=& ex\_hi \times (rp\_lo + P\_r \times (rp\_hi)^3) + \\
& & ex\_lo \times (1 + rp\_hi + \frac{1}{2}(rp\_hi)^2) +  \\
& & R5\_lo + R3\_lo + \epsilon_{-80} \\
\end{array}
$
\end{prop}
\\ 

\ref{exp:code:38}&
Add12 algorithm guarantee :
\begin{prop}
  \label{chap3:exp:prop38}
  $|R9| \leq 2^{-19}$ and $|R8| \leq 2^{-73}$,

  $R7 + R5\_hi = R9 + R8$ exactly
\end{prop}\\
\ref{exp:code:39}&
Add12 algorithm guarantee :
\begin{prop}
  \label{chap3:exp:prop39}
  $|R10| \leq 2^{-9}$ and $|tmp| \leq 2^{-63}$,

  $R3\_hi + R9 = R10 + tmp$ exactly
\end{prop}\\

\ref{exp:code:40}&
By using properties \pref{chap3:exp:prop38} and \pref{chap3:exp:prop39} :
\begin{prop}
  \label{chap3:exp:prop40}
  $|R8 + tmp| \leq 2^{-62}$,

  $R8 = R8 + tmp + \epsilon_{-116}$
\end{prop}\\
\ref{exp:code:41}&
Add12 algorithm guarantee :
\begin{prop}
  \label{chap3:exp:prop41}
  $|R11| \leq 2^{0}$ and $|tmp| \leq 2^{-54}$,

  $R11 + tmp = ex\_hi + R10$ exactly
\end{prop}\\
\ref{exp:code:42}&
By using properties \pref{chap3:exp:prop40} and \pref{chap3:exp:prop41} :
\begin{prop}
  \label{chap3:exp:prop42}
  $|R8 + tmp| \leq 2^{-53}$,

  $R8 = R8 + tmp + \epsilon_{-107}$
\end{prop}\\
\ref{exp:code:43}&
Add12 algorithm guarantee :
\begin{prop}
  \label{chap3:exp:prop43}
  $|R11| \leq 2^{1}$ and $|R8| \leq 2^{-53}$,

  $R11 + R8 = R11 + R8$ exactly
\end{prop}\\

\end{longtable}
\end{preuve}


Therefore we have :
$$
\begin{array}{rcll}
R11 + R8  
&=& R11 + tmp + R8 + \epsilon_{-107} 
& \mbox{\pref{chap3:exp:prop42}} \\

&=& ex\_hi + R10 + R8 + \epsilon_{-107} 
& \mbox{\pref{chap3:exp:prop41}} \\

&=& ex\_hi + R10 + tmp + R8 + \epsilon_{-107} + \epsilon_{-116} 
& \mbox{\pref{chap3:exp:prop40}} \\

&=& ex\_hi + R3\_hi + R9 + R8 + \epsilon_{-106}
& \mbox{\pref{chap3:exp:prop39}} \\

&=& ex\_hi + R3\_hi + R7 + R5\_hi + \epsilon_{-106}
& \mbox{\pref{chap3:exp:prop38}} \\

&=&  ex\_hi + R3\_hi + R5\_hi + &\\
& & ex\_hi \times (rp\_lo + P\_r \times (rp\_hi)^3) + &\\
& & ex\_lo \times (1 + rp\_hi + \frac{1}{2}(rp\_hi)^2) +  &\\
& & R5\_lo + R3\_lo + \epsilon_{-80} + \epsilon_{-106} 
&  \mbox{\pref{chap3:exp:prop37}} \\

&=& (R3\_hi + R3\_lo) + (R5\_hi + R5\_lo) + &\\
& & ex\_hi \times (1 + rp\_lo + P\_r \times (rp\_hi)^3) + &\\
& & ex\_lo \times (1 + rp\_hi + \frac{1}{2}(rp\_hi)^2) 
+ \epsilon_{-80} + \epsilon_{-106} &\\

&=& (R3\_hi + R3\_lo) + &\\
& & ex\_hi \times (1 + rp\_lo + \frac{1}{2}(rp\_hi)^2 + P\_r \times (rp\_hi)^3) + &\\
& & ex\_lo \times (1 + rp\_hi + \frac{1}{2}(rp\_hi)^2) 
+ \epsilon_{-74} + \epsilon_{-79} 
&  \mbox{\pref{chap3:exp:prop34}} \\

&=& ex\_hi \times (1 + rp\_hi + rp\_lo + \frac{1}{2}(rp\_hi)^2 + P\_r \times (rp\_hi)^3) + &\\
& & ex\_lo \times (1 + rp\_hi +          \frac{1}{2}(rp\_hi)^2) + &\\
& & \epsilon_{-74} + \epsilon_{-79} 
&  \mbox{\pref{chap3:exp:prop32}} \\
\end{array}
$$

Which combine with property \pref{chap3:exp:reconstruction} gives us :

$$
\begin{array}{rcl}
R11 + R8  
&=& (ex\_hi + ex\_lo) . (1 +r + \frac{1}{2}.r^2 + r^3.P\_r)\\
& & +\epsilon_{-73} + \epsilon_{-78} \\
\end{array}
$$
By construction of values $(ex\_hi + ex\_lo)$, we have $(ex\_hi + ex\_lo) > \exp(-1) > 2^{-2}$
then,

\begin{prop}
 \label{chap3:exp:evalr11r8}
$|R11 + R8| > 2^{-2}$

and

$
\exp(x) = 2^k . (R11 + R8 + \epsilon_{-73} + \epsilon_{-78}).(1 + \epsilon_{-69}).(1 + \epsilon_{-109})
$
\end{prop}

 
\begin{landscape}

\begin{figure}[ht] \begin{center}
% 
% Dessin xfig : magnification : 0.24 %
% 
    % PSTricks TeX macro
% Title: /home/ddefour/ens/Personnel/these/fig_chap4/reconstruction
% Creator: Dia v0.90
% CreationDate: Tue Jun 17 23:08:32 2003
% For: a user
% \usepackage{pstricks}
% The following commands are not supported in PSTricks at present
% We define them conditionally, so when they are implemented,
% this pstricks file will use them.
\ifx\setlinejoinmode\undefined
  \newcommand{\setlinejoinmode}[1]{}
\fi
\ifx\setlinecaps\undefined
  \newcommand{\setlinecaps}[1]{}
\fi
% This way define your own fonts mapping (for example with ifthen)
\ifx\setfont\undefined
  \newcommand{\setfont}[2]{}
\fi
\pspicture(15.950000,-26.050000)(106.840500,26.050000)
\scalebox{1.000000 -1.000000}{
\newrgbcolor{dialinecolor}{0.000000 0.000000 0.000000}
\psset{linecolor=dialinecolor}
\newrgbcolor{diafillcolor}{1.000000 1.000000 1.000000}
\psset{fillcolor=diafillcolor}
\psset{linewidth=0.100000}
\psset{linestyle=solid}
\psset{linestyle=solid}
\setlinecaps{0}
\setlinejoinmode{0}
\setlinecaps{0}
\setlinejoinmode{0}
\psset{linestyle=solid}
\newrgbcolor{dialinecolor}{1.000000 1.000000 1.000000}
\psset{linecolor=dialinecolor}
\psellipse*(26.582220,0.147935)(1.582220,1.147935)
\newrgbcolor{dialinecolor}{0.000000 0.000000 0.000000}
\psset{linecolor=dialinecolor}
\psellipse(26.582220,0.147935)(1.582220,1.147935)
\setfont{Courier}{0.800000}
\newrgbcolor{dialinecolor}{0.000000 0.000000 0.000000}
\psset{linecolor=dialinecolor}
\rput(26.582220,0.366803){\scalebox{1 -1}{$1/2$}}
\psset{linewidth=0.100000}
\psset{linestyle=solid}
\psset{linestyle=solid}
\setlinecaps{0}
\setlinejoinmode{0}
\setlinecaps{0}
\setlinejoinmode{0}
\psset{linestyle=solid}
\newrgbcolor{dialinecolor}{1.000000 1.000000 1.000000}
\psset{linecolor=dialinecolor}
\psellipse*(22.000000,5.000000)(1.000000,1.000000)
\newrgbcolor{dialinecolor}{0.000000 0.000000 0.000000}
\psset{linecolor=dialinecolor}
\psellipse(22.000000,5.000000)(1.000000,1.000000)
\setlinecaps{0}
\setlinejoinmode{0}
\psset{linestyle=solid}
\newrgbcolor{dialinecolor}{0.000000 0.000000 0.000000}
\psset{linecolor=dialinecolor}
\psline(21.292900,4.292900)(22.707100,5.707100)
\setlinecaps{0}
\setlinejoinmode{0}
\psset{linestyle=solid}
\newrgbcolor{dialinecolor}{0.000000 0.000000 0.000000}
\psset{linecolor=dialinecolor}
\psline(21.292900,5.707100)(22.707100,4.292900)
\newrgbcolor{dialinecolor}{1.000000 1.000000 1.000000}
\psset{linecolor=dialinecolor}
\pspolygon*(16.000000,6.000000)(16.000000,8.000000)(20.000000,8.000000)(20.000000,6.000000)
\psset{linewidth=0.100000}
\psset{linestyle=solid}
\psset{linestyle=solid}
\setlinejoinmode{0}
\newrgbcolor{dialinecolor}{0.000000 0.000000 0.000000}
\psset{linecolor=dialinecolor}
\pspolygon(16.000000,6.000000)(16.000000,8.000000)(20.000000,8.000000)(20.000000,6.000000)
\setfont{Times-Bold}{1.000000}
\newrgbcolor{dialinecolor}{0.000000 0.000000 0.000000}
\psset{linecolor=dialinecolor}
\rput(18.000000,7.000000){\scalebox{1 -1}{$rp\_hi$}}
\psset{linewidth=0.100000}
\psset{linestyle=solid}
\psset{linestyle=solid}
\setlinejoinmode{0}
\setlinecaps{0}
\newrgbcolor{dialinecolor}{0.000000 0.000000 0.000000}
\psset{linecolor=dialinecolor}
\psline(20.000000,2.000000)(20.000000,2.000000)(22.000000,2.000000)(22.000000,4.000000)
\psset{linewidth=0.100000}
\psset{linestyle=solid}
\psset{linestyle=solid}
\setlinejoinmode{0}
\setlinecaps{0}
\newrgbcolor{dialinecolor}{0.000000 0.000000 0.000000}
\psset{linecolor=dialinecolor}
\psline(20.000000,8.000000)(20.000000,8.000000)(22.000000,8.000000)(22.000000,6.000000)
\newrgbcolor{dialinecolor}{1.000000 1.000000 1.000000}
\psset{linecolor=dialinecolor}
\pspolygon*(16.000000,0.000000)(16.000000,2.000000)(20.000000,2.000000)(20.000000,0.000000)
\psset{linewidth=0.100000}
\psset{linestyle=solid}
\psset{linestyle=solid}
\setlinejoinmode{0}
\newrgbcolor{dialinecolor}{0.000000 0.000000 0.000000}
\psset{linecolor=dialinecolor}
\pspolygon(16.000000,0.000000)(16.000000,2.000000)(20.000000,2.000000)(20.000000,0.000000)
\setfont{Times-Bold}{1.000000}
\newrgbcolor{dialinecolor}{0.000000 0.000000 0.000000}
\psset{linecolor=dialinecolor}
\rput(18.000000,1.000000){\scalebox{1 -1}{$rp\_hi$}}
\newrgbcolor{dialinecolor}{1.000000 1.000000 1.000000}
\psset{linecolor=dialinecolor}
\pspolygon*(24.000000,3.000000)(24.000000,5.000000)(28.000000,5.000000)(28.000000,3.000000)
\psset{linewidth=0.100000}
\psset{linestyle=solid}
\psset{linestyle=solid}
\setlinejoinmode{0}
\newrgbcolor{dialinecolor}{0.000000 0.000000 0.000000}
\psset{linecolor=dialinecolor}
\pspolygon(24.000000,3.000000)(24.000000,5.000000)(28.000000,5.000000)(28.000000,3.000000)
\setfont{Times-Bold}{1.000000}
\newrgbcolor{dialinecolor}{0.000000 0.000000 0.000000}
\psset{linecolor=dialinecolor}
\rput(26.000000,4.000000){\scalebox{1 -1}{$R1$}}
\newrgbcolor{dialinecolor}{1.000000 1.000000 1.000000}
\psset{linecolor=dialinecolor}
\pspolygon*(24.000000,15.000000)(24.000000,17.000000)(28.000000,17.000000)(28.000000,15.000000)
\psset{linewidth=0.100000}
\psset{linestyle=solid}
\psset{linestyle=solid}
\setlinejoinmode{0}
\newrgbcolor{dialinecolor}{0.000000 0.000000 0.000000}
\psset{linecolor=dialinecolor}
\pspolygon(24.000000,15.000000)(24.000000,17.000000)(28.000000,17.000000)(28.000000,15.000000)
\setfont{Times-Bold}{1.000000}
\newrgbcolor{dialinecolor}{0.000000 0.000000 0.000000}
\psset{linecolor=dialinecolor}
\rput(26.000000,16.000000){\scalebox{1 -1}{$rp\_hi$}}
\psset{linewidth=0.100000}
\psset{linestyle=solid}
\psset{linestyle=solid}
\setlinecaps{0}
\setlinejoinmode{0}
\setlinecaps{0}
\setlinejoinmode{0}
\psset{linestyle=solid}
\newrgbcolor{dialinecolor}{1.000000 1.000000 1.000000}
\psset{linecolor=dialinecolor}
\psellipse*(30.000000,15.000000)(1.000000,1.000000)
\newrgbcolor{dialinecolor}{0.000000 0.000000 0.000000}
\psset{linecolor=dialinecolor}
\psellipse(30.000000,15.000000)(1.000000,1.000000)
\setlinecaps{0}
\setlinejoinmode{0}
\psset{linestyle=solid}
\newrgbcolor{dialinecolor}{0.000000 0.000000 0.000000}
\psset{linecolor=dialinecolor}
\psline(29.292900,14.292900)(30.707100,15.707100)
\setlinecaps{0}
\setlinejoinmode{0}
\psset{linestyle=solid}
\newrgbcolor{dialinecolor}{0.000000 0.000000 0.000000}
\psset{linecolor=dialinecolor}
\psline(29.292900,15.707100)(30.707100,14.292900)
\psset{linewidth=0.100000}
\psset{linestyle=solid}
\psset{linestyle=solid}
\setlinejoinmode{0}
\setlinecaps{0}
\newrgbcolor{dialinecolor}{0.000000 0.000000 0.000000}
\psset{linecolor=dialinecolor}
\psline(28.000000,4.000000)(28.000000,5.000000)(30.000000,5.000000)(30.000000,14.000000)
\psset{linewidth=0.100000}
\psset{linestyle=solid}
\psset{linestyle=solid}
\setlinejoinmode{0}
\setlinecaps{0}
\newrgbcolor{dialinecolor}{0.000000 0.000000 0.000000}
\psset{linecolor=dialinecolor}
\psline(28.000000,17.000000)(28.000000,17.000000)(30.000000,17.000000)(30.000000,16.000000)
\psset{linewidth=0.100000}
\psset{linestyle=solid}
\psset{linestyle=solid}
\setlinejoinmode{0}
\setlinecaps{0}
\newrgbcolor{dialinecolor}{0.000000 0.000000 0.000000}
\psset{linecolor=dialinecolor}
\psline(23.000000,5.000000)(23.000000,5.000000)(24.000000,5.000000)(24.000000,4.000000)
\newrgbcolor{dialinecolor}{1.000000 1.000000 1.000000}
\psset{linecolor=dialinecolor}
\pspolygon*(32.000000,13.000000)(32.000000,15.000000)(36.000000,15.000000)(36.000000,13.000000)
\psset{linewidth=0.100000}
\psset{linestyle=solid}
\psset{linestyle=solid}
\setlinejoinmode{0}
\newrgbcolor{dialinecolor}{0.000000 0.000000 0.000000}
\psset{linecolor=dialinecolor}
\pspolygon(32.000000,13.000000)(32.000000,15.000000)(36.000000,15.000000)(36.000000,13.000000)
\setfont{Times-Bold}{1.000000}
\newrgbcolor{dialinecolor}{0.000000 0.000000 0.000000}
\psset{linecolor=dialinecolor}
\rput(34.000000,14.000000){\scalebox{1 -1}{$rp\_hi$}}
\psset{linewidth=0.100000}
\psset{linestyle=solid}
\psset{linestyle=solid}
\setlinejoinmode{0}
\setlinecaps{0}
\newrgbcolor{dialinecolor}{0.000000 0.000000 0.000000}
\psset{linecolor=dialinecolor}
\psline(31.000000,15.000000)(31.000000,15.000000)(32.000000,15.000000)(32.000000,15.000000)
\newrgbcolor{dialinecolor}{1.000000 1.000000 1.000000}
\psset{linecolor=dialinecolor}
\pspolygon*(32.000000,19.000000)(32.000000,21.000000)(36.000000,21.000000)(36.000000,19.000000)
\psset{linewidth=0.100000}
\psset{linestyle=solid}
\psset{linestyle=solid}
\setlinejoinmode{0}
\newrgbcolor{dialinecolor}{0.000000 0.000000 0.000000}
\psset{linecolor=dialinecolor}
\pspolygon(32.000000,19.000000)(32.000000,21.000000)(36.000000,21.000000)(36.000000,19.000000)
\setfont{Times-Bold}{1.000000}
\newrgbcolor{dialinecolor}{0.000000 0.000000 0.000000}
\psset{linecolor=dialinecolor}
\rput(34.000000,20.000000){\scalebox{1 -1}{$P\_r$}}
\psset{linewidth=0.100000}
\psset{linestyle=solid}
\psset{linestyle=solid}
\setlinecaps{0}
\setlinejoinmode{0}
\setlinecaps{0}
\setlinejoinmode{0}
\psset{linestyle=solid}
\newrgbcolor{dialinecolor}{1.000000 1.000000 1.000000}
\psset{linecolor=dialinecolor}
\psellipse*(38.000000,18.000000)(1.000000,1.000000)
\newrgbcolor{dialinecolor}{0.000000 0.000000 0.000000}
\psset{linecolor=dialinecolor}
\psellipse(38.000000,18.000000)(1.000000,1.000000)
\setlinecaps{0}
\setlinejoinmode{0}
\psset{linestyle=solid}
\newrgbcolor{dialinecolor}{0.000000 0.000000 0.000000}
\psset{linecolor=dialinecolor}
\psline(37.292900,17.292900)(38.707100,18.707100)
\setlinecaps{0}
\setlinejoinmode{0}
\psset{linestyle=solid}
\newrgbcolor{dialinecolor}{0.000000 0.000000 0.000000}
\psset{linecolor=dialinecolor}
\psline(37.292900,18.707100)(38.707100,17.292900)
\psset{linewidth=0.100000}
\psset{linestyle=solid}
\psset{linestyle=solid}
\setlinejoinmode{0}
\setlinecaps{0}
\newrgbcolor{dialinecolor}{0.000000 0.000000 0.000000}
\psset{linecolor=dialinecolor}
\psline(36.000000,15.000000)(36.000000,15.000000)(38.000000,15.000000)(38.000000,17.000000)
\psset{linewidth=0.100000}
\psset{linestyle=solid}
\psset{linestyle=solid}
\setlinejoinmode{0}
\setlinecaps{0}
\newrgbcolor{dialinecolor}{0.000000 0.000000 0.000000}
\psset{linecolor=dialinecolor}
\psline(36.000000,21.000000)(36.000000,21.000000)(38.000000,21.000000)(38.000000,19.000000)
\newrgbcolor{dialinecolor}{1.000000 1.000000 1.000000}
\psset{linecolor=dialinecolor}
\pspolygon*(41.000000,16.000000)(41.000000,18.000000)(45.000000,18.000000)(45.000000,16.000000)
\psset{linewidth=0.100000}
\psset{linestyle=solid}
\psset{linestyle=solid}
\setlinejoinmode{0}
\newrgbcolor{dialinecolor}{0.000000 0.000000 0.000000}
\psset{linecolor=dialinecolor}
\pspolygon(41.000000,16.000000)(41.000000,18.000000)(45.000000,18.000000)(45.000000,16.000000)
\setfont{Times-Bold}{1.000000}
\newrgbcolor{dialinecolor}{0.000000 0.000000 0.000000}
\psset{linecolor=dialinecolor}
\rput(43.000000,17.000000){\scalebox{1 -1}{$R2$}}
\psset{linewidth=0.100000}
\psset{linestyle=solid}
\psset{linestyle=solid}
\setlinecaps{0}
\setlinejoinmode{0}
\setlinecaps{0}
\setlinejoinmode{0}
\psset{linestyle=solid}
\newrgbcolor{dialinecolor}{1.000000 1.000000 1.000000}
\psset{linecolor=dialinecolor}
\psellipse*(30.000000,3.000000)(1.000000,1.000000)
\newrgbcolor{dialinecolor}{0.000000 0.000000 0.000000}
\psset{linecolor=dialinecolor}
\psellipse(30.000000,3.000000)(1.000000,1.000000)
\setlinecaps{0}
\setlinejoinmode{0}
\psset{linestyle=solid}
\newrgbcolor{dialinecolor}{0.000000 0.000000 0.000000}
\psset{linecolor=dialinecolor}
\psline(29.292900,2.292900)(30.707100,3.707100)
\setlinecaps{0}
\setlinejoinmode{0}
\psset{linestyle=solid}
\newrgbcolor{dialinecolor}{0.000000 0.000000 0.000000}
\psset{linecolor=dialinecolor}
\psline(29.292900,3.707100)(30.707100,2.292900)
\psset{linewidth=0.100000}
\psset{linestyle=solid}
\psset{linestyle=solid}
\setlinejoinmode{0}
\setlinecaps{0}
\newrgbcolor{dialinecolor}{0.000000 0.000000 0.000000}
\psset{linecolor=dialinecolor}
\psline(30.000000,4.000000)(30.000000,5.000000)(28.000000,5.000000)(28.000000,4.000000)
\psset{linewidth=0.100000}
\psset{linestyle=solid}
\psset{linestyle=solid}
\setlinejoinmode{0}
\setlinecaps{0}
\newrgbcolor{dialinecolor}{0.000000 0.000000 0.000000}
\psset{linecolor=dialinecolor}
\psline(28.164440,0.147935)(28.164440,0.000000)(30.000000,0.000000)(30.000000,2.000000)
\newrgbcolor{dialinecolor}{1.000000 1.000000 1.000000}
\psset{linecolor=dialinecolor}
\pspolygon*(32.000000,1.000000)(32.000000,3.000000)(36.000000,3.000000)(36.000000,1.000000)
\psset{linewidth=0.100000}
\psset{linestyle=solid}
\psset{linestyle=solid}
\setlinejoinmode{0}
\newrgbcolor{dialinecolor}{0.000000 0.000000 0.000000}
\psset{linecolor=dialinecolor}
\pspolygon(32.000000,1.000000)(32.000000,3.000000)(36.000000,3.000000)(36.000000,1.000000)
\setfont{Times-Bold}{1.000000}
\newrgbcolor{dialinecolor}{0.000000 0.000000 0.000000}
\psset{linecolor=dialinecolor}
\rput(34.000000,2.000000){\scalebox{1 -1}{$R1$}}
\psset{linewidth=0.100000}
\psset{linestyle=solid}
\psset{linestyle=solid}
\setlinejoinmode{0}
\setlinecaps{0}
\newrgbcolor{dialinecolor}{0.000000 0.000000 0.000000}
\psset{linecolor=dialinecolor}
\psline(31.000000,3.000000)(31.000000,3.000000)(32.000000,3.000000)(32.000000,3.000000)
\newrgbcolor{dialinecolor}{1.000000 1.000000 1.000000}
\psset{linecolor=dialinecolor}
\pspolygon*(24.000000,-17.000000)(24.000000,-15.000000)(28.000000,-15.000000)(28.000000,-17.000000)
\psset{linewidth=0.100000}
\psset{linestyle=solid}
\psset{linestyle=solid}
\setlinejoinmode{0}
\newrgbcolor{dialinecolor}{0.000000 0.000000 0.000000}
\psset{linecolor=dialinecolor}
\pspolygon(24.000000,-17.000000)(24.000000,-15.000000)(28.000000,-15.000000)(28.000000,-17.000000)
\setfont{Times-Bold}{1.000000}
\newrgbcolor{dialinecolor}{0.000000 0.000000 0.000000}
\psset{linecolor=dialinecolor}
\rput(26.000000,-16.000000){\scalebox{1 -1}{$ex\_hi$}}
\newrgbcolor{dialinecolor}{1.000000 1.000000 1.000000}
\psset{linecolor=dialinecolor}
\pspolygon*(24.000000,-11.000000)(24.000000,-9.000000)(28.000000,-9.000000)(28.000000,-11.000000)
\psset{linewidth=0.100000}
\psset{linestyle=solid}
\psset{linestyle=solid}
\setlinejoinmode{0}
\newrgbcolor{dialinecolor}{0.000000 0.000000 0.000000}
\psset{linecolor=dialinecolor}
\pspolygon(24.000000,-11.000000)(24.000000,-9.000000)(28.000000,-9.000000)(28.000000,-11.000000)
\setfont{Times-Bold}{1.000000}
\newrgbcolor{dialinecolor}{0.000000 0.000000 0.000000}
\psset{linecolor=dialinecolor}
\rput(26.000000,-10.000000){\scalebox{1 -1}{$rp\_lo$}}
\psset{linewidth=0.100000}
\psset{linestyle=solid}
\psset{linestyle=solid}
\setlinecaps{0}
\setlinejoinmode{0}
\setlinecaps{0}
\setlinejoinmode{0}
\psset{linestyle=solid}
\newrgbcolor{dialinecolor}{1.000000 1.000000 1.000000}
\psset{linecolor=dialinecolor}
\psellipse*(30.000000,-12.000000)(1.000000,1.000000)
\newrgbcolor{dialinecolor}{0.000000 0.000000 0.000000}
\psset{linecolor=dialinecolor}
\psellipse(30.000000,-12.000000)(1.000000,1.000000)
\setlinecaps{0}
\setlinejoinmode{0}
\psset{linestyle=solid}
\newrgbcolor{dialinecolor}{0.000000 0.000000 0.000000}
\psset{linecolor=dialinecolor}
\psline(29.292900,-12.707100)(30.707100,-11.292900)
\setlinecaps{0}
\setlinejoinmode{0}
\psset{linestyle=solid}
\newrgbcolor{dialinecolor}{0.000000 0.000000 0.000000}
\psset{linecolor=dialinecolor}
\psline(29.292900,-11.292900)(30.707100,-12.707100)
\psset{linewidth=0.100000}
\psset{linestyle=solid}
\psset{linestyle=solid}
\setlinejoinmode{0}
\setlinecaps{0}
\newrgbcolor{dialinecolor}{0.000000 0.000000 0.000000}
\psset{linecolor=dialinecolor}
\psline(28.000000,-15.000000)(28.000000,-15.000000)(30.000000,-15.000000)(30.000000,-13.000000)
\psset{linewidth=0.100000}
\psset{linestyle=solid}
\psset{linestyle=solid}
\setlinejoinmode{0}
\setlinecaps{0}
\newrgbcolor{dialinecolor}{0.000000 0.000000 0.000000}
\psset{linecolor=dialinecolor}
\psline(28.000000,-9.000000)(28.000000,-9.000000)(30.000000,-9.000000)(30.000000,-11.000000)
\newrgbcolor{dialinecolor}{1.000000 1.000000 1.000000}
\psset{linecolor=dialinecolor}
\pspolygon*(40.000000,-14.000000)(40.000000,-12.000000)(44.000000,-12.000000)(44.000000,-14.000000)
\psset{linewidth=0.100000}
\psset{linestyle=solid}
\psset{linestyle=solid}
\setlinejoinmode{0}
\newrgbcolor{dialinecolor}{0.000000 0.000000 0.000000}
\psset{linecolor=dialinecolor}
\pspolygon(40.000000,-14.000000)(40.000000,-12.000000)(44.000000,-12.000000)(44.000000,-14.000000)
\setfont{Times-Bold}{1.000000}
\newrgbcolor{dialinecolor}{0.000000 0.000000 0.000000}
\psset{linecolor=dialinecolor}
\rput(42.000000,-13.000000){\scalebox{1 -1}{$R4$}}
\psset{linewidth=0.100000}
\psset{linestyle=solid}
\psset{linestyle=solid}
\setlinejoinmode{0}
\setlinecaps{0}
\newrgbcolor{dialinecolor}{0.000000 0.000000 0.000000}
\psset{linecolor=dialinecolor}
\psline(31.000000,-12.000000)(31.000000,-12.000000)(40.000000,-12.000000)(40.000000,-13.000000)
\newrgbcolor{dialinecolor}{1.000000 1.000000 1.000000}
\psset{linecolor=dialinecolor}
\pspolygon*(32.000000,7.000000)(32.000000,9.000000)(36.000000,9.000000)(36.000000,7.000000)
\psset{linewidth=0.100000}
\psset{linestyle=solid}
\psset{linestyle=solid}
\setlinejoinmode{0}
\newrgbcolor{dialinecolor}{0.000000 0.000000 0.000000}
\psset{linecolor=dialinecolor}
\pspolygon(32.000000,7.000000)(32.000000,9.000000)(36.000000,9.000000)(36.000000,7.000000)
\setfont{Times-Bold}{1.000000}
\newrgbcolor{dialinecolor}{0.000000 0.000000 0.000000}
\psset{linecolor=dialinecolor}
\rput(34.000000,8.000000){\scalebox{1 -1}{$ex\_hi$}}
\psset{linewidth=0.100000}
\psset{linestyle=solid}
\psset{linestyle=solid}
\setlinejoinmode{0}
\setlinecaps{0}
\newrgbcolor{dialinecolor}{0.000000 0.000000 0.000000}
\psset{linecolor=dialinecolor}
\psline(36.000000,9.000000)(36.000000,9.000000)(36.585786,9.000000)(36.585786,6.707107)
\newrgbcolor{dialinecolor}{1.000000 1.000000 1.000000}
\psset{linecolor=dialinecolor}
\psellipse*(38.000000,6.000000)(2.000000,1.000000)
\psset{linewidth=0.100000}
\psset{linestyle=solid}
\psset{linestyle=solid}
\newrgbcolor{dialinecolor}{0.000000 0.000000 0.000000}
\psset{linecolor=dialinecolor}
\psellipse(38.000000,6.000000)(2.000000,1.000000)
\setfont{Courier}{0.700000}
\newrgbcolor{dialinecolor}{0.000000 0.000000 0.000000}
\psset{linecolor=dialinecolor}
\rput(38.000000,6.000000){\scalebox{1 -1}{Dekker}}
\psset{linewidth=0.100000}
\psset{linestyle=solid}
\psset{linestyle=solid}
\setlinejoinmode{0}
\setlinecaps{0}
\newrgbcolor{dialinecolor}{0.000000 0.000000 0.000000}
\psset{linecolor=dialinecolor}
\psline(36.000000,3.000000)(36.000000,3.000000)(36.585786,3.000000)(36.585786,5.292893)
\newrgbcolor{dialinecolor}{1.000000 1.000000 1.000000}
\psset{linecolor=dialinecolor}
\pspolygon*(40.000000,1.000000)(40.000000,3.000000)(44.000000,3.000000)(44.000000,1.000000)
\psset{linewidth=0.100000}
\psset{linestyle=solid}
\psset{linestyle=solid}
\setlinejoinmode{0}
\newrgbcolor{dialinecolor}{0.000000 0.000000 0.000000}
\psset{linecolor=dialinecolor}
\pspolygon(40.000000,1.000000)(40.000000,3.000000)(44.000000,3.000000)(44.000000,1.000000)
\setfont{Times-Bold}{1.000000}
\newrgbcolor{dialinecolor}{0.000000 0.000000 0.000000}
\psset{linecolor=dialinecolor}
\rput(42.000000,2.000000){\scalebox{1 -1}{$R5\_lo$}}
\newrgbcolor{dialinecolor}{1.000000 1.000000 1.000000}
\psset{linecolor=dialinecolor}
\pspolygon*(40.000000,7.000000)(40.000000,9.000000)(44.000000,9.000000)(44.000000,7.000000)
\psset{linewidth=0.100000}
\psset{linestyle=solid}
\psset{linestyle=solid}
\setlinejoinmode{0}
\newrgbcolor{dialinecolor}{0.000000 0.000000 0.000000}
\psset{linecolor=dialinecolor}
\pspolygon(40.000000,7.000000)(40.000000,9.000000)(44.000000,9.000000)(44.000000,7.000000)
\setfont{Times-Bold}{1.000000}
\newrgbcolor{dialinecolor}{0.000000 0.000000 0.000000}
\psset{linecolor=dialinecolor}
\rput(42.000000,8.000000){\scalebox{1 -1}{$R5\_hi$}}
\psset{linewidth=0.100000}
\psset{linestyle=solid}
\psset{linestyle=solid}
\setlinejoinmode{0}
\setlinecaps{0}
\newrgbcolor{dialinecolor}{0.000000 0.000000 0.000000}
\psset{linecolor=dialinecolor}
\psline(40.000000,3.000000)(40.000000,3.000000)(39.414214,3.000000)(39.414214,5.292893)
\psset{linewidth=0.100000}
\psset{linestyle=solid}
\psset{linestyle=solid}
\setlinejoinmode{0}
\setlinecaps{0}
\newrgbcolor{dialinecolor}{0.000000 0.000000 0.000000}
\psset{linecolor=dialinecolor}
\psline(40.000000,9.000000)(40.000000,9.000000)(39.414214,9.000000)(39.414214,6.707107)
\newrgbcolor{dialinecolor}{1.000000 1.000000 1.000000}
\psset{linecolor=dialinecolor}
\pspolygon*(32.000000,-5.000000)(32.000000,-3.000000)(36.000000,-3.000000)(36.000000,-5.000000)
\psset{linewidth=0.100000}
\psset{linestyle=solid}
\psset{linestyle=solid}
\setlinejoinmode{0}
\newrgbcolor{dialinecolor}{0.000000 0.000000 0.000000}
\psset{linecolor=dialinecolor}
\pspolygon(32.000000,-5.000000)(32.000000,-3.000000)(36.000000,-3.000000)(36.000000,-5.000000)
\setfont{Times-Bold}{1.000000}
\newrgbcolor{dialinecolor}{0.000000 0.000000 0.000000}
\psset{linecolor=dialinecolor}
\rput(34.000000,-4.000000){\scalebox{1 -1}{$rp\_hi$}}
\psset{linewidth=0.100000}
\psset{linestyle=solid}
\psset{linestyle=solid}
\setlinecaps{0}
\setlinejoinmode{0}
\setlinecaps{0}
\setlinejoinmode{0}
\psset{linestyle=solid}
\newrgbcolor{dialinecolor}{1.000000 1.000000 1.000000}
\psset{linecolor=dialinecolor}
\psellipse*(38.000000,0.000000)(1.000000,1.000000)
\newrgbcolor{dialinecolor}{0.000000 0.000000 0.000000}
\psset{linecolor=dialinecolor}
\psellipse(38.000000,0.000000)(1.000000,1.000000)
\setlinecaps{0}
\setlinejoinmode{0}
\psset{linestyle=solid}
\newrgbcolor{dialinecolor}{0.000000 0.000000 0.000000}
\psset{linecolor=dialinecolor}
\psline(38.000000,-1.000000)(38.000000,1.000000)
\setlinecaps{0}
\setlinejoinmode{0}
\psset{linestyle=solid}
\newrgbcolor{dialinecolor}{0.000000 0.000000 0.000000}
\psset{linecolor=dialinecolor}
\psline(37.000000,0.000000)(39.000000,0.000000)
\psset{linewidth=0.100000}
\psset{linestyle=solid}
\psset{linestyle=solid}
\setlinejoinmode{0}
\setlinecaps{0}
\newrgbcolor{dialinecolor}{0.000000 0.000000 0.000000}
\psset{linecolor=dialinecolor}
\psline(36.000000,-4.000000)(36.000000,-3.000000)(38.000000,-3.000000)(38.000000,-1.000000)
\psset{linewidth=0.100000}
\psset{linestyle=solid}
\psset{linestyle=solid}
\setlinejoinmode{0}
\setlinecaps{0}
\newrgbcolor{dialinecolor}{0.000000 0.000000 0.000000}
\psset{linecolor=dialinecolor}
\psline(36.000000,3.000000)(36.000000,3.000000)(38.000000,3.000000)(38.000000,1.000000)
\newrgbcolor{dialinecolor}{1.000000 1.000000 1.000000}
\psset{linecolor=dialinecolor}
\pspolygon*(32.000000,-11.000000)(32.000000,-9.000000)(36.000000,-9.000000)(36.000000,-11.000000)
\psset{linewidth=0.100000}
\psset{linestyle=solid}
\psset{linestyle=solid}
\setlinejoinmode{0}
\newrgbcolor{dialinecolor}{0.000000 0.000000 0.000000}
\psset{linecolor=dialinecolor}
\pspolygon(32.000000,-11.000000)(32.000000,-9.000000)(36.000000,-9.000000)(36.000000,-11.000000)
\setfont{Times-Bold}{1.000000}
\newrgbcolor{dialinecolor}{0.000000 0.000000 0.000000}
\psset{linecolor=dialinecolor}
\rput(34.000000,-10.000000){\scalebox{1 -1}{$ex\_lo$}}
\psset{linewidth=0.100000}
\psset{linestyle=solid}
\psset{linestyle=solid}
\setlinejoinmode{0}
\setlinecaps{0}
\newrgbcolor{dialinecolor}{0.000000 0.000000 0.000000}
\psset{linecolor=dialinecolor}
\psline(36.000000,-10.000000)(36.000000,-9.000000)(41.000000,-9.000000)(41.000000,-4.000000)
\psset{linewidth=0.100000}
\psset{linestyle=solid}
\psset{linestyle=solid}
\setlinejoinmode{0}
\setlinecaps{0}
\newrgbcolor{dialinecolor}{0.000000 0.000000 0.000000}
\psset{linecolor=dialinecolor}
\psline(39.000000,0.000000)(39.000000,0.000000)(41.000000,0.000000)(41.000000,-2.000000)
\psset{linewidth=0.100000}
\psset{linestyle=solid}
\psset{linestyle=solid}
\setlinecaps{0}
\setlinejoinmode{0}
\setlinecaps{0}
\setlinejoinmode{0}
\psset{linestyle=solid}
\newrgbcolor{dialinecolor}{1.000000 1.000000 1.000000}
\psset{linecolor=dialinecolor}
\psellipse*(46.000000,-6.000000)(1.000000,1.000000)
\newrgbcolor{dialinecolor}{0.000000 0.000000 0.000000}
\psset{linecolor=dialinecolor}
\psellipse(46.000000,-6.000000)(1.000000,1.000000)
\setlinecaps{0}
\setlinejoinmode{0}
\psset{linestyle=solid}
\newrgbcolor{dialinecolor}{0.000000 0.000000 0.000000}
\psset{linecolor=dialinecolor}
\psline(46.000000,-7.000000)(46.000000,-5.000000)
\setlinecaps{0}
\setlinejoinmode{0}
\psset{linestyle=solid}
\newrgbcolor{dialinecolor}{0.000000 0.000000 0.000000}
\psset{linecolor=dialinecolor}
\psline(45.000000,-6.000000)(47.000000,-6.000000)
\psset{linewidth=0.100000}
\psset{linestyle=solid}
\psset{linestyle=solid}
\setlinejoinmode{0}
\setlinecaps{0}
\newrgbcolor{dialinecolor}{0.000000 0.000000 0.000000}
\psset{linecolor=dialinecolor}
\psline(44.000000,-13.000000)(44.000000,-12.000000)(46.000000,-12.000000)(46.000000,-7.000000)
\psset{linewidth=0.100000}
\psset{linestyle=solid}
\psset{linestyle=solid}
\setlinejoinmode{0}
\setlinecaps{0}
\newrgbcolor{dialinecolor}{0.000000 0.000000 0.000000}
\psset{linecolor=dialinecolor}
\psline(42.000000,-3.000000)(42.000000,-3.000000)(46.000000,-3.000000)(46.000000,-5.000000)
\psset{linewidth=0.100000}
\psset{linestyle=solid}
\psset{linestyle=solid}
\setlinecaps{0}
\setlinejoinmode{0}
\setlinecaps{0}
\setlinejoinmode{0}
\psset{linestyle=solid}
\newrgbcolor{dialinecolor}{1.000000 1.000000 1.000000}
\psset{linecolor=dialinecolor}
\psellipse*(41.000000,-3.000000)(1.000000,1.000000)
\newrgbcolor{dialinecolor}{0.000000 0.000000 0.000000}
\psset{linecolor=dialinecolor}
\psellipse(41.000000,-3.000000)(1.000000,1.000000)
\setlinecaps{0}
\setlinejoinmode{0}
\psset{linestyle=solid}
\newrgbcolor{dialinecolor}{0.000000 0.000000 0.000000}
\psset{linecolor=dialinecolor}
\psline(40.292900,-3.707100)(41.707100,-2.292900)
\setlinecaps{0}
\setlinejoinmode{0}
\psset{linestyle=solid}
\newrgbcolor{dialinecolor}{0.000000 0.000000 0.000000}
\psset{linecolor=dialinecolor}
\psline(40.292900,-2.292900)(41.707100,-3.707100)
\newrgbcolor{dialinecolor}{1.000000 1.000000 1.000000}
\psset{linecolor=dialinecolor}
\pspolygon*(48.000000,-8.000000)(48.000000,-6.000000)(52.000000,-6.000000)(52.000000,-8.000000)
\psset{linewidth=0.100000}
\psset{linestyle=solid}
\psset{linestyle=solid}
\setlinejoinmode{0}
\newrgbcolor{dialinecolor}{0.000000 0.000000 0.000000}
\psset{linecolor=dialinecolor}
\pspolygon(48.000000,-8.000000)(48.000000,-6.000000)(52.000000,-6.000000)(52.000000,-8.000000)
\setfont{Times-Bold}{1.000000}
\newrgbcolor{dialinecolor}{0.000000 0.000000 0.000000}
\psset{linecolor=dialinecolor}
\rput(50.000000,-7.000000){\scalebox{1 -1}{$R6$}}
\psset{linewidth=0.100000}
\psset{linestyle=solid}
\psset{linestyle=solid}
\setlinejoinmode{0}
\setlinecaps{0}
\newrgbcolor{dialinecolor}{0.000000 0.000000 0.000000}
\psset{linecolor=dialinecolor}
\psline(47.000000,-6.000000)(47.000000,-6.000000)(48.000000,-6.000000)(48.000000,-6.000000)
\newrgbcolor{dialinecolor}{1.000000 1.000000 1.000000}
\psset{linecolor=dialinecolor}
\pspolygon*(41.000000,22.000000)(41.000000,24.000000)(45.000000,24.000000)(45.000000,22.000000)
\psset{linewidth=0.100000}
\psset{linestyle=solid}
\psset{linestyle=solid}
\setlinejoinmode{0}
\newrgbcolor{dialinecolor}{0.000000 0.000000 0.000000}
\psset{linecolor=dialinecolor}
\pspolygon(41.000000,22.000000)(41.000000,24.000000)(45.000000,24.000000)(45.000000,22.000000)
\setfont{Times-Bold}{1.000000}
\newrgbcolor{dialinecolor}{0.000000 0.000000 0.000000}
\psset{linecolor=dialinecolor}
\rput(43.000000,23.000000){\scalebox{1 -1}{$ex\_hi$}}
\psset{linewidth=0.100000}
\psset{linestyle=solid}
\psset{linestyle=solid}
\setlinecaps{0}
\setlinejoinmode{0}
\setlinecaps{0}
\setlinejoinmode{0}
\psset{linestyle=solid}
\newrgbcolor{dialinecolor}{1.000000 1.000000 1.000000}
\psset{linecolor=dialinecolor}
\psellipse*(47.000000,21.000000)(1.000000,1.000000)
\newrgbcolor{dialinecolor}{0.000000 0.000000 0.000000}
\psset{linecolor=dialinecolor}
\psellipse(47.000000,21.000000)(1.000000,1.000000)
\setlinecaps{0}
\setlinejoinmode{0}
\psset{linestyle=solid}
\newrgbcolor{dialinecolor}{0.000000 0.000000 0.000000}
\psset{linecolor=dialinecolor}
\psline(46.292900,20.292900)(47.707100,21.707100)
\setlinecaps{0}
\setlinejoinmode{0}
\psset{linestyle=solid}
\newrgbcolor{dialinecolor}{0.000000 0.000000 0.000000}
\psset{linecolor=dialinecolor}
\psline(46.292900,21.707100)(47.707100,20.292900)
\newrgbcolor{dialinecolor}{1.000000 1.000000 1.000000}
\psset{linecolor=dialinecolor}
\pspolygon*(49.000000,19.000000)(49.000000,21.000000)(53.000000,21.000000)(53.000000,19.000000)
\psset{linewidth=0.100000}
\psset{linestyle=solid}
\psset{linestyle=solid}
\setlinejoinmode{0}
\newrgbcolor{dialinecolor}{0.000000 0.000000 0.000000}
\psset{linecolor=dialinecolor}
\pspolygon(49.000000,19.000000)(49.000000,21.000000)(53.000000,21.000000)(53.000000,19.000000)
\setfont{Times-Bold}{1.000000}
\newrgbcolor{dialinecolor}{0.000000 0.000000 0.000000}
\psset{linecolor=dialinecolor}
\rput(51.000000,20.000000){\scalebox{1 -1}{$R7$}}
\psset{linewidth=0.100000}
\psset{linestyle=solid}
\psset{linestyle=solid}
\setlinejoinmode{0}
\setlinecaps{0}
\newrgbcolor{dialinecolor}{0.000000 0.000000 0.000000}
\psset{linecolor=dialinecolor}
\psline(45.000000,18.000000)(45.000000,18.000000)(47.000000,18.000000)(47.000000,20.000000)
\psset{linewidth=0.100000}
\psset{linestyle=solid}
\psset{linestyle=solid}
\setlinejoinmode{0}
\setlinecaps{0}
\newrgbcolor{dialinecolor}{0.000000 0.000000 0.000000}
\psset{linecolor=dialinecolor}
\psline(45.000000,24.000000)(45.000000,24.000000)(47.000000,24.000000)(47.000000,22.000000)
\psset{linewidth=0.100000}
\psset{linestyle=solid}
\psset{linestyle=solid}
\setlinejoinmode{0}
\setlinecaps{0}
\newrgbcolor{dialinecolor}{0.000000 0.000000 0.000000}
\psset{linecolor=dialinecolor}
\psline(48.000000,21.000000)(48.000000,21.000000)(49.000000,21.000000)(49.000000,20.000000)
\psset{linewidth=0.100000}
\psset{linestyle=solid}
\psset{linestyle=solid}
\setlinecaps{0}
\setlinejoinmode{0}
\setlinecaps{0}
\setlinejoinmode{0}
\psset{linestyle=solid}
\newrgbcolor{dialinecolor}{1.000000 1.000000 1.000000}
\psset{linecolor=dialinecolor}
\psellipse*(54.000000,-3.000000)(1.000000,1.000000)
\newrgbcolor{dialinecolor}{0.000000 0.000000 0.000000}
\psset{linecolor=dialinecolor}
\psellipse(54.000000,-3.000000)(1.000000,1.000000)
\setlinecaps{0}
\setlinejoinmode{0}
\psset{linestyle=solid}
\newrgbcolor{dialinecolor}{0.000000 0.000000 0.000000}
\psset{linecolor=dialinecolor}
\psline(54.000000,-4.000000)(54.000000,-2.000000)
\setlinecaps{0}
\setlinejoinmode{0}
\psset{linestyle=solid}
\newrgbcolor{dialinecolor}{0.000000 0.000000 0.000000}
\psset{linecolor=dialinecolor}
\psline(53.000000,-3.000000)(55.000000,-3.000000)
\psset{linewidth=0.100000}
\psset{linestyle=solid}
\psset{linestyle=solid}
\setlinejoinmode{0}
\setlinecaps{0}
\newrgbcolor{dialinecolor}{0.000000 0.000000 0.000000}
\psset{linecolor=dialinecolor}
\psline(52.000000,-6.000000)(52.000000,-6.000000)(54.000000,-6.000000)(54.000000,-4.000000)
\psset{linewidth=0.100000}
\psset{linestyle=solid}
\psset{linestyle=solid}
\setlinejoinmode{0}
\setlinecaps{0}
\newrgbcolor{dialinecolor}{0.000000 0.000000 0.000000}
\psset{linecolor=dialinecolor}
\psline(44.000000,3.000000)(44.000000,3.000000)(54.000000,3.000000)(54.000000,-2.000000)
\newrgbcolor{dialinecolor}{1.000000 1.000000 1.000000}
\psset{linecolor=dialinecolor}
\pspolygon*(40.000000,-26.000000)(40.000000,-24.000000)(44.000000,-24.000000)(44.000000,-26.000000)
\psset{linewidth=0.100000}
\psset{linestyle=solid}
\psset{linestyle=solid}
\setlinejoinmode{0}
\newrgbcolor{dialinecolor}{0.000000 0.000000 0.000000}
\psset{linecolor=dialinecolor}
\pspolygon(40.000000,-26.000000)(40.000000,-24.000000)(44.000000,-24.000000)(44.000000,-26.000000)
\setfont{Times-Bold}{1.000000}
\newrgbcolor{dialinecolor}{0.000000 0.000000 0.000000}
\psset{linecolor=dialinecolor}
\rput(42.000000,-25.000000){\scalebox{1 -1}{$rp\_hi$}}
\newrgbcolor{dialinecolor}{1.000000 1.000000 1.000000}
\psset{linecolor=dialinecolor}
\pspolygon*(40.000000,-20.000000)(40.000000,-18.000000)(44.000000,-18.000000)(44.000000,-20.000000)
\psset{linewidth=0.100000}
\psset{linestyle=solid}
\psset{linestyle=solid}
\setlinejoinmode{0}
\newrgbcolor{dialinecolor}{0.000000 0.000000 0.000000}
\psset{linecolor=dialinecolor}
\pspolygon(40.000000,-20.000000)(40.000000,-18.000000)(44.000000,-18.000000)(44.000000,-20.000000)
\setfont{Times-Bold}{1.000000}
\newrgbcolor{dialinecolor}{0.000000 0.000000 0.000000}
\psset{linecolor=dialinecolor}
\rput(42.000000,-19.000000){\scalebox{1 -1}{$ex\_hi$}}
\psset{linewidth=0.100000}
\psset{linestyle=solid}
\psset{linestyle=solid}
\setlinejoinmode{0}
\setlinecaps{0}
\newrgbcolor{dialinecolor}{0.000000 0.000000 0.000000}
\psset{linecolor=dialinecolor}
\psline(44.000000,-24.000000)(44.000000,-24.000000)(44.585786,-24.000000)(44.585786,-21.707107)
\psset{linewidth=0.100000}
\psset{linestyle=solid}
\psset{linestyle=solid}
\setlinejoinmode{0}
\setlinecaps{0}
\newrgbcolor{dialinecolor}{0.000000 0.000000 0.000000}
\psset{linecolor=dialinecolor}
\psline(44.000000,-19.000000)(44.000000,-19.000000)(44.585786,-19.000000)(44.585786,-20.292893)
\newrgbcolor{dialinecolor}{1.000000 1.000000 1.000000}
\psset{linecolor=dialinecolor}
\psellipse*(46.000000,-21.000000)(2.000000,1.000000)
\psset{linewidth=0.100000}
\psset{linestyle=solid}
\psset{linestyle=solid}
\newrgbcolor{dialinecolor}{0.000000 0.000000 0.000000}
\psset{linecolor=dialinecolor}
\psellipse(46.000000,-21.000000)(2.000000,1.000000)
\setfont{Courier}{0.700000}
\newrgbcolor{dialinecolor}{0.000000 0.000000 0.000000}
\psset{linecolor=dialinecolor}
\rput(46.000000,-21.000000){\scalebox{1 -1}{Dekker}}
\newrgbcolor{dialinecolor}{1.000000 1.000000 1.000000}
\psset{linecolor=dialinecolor}
\pspolygon*(48.000000,-26.000000)(48.000000,-24.000000)(52.000000,-24.000000)(52.000000,-26.000000)
\psset{linewidth=0.100000}
\psset{linestyle=solid}
\psset{linestyle=solid}
\setlinejoinmode{0}
\newrgbcolor{dialinecolor}{0.000000 0.000000 0.000000}
\psset{linecolor=dialinecolor}
\pspolygon(48.000000,-26.000000)(48.000000,-24.000000)(52.000000,-24.000000)(52.000000,-26.000000)
\setfont{Times-Bold}{1.000000}
\newrgbcolor{dialinecolor}{0.000000 0.000000 0.000000}
\psset{linecolor=dialinecolor}
\rput(50.000000,-25.000000){\scalebox{1 -1}{$R3\_lo$}}
\newrgbcolor{dialinecolor}{1.000000 1.000000 1.000000}
\psset{linecolor=dialinecolor}
\pspolygon*(48.000000,-20.000000)(48.000000,-18.000000)(52.000000,-18.000000)(52.000000,-20.000000)
\psset{linewidth=0.100000}
\psset{linestyle=solid}
\psset{linestyle=solid}
\setlinejoinmode{0}
\newrgbcolor{dialinecolor}{0.000000 0.000000 0.000000}
\psset{linecolor=dialinecolor}
\pspolygon(48.000000,-20.000000)(48.000000,-18.000000)(52.000000,-18.000000)(52.000000,-20.000000)
\setfont{Times-Bold}{1.000000}
\newrgbcolor{dialinecolor}{0.000000 0.000000 0.000000}
\psset{linecolor=dialinecolor}
\rput(50.000000,-19.000000){\scalebox{1 -1}{$R3\_lo$}}
\psset{linewidth=0.100000}
\psset{linestyle=solid}
\psset{linestyle=solid}
\setlinejoinmode{0}
\setlinecaps{0}
\newrgbcolor{dialinecolor}{0.000000 0.000000 0.000000}
\psset{linecolor=dialinecolor}
\psline(48.000000,-24.000000)(48.000000,-24.000000)(47.414214,-24.000000)(47.414214,-21.707107)
\psset{linewidth=0.100000}
\psset{linestyle=solid}
\psset{linestyle=solid}
\setlinejoinmode{0}
\setlinecaps{0}
\newrgbcolor{dialinecolor}{0.000000 0.000000 0.000000}
\psset{linecolor=dialinecolor}
\psline(48.000000,-19.000000)(48.000000,-19.000000)(47.414214,-19.000000)(47.414214,-20.292893)
\newrgbcolor{dialinecolor}{1.000000 1.000000 1.000000}
\psset{linecolor=dialinecolor}
\pspolygon*(48.000000,-14.000000)(48.000000,-12.000000)(52.000000,-12.000000)(52.000000,-14.000000)
\psset{linewidth=0.100000}
\psset{linestyle=solid}
\psset{linestyle=solid}
\setlinejoinmode{0}
\newrgbcolor{dialinecolor}{0.000000 0.000000 0.000000}
\psset{linecolor=dialinecolor}
\pspolygon(48.000000,-14.000000)(48.000000,-12.000000)(52.000000,-12.000000)(52.000000,-14.000000)
\setfont{Times-Bold}{1.000000}
\newrgbcolor{dialinecolor}{0.000000 0.000000 0.000000}
\psset{linecolor=dialinecolor}
\rput(50.000000,-13.000000){\scalebox{1 -1}{$ex\_lo$}}
\psset{linewidth=0.100000}
\psset{linestyle=solid}
\psset{linestyle=solid}
\setlinecaps{0}
\setlinejoinmode{0}
\setlinecaps{0}
\setlinejoinmode{0}
\psset{linestyle=solid}
\newrgbcolor{dialinecolor}{1.000000 1.000000 1.000000}
\psset{linecolor=dialinecolor}
\psellipse*(54.000000,-15.000000)(1.000000,1.000000)
\newrgbcolor{dialinecolor}{0.000000 0.000000 0.000000}
\psset{linecolor=dialinecolor}
\psellipse(54.000000,-15.000000)(1.000000,1.000000)
\setlinecaps{0}
\setlinejoinmode{0}
\psset{linestyle=solid}
\newrgbcolor{dialinecolor}{0.000000 0.000000 0.000000}
\psset{linecolor=dialinecolor}
\psline(54.000000,-16.000000)(54.000000,-14.000000)
\setlinecaps{0}
\setlinejoinmode{0}
\psset{linestyle=solid}
\newrgbcolor{dialinecolor}{0.000000 0.000000 0.000000}
\psset{linecolor=dialinecolor}
\psline(53.000000,-15.000000)(55.000000,-15.000000)
\psset{linewidth=0.100000}
\psset{linestyle=solid}
\psset{linestyle=solid}
\setlinejoinmode{0}
\setlinecaps{0}
\newrgbcolor{dialinecolor}{0.000000 0.000000 0.000000}
\psset{linecolor=dialinecolor}
\psline(52.000000,-19.000000)(52.000000,-18.000000)(54.000000,-18.000000)(54.000000,-14.000000)
\psset{linewidth=0.100000}
\psset{linestyle=solid}
\psset{linestyle=solid}
\setlinejoinmode{0}
\setlinecaps{0}
\newrgbcolor{dialinecolor}{0.000000 0.000000 0.000000}
\psset{linecolor=dialinecolor}
\psline(52.000000,-13.000000)(52.000000,-12.000000)(54.000000,-12.000000)(54.000000,-16.000000)
\psset{linewidth=0.100000}
\psset{linestyle=solid}
\psset{linestyle=solid}
\setlinecaps{0}
\setlinejoinmode{0}
\setlinecaps{0}
\setlinejoinmode{0}
\psset{linestyle=solid}
\newrgbcolor{dialinecolor}{1.000000 1.000000 1.000000}
\psset{linecolor=dialinecolor}
\psellipse*(56.000000,-10.000000)(1.000000,1.000000)
\newrgbcolor{dialinecolor}{0.000000 0.000000 0.000000}
\psset{linecolor=dialinecolor}
\psellipse(56.000000,-10.000000)(1.000000,1.000000)
\setlinecaps{0}
\setlinejoinmode{0}
\psset{linestyle=solid}
\newrgbcolor{dialinecolor}{0.000000 0.000000 0.000000}
\psset{linecolor=dialinecolor}
\psline(56.000000,-11.000000)(56.000000,-9.000000)
\setlinecaps{0}
\setlinejoinmode{0}
\psset{linestyle=solid}
\newrgbcolor{dialinecolor}{0.000000 0.000000 0.000000}
\psset{linecolor=dialinecolor}
\psline(55.000000,-10.000000)(57.000000,-10.000000)
\psset{linewidth=0.100000}
\psset{linestyle=solid}
\psset{linestyle=solid}
\setlinejoinmode{0}
\setlinecaps{0}
\newrgbcolor{dialinecolor}{0.000000 0.000000 0.000000}
\psset{linecolor=dialinecolor}
\psline(55.000000,-15.000000)(55.000000,-15.000000)(56.000000,-15.000000)(56.000000,-11.000000)
\psset{linewidth=0.100000}
\psset{linestyle=solid}
\psset{linestyle=solid}
\setlinejoinmode{0}
\setlinecaps{0}
\newrgbcolor{dialinecolor}{0.000000 0.000000 0.000000}
\psset{linecolor=dialinecolor}
\psline(56.000000,-9.000000)(56.000000,-3.000000)(55.000000,-3.000000)(55.000000,-3.000000)
\psset{linewidth=0.100000}
\psset{linestyle=solid}
\psset{linestyle=solid}
\setlinecaps{0}
\setlinejoinmode{0}
\setlinecaps{0}
\setlinejoinmode{0}
\psset{linestyle=solid}
\newrgbcolor{dialinecolor}{1.000000 1.000000 1.000000}
\psset{linecolor=dialinecolor}
\psellipse*(58.000000,-2.000000)(1.000000,1.000000)
\newrgbcolor{dialinecolor}{0.000000 0.000000 0.000000}
\psset{linecolor=dialinecolor}
\psellipse(58.000000,-2.000000)(1.000000,1.000000)
\setlinecaps{0}
\setlinejoinmode{0}
\psset{linestyle=solid}
\newrgbcolor{dialinecolor}{0.000000 0.000000 0.000000}
\psset{linecolor=dialinecolor}
\psline(58.000000,-3.000000)(58.000000,-1.000000)
\setlinecaps{0}
\setlinejoinmode{0}
\psset{linestyle=solid}
\newrgbcolor{dialinecolor}{0.000000 0.000000 0.000000}
\psset{linecolor=dialinecolor}
\psline(57.000000,-2.000000)(59.000000,-2.000000)
\psset{linewidth=0.100000}
\psset{linestyle=solid}
\psset{linestyle=solid}
\setlinejoinmode{0}
\setlinecaps{0}
\newrgbcolor{dialinecolor}{0.000000 0.000000 0.000000}
\psset{linecolor=dialinecolor}
\psline(53.000000,21.000000)(53.000000,18.000000)(58.000000,18.000000)(58.000000,-1.000000)
\psset{linewidth=0.100000}
\psset{linestyle=solid}
\psset{linestyle=solid}
\setlinejoinmode{0}
\setlinecaps{0}
\newrgbcolor{dialinecolor}{0.000000 0.000000 0.000000}
\psset{linecolor=dialinecolor}
\psline(57.000000,-10.000000)(57.000000,-10.000000)(58.000000,-10.000000)(58.000000,-3.000000)
\newrgbcolor{dialinecolor}{1.000000 1.000000 1.000000}
\psset{linecolor=dialinecolor}
\pspolygon*(60.000000,-3.000000)(60.000000,-1.000000)(64.000000,-1.000000)(64.000000,-3.000000)
\psset{linewidth=0.100000}
\psset{linestyle=solid}
\psset{linestyle=solid}
\setlinejoinmode{0}
\newrgbcolor{dialinecolor}{0.000000 0.000000 0.000000}
\psset{linecolor=dialinecolor}
\pspolygon(60.000000,-3.000000)(60.000000,-1.000000)(64.000000,-1.000000)(64.000000,-3.000000)
\setfont{Times-Bold}{1.000000}
\newrgbcolor{dialinecolor}{0.000000 0.000000 0.000000}
\psset{linecolor=dialinecolor}
\rput(62.000000,-2.000000){\scalebox{1 -1}{$R7$}}
\psset{linewidth=0.100000}
\psset{linestyle=solid}
\psset{linestyle=solid}
\setlinejoinmode{0}
\setlinecaps{0}
\newrgbcolor{dialinecolor}{0.000000 0.000000 0.000000}
\psset{linecolor=dialinecolor}
\psline(59.000000,-2.000000)(59.000000,-1.930000)(60.000000,-1.930000)(60.000000,-2.000000)
\newrgbcolor{dialinecolor}{1.000000 1.000000 1.000000}
\psset{linecolor=dialinecolor}
\psellipse*(66.000000,2.000000)(2.000000,1.000000)
\psset{linewidth=0.100000}
\psset{linestyle=solid}
\psset{linestyle=solid}
\newrgbcolor{dialinecolor}{0.000000 0.000000 0.000000}
\psset{linecolor=dialinecolor}
\psellipse(66.000000,2.000000)(2.000000,1.000000)
\setfont{Courier}{0.700000}
\newrgbcolor{dialinecolor}{0.000000 0.000000 0.000000}
\psset{linecolor=dialinecolor}
\rput(66.000000,2.000000){\scalebox{1 -1}{Fast2Sum}}
\psset{linewidth=0.100000}
\psset{linestyle=solid}
\psset{linestyle=solid}
\setlinejoinmode{0}
\setlinecaps{0}
\newrgbcolor{dialinecolor}{0.000000 0.000000 0.000000}
\psset{linecolor=dialinecolor}
\psline(64.000000,-2.000000)(64.000000,-2.000000)(64.585800,-2.000000)(64.585800,1.292890)
\psset{linewidth=0.100000}
\psset{linestyle=solid}
\psset{linestyle=solid}
\setlinejoinmode{0}
\setlinecaps{0}
\newrgbcolor{dialinecolor}{0.000000 0.000000 0.000000}
\psset{linecolor=dialinecolor}
\psline(44.000000,9.000000)(44.000000,9.000000)(64.585800,9.000000)(64.585800,2.707110)
\newrgbcolor{dialinecolor}{1.000000 1.000000 1.000000}
\psset{linecolor=dialinecolor}
\pspolygon*(68.000000,-3.000000)(68.000000,-1.000000)(72.000000,-1.000000)(72.000000,-3.000000)
\psset{linewidth=0.100000}
\psset{linestyle=solid}
\psset{linestyle=solid}
\setlinejoinmode{0}
\newrgbcolor{dialinecolor}{0.000000 0.000000 0.000000}
\psset{linecolor=dialinecolor}
\pspolygon(68.000000,-3.000000)(68.000000,-1.000000)(72.000000,-1.000000)(72.000000,-3.000000)
\setfont{Times-Bold}{1.000000}
\newrgbcolor{dialinecolor}{0.000000 0.000000 0.000000}
\psset{linecolor=dialinecolor}
\rput(70.000000,-2.000000){\scalebox{1 -1}{$R9$}}
\newrgbcolor{dialinecolor}{1.000000 1.000000 1.000000}
\psset{linecolor=dialinecolor}
\pspolygon*(68.000000,3.000000)(68.000000,5.000000)(72.000000,5.000000)(72.000000,3.000000)
\psset{linewidth=0.100000}
\psset{linestyle=solid}
\psset{linestyle=solid}
\setlinejoinmode{0}
\newrgbcolor{dialinecolor}{0.000000 0.000000 0.000000}
\psset{linecolor=dialinecolor}
\pspolygon(68.000000,3.000000)(68.000000,5.000000)(72.000000,5.000000)(72.000000,3.000000)
\setfont{Times-Bold}{1.000000}
\newrgbcolor{dialinecolor}{0.000000 0.000000 0.000000}
\psset{linecolor=dialinecolor}
\rput(70.000000,4.000000){\scalebox{1 -1}{$R8$}}
\psset{linewidth=0.100000}
\psset{linestyle=solid}
\psset{linestyle=solid}
\setlinejoinmode{0}
\setlinecaps{0}
\newrgbcolor{dialinecolor}{0.000000 0.000000 0.000000}
\psset{linecolor=dialinecolor}
\psline(68.000000,-1.000000)(68.000000,-1.000000)(67.414200,-1.000000)(67.414200,1.292890)
\psset{linewidth=0.100000}
\psset{linestyle=solid}
\psset{linestyle=solid}
\setlinejoinmode{0}
\setlinecaps{0}
\newrgbcolor{dialinecolor}{0.000000 0.000000 0.000000}
\psset{linecolor=dialinecolor}
\psline(68.000000,4.000000)(68.000000,4.000000)(67.414200,4.000000)(67.414200,2.707110)
\newrgbcolor{dialinecolor}{1.000000 1.000000 1.000000}
\psset{linecolor=dialinecolor}
\psellipse*(74.000000,-10.000000)(2.000000,1.000000)
\psset{linewidth=0.100000}
\psset{linestyle=solid}
\psset{linestyle=solid}
\newrgbcolor{dialinecolor}{0.000000 0.000000 0.000000}
\psset{linecolor=dialinecolor}
\psellipse(74.000000,-10.000000)(2.000000,1.000000)
\setfont{Courier}{0.700000}
\newrgbcolor{dialinecolor}{0.000000 0.000000 0.000000}
\psset{linecolor=dialinecolor}
\rput(74.000000,-10.000000){\scalebox{1 -1}{Fast2Sum}}
\newrgbcolor{dialinecolor}{1.000000 1.000000 1.000000}
\psset{linecolor=dialinecolor}
\pspolygon*(76.000000,-15.000000)(76.000000,-13.000000)(80.000000,-13.000000)(80.000000,-15.000000)
\psset{linewidth=0.100000}
\psset{linestyle=solid}
\psset{linestyle=solid}
\setlinejoinmode{0}
\newrgbcolor{dialinecolor}{0.000000 0.000000 0.000000}
\psset{linecolor=dialinecolor}
\pspolygon(76.000000,-15.000000)(76.000000,-13.000000)(80.000000,-13.000000)(80.000000,-15.000000)
\setfont{Times-Bold}{1.000000}
\newrgbcolor{dialinecolor}{0.000000 0.000000 0.000000}
\psset{linecolor=dialinecolor}
\rput(78.000000,-14.000000){\scalebox{1 -1}{$R10$}}
\newrgbcolor{dialinecolor}{1.000000 1.000000 1.000000}
\psset{linecolor=dialinecolor}
\pspolygon*(76.000000,-9.000000)(76.000000,-7.000000)(80.000000,-7.000000)(80.000000,-9.000000)
\psset{linewidth=0.100000}
\psset{linestyle=solid}
\psset{linestyle=solid}
\setlinejoinmode{0}
\newrgbcolor{dialinecolor}{0.000000 0.000000 0.000000}
\psset{linecolor=dialinecolor}
\pspolygon(76.000000,-9.000000)(76.000000,-7.000000)(80.000000,-7.000000)(80.000000,-9.000000)
\setfont{Times-Bold}{1.000000}
\newrgbcolor{dialinecolor}{0.000000 0.000000 0.000000}
\psset{linecolor=dialinecolor}
\rput(78.000000,-8.000000){\scalebox{1 -1}{$tmp$}}
\psset{linewidth=0.100000}
\psset{linestyle=solid}
\psset{linestyle=solid}
\setlinejoinmode{0}
\setlinecaps{0}
\newrgbcolor{dialinecolor}{0.000000 0.000000 0.000000}
\psset{linecolor=dialinecolor}
\psline(76.000000,-14.000000)(76.000000,-13.000000)(75.414200,-13.000000)(75.414200,-10.707100)
\psset{linewidth=0.100000}
\psset{linestyle=solid}
\psset{linestyle=solid}
\setlinejoinmode{0}
\setlinecaps{0}
\newrgbcolor{dialinecolor}{0.000000 0.000000 0.000000}
\psset{linecolor=dialinecolor}
\psline(76.000000,-7.000000)(76.000000,-7.000000)(75.414200,-7.000000)(75.414200,-9.292890)
\psset{linewidth=0.100000}
\psset{linestyle=solid}
\psset{linestyle=solid}
\setlinejoinmode{0}
\setlinecaps{0}
\newrgbcolor{dialinecolor}{0.000000 0.000000 0.000000}
\psset{linecolor=dialinecolor}
\psline(72.000000,-2.000000)(72.000000,-1.000000)(72.585800,-1.000000)(72.585800,-9.292890)
\psset{linewidth=0.100000}
\psset{linestyle=solid}
\psset{linestyle=solid}
\setlinejoinmode{0}
\setlinecaps{0}
\newrgbcolor{dialinecolor}{0.000000 0.000000 0.000000}
\psset{linecolor=dialinecolor}
\psline(52.000000,-24.000000)(52.000000,-23.000000)(72.585800,-23.000000)(72.585800,-10.707100)
\psset{linewidth=0.100000}
\psset{linestyle=solid}
\psset{linestyle=solid}
\setlinecaps{0}
\setlinejoinmode{0}
\setlinecaps{0}
\setlinejoinmode{0}
\psset{linestyle=solid}
\newrgbcolor{dialinecolor}{1.000000 1.000000 1.000000}
\psset{linecolor=dialinecolor}
\psellipse*(82.000000,-2.000000)(1.000000,1.000000)
\newrgbcolor{dialinecolor}{0.000000 0.000000 0.000000}
\psset{linecolor=dialinecolor}
\psellipse(82.000000,-2.000000)(1.000000,1.000000)
\setlinecaps{0}
\setlinejoinmode{0}
\psset{linestyle=solid}
\newrgbcolor{dialinecolor}{0.000000 0.000000 0.000000}
\psset{linecolor=dialinecolor}
\psline(82.000000,-3.000000)(82.000000,-1.000000)
\setlinecaps{0}
\setlinejoinmode{0}
\psset{linestyle=solid}
\newrgbcolor{dialinecolor}{0.000000 0.000000 0.000000}
\psset{linecolor=dialinecolor}
\psline(81.000000,-2.000000)(83.000000,-2.000000)
\psset{linewidth=0.100000}
\psset{linestyle=solid}
\psset{linestyle=solid}
\setlinejoinmode{0}
\setlinecaps{0}
\newrgbcolor{dialinecolor}{0.000000 0.000000 0.000000}
\psset{linecolor=dialinecolor}
\psline(72.000000,4.000000)(72.000000,4.000000)(82.000000,4.000000)(82.000000,-1.000000)
\psset{linewidth=0.100000}
\psset{linestyle=solid}
\psset{linestyle=solid}
\setlinejoinmode{0}
\setlinecaps{0}
\newrgbcolor{dialinecolor}{0.000000 0.000000 0.000000}
\psset{linecolor=dialinecolor}
\psline(80.000000,-7.000000)(80.000000,-7.000000)(82.000000,-7.000000)(82.000000,-3.000000)
\newrgbcolor{dialinecolor}{1.000000 1.000000 1.000000}
\psset{linecolor=dialinecolor}
\pspolygon*(84.000000,-4.000000)(84.000000,-2.000000)(88.000000,-2.000000)(88.000000,-4.000000)
\psset{linewidth=0.100000}
\psset{linestyle=solid}
\psset{linestyle=solid}
\setlinejoinmode{0}
\newrgbcolor{dialinecolor}{0.000000 0.000000 0.000000}
\psset{linecolor=dialinecolor}
\pspolygon(84.000000,-4.000000)(84.000000,-2.000000)(88.000000,-2.000000)(88.000000,-4.000000)
\setfont{Times-Bold}{1.000000}
\newrgbcolor{dialinecolor}{0.000000 0.000000 0.000000}
\psset{linecolor=dialinecolor}
\rput(86.000000,-3.000000){\scalebox{1 -1}{$R8$}}
\psset{linewidth=0.100000}
\psset{linestyle=solid}
\psset{linestyle=solid}
\setlinejoinmode{0}
\setlinecaps{0}
\newrgbcolor{dialinecolor}{0.000000 0.000000 0.000000}
\psset{linecolor=dialinecolor}
\psline(83.000000,-2.000000)(83.000000,-2.000000)(84.000000,-2.000000)(84.000000,-3.000000)
\newrgbcolor{dialinecolor}{1.000000 1.000000 1.000000}
\psset{linecolor=dialinecolor}
\pspolygon*(76.000000,-21.000000)(76.000000,-19.000000)(80.000000,-19.000000)(80.000000,-21.000000)
\psset{linewidth=0.100000}
\psset{linestyle=solid}
\psset{linestyle=solid}
\setlinejoinmode{0}
\newrgbcolor{dialinecolor}{0.000000 0.000000 0.000000}
\psset{linecolor=dialinecolor}
\pspolygon(76.000000,-21.000000)(76.000000,-19.000000)(80.000000,-19.000000)(80.000000,-21.000000)
\setfont{Times-Bold}{1.000000}
\newrgbcolor{dialinecolor}{0.000000 0.000000 0.000000}
\psset{linecolor=dialinecolor}
\rput(78.000000,-20.000000){\scalebox{1 -1}{$ex\_hi$}}
\newrgbcolor{dialinecolor}{1.000000 1.000000 1.000000}
\psset{linecolor=dialinecolor}
\psellipse*(82.000000,-16.000000)(2.000000,1.000000)
\psset{linewidth=0.100000}
\psset{linestyle=solid}
\psset{linestyle=solid}
\newrgbcolor{dialinecolor}{0.000000 0.000000 0.000000}
\psset{linecolor=dialinecolor}
\psellipse(82.000000,-16.000000)(2.000000,1.000000)
\setfont{Courier}{0.700000}
\newrgbcolor{dialinecolor}{0.000000 0.000000 0.000000}
\psset{linecolor=dialinecolor}
\rput(82.000000,-16.000000){\scalebox{1 -1}{Fast2Sum}}
\newrgbcolor{dialinecolor}{1.000000 1.000000 1.000000}
\psset{linecolor=dialinecolor}
\pspolygon*(84.000000,-21.000000)(84.000000,-19.000000)(88.000000,-19.000000)(88.000000,-21.000000)
\psset{linewidth=0.100000}
\psset{linestyle=solid}
\psset{linestyle=solid}
\setlinejoinmode{0}
\newrgbcolor{dialinecolor}{0.000000 0.000000 0.000000}
\psset{linecolor=dialinecolor}
\pspolygon(84.000000,-21.000000)(84.000000,-19.000000)(88.000000,-19.000000)(88.000000,-21.000000)
\setfont{Times-Bold}{1.000000}
\newrgbcolor{dialinecolor}{0.000000 0.000000 0.000000}
\psset{linecolor=dialinecolor}
\rput(86.000000,-20.000000){\scalebox{1 -1}{$R11$}}
\newrgbcolor{dialinecolor}{1.000000 1.000000 1.000000}
\psset{linecolor=dialinecolor}
\pspolygon*(84.000000,-15.000000)(84.000000,-13.000000)(88.000000,-13.000000)(88.000000,-15.000000)
\psset{linewidth=0.100000}
\psset{linestyle=solid}
\psset{linestyle=solid}
\setlinejoinmode{0}
\newrgbcolor{dialinecolor}{0.000000 0.000000 0.000000}
\psset{linecolor=dialinecolor}
\pspolygon(84.000000,-15.000000)(84.000000,-13.000000)(88.000000,-13.000000)(88.000000,-15.000000)
\setfont{Times-Bold}{1.000000}
\newrgbcolor{dialinecolor}{0.000000 0.000000 0.000000}
\psset{linecolor=dialinecolor}
\rput(86.000000,-14.000000){\scalebox{1 -1}{$tmp$}}
\psset{linewidth=0.100000}
\psset{linestyle=solid}
\psset{linestyle=solid}
\setlinejoinmode{0}
\setlinecaps{0}
\newrgbcolor{dialinecolor}{0.000000 0.000000 0.000000}
\psset{linecolor=dialinecolor}
\psline(84.000000,-19.000000)(84.000000,-19.000000)(83.414214,-19.000000)(83.414214,-16.707107)
\psset{linewidth=0.100000}
\psset{linestyle=solid}
\psset{linestyle=solid}
\setlinejoinmode{0}
\setlinecaps{0}
\newrgbcolor{dialinecolor}{0.000000 0.000000 0.000000}
\psset{linecolor=dialinecolor}
\psline(84.000000,-14.000000)(84.000000,-14.000000)(83.414214,-14.000000)(83.414214,-15.292893)
\psset{linewidth=0.100000}
\psset{linestyle=solid}
\psset{linestyle=solid}
\setlinejoinmode{0}
\setlinecaps{0}
\newrgbcolor{dialinecolor}{0.000000 0.000000 0.000000}
\psset{linecolor=dialinecolor}
\psline(88.000000,-14.000000)(88.000000,-12.000000)(90.000000,-12.000000)(90.000000,-8.000000)
\psset{linewidth=0.100000}
\psset{linestyle=solid}
\psset{linestyle=solid}
\setlinejoinmode{0}
\setlinecaps{0}
\newrgbcolor{dialinecolor}{0.000000 0.000000 0.000000}
\psset{linecolor=dialinecolor}
\psline(80.000000,-19.000000)(80.000000,-19.000000)(80.585786,-19.000000)(80.585786,-16.707107)
\psset{linewidth=0.100000}
\psset{linestyle=solid}
\psset{linestyle=solid}
\setlinejoinmode{0}
\setlinecaps{0}
\newrgbcolor{dialinecolor}{0.000000 0.000000 0.000000}
\psset{linecolor=dialinecolor}
\psline(80.000000,-14.000000)(80.000000,-14.000000)(80.585786,-14.000000)(80.585786,-15.292893)
\psset{linewidth=0.100000}
\psset{linestyle=solid}
\psset{linestyle=solid}
\setlinecaps{0}
\setlinejoinmode{0}
\setlinecaps{0}
\setlinejoinmode{0}
\psset{linestyle=solid}
\newrgbcolor{dialinecolor}{1.000000 1.000000 1.000000}
\psset{linecolor=dialinecolor}
\psellipse*(90.000000,-7.000000)(1.000000,1.000000)
\newrgbcolor{dialinecolor}{0.000000 0.000000 0.000000}
\psset{linecolor=dialinecolor}
\psellipse(90.000000,-7.000000)(1.000000,1.000000)
\setlinecaps{0}
\setlinejoinmode{0}
\psset{linestyle=solid}
\newrgbcolor{dialinecolor}{0.000000 0.000000 0.000000}
\psset{linecolor=dialinecolor}
\psline(90.000000,-8.000000)(90.000000,-6.000000)
\setlinecaps{0}
\setlinejoinmode{0}
\psset{linestyle=solid}
\newrgbcolor{dialinecolor}{0.000000 0.000000 0.000000}
\psset{linecolor=dialinecolor}
\psline(89.000000,-7.000000)(91.000000,-7.000000)
\psset{linewidth=0.100000}
\psset{linestyle=solid}
\psset{linestyle=solid}
\setlinejoinmode{0}
\setlinecaps{0}
\newrgbcolor{dialinecolor}{0.000000 0.000000 0.000000}
\psset{linecolor=dialinecolor}
\psline(88.000000,-3.000000)(88.000000,-2.000000)(90.000000,-2.000000)(90.000000,-6.000000)
\newrgbcolor{dialinecolor}{1.000000 1.000000 1.000000}
\psset{linecolor=dialinecolor}
\pspolygon*(92.000000,-9.000000)(92.000000,-7.000000)(96.000000,-7.000000)(96.000000,-9.000000)
\psset{linewidth=0.100000}
\psset{linestyle=solid}
\psset{linestyle=solid}
\setlinejoinmode{0}
\newrgbcolor{dialinecolor}{0.000000 0.000000 0.000000}
\psset{linecolor=dialinecolor}
\pspolygon(92.000000,-9.000000)(92.000000,-7.000000)(96.000000,-7.000000)(96.000000,-9.000000)
\setfont{Times-Bold}{1.000000}
\newrgbcolor{dialinecolor}{0.000000 0.000000 0.000000}
\psset{linecolor=dialinecolor}
\rput(94.000000,-8.000000){\scalebox{1 -1}{$R8$}}
\psset{linewidth=0.100000}
\psset{linestyle=solid}
\psset{linestyle=solid}
\setlinejoinmode{0}
\setlinecaps{0}
\newrgbcolor{dialinecolor}{0.000000 0.000000 0.000000}
\psset{linecolor=dialinecolor}
\psline(91.000000,-7.000000)(91.000000,-7.000000)(92.000000,-7.000000)(92.000000,-8.000000)
\newrgbcolor{dialinecolor}{1.000000 1.000000 1.000000}
\psset{linecolor=dialinecolor}
\psellipse*(98.000000,-12.000000)(2.000000,1.000000)
\psset{linewidth=0.100000}
\psset{linestyle=solid}
\psset{linestyle=solid}
\newrgbcolor{dialinecolor}{0.000000 0.000000 0.000000}
\psset{linecolor=dialinecolor}
\psellipse(98.000000,-12.000000)(2.000000,1.000000)
\setfont{Courier}{0.700000}
\newrgbcolor{dialinecolor}{0.000000 0.000000 0.000000}
\psset{linecolor=dialinecolor}
\rput(98.000000,-12.000000){\scalebox{1 -1}{Fast2Sum}}
\newrgbcolor{dialinecolor}{1.000000 1.000000 1.000000}
\psset{linecolor=dialinecolor}
\pspolygon*(100.000000,-17.000000)(100.000000,-15.000000)(104.000000,-15.000000)(104.000000,-17.000000)
\psset{linewidth=0.100000}
\psset{linestyle=solid}
\psset{linestyle=solid}
\setlinejoinmode{0}
\newrgbcolor{dialinecolor}{0.000000 0.000000 0.000000}
\psset{linecolor=dialinecolor}
\pspolygon(100.000000,-17.000000)(100.000000,-15.000000)(104.000000,-15.000000)(104.000000,-17.000000)
\setfont{Times-Bold}{1.000000}
\newrgbcolor{dialinecolor}{0.000000 0.000000 0.000000}
\psset{linecolor=dialinecolor}
\rput(102.000000,-16.000000){\scalebox{1 -1}{$R11$}}
\newrgbcolor{dialinecolor}{1.000000 1.000000 1.000000}
\psset{linecolor=dialinecolor}
\pspolygon*(100.000000,-11.000000)(100.000000,-9.000000)(104.000000,-9.000000)(104.000000,-11.000000)
\psset{linewidth=0.100000}
\psset{linestyle=solid}
\psset{linestyle=solid}
\setlinejoinmode{0}
\newrgbcolor{dialinecolor}{0.000000 0.000000 0.000000}
\psset{linecolor=dialinecolor}
\pspolygon(100.000000,-11.000000)(100.000000,-9.000000)(104.000000,-9.000000)(104.000000,-11.000000)
\setfont{Times-Bold}{1.000000}
\newrgbcolor{dialinecolor}{0.000000 0.000000 0.000000}
\psset{linecolor=dialinecolor}
\rput(102.000000,-10.000000){\scalebox{1 -1}{$R8$}}
\psset{linewidth=0.100000}
\psset{linestyle=solid}
\psset{linestyle=solid}
\setlinejoinmode{0}
\setlinecaps{0}
\newrgbcolor{dialinecolor}{0.000000 0.000000 0.000000}
\psset{linecolor=dialinecolor}
\psline(100.000000,-16.000000)(100.000000,-15.000000)(99.414200,-15.000000)(99.414200,-12.707100)
\psset{linewidth=0.100000}
\psset{linestyle=solid}
\psset{linestyle=solid}
\setlinejoinmode{0}
\setlinecaps{0}
\newrgbcolor{dialinecolor}{0.000000 0.000000 0.000000}
\psset{linecolor=dialinecolor}
\psline(100.000000,-9.000000)(100.000000,-9.000000)(99.414200,-9.000000)(99.414200,-11.292900)
\psset{linewidth=0.100000}
\psset{linestyle=solid}
\psset{linestyle=solid}
\setlinejoinmode{0}
\setlinecaps{0}
\newrgbcolor{dialinecolor}{0.000000 0.000000 0.000000}
\psset{linecolor=dialinecolor}
\psline(88.000000,-19.000000)(88.000000,-19.000000)(96.585800,-19.000000)(96.585800,-12.707100)
\psset{linewidth=0.100000}
\psset{linestyle=solid}
\psset{linestyle=solid}
\setlinejoinmode{0}
\setlinecaps{0}
\newrgbcolor{dialinecolor}{0.000000 0.000000 0.000000}
\psset{linecolor=dialinecolor}
\psline(96.000000,-8.000000)(96.000000,-7.000000)(96.585800,-7.000000)(96.585800,-11.292900)
\newrgbcolor{dialinecolor}{1.000000 1.000000 1.000000}
\psset{linecolor=dialinecolor}
\pspolygon*(16.000000,8.000000)(16.000000,9.000000)(20.000000,9.000000)(20.000000,8.000000)
\psset{linewidth=0.100000}
\psset{linestyle=solid}
\psset{linestyle=solid}
\setlinejoinmode{0}
\newrgbcolor{dialinecolor}{0.000000 0.000000 0.000000}
\psset{linecolor=dialinecolor}
\pspolygon(16.000000,8.000000)(16.000000,9.000000)(20.000000,9.000000)(20.000000,8.000000)
\newrgbcolor{dialinecolor}{1.000000 1.000000 1.000000}
\psset{linecolor=dialinecolor}
\pspolygon*(16.000000,8.000000)(16.000000,9.000000)(20.000000,9.000000)(20.000000,8.000000)
\psset{linewidth=0.100000}
\psset{linestyle=solid}
\psset{linestyle=solid}
\setlinejoinmode{0}
\newrgbcolor{dialinecolor}{0.000000 0.000000 0.000000}
\psset{linecolor=dialinecolor}
\pspolygon(16.000000,8.000000)(16.000000,9.000000)(20.000000,9.000000)(20.000000,8.000000)
\newrgbcolor{dialinecolor}{1.000000 1.000000 1.000000}
\psset{linecolor=dialinecolor}
\pspolygon*(16.000000,2.000000)(16.000000,3.000000)(20.000000,3.000000)(20.000000,2.000000)
\psset{linewidth=0.100000}
\psset{linestyle=solid}
\psset{linestyle=solid}
\setlinejoinmode{0}
\newrgbcolor{dialinecolor}{0.000000 0.000000 0.000000}
\psset{linecolor=dialinecolor}
\pspolygon(16.000000,2.000000)(16.000000,3.000000)(20.000000,3.000000)(20.000000,2.000000)
\newrgbcolor{dialinecolor}{1.000000 1.000000 1.000000}
\psset{linecolor=dialinecolor}
\pspolygon*(16.000000,3.000000)(16.000000,4.000000)(20.000000,4.000000)(20.000000,3.000000)
\psset{linewidth=0.100000}
\psset{linestyle=solid}
\psset{linestyle=solid}
\setlinejoinmode{0}
\newrgbcolor{dialinecolor}{0.000000 0.000000 0.000000}
\psset{linecolor=dialinecolor}
\pspolygon(16.000000,3.000000)(16.000000,4.000000)(20.000000,4.000000)(20.000000,3.000000)
\newrgbcolor{dialinecolor}{1.000000 1.000000 1.000000}
\psset{linecolor=dialinecolor}
\pspolygon*(100.000000,-15.000000)(100.000000,-14.000000)(104.000000,-14.000000)(104.000000,-15.000000)
\psset{linewidth=0.100000}
\psset{linestyle=solid}
\psset{linestyle=solid}
\setlinejoinmode{0}
\newrgbcolor{dialinecolor}{0.000000 0.000000 0.000000}
\psset{linecolor=dialinecolor}
\pspolygon(100.000000,-15.000000)(100.000000,-14.000000)(104.000000,-14.000000)(104.000000,-15.000000)
\newrgbcolor{dialinecolor}{1.000000 1.000000 1.000000}
\psset{linecolor=dialinecolor}
\pspolygon*(100.000000,-14.000000)(100.000000,-13.000000)(104.000000,-13.000000)(104.000000,-14.000000)
\psset{linewidth=0.100000}
\psset{linestyle=solid}
\psset{linestyle=solid}
\setlinejoinmode{0}
\newrgbcolor{dialinecolor}{0.000000 0.000000 0.000000}
\psset{linecolor=dialinecolor}
\pspolygon(100.000000,-14.000000)(100.000000,-13.000000)(104.000000,-13.000000)(104.000000,-14.000000)
\newrgbcolor{dialinecolor}{1.000000 1.000000 1.000000}
\psset{linecolor=dialinecolor}
\pspolygon*(24.000000,5.000000)(24.000000,6.000000)(28.000000,6.000000)(28.000000,5.000000)
\psset{linewidth=0.100000}
\psset{linestyle=solid}
\psset{linestyle=solid}
\setlinejoinmode{0}
\newrgbcolor{dialinecolor}{0.000000 0.000000 0.000000}
\psset{linecolor=dialinecolor}
\pspolygon(24.000000,5.000000)(24.000000,6.000000)(28.000000,6.000000)(28.000000,5.000000)
\newrgbcolor{dialinecolor}{1.000000 1.000000 1.000000}
\psset{linecolor=dialinecolor}
\pspolygon*(24.000000,6.000000)(24.000000,7.000000)(28.000000,7.000000)(28.000000,6.000000)
\psset{linewidth=0.100000}
\psset{linestyle=solid}
\psset{linestyle=solid}
\setlinejoinmode{0}
\newrgbcolor{dialinecolor}{0.000000 0.000000 0.000000}
\psset{linecolor=dialinecolor}
\pspolygon(24.000000,6.000000)(24.000000,7.000000)(28.000000,7.000000)(28.000000,6.000000)
\newrgbcolor{dialinecolor}{1.000000 1.000000 1.000000}
\psset{linecolor=dialinecolor}
\pspolygon*(24.000000,17.000000)(24.000000,18.000000)(28.000000,18.000000)(28.000000,17.000000)
\psset{linewidth=0.100000}
\psset{linestyle=solid}
\psset{linestyle=solid}
\setlinejoinmode{0}
\newrgbcolor{dialinecolor}{0.000000 0.000000 0.000000}
\psset{linecolor=dialinecolor}
\pspolygon(24.000000,17.000000)(24.000000,18.000000)(28.000000,18.000000)(28.000000,17.000000)
\newrgbcolor{dialinecolor}{1.000000 1.000000 1.000000}
\psset{linecolor=dialinecolor}
\pspolygon*(24.000000,18.000000)(24.000000,19.000000)(28.000000,19.000000)(28.000000,18.000000)
\psset{linewidth=0.100000}
\psset{linestyle=solid}
\psset{linestyle=solid}
\setlinejoinmode{0}
\newrgbcolor{dialinecolor}{0.000000 0.000000 0.000000}
\psset{linecolor=dialinecolor}
\pspolygon(24.000000,18.000000)(24.000000,19.000000)(28.000000,19.000000)(28.000000,18.000000)
\newrgbcolor{dialinecolor}{1.000000 1.000000 1.000000}
\psset{linecolor=dialinecolor}
\pspolygon*(32.000000,21.000000)(32.000000,22.000000)(36.000000,22.000000)(36.000000,21.000000)
\psset{linewidth=0.100000}
\psset{linestyle=solid}
\psset{linestyle=solid}
\setlinejoinmode{0}
\newrgbcolor{dialinecolor}{0.000000 0.000000 0.000000}
\psset{linecolor=dialinecolor}
\pspolygon(32.000000,21.000000)(32.000000,22.000000)(36.000000,22.000000)(36.000000,21.000000)
\newrgbcolor{dialinecolor}{1.000000 1.000000 1.000000}
\psset{linecolor=dialinecolor}
\pspolygon*(32.000000,22.000000)(32.000000,23.000000)(36.000000,23.000000)(36.000000,22.000000)
\psset{linewidth=0.100000}
\psset{linestyle=solid}
\psset{linestyle=solid}
\setlinejoinmode{0}
\newrgbcolor{dialinecolor}{0.000000 0.000000 0.000000}
\psset{linecolor=dialinecolor}
\pspolygon(32.000000,22.000000)(32.000000,23.000000)(36.000000,23.000000)(36.000000,22.000000)
\newrgbcolor{dialinecolor}{1.000000 1.000000 1.000000}
\psset{linecolor=dialinecolor}
\pspolygon*(41.000000,24.000000)(41.000000,25.000000)(45.000000,25.000000)(45.000000,24.000000)
\psset{linewidth=0.100000}
\psset{linestyle=solid}
\psset{linestyle=solid}
\setlinejoinmode{0}
\newrgbcolor{dialinecolor}{0.000000 0.000000 0.000000}
\psset{linecolor=dialinecolor}
\pspolygon(41.000000,24.000000)(41.000000,25.000000)(45.000000,25.000000)(45.000000,24.000000)
\newrgbcolor{dialinecolor}{1.000000 1.000000 1.000000}
\psset{linecolor=dialinecolor}
\pspolygon*(41.000000,25.000000)(41.000000,26.000000)(45.000000,26.000000)(45.000000,25.000000)
\psset{linewidth=0.100000}
\psset{linestyle=solid}
\psset{linestyle=solid}
\setlinejoinmode{0}
\newrgbcolor{dialinecolor}{0.000000 0.000000 0.000000}
\psset{linecolor=dialinecolor}
\pspolygon(41.000000,25.000000)(41.000000,26.000000)(45.000000,26.000000)(45.000000,25.000000)
\newrgbcolor{dialinecolor}{1.000000 1.000000 1.000000}
\psset{linecolor=dialinecolor}
\pspolygon*(49.000000,21.000000)(49.000000,22.000000)(53.000000,22.000000)(53.000000,21.000000)
\psset{linewidth=0.100000}
\psset{linestyle=solid}
\psset{linestyle=solid}
\setlinejoinmode{0}
\newrgbcolor{dialinecolor}{0.000000 0.000000 0.000000}
\psset{linecolor=dialinecolor}
\pspolygon(49.000000,21.000000)(49.000000,22.000000)(53.000000,22.000000)(53.000000,21.000000)
\newrgbcolor{dialinecolor}{1.000000 1.000000 1.000000}
\psset{linecolor=dialinecolor}
\pspolygon*(49.000000,22.000000)(49.000000,23.000000)(53.000000,23.000000)(53.000000,22.000000)
\psset{linewidth=0.100000}
\psset{linestyle=solid}
\psset{linestyle=solid}
\setlinejoinmode{0}
\newrgbcolor{dialinecolor}{0.000000 0.000000 0.000000}
\psset{linecolor=dialinecolor}
\pspolygon(49.000000,22.000000)(49.000000,23.000000)(53.000000,23.000000)(53.000000,22.000000)
\newrgbcolor{dialinecolor}{1.000000 1.000000 1.000000}
\psset{linecolor=dialinecolor}
\pspolygon*(48.000000,-6.000000)(48.000000,-5.000000)(52.000000,-5.000000)(52.000000,-6.000000)
\psset{linewidth=0.100000}
\psset{linestyle=solid}
\psset{linestyle=solid}
\setlinejoinmode{0}
\newrgbcolor{dialinecolor}{0.000000 0.000000 0.000000}
\psset{linecolor=dialinecolor}
\pspolygon(48.000000,-6.000000)(48.000000,-5.000000)(52.000000,-5.000000)(52.000000,-6.000000)
\newrgbcolor{dialinecolor}{1.000000 1.000000 1.000000}
\psset{linecolor=dialinecolor}
\pspolygon*(48.000000,-5.000000)(48.000000,-4.000000)(52.000000,-4.000000)(52.000000,-5.000000)
\psset{linewidth=0.100000}
\psset{linestyle=solid}
\psset{linestyle=solid}
\setlinejoinmode{0}
\newrgbcolor{dialinecolor}{0.000000 0.000000 0.000000}
\psset{linecolor=dialinecolor}
\pspolygon(48.000000,-5.000000)(48.000000,-4.000000)(52.000000,-4.000000)(52.000000,-5.000000)
\newrgbcolor{dialinecolor}{1.000000 1.000000 1.000000}
\psset{linecolor=dialinecolor}
\pspolygon*(60.000000,-1.000000)(60.000000,0.000000)(64.000000,0.000000)(64.000000,-1.000000)
\psset{linewidth=0.100000}
\psset{linestyle=solid}
\psset{linestyle=solid}
\setlinejoinmode{0}
\newrgbcolor{dialinecolor}{0.000000 0.000000 0.000000}
\psset{linecolor=dialinecolor}
\pspolygon(60.000000,-1.000000)(60.000000,0.000000)(64.000000,0.000000)(64.000000,-1.000000)
\newrgbcolor{dialinecolor}{1.000000 1.000000 1.000000}
\psset{linecolor=dialinecolor}
\pspolygon*(60.000000,0.000000)(60.000000,1.000000)(64.000000,1.000000)(64.000000,0.000000)
\psset{linewidth=0.100000}
\psset{linestyle=solid}
\psset{linestyle=solid}
\setlinejoinmode{0}
\newrgbcolor{dialinecolor}{0.000000 0.000000 0.000000}
\psset{linecolor=dialinecolor}
\pspolygon(60.000000,0.000000)(60.000000,1.000000)(64.000000,1.000000)(64.000000,0.000000)
\newrgbcolor{dialinecolor}{1.000000 1.000000 1.000000}
\psset{linecolor=dialinecolor}
\pspolygon*(48.000000,-12.000000)(48.000000,-11.000000)(52.000000,-11.000000)(52.000000,-12.000000)
\psset{linewidth=0.100000}
\psset{linestyle=solid}
\psset{linestyle=solid}
\setlinejoinmode{0}
\newrgbcolor{dialinecolor}{0.000000 0.000000 0.000000}
\psset{linecolor=dialinecolor}
\pspolygon(48.000000,-12.000000)(48.000000,-11.000000)(52.000000,-11.000000)(52.000000,-12.000000)
\newrgbcolor{dialinecolor}{1.000000 1.000000 1.000000}
\psset{linecolor=dialinecolor}
\pspolygon*(48.000000,-11.000000)(48.000000,-10.000000)(52.000000,-10.000000)(52.000000,-11.000000)
\psset{linewidth=0.100000}
\psset{linestyle=solid}
\psset{linestyle=solid}
\setlinejoinmode{0}
\newrgbcolor{dialinecolor}{0.000000 0.000000 0.000000}
\psset{linecolor=dialinecolor}
\pspolygon(48.000000,-11.000000)(48.000000,-10.000000)(52.000000,-10.000000)(52.000000,-11.000000)
\newrgbcolor{dialinecolor}{1.000000 1.000000 1.000000}
\psset{linecolor=dialinecolor}
\pspolygon*(68.000000,5.000000)(68.000000,6.000000)(72.000000,6.000000)(72.000000,5.000000)
\psset{linewidth=0.100000}
\psset{linestyle=solid}
\psset{linestyle=solid}
\setlinejoinmode{0}
\newrgbcolor{dialinecolor}{0.000000 0.000000 0.000000}
\psset{linecolor=dialinecolor}
\pspolygon(68.000000,5.000000)(68.000000,6.000000)(72.000000,6.000000)(72.000000,5.000000)
\newrgbcolor{dialinecolor}{1.000000 1.000000 1.000000}
\psset{linecolor=dialinecolor}
\pspolygon*(68.000000,6.000000)(68.000000,7.000000)(72.000000,7.000000)(72.000000,6.000000)
\psset{linewidth=0.100000}
\psset{linestyle=solid}
\psset{linestyle=solid}
\setlinejoinmode{0}
\newrgbcolor{dialinecolor}{0.000000 0.000000 0.000000}
\psset{linecolor=dialinecolor}
\pspolygon(68.000000,6.000000)(68.000000,7.000000)(72.000000,7.000000)(72.000000,6.000000)
\newrgbcolor{dialinecolor}{1.000000 1.000000 1.000000}
\psset{linecolor=dialinecolor}
\pspolygon*(40.000000,-12.000000)(40.000000,-11.000000)(44.000000,-11.000000)(44.000000,-12.000000)
\psset{linewidth=0.100000}
\psset{linestyle=solid}
\psset{linestyle=solid}
\setlinejoinmode{0}
\newrgbcolor{dialinecolor}{0.000000 0.000000 0.000000}
\psset{linecolor=dialinecolor}
\pspolygon(40.000000,-12.000000)(40.000000,-11.000000)(44.000000,-11.000000)(44.000000,-12.000000)
\newrgbcolor{dialinecolor}{1.000000 1.000000 1.000000}
\psset{linecolor=dialinecolor}
\pspolygon*(40.000000,-11.000000)(40.000000,-10.000000)(44.000000,-10.000000)(44.000000,-11.000000)
\psset{linewidth=0.100000}
\psset{linestyle=solid}
\psset{linestyle=solid}
\setlinejoinmode{0}
\newrgbcolor{dialinecolor}{0.000000 0.000000 0.000000}
\psset{linecolor=dialinecolor}
\pspolygon(40.000000,-11.000000)(40.000000,-10.000000)(44.000000,-10.000000)(44.000000,-11.000000)
\newrgbcolor{dialinecolor}{1.000000 1.000000 1.000000}
\psset{linecolor=dialinecolor}
\pspolygon*(40.000000,-18.000000)(40.000000,-17.000000)(44.000000,-17.000000)(44.000000,-18.000000)
\psset{linewidth=0.100000}
\psset{linestyle=solid}
\psset{linestyle=solid}
\setlinejoinmode{0}
\newrgbcolor{dialinecolor}{0.000000 0.000000 0.000000}
\psset{linecolor=dialinecolor}
\pspolygon(40.000000,-18.000000)(40.000000,-17.000000)(44.000000,-17.000000)(44.000000,-18.000000)
\newrgbcolor{dialinecolor}{1.000000 1.000000 1.000000}
\psset{linecolor=dialinecolor}
\pspolygon*(40.000000,-17.000000)(40.000000,-16.000000)(44.000000,-16.000000)(44.000000,-17.000000)
\psset{linewidth=0.100000}
\psset{linestyle=solid}
\psset{linestyle=solid}
\setlinejoinmode{0}
\newrgbcolor{dialinecolor}{0.000000 0.000000 0.000000}
\psset{linecolor=dialinecolor}
\pspolygon(40.000000,-17.000000)(40.000000,-16.000000)(44.000000,-16.000000)(44.000000,-17.000000)
\newrgbcolor{dialinecolor}{1.000000 1.000000 1.000000}
\psset{linecolor=dialinecolor}
\pspolygon*(76.000000,-7.000000)(76.000000,-6.000000)(80.000000,-6.000000)(80.000000,-7.000000)
\psset{linewidth=0.100000}
\psset{linestyle=solid}
\psset{linestyle=solid}
\setlinejoinmode{0}
\newrgbcolor{dialinecolor}{0.000000 0.000000 0.000000}
\psset{linecolor=dialinecolor}
\pspolygon(76.000000,-7.000000)(76.000000,-6.000000)(80.000000,-6.000000)(80.000000,-7.000000)
\newrgbcolor{dialinecolor}{1.000000 1.000000 1.000000}
\psset{linecolor=dialinecolor}
\pspolygon*(76.000000,-6.000000)(76.000000,-5.000000)(80.000000,-5.000000)(80.000000,-6.000000)
\psset{linewidth=0.100000}
\psset{linestyle=solid}
\psset{linestyle=solid}
\setlinejoinmode{0}
\newrgbcolor{dialinecolor}{0.000000 0.000000 0.000000}
\psset{linecolor=dialinecolor}
\pspolygon(76.000000,-6.000000)(76.000000,-5.000000)(80.000000,-5.000000)(80.000000,-6.000000)
\newrgbcolor{dialinecolor}{1.000000 1.000000 1.000000}
\psset{linecolor=dialinecolor}
\pspolygon*(84.000000,-2.000000)(84.000000,-1.000000)(88.000000,-1.000000)(88.000000,-2.000000)
\psset{linewidth=0.100000}
\psset{linestyle=solid}
\psset{linestyle=solid}
\setlinejoinmode{0}
\newrgbcolor{dialinecolor}{0.000000 0.000000 0.000000}
\psset{linecolor=dialinecolor}
\pspolygon(84.000000,-2.000000)(84.000000,-1.000000)(88.000000,-1.000000)(88.000000,-2.000000)
\newrgbcolor{dialinecolor}{1.000000 1.000000 1.000000}
\psset{linecolor=dialinecolor}
\pspolygon*(84.000000,-1.000000)(84.000000,0.000000)(88.000000,0.000000)(88.000000,-1.000000)
\psset{linewidth=0.100000}
\psset{linestyle=solid}
\psset{linestyle=solid}
\setlinejoinmode{0}
\newrgbcolor{dialinecolor}{0.000000 0.000000 0.000000}
\psset{linecolor=dialinecolor}
\pspolygon(84.000000,-1.000000)(84.000000,0.000000)(88.000000,0.000000)(88.000000,-1.000000)
\newrgbcolor{dialinecolor}{1.000000 1.000000 1.000000}
\psset{linecolor=dialinecolor}
\pspolygon*(84.000000,-13.000000)(84.000000,-12.000000)(88.000000,-12.000000)(88.000000,-13.000000)
\psset{linewidth=0.100000}
\psset{linestyle=solid}
\psset{linestyle=solid}
\setlinejoinmode{0}
\newrgbcolor{dialinecolor}{0.000000 0.000000 0.000000}
\psset{linecolor=dialinecolor}
\pspolygon(84.000000,-13.000000)(84.000000,-12.000000)(88.000000,-12.000000)(88.000000,-13.000000)
\newrgbcolor{dialinecolor}{1.000000 1.000000 1.000000}
\psset{linecolor=dialinecolor}
\pspolygon*(84.000000,-12.000000)(84.000000,-11.000000)(88.000000,-11.000000)(88.000000,-12.000000)
\psset{linewidth=0.100000}
\psset{linestyle=solid}
\psset{linestyle=solid}
\setlinejoinmode{0}
\newrgbcolor{dialinecolor}{0.000000 0.000000 0.000000}
\psset{linecolor=dialinecolor}
\pspolygon(84.000000,-12.000000)(84.000000,-11.000000)(88.000000,-11.000000)(88.000000,-12.000000)
\newrgbcolor{dialinecolor}{1.000000 1.000000 1.000000}
\psset{linecolor=dialinecolor}
\pspolygon*(92.000000,-7.000000)(92.000000,-6.000000)(96.000000,-6.000000)(96.000000,-7.000000)
\psset{linewidth=0.100000}
\psset{linestyle=solid}
\psset{linestyle=solid}
\setlinejoinmode{0}
\newrgbcolor{dialinecolor}{0.000000 0.000000 0.000000}
\psset{linecolor=dialinecolor}
\pspolygon(92.000000,-7.000000)(92.000000,-6.000000)(96.000000,-6.000000)(96.000000,-7.000000)
\newrgbcolor{dialinecolor}{1.000000 1.000000 1.000000}
\psset{linecolor=dialinecolor}
\pspolygon*(92.000000,-6.000000)(92.000000,-5.000000)(96.000000,-5.000000)(96.000000,-6.000000)
\psset{linewidth=0.100000}
\psset{linestyle=solid}
\psset{linestyle=solid}
\setlinejoinmode{0}
\newrgbcolor{dialinecolor}{0.000000 0.000000 0.000000}
\psset{linecolor=dialinecolor}
\pspolygon(92.000000,-6.000000)(92.000000,-5.000000)(96.000000,-5.000000)(96.000000,-6.000000)
\newrgbcolor{dialinecolor}{1.000000 1.000000 1.000000}
\psset{linecolor=dialinecolor}
\pspolygon*(100.000000,-9.000000)(100.000000,-8.000000)(104.000000,-8.000000)(104.000000,-9.000000)
\psset{linewidth=0.100000}
\psset{linestyle=solid}
\psset{linestyle=solid}
\setlinejoinmode{0}
\newrgbcolor{dialinecolor}{0.000000 0.000000 0.000000}
\psset{linecolor=dialinecolor}
\pspolygon(100.000000,-9.000000)(100.000000,-8.000000)(104.000000,-8.000000)(104.000000,-9.000000)
\newrgbcolor{dialinecolor}{1.000000 1.000000 1.000000}
\psset{linecolor=dialinecolor}
\pspolygon*(100.000000,-8.000000)(100.000000,-7.000000)(104.000000,-7.000000)(104.000000,-8.000000)
\psset{linewidth=0.100000}
\psset{linestyle=solid}
\psset{linestyle=solid}
\setlinejoinmode{0}
\newrgbcolor{dialinecolor}{0.000000 0.000000 0.000000}
\psset{linecolor=dialinecolor}
\pspolygon(100.000000,-8.000000)(100.000000,-7.000000)(104.000000,-7.000000)(104.000000,-8.000000)
\newrgbcolor{dialinecolor}{1.000000 1.000000 1.000000}
\psset{linecolor=dialinecolor}
\pspolygon*(41.000000,18.000000)(41.000000,19.000000)(45.000000,19.000000)(45.000000,18.000000)
\psset{linewidth=0.100000}
\psset{linestyle=solid}
\psset{linestyle=solid}
\setlinejoinmode{0}
\newrgbcolor{dialinecolor}{0.000000 0.000000 0.000000}
\psset{linecolor=dialinecolor}
\pspolygon(41.000000,18.000000)(41.000000,19.000000)(45.000000,19.000000)(45.000000,18.000000)
\newrgbcolor{dialinecolor}{1.000000 1.000000 1.000000}
\psset{linecolor=dialinecolor}
\pspolygon*(41.000000,19.000000)(41.000000,20.000000)(45.000000,20.000000)(45.000000,19.000000)
\psset{linewidth=0.100000}
\psset{linestyle=solid}
\psset{linestyle=solid}
\setlinejoinmode{0}
\newrgbcolor{dialinecolor}{0.000000 0.000000 0.000000}
\psset{linecolor=dialinecolor}
\pspolygon(41.000000,19.000000)(41.000000,20.000000)(45.000000,20.000000)(45.000000,19.000000)
\newrgbcolor{dialinecolor}{1.000000 1.000000 1.000000}
\psset{linecolor=dialinecolor}
\pspolygon*(32.000000,15.000000)(32.000000,16.000000)(36.000000,16.000000)(36.000000,15.000000)
\psset{linewidth=0.100000}
\psset{linestyle=solid}
\psset{linestyle=solid}
\setlinejoinmode{0}
\newrgbcolor{dialinecolor}{0.000000 0.000000 0.000000}
\psset{linecolor=dialinecolor}
\pspolygon(32.000000,15.000000)(32.000000,16.000000)(36.000000,16.000000)(36.000000,15.000000)
\newrgbcolor{dialinecolor}{1.000000 1.000000 1.000000}
\psset{linecolor=dialinecolor}
\pspolygon*(32.000000,16.000000)(32.000000,17.000000)(36.000000,17.000000)(36.000000,16.000000)
\psset{linewidth=0.100000}
\psset{linestyle=solid}
\psset{linestyle=solid}
\setlinejoinmode{0}
\newrgbcolor{dialinecolor}{0.000000 0.000000 0.000000}
\psset{linecolor=dialinecolor}
\pspolygon(32.000000,16.000000)(32.000000,17.000000)(36.000000,17.000000)(36.000000,16.000000)
\newrgbcolor{dialinecolor}{1.000000 1.000000 1.000000}
\psset{linecolor=dialinecolor}
\pspolygon*(40.000000,9.000000)(40.000000,10.000000)(44.000000,10.000000)(44.000000,9.000000)
\psset{linewidth=0.100000}
\psset{linestyle=solid}
\psset{linestyle=solid}
\setlinejoinmode{0}
\newrgbcolor{dialinecolor}{0.000000 0.000000 0.000000}
\psset{linecolor=dialinecolor}
\pspolygon(40.000000,9.000000)(40.000000,10.000000)(44.000000,10.000000)(44.000000,9.000000)
\newrgbcolor{dialinecolor}{1.000000 1.000000 1.000000}
\psset{linecolor=dialinecolor}
\pspolygon*(40.000000,10.000000)(40.000000,11.000000)(44.000000,11.000000)(44.000000,10.000000)
\psset{linewidth=0.100000}
\psset{linestyle=solid}
\psset{linestyle=solid}
\setlinejoinmode{0}
\newrgbcolor{dialinecolor}{0.000000 0.000000 0.000000}
\psset{linecolor=dialinecolor}
\pspolygon(40.000000,10.000000)(40.000000,11.000000)(44.000000,11.000000)(44.000000,10.000000)
\newrgbcolor{dialinecolor}{1.000000 1.000000 1.000000}
\psset{linecolor=dialinecolor}
\pspolygon*(32.000000,9.000000)(32.000000,10.000000)(36.000000,10.000000)(36.000000,9.000000)
\psset{linewidth=0.100000}
\psset{linestyle=solid}
\psset{linestyle=solid}
\setlinejoinmode{0}
\newrgbcolor{dialinecolor}{0.000000 0.000000 0.000000}
\psset{linecolor=dialinecolor}
\pspolygon(32.000000,9.000000)(32.000000,10.000000)(36.000000,10.000000)(36.000000,9.000000)
\newrgbcolor{dialinecolor}{1.000000 1.000000 1.000000}
\psset{linecolor=dialinecolor}
\pspolygon*(32.000000,10.000000)(32.000000,11.000000)(36.000000,11.000000)(36.000000,10.000000)
\psset{linewidth=0.100000}
\psset{linestyle=solid}
\psset{linestyle=solid}
\setlinejoinmode{0}
\newrgbcolor{dialinecolor}{0.000000 0.000000 0.000000}
\psset{linecolor=dialinecolor}
\pspolygon(32.000000,10.000000)(32.000000,11.000000)(36.000000,11.000000)(36.000000,10.000000)
\newrgbcolor{dialinecolor}{1.000000 1.000000 1.000000}
\psset{linecolor=dialinecolor}
\pspolygon*(40.000000,-24.000000)(40.000000,-23.000000)(44.000000,-23.000000)(44.000000,-24.000000)
\psset{linewidth=0.100000}
\psset{linestyle=solid}
\psset{linestyle=solid}
\setlinejoinmode{0}
\newrgbcolor{dialinecolor}{0.000000 0.000000 0.000000}
\psset{linecolor=dialinecolor}
\pspolygon(40.000000,-24.000000)(40.000000,-23.000000)(44.000000,-23.000000)(44.000000,-24.000000)
\newrgbcolor{dialinecolor}{1.000000 1.000000 1.000000}
\psset{linecolor=dialinecolor}
\pspolygon*(40.000000,-23.000000)(40.000000,-22.000000)(44.000000,-22.000000)(44.000000,-23.000000)
\psset{linewidth=0.100000}
\psset{linestyle=solid}
\psset{linestyle=solid}
\setlinejoinmode{0}
\newrgbcolor{dialinecolor}{0.000000 0.000000 0.000000}
\psset{linecolor=dialinecolor}
\pspolygon(40.000000,-23.000000)(40.000000,-22.000000)(44.000000,-22.000000)(44.000000,-23.000000)
\newrgbcolor{dialinecolor}{1.000000 1.000000 1.000000}
\psset{linecolor=dialinecolor}
\pspolygon*(48.000000,-24.000000)(48.000000,-23.000000)(52.000000,-23.000000)(52.000000,-24.000000)
\psset{linewidth=0.100000}
\psset{linestyle=solid}
\psset{linestyle=solid}
\setlinejoinmode{0}
\newrgbcolor{dialinecolor}{0.000000 0.000000 0.000000}
\psset{linecolor=dialinecolor}
\pspolygon(48.000000,-24.000000)(48.000000,-23.000000)(52.000000,-23.000000)(52.000000,-24.000000)
\newrgbcolor{dialinecolor}{1.000000 1.000000 1.000000}
\psset{linecolor=dialinecolor}
\pspolygon*(48.000000,-23.000000)(48.000000,-22.000000)(52.000000,-22.000000)(52.000000,-23.000000)
\psset{linewidth=0.100000}
\psset{linestyle=solid}
\psset{linestyle=solid}
\setlinejoinmode{0}
\newrgbcolor{dialinecolor}{0.000000 0.000000 0.000000}
\psset{linecolor=dialinecolor}
\pspolygon(48.000000,-23.000000)(48.000000,-22.000000)(52.000000,-22.000000)(52.000000,-23.000000)
\newrgbcolor{dialinecolor}{1.000000 1.000000 1.000000}
\psset{linecolor=dialinecolor}
\pspolygon*(24.000000,-15.000000)(24.000000,-14.000000)(28.000000,-14.000000)(28.000000,-15.000000)
\psset{linewidth=0.100000}
\psset{linestyle=solid}
\psset{linestyle=solid}
\setlinejoinmode{0}
\newrgbcolor{dialinecolor}{0.000000 0.000000 0.000000}
\psset{linecolor=dialinecolor}
\pspolygon(24.000000,-15.000000)(24.000000,-14.000000)(28.000000,-14.000000)(28.000000,-15.000000)
\newrgbcolor{dialinecolor}{1.000000 1.000000 1.000000}
\psset{linecolor=dialinecolor}
\pspolygon*(24.000000,-14.000000)(24.000000,-13.000000)(28.000000,-13.000000)(28.000000,-14.000000)
\psset{linewidth=0.100000}
\psset{linestyle=solid}
\psset{linestyle=solid}
\setlinejoinmode{0}
\newrgbcolor{dialinecolor}{0.000000 0.000000 0.000000}
\psset{linecolor=dialinecolor}
\pspolygon(24.000000,-14.000000)(24.000000,-13.000000)(28.000000,-13.000000)(28.000000,-14.000000)
\newrgbcolor{dialinecolor}{1.000000 1.000000 1.000000}
\psset{linecolor=dialinecolor}
\pspolygon*(24.000000,-9.000000)(24.000000,-8.000000)(28.000000,-8.000000)(28.000000,-9.000000)
\psset{linewidth=0.100000}
\psset{linestyle=solid}
\psset{linestyle=solid}
\setlinejoinmode{0}
\newrgbcolor{dialinecolor}{0.000000 0.000000 0.000000}
\psset{linecolor=dialinecolor}
\pspolygon(24.000000,-9.000000)(24.000000,-8.000000)(28.000000,-8.000000)(28.000000,-9.000000)
\newrgbcolor{dialinecolor}{1.000000 1.000000 1.000000}
\psset{linecolor=dialinecolor}
\pspolygon*(24.000000,-8.000000)(24.000000,-7.000000)(28.000000,-7.000000)(28.000000,-8.000000)
\psset{linewidth=0.100000}
\psset{linestyle=solid}
\psset{linestyle=solid}
\setlinejoinmode{0}
\newrgbcolor{dialinecolor}{0.000000 0.000000 0.000000}
\psset{linecolor=dialinecolor}
\pspolygon(24.000000,-8.000000)(24.000000,-7.000000)(28.000000,-7.000000)(28.000000,-8.000000)
\newrgbcolor{dialinecolor}{1.000000 1.000000 1.000000}
\psset{linecolor=dialinecolor}
\pspolygon*(32.000000,-9.000000)(32.000000,-8.000000)(36.000000,-8.000000)(36.000000,-9.000000)
\psset{linewidth=0.100000}
\psset{linestyle=solid}
\psset{linestyle=solid}
\setlinejoinmode{0}
\newrgbcolor{dialinecolor}{0.000000 0.000000 0.000000}
\psset{linecolor=dialinecolor}
\pspolygon(32.000000,-9.000000)(32.000000,-8.000000)(36.000000,-8.000000)(36.000000,-9.000000)
\newrgbcolor{dialinecolor}{1.000000 1.000000 1.000000}
\psset{linecolor=dialinecolor}
\pspolygon*(32.000000,-8.000000)(32.000000,-7.000000)(36.000000,-7.000000)(36.000000,-8.000000)
\psset{linewidth=0.100000}
\psset{linestyle=solid}
\psset{linestyle=solid}
\setlinejoinmode{0}
\newrgbcolor{dialinecolor}{0.000000 0.000000 0.000000}
\psset{linecolor=dialinecolor}
\pspolygon(32.000000,-8.000000)(32.000000,-7.000000)(36.000000,-7.000000)(36.000000,-8.000000)
\newrgbcolor{dialinecolor}{1.000000 1.000000 1.000000}
\psset{linecolor=dialinecolor}
\pspolygon*(32.000000,-3.000000)(32.000000,-2.000000)(36.000000,-2.000000)(36.000000,-3.000000)
\psset{linewidth=0.100000}
\psset{linestyle=solid}
\psset{linestyle=solid}
\setlinejoinmode{0}
\newrgbcolor{dialinecolor}{0.000000 0.000000 0.000000}
\psset{linecolor=dialinecolor}
\pspolygon(32.000000,-3.000000)(32.000000,-2.000000)(36.000000,-2.000000)(36.000000,-3.000000)
\newrgbcolor{dialinecolor}{1.000000 1.000000 1.000000}
\psset{linecolor=dialinecolor}
\pspolygon*(32.000000,-2.000000)(32.000000,-1.000000)(36.000000,-1.000000)(36.000000,-2.000000)
\psset{linewidth=0.100000}
\psset{linestyle=solid}
\psset{linestyle=solid}
\setlinejoinmode{0}
\newrgbcolor{dialinecolor}{0.000000 0.000000 0.000000}
\psset{linecolor=dialinecolor}
\pspolygon(32.000000,-2.000000)(32.000000,-1.000000)(36.000000,-1.000000)(36.000000,-2.000000)
\newrgbcolor{dialinecolor}{1.000000 1.000000 1.000000}
\psset{linecolor=dialinecolor}
\pspolygon*(48.000000,-18.000000)(48.000000,-17.000000)(52.000000,-17.000000)(52.000000,-18.000000)
\psset{linewidth=0.100000}
\psset{linestyle=solid}
\psset{linestyle=solid}
\setlinejoinmode{0}
\newrgbcolor{dialinecolor}{0.000000 0.000000 0.000000}
\psset{linecolor=dialinecolor}
\pspolygon(48.000000,-18.000000)(48.000000,-17.000000)(52.000000,-17.000000)(52.000000,-18.000000)
\newrgbcolor{dialinecolor}{1.000000 1.000000 1.000000}
\psset{linecolor=dialinecolor}
\pspolygon*(48.000000,-17.000000)(48.000000,-16.000000)(52.000000,-16.000000)(52.000000,-17.000000)
\psset{linewidth=0.100000}
\psset{linestyle=solid}
\psset{linestyle=solid}
\setlinejoinmode{0}
\newrgbcolor{dialinecolor}{0.000000 0.000000 0.000000}
\psset{linecolor=dialinecolor}
\pspolygon(48.000000,-17.000000)(48.000000,-16.000000)(52.000000,-16.000000)(52.000000,-17.000000)
\newrgbcolor{dialinecolor}{1.000000 1.000000 1.000000}
\psset{linecolor=dialinecolor}
\pspolygon*(32.000000,3.000000)(32.000000,4.000000)(36.000000,4.000000)(36.000000,3.000000)
\psset{linewidth=0.100000}
\psset{linestyle=solid}
\psset{linestyle=solid}
\setlinejoinmode{0}
\newrgbcolor{dialinecolor}{0.000000 0.000000 0.000000}
\psset{linecolor=dialinecolor}
\pspolygon(32.000000,3.000000)(32.000000,4.000000)(36.000000,4.000000)(36.000000,3.000000)
\newrgbcolor{dialinecolor}{1.000000 1.000000 1.000000}
\psset{linecolor=dialinecolor}
\pspolygon*(32.000000,4.000000)(32.000000,5.000000)(36.000000,5.000000)(36.000000,4.000000)
\psset{linewidth=0.100000}
\psset{linestyle=solid}
\psset{linestyle=solid}
\setlinejoinmode{0}
\newrgbcolor{dialinecolor}{0.000000 0.000000 0.000000}
\psset{linecolor=dialinecolor}
\pspolygon(32.000000,4.000000)(32.000000,5.000000)(36.000000,5.000000)(36.000000,4.000000)
\newrgbcolor{dialinecolor}{1.000000 1.000000 1.000000}
\psset{linecolor=dialinecolor}
\pspolygon*(40.000000,3.000000)(40.000000,4.000000)(44.000000,4.000000)(44.000000,3.000000)
\psset{linewidth=0.100000}
\psset{linestyle=solid}
\psset{linestyle=solid}
\setlinejoinmode{0}
\newrgbcolor{dialinecolor}{0.000000 0.000000 0.000000}
\psset{linecolor=dialinecolor}
\pspolygon(40.000000,3.000000)(40.000000,4.000000)(44.000000,4.000000)(44.000000,3.000000)
\newrgbcolor{dialinecolor}{1.000000 1.000000 1.000000}
\psset{linecolor=dialinecolor}
\pspolygon*(40.000000,4.000000)(40.000000,5.000000)(44.000000,5.000000)(44.000000,4.000000)
\psset{linewidth=0.100000}
\psset{linestyle=solid}
\psset{linestyle=solid}
\setlinejoinmode{0}
\newrgbcolor{dialinecolor}{0.000000 0.000000 0.000000}
\psset{linecolor=dialinecolor}
\pspolygon(40.000000,4.000000)(40.000000,5.000000)(44.000000,5.000000)(44.000000,4.000000)
\newrgbcolor{dialinecolor}{1.000000 1.000000 1.000000}
\psset{linecolor=dialinecolor}
\pspolygon*(68.000000,-1.000000)(68.000000,0.000000)(72.000000,0.000000)(72.000000,-1.000000)
\psset{linewidth=0.100000}
\psset{linestyle=solid}
\psset{linestyle=solid}
\setlinejoinmode{0}
\newrgbcolor{dialinecolor}{0.000000 0.000000 0.000000}
\psset{linecolor=dialinecolor}
\pspolygon(68.000000,-1.000000)(68.000000,0.000000)(72.000000,0.000000)(72.000000,-1.000000)
\newrgbcolor{dialinecolor}{1.000000 1.000000 1.000000}
\psset{linecolor=dialinecolor}
\pspolygon*(68.000000,0.000000)(68.000000,1.000000)(72.000000,1.000000)(72.000000,0.000000)
\psset{linewidth=0.100000}
\psset{linestyle=solid}
\psset{linestyle=solid}
\setlinejoinmode{0}
\newrgbcolor{dialinecolor}{0.000000 0.000000 0.000000}
\psset{linecolor=dialinecolor}
\pspolygon(68.000000,0.000000)(68.000000,1.000000)(72.000000,1.000000)(72.000000,0.000000)
\newrgbcolor{dialinecolor}{1.000000 1.000000 1.000000}
\psset{linecolor=dialinecolor}
\pspolygon*(76.000000,-13.000000)(76.000000,-12.000000)(80.000000,-12.000000)(80.000000,-13.000000)
\psset{linewidth=0.100000}
\psset{linestyle=solid}
\psset{linestyle=solid}
\setlinejoinmode{0}
\newrgbcolor{dialinecolor}{0.000000 0.000000 0.000000}
\psset{linecolor=dialinecolor}
\pspolygon(76.000000,-13.000000)(76.000000,-12.000000)(80.000000,-12.000000)(80.000000,-13.000000)
\newrgbcolor{dialinecolor}{1.000000 1.000000 1.000000}
\psset{linecolor=dialinecolor}
\pspolygon*(76.000000,-12.000000)(76.000000,-11.000000)(80.000000,-11.000000)(80.000000,-12.000000)
\psset{linewidth=0.100000}
\psset{linestyle=solid}
\psset{linestyle=solid}
\setlinejoinmode{0}
\newrgbcolor{dialinecolor}{0.000000 0.000000 0.000000}
\psset{linecolor=dialinecolor}
\pspolygon(76.000000,-12.000000)(76.000000,-11.000000)(80.000000,-11.000000)(80.000000,-12.000000)
\newrgbcolor{dialinecolor}{1.000000 1.000000 1.000000}
\psset{linecolor=dialinecolor}
\pspolygon*(76.000000,-19.000000)(76.000000,-18.000000)(80.000000,-18.000000)(80.000000,-19.000000)
\psset{linewidth=0.100000}
\psset{linestyle=solid}
\psset{linestyle=solid}
\setlinejoinmode{0}
\newrgbcolor{dialinecolor}{0.000000 0.000000 0.000000}
\psset{linecolor=dialinecolor}
\pspolygon(76.000000,-19.000000)(76.000000,-18.000000)(80.000000,-18.000000)(80.000000,-19.000000)
\newrgbcolor{dialinecolor}{1.000000 1.000000 1.000000}
\psset{linecolor=dialinecolor}
\pspolygon*(76.000000,-18.000000)(76.000000,-17.000000)(80.000000,-17.000000)(80.000000,-18.000000)
\psset{linewidth=0.100000}
\psset{linestyle=solid}
\psset{linestyle=solid}
\setlinejoinmode{0}
\newrgbcolor{dialinecolor}{0.000000 0.000000 0.000000}
\psset{linecolor=dialinecolor}
\pspolygon(76.000000,-18.000000)(76.000000,-17.000000)(80.000000,-17.000000)(80.000000,-18.000000)
\newrgbcolor{dialinecolor}{1.000000 1.000000 1.000000}
\psset{linecolor=dialinecolor}
\pspolygon*(84.000000,-19.000000)(84.000000,-18.000000)(88.000000,-18.000000)(88.000000,-19.000000)
\psset{linewidth=0.100000}
\psset{linestyle=solid}
\psset{linestyle=solid}
\setlinejoinmode{0}
\newrgbcolor{dialinecolor}{0.000000 0.000000 0.000000}
\psset{linecolor=dialinecolor}
\pspolygon(84.000000,-19.000000)(84.000000,-18.000000)(88.000000,-18.000000)(88.000000,-19.000000)
\newrgbcolor{dialinecolor}{1.000000 1.000000 1.000000}
\psset{linecolor=dialinecolor}
\pspolygon*(84.000000,-18.000000)(84.000000,-17.000000)(88.000000,-17.000000)(88.000000,-18.000000)
\psset{linewidth=0.100000}
\psset{linestyle=solid}
\psset{linestyle=solid}
\setlinejoinmode{0}
\newrgbcolor{dialinecolor}{0.000000 0.000000 0.000000}
\psset{linecolor=dialinecolor}
\pspolygon(84.000000,-18.000000)(84.000000,-17.000000)(88.000000,-17.000000)(88.000000,-18.000000)
\newrgbcolor{dialinecolor}{1.000000 1.000000 1.000000}
\psset{linecolor=dialinecolor}
\pspolygon*(16.000000,9.000000)(16.000000,10.000000)(20.000000,10.000000)(20.000000,9.000000)
\psset{linewidth=0.100000}
\psset{linestyle=solid}
\psset{linestyle=solid}
\setlinejoinmode{0}
\newrgbcolor{dialinecolor}{0.000000 0.000000 0.000000}
\psset{linecolor=dialinecolor}
\pspolygon(16.000000,9.000000)(16.000000,10.000000)(20.000000,10.000000)(20.000000,9.000000)
\psset{linewidth=0.100000}
\psset{linestyle=solid}
\psset{linestyle=solid}
\setlinecaps{0}
\newrgbcolor{dialinecolor}{0.000000 0.000000 0.000000}
\psset{linecolor=dialinecolor}
\psline(41.000000,18.000000)(39.000000,18.000000)
\setfont{Times-Bold}{1.000000}
\newrgbcolor{dialinecolor}{0.000000 0.000000 0.000000}
\psset{linecolor=dialinecolor}
\rput(102.000000,-7.000000){\scalebox{1 -1}{$(1+\epsilon_{-60})$}}
}\endpspicture
    \caption{\'Etape de reconstruction}
    \label{chap4:fig:reconstruction}
\end{center}\end{figure}

\end{landscape}




\subsection{Test if correct rounding is possible}

We now have to round correctly the result and multiply it by $2^{k}$,
with $k$ an integer. This multiplication is exact, therefore we have :

$$
\begin{array}{rcl}
\exp(x) &=& \circ \left( 2^k . (R11 + R8 + \epsilon_{-73} + \epsilon_{-78}).(1 + \epsilon_{-69}).(1 + \epsilon_{-109}) \right) \\
        &=&  2^k . \circ \left((R11 + R8 + \epsilon_{-73} + \epsilon_{-78}).(1 + \epsilon_{-69}).(1 + \epsilon_{-109}) \right) \\
\end{array}
$$

if the result do not represent a denormalised number. Indeed, if the
final result belongs to denormalised numbers, then the precision of
the results is less than the $53$ bits of a normalized number. Let's
take an exemple. Let $a=1.75$ be a floating-point number exactly
representable in double precision format (we have $\circ(a) = a$). Let
multiply this number by $2^{-1074}$, the exact result is $1.75 \times
2^{-1074}$ which is different of the result rounded to nearest in
double precision $\circ(1.75 \times 2^{-1074})=2^{-1073}$.

Then in the case the result represent a denormalized number, we have to use a special procedure.

\subsection{Rounding to nearest}
\begin{lstlisting}[caption={Test if rounding to nearest is possible}]
static const union{int i[2]; double d;}
#ifdef BIG_ENDIAN
 _two1000    = {0x7E700000, 0x00000000}, /*  1.07150860718626732095e301  */
 _twom1000   = {0x01700000, 0x00000000}, /*  9.33263618503218878990e-302 */
 _errn       = {0x3FF00080, 0x00000000}, /*  1.00012207031250000000e0    */
 _twom75     = {0x3B400000, 0x00000000}; /*  2.64697796016968855958e-23  */
#else
 ...
#endif
#define two1000    _two1000.d
#define twom1000   _twom1000.d
#define errn       _errn.d
#define twom75     _twom75.d

int errd        = 71303168;                    /* 68 * 2^20 */  

double R11_new, R11_err, R13, st2mem;
int    exp_R11;

union {int i[2]; long long int l; double d;} R12;


/* R�sult = (R11 + R8) */
if (R11 == (R11 + R8 * errn)){             (*@ \label{exp:code:44} @*)
  if (k > -1020){                          (*@ \label{exp:code:45} @*)
    if (k < 1020){                        
      HI(R11) += (k<<20);                 
      return R11;
    }else {
      /* we are close to  + Inf */
      HI(R11) += ((k-1000)<<20);           
      return R11*two1000;                
    }                                      (*@ \label{exp:code:46} @*)
  }else {
    /* We consider denormalized number */
    HI(R11_new) = HI(R11)+((k+1000)<<20);  (*@ \label{exp:code:47} @*)
    LO(R11_new) = LO(R11);
    R12.d       = R11_new * twom1000;      (*@ \label{exp:code:48} @*)

    HI(st2mem) = R12.i[HI];         (*@ \label{exp:code:49} @*)
    LO(st2mem) = R12.i[LO];         (*@ \label{exp:code:50} @*)

    R11_err -= st2mem * two1000;           (*@ \label{exp:code:51} @*)
    HI(R13) = HI(R11_err) & 0x7fffffff;    (*@ \label{exp:code:52} @*)
    LO(R13) = LO(R11_err);                 (*@ \label{exp:code:53} @*)

    if (R13 == two_m75){                   (*@ \label{exp:code:54} @*)
      exp_R11 = (HI(R11) & 0x7ff00000) - errd;
      if ((HI(R8) & 0x7ff00000) < exp_R11){(*@ \label{exp:code:55} @*)
        /* Difficult rounding ! */
        sn_exp(x);
      }
      /* The error term is exactly 1/2 ulp */
      if ((HI(R11_err) > 0) && (HI(R8) > 0)) R12.l +=1;   (*@ \label{exp:code:56} @*)
      else
      if ((HI(R11_err) < 0) && (HI(R8) < 0)) R12.l -=1;   (*@ \label{exp:code:57} @*)
    }    
    return R12.d;
  }
}else {
  /* Challenging case */
  sn_exp(x);
}

\end{lstlisting}


\begin{preuve}
\begin{longtable}[c]{@{line }p{0.08\textwidth}p{0.81\textwidth}}
% en cas de cesure c'est ce que l'on place 
% en debut de page
\endhead
% en fin de page
\endfoot 
% en fin de derniere page
\endlastfoot 
\ref{exp:code:44} &
This test is used to know whether we can round the result or not, as shown in~ref{section:testrounding}.

By using property \pref{chap3:exp:evalr11r8} we have :
$$
|(R11 + R8) \times 2^k - \exp(x)| \leq 2^{-68}
$$
where $x\in[A, B]$, that leads to $errn = 1+ 2^{-68} \times 2^{55} = 1 + 2^{-13}$.
This test is true if we are able round correctly the result, else we need to call multiprecision procedure.
\\
\ref{exp:code:45}-\ref{exp:code:46}&
We are able to round, now we need to perform the multiplication $R11 \times 2^k$ exactly.
To keep good performance we do this multiplication by using integer addition on the exponent of $R11$. 
However this operation is valide and exact, only if we do not create a denormalized or infinity number. This is the reason why we perform a test on the value of $k$.

$(R11 + R8) > 2^{-2}$ then $2^{k}.(R11 + R8)$ will not lead to a denormalized number if  $k> -1020$.

$(R11 + R8) < 2^{3}$ then $2^{k}.(R11 + R8)$ will not lead to an overflow if $k< 1020$.
The we have $R11 = R11 \oplus R8$

In the case, we may give an overflow as result, we make the value of $k$ smaller, by subtracting $1000$ to it that will prevent the apparition of exception cases during the multiplication by $k$. This result is exactly multiplies (powerof 2) by the floating-point number $twom1000=2^{-1000}$, and leave to the group composed of the system, compiler and processor overflows handling.
\\
\ref{exp:code:47}&
The result mcan be a denormalized number. We need to use a specific test to check whether if we are able to round properly the result.
$$
R11 = R11 \times 2^{k+1000}
$$
\\
\ref{exp:code:48}&
In rounding to nearest we get :
$$
R12 = R11 \otimes 2^{-1000} = R11 \times 2^{-1000} + \epsilon_{-1075}
$$
The error terms $\epsilon_{-1075}$ come from the possible truncationwhen the result belong to denormalized number.
\\
\ref{exp:code:49},\ref{exp:code:50}&
These lines prevent the compiler to perform ``dangerous'' optimizations, meanwhile prevent to have extra precision by forcing $R12$ to transit through memory. These lines could be removed for exemple by using gcc and the flag  {\tt -ffloat-store} that will have the same effect. However this flag force each floating-point instruction to transit through memory, and have as consequences to severely degrade performance of the resulting program. The solution to keep good performance is to manually force a data to transit through memory, in order to have an IEEE compliant behavior for denormalized number.
\\
\ref{exp:code:51}&
Let
$$
R11\_err = R11 \ominus R12 \otimes 2^{1000}
$$
By the Sterbenz lemma, and by the fact that a multiplication by a power of 2 is exact we have :
$$
R11\_err = R11 - R12 \times 2^{1000}
$$
This operation will put within  $R11\_err$ the error done during the multiplication of $R11$ by $2^{-1000}$ (line \ref{exp:code:48}).
\\
\ref{exp:code:52},\ref{exp:code:53}&
Remove the sign information of  $R11\_err$
$$
R13 = |R11\_err|
$$
\\
\ref{exp:code:54}&
Test if $R13$ is exactly equal to the absolute error (which is also the relative error) done during the rounding process, in rounding to nearest, whitin denormalized number, times $2^{1000}$ :
  $$2^{-1075} \times 2^{1000} = 2^{-75}$$

Now we want to prove that if error term is strictly less than $1/2 ulp (R12)$ then $R12$ correspond to the correct rounding of $R11 + R8$.
 
If $|R11\_err| < \frac{1}{2} ulp (R12)$ then

$|R11\_err| \leq \frac{1}{2} ulp (R12) - ulp(R11)$ and $|R8| \leq \frac{1}{2} ulp (R11)$ therefore :

$$
|R11\_err + R8| \leq \frac{1}{2} ulp (R12) - ulp(R11) + \frac{1}{2} ulp(R11)
$$
then :
$$
|R11\_err + R8| < \frac{1}{2} ulp (R12) 
$$

In that case $R12$ represent the correct rounding of $R11 + R8$ if $|R11\_err| < \frac{1}{2} ulp (R12)$.



However, if $|R11\_err| = \frac{1}{2} ulp (R12) = 2^{-75}$, during the multiplication $R11 \times 2^{-1000}$,  the result is rounded to odd/even due to the presence of an ambiguous value. It mean that $R12$ may not represent the rounding to nearest result of $R11+R8$, we need to perform a correction :


\begin{itemize}
\item 
If $R8 > 0$, and $R11\_err = - \frac{1}{2} ulp (R12)$, then the round to odd/even was done on the correct side.
\item 
If $R8 > 0$, and $R11\_err = \frac{1}{2} ulp (R12)$, then the round to odd/even wasn't done on the correct side. We need to perform a correction by adding $1 ulp$ to $R12$ 
(figure \ref{chap4:fig:rnd_to_nearest}, case a)
\item 
If $R8 < 0$, and $R11\_err = - \frac{1}{2} ulp (R12)$, then the round to odd/even wasn't done on the correct side. We need to perform a correction by subtracting $1 ulp$ to $R12$  
(figure \ref{chap4:fig:rnd_to_nearest}, case b).


\item 
If $R8 < 0$, and $R11\_err = \frac{1}{2} ulp (R12)$, then the round to odd/even was done on the correct side.
\end{itemize}

\\
\ref{exp:code:55}&
When we are in presence of a consecutive sequence of  $0$ or $1$ straddling $R11$ and  $R8$, then the test done at line \ref{exp:code:44} will not detect a possible problem. This problem will arise only will denormalized number, when $R11\_err$ is close to $\frac{1}{2} ulp (R12)$. 

Then we have to detect if in that case (denormalized, $|R11\_err| = \frac{1}{2} ulp(R12)$) we have enought precision to correctly round the result. We use a test similar to the one used to test wheteher we can round with rounding toward $\pm \infty$. Indeed, problematic cases arise when  $ \frac{1}{2} ulp(R12) - 2^{-68}.R11 \leq |R11\_err + R8| \leq \frac{1}{2} ulp(R12) + 2^{-68}.R11 ) $.
\\
\ref{exp:code:56},\ref{exp:code:57}&
Test is we are in presence of one of the previous cases describe previously, and correct the result by adding or subtracting $1 ulp$ by using the ``continuity'' of the representation of floating-point number.
\end{longtable}
\end{preuve}

%\begin{figure}[!htb]  
%  \begin{center}
%    \includegraphics[width=0.5\textwidth]{rnd_to_nearest_pp.eps}
%   \end{center} 
%\end{figure}

%\begin{figure}[!htb]  
%  \begin{center}
%    \includegraphics[width=0.5\textwidth]{rnd_to_nearest_mm.eps}
%  \end{center} 
%\end{figure}

\begin{figure}[ht] \begin{center}
\input{fig_exp/rnd_to_nearest.pstex_t}
\caption{Description of problem with rounding to nearest of a denormalized number.
  \label{chap4:fig:rnd_to_nearest}}
\end{center}\end{figure}


 
\subsection{Rounding toward $+ \infty$}
\begin{lstlisting}[caption={Test if rounding toward $+ \infty$ is possible}]
static const union{int i[2]; double d;}
#ifdef BIG_ENDIAN
 _two1000    = {0x7E700000, 0x00000000}, /*  1.07150860718626732095e301  */
 _twom1000   = {0x01700000, 0x00000000}, /*  9.33263618503218878990e-302 */
#else
 ...
#endif
#define two1000    _two1000.d
#define twom1000   _twom1000.d


int errd        = 71303168;                    /* 68 * 2^20 */  

int    exp_R11;
union {int i[2]; long long int l; double d;} R12;

/* Result = (R11 + R8) */

exp_R11 = (HI(R11) & 0x7ff00000) - errd;

if ((HI(R8) & 0x7ff00000) > exp_R11){      (*@ \label{exp:code:pinf:0} @*)
  /* We are able to round the result */
  if (k > -1020){                          
    if (k < 1020){                         
      HI(R11) += (k<<20);                  
    }else {
      /* We are close to + Inf */
      HI(R11) += ((k-1000)<<20);           
      R11 *= two1000;
    }
    if (HI(R8) > 0){                       (*@ \label{exp:code:pinf:1} @*)
      R12.d  = R11;
      R12.l += 1;
      R11    = R12.d;
    }
    return R11;
  }else {
    /* We are with denormalized number */
    HI(R11) += ((k+1000)<<20);          
    R12.d    = R11 * twom1000;              (*@ \label{exp:code:pinf:2} @*)

    HI(st2mem) = R12.i[HI] ;     
    LO(st2mem) = R12.i[LO]       

    R11 -= st2mem * two1000;                (*@ \label{exp:code:pinf:3} @*)
    if ((HI(R11) > 0)||((HI(R11) == 0)&&(HI(R8) > 0))) R12.l += 1;(*@ \label{exp:code:pinf:5} @*)

    return R12.d;
  }
}else {
  /* Difficult case */
  su_exp(x);
}
\end{lstlisting}


\begin{preuve}

The program used to check whether correct rounding toward $+ \infty$ is possible is similar to the one used with rounding to nearest.

\begin{longtable}[c]{@{line }p{0.08\textwidth}p{0.81\textwidth}}
% en cas de cesure c'est ce que l'on place 
% en debut de page
\endhead
% en fin de page
\endfoot 
% en fin de derniere page
\endlastfoot 
\ref{exp:code:pinf:0} & 
This test is valid even if the final result is a denormalized number.
\\

\ref{exp:code:pinf:1} &
We add $1 ulp$ to the result if $R8$ is positive. 
\\

\ref{exp:code:pinf:3} &
Like for rounding to nearest, the quantity $R11$ represent the rounding error that come from the operation $R11* twom1000$ in line \ref{exp:code:pinf:2}.
\\

\ref{exp:code:pinf:5} &
This test check whether the error from line \ref{exp:code:pinf:2} is strictly positive or if it is equal to zero and if $R8$ is strictly positive. If we are in one of these two cases, by definition of rounding toward $+ \infty$, we need to add $1 ulp$ to the result.
\end{longtable}
\end{preuve}


 
\subsection{Rounding toward $- \infty$}

\begin{lstlisting}[caption={Test if rounding toward $- \infty$ is possible},firstnumber = 144]
static const union{int i[2]; double d;}
#ifdef BIG_ENDIAN
 _two1000    = {0x7E700000, 0x00000000}, /*  1.07150860718626732095e301  */
 _twom1000   = {0x01700000, 0x00000000}, /*  9.33263618503218878990e-302 */
#else
 ...
#endif
#define two1000    _two1000.d
#define twom1000   _twom1000.d

int errd        = 71303168;                    /* 68 * 2^20 */  

int    exp_R11;
union {int i[2]; long long int l; double d} R12;
 
/* R�sult = (R11 + R8) */

exp_R11 = (HI(R11) & 0x7ff00000) - errd;

if ((HI(R8) & 0x7ff00000) > exp_R11){
  /* We are able to round the result */
  if (k > -1020){                          
    if (k < 1020){                         
      HI(R11) += (k<<20);                  
    }else {
      /* We are close to + Inf */
      HI(R11) += ((k-1000)<<20);           
      R11 *= two1000;
    }
    if (HI(R8) > 0){
      R12.d  = R11;
      R12.l += 1;
      R11    = R12.d;
    }
    return R11;
  }else {
    /* We are with denormalized number */
    HI(R11) += ((k+1000)<<20);             
    R12.d    = R11 * twom1000;             

    HI(st2mem) = R12.i[HI];                  
    LO(st2mem) = R12.i[LO];                  

    R11 -= st2mem * two1000;               
    if ((HI(R11) < 0)||((HI(R11) == 0)&&(HI(R8) < 0))) R12.l -= 1;

    return R12.d;
  }
}else {
  /* Difficult case */
  su_exp(x);
}
\end{lstlisting}



\begin{preuve}

The program used to check whether correct rounding toward $- \infty$ is possible is similar to the previous one and it work for similar reasons.
\end{preuve}


 
\subsection{Rounding toward $0$} 
The program used to check whether correct rounding toward $0$ is possible is identical to the one used for rounding to $- \infty$ because $\exp(x)$ is a positive function.

%%
%% 
%%
\section{Accurate phase}
\label{section:accurate_phase}
When the previous computation failed, it mean that the rounding of the result is difficult to get. We need to use more accurate methods :
\begin{itemize}
\item
 $sn\_exp$ with rounding to nearest,
\item
 $su\_exp$ with rounding toward $+\infty$,
\item
 $sd\_exp$ with rounding toward $-\infty$,
\end{itemize}

These methods are based on SCS library \cite{SCSweb}, with $30$ bits of precision per digits and $8$ digits per vector. The guarantee precision with this format is $211$ bits at least. Even if there is no proof for these operators yet, the proof for correct rounding of the exponential only request the following property  that are easy to check and/or satisfy :

\begin{propriete}(Addition)
Let $a \boxplus b$ represent the multiprecision operation performing an addition between $a$ and $b$ with at least $210$ bits of precision for the result. Like for double precision floating point number, the scs addition may leads to a cancellation. We have :
$$
a+ b = (a \boxplus b).(1+\epsilon_{-211})
$$
\end{propriete}

\begin{propriete}(Multiplication)
Let $a \boxtimes b$ represent the multiprecision operation performing a multiplication between $a$ and $b$ with at least $210$ bits of precision for the result. This operation do not produce a cancellation.
$$
a\times b = (a \boxtimes b).(1+\epsilon_{-211})
$$
\end{propriete}



\subsection{Overview of the algorithm}
Here is the algorithm used for the second part of the evaluation :
\begin{enumerate}
\item
{\bf No special case handeling} \\
For this part, we assume that test for special case handeling have been done.
\item
{\bf Range reduction} \\
We compute the reduced argument $r$ and the integer $k$ such that :
$$
r = \frac{x - k . \ln(2)}{512}
$$

with
$\frac{-\ln(2)}{1024} \leq r \leq \frac{-\ln(2)}{1024}$

such that
$$
\exp(x) = \exp(r)^{512} \times 2^{k}
$$


\item
{\bf Polynomial evaluation} \\
We compute the polynom $P(r)$ of degree 10 :
$$
\exp(r) = (1 + r + P(r)).(1+ \epsilon_{-164})
$$

\item
{\bf Powering the result} \\
$\exp(r)^{512} = \left(\left(\left(\left(\left(\left(\left(\left(\exp(r)^2\right)^2\right)^2\right)^2\right)^2\right)^2\right)^2\right)^2\right)^2$



\item
{\bf Reconstruction} \\
$$
\exp(x) = \exp(r)^{512}. 2^{k} . (1+\epsilon)
$$
avec $|\epsilon| \leq 2^{-163} $
\end{enumerate}

We have choosen this evaluation scheme, because the reconstruction step use the squaring multiprecision operator. This operator facilitate the error computation and is very economic : its cost is $0.7$ times the one of a true multiprecision multiplication.

We will notice that there exist an alternative to the squaring solution. We can tabulate values $2^{\frac{N}{512}}$ for $N=1,2,\ldots,511$ and use the formulae $\exp(x) = \exp(r) \times 2^{N} \times 2^{\frac{M}{512}}$ with $k=M+N/512$. However we prefer the squaring method that do not request the storage of SCS number and the associated quantity of memory.



\subsection{Function's call}
With Lef\`evre worst cases on table makers dilemma we get the following theorem :

\begin{theorem}(Correct rounding for the exponential)
Let $y$ be the exact value of the exponential of a floating-point number in double precision $x$. Let $y^*$ be an approximation of $y$ such that the distance between $y$ and $y^*$ be bounded by $\epsilon$. Then if $\epsilon \leq 2^{-157}$, for each of the four rounding mode, rounding $y^*$ is equivalent to rounding $y$;
\end{theorem}

To round the multiprecision result in SCS format depending on the rounding mode, we use the following procedure (\texttt{scs\_get\_d}, \texttt{scs\_get\_d\_pinf}, \texttt{scs\_get\_d\_minf}). 

\subsubsection{Rounding to nearest}

\begin{lstlisting}[caption={Compute the rounding to nearest of the exponential in multiprecision},firstnumber=1]
double sn_exp(double x){ 
  scs_t res_scs;
  scs_db_number res;

  exp_SC(res_scs, x);
  scs_get_d(&res.d, res_scs);  res.d = x;

  return res.d;
}
\end{lstlisting}



\subsubsection{Rounding toward $+ \infty$}

\begin{lstlisting}[caption={Compute the rounding toward $+ \infty$ of the exponential in multiprecision},firstnumber=1]
double su_exp(double x){ 
  scs_t res_scs;
  scs_db_number res;
  
  exp_SC(res_scs, x);
  scs_get_d_pinf(&res.d, res_scs);
  return res.d;
}
\end{lstlisting}



\subsubsection{Rounding toward $- \infty$}

\begin{lstlisting}[caption={Compute the rounding toward $- \infty$ of the exponential in multiprecision},firstnumber=1]
double sd_exp(double x){ 
  scs_t res_scs;
  scs_db_number res;
  
  exp_SC(res_scs, x);
  scs_get_d_minf(&res.d, res_scs);
  return res.d;
}
\end{lstlisting}



\subsection{Software}
The function $exp\_SC$ approximate the exponential of $x$ with $163$ bits of precision and put the result in $res\_scs$.

\begin{lstlisting}[caption={Compute the exponential in multiprecision},firstnumber=1]
void exp_SC(scs_ptr res_scs, double x){
  scs_t sc1, red;
  scs_db_number db;
  int i, k;


  /* db.d = x/512  (= 2^9)  */
  
  db.d = x;                                    (*@ \label{exp:mp:0} @*)
  db.i[HI] -= (9 << 20);                (*@ \label{exp:mp:1} @*) 
  scs_set_d(sc1, db.d);                        (*@ \label{exp:mp:2} @*)
  
  
  DOUBLE2INT(k, (db.d * iln2_o512.d));         (*@ \label{exp:mp:3} @*) 
 
  /* 1) Range reduction */
  
  scs_set(red,     sc_ln2_o512_ptr_1);;        (*@ \label{exp:mp:4} @*)             
  scs_set(red_low, sc_ln2_o512_ptr_2);         (*@ \label{exp:mp:4b} @*)      
  if (k>0){
    scs_mul_ui(red,      (unsigned int) k);
    scs_mul_ui(red_low,  (unsigned int) k);
  }else {
    scs_mul_ui(red,      (unsigned int)(-k));
    scs_mul_ui(red_low,  (unsigned int)(-k));
    red->sign *= -1;
    red_low->sign *=-1;
  }                                            (*@ \label{exp:mp:5} @*)

  scs_sub(red, sc1, red);                      (*@ \label{exp:mp:6} @*)
  scs_sub(red, red, red_low);                  (*@ \label{exp:mp:6b} @*)

  
  /* 2) Polynomial evaluation */               (*@ \label{exp:mp:7} @*)
   
  scs_mul(res_scs, constant_poly_ptr[0], red);
  for(i=1; i< 10; i++){                       
    scs_add(res_scs, constant_poly_ptr[i], res_scs);
    scs_mul(res_scs, red, res_scs);            
  }
  
  scs_add(res_scs, SCS_ONE, res_scs);          
  scs_mul(res_scs, red, res_scs);
  scs_add(res_scs, SCS_ONE, res_scs);          (*@ \label{exp:mp:8} @*)  

  /* 3) Powering the result exp(r)^512  */
  
  for(i=0; i<9; i++){                          (*@ \label{exp:mp:9} @*)
    scs_square(res_scs, res_scs);
  }

  /* 4) Multiplication by 2^k */
  
  res_scs->index += (int)(k/30);               (*@ \label{exp:mp:10} @*)
  if ((k%30) > 0) 
    scs_mul_ui(res_scs, (unsigned int) (1<<((k%30))));
  else if ((k%30) < 0){
    res_scs->index --;
    scs_mul_ui(res_scs, (unsigned int) (1<<((30+(k%30)))));
  }                                            (*@ \label{exp:mp:11} @*)

}

\end{lstlisting}



\begin{preuve}
\begin{longtable}[c]{@{line }p{0.08\textwidth}p{0.81\textwidth}}
% en cas de cesure c'est ce que l'on place 
% en debut de page
\endhead
% en fin de page
\endfoot 
% en fin de derniere page
\endlastfoot 
\ref{exp:mp:0}& $db.d = x$
\\
\ref{exp:mp:1}& 
This operation divide  $db.d$ by $512=2^9$ and is valid under the condition that $db.d$, and consequently $x$, do not represent a special values (denormalized, infinity, NaN). Condition that is satisfied because special case have been treated during the quick phase.

\\
\ref{exp:mp:2}& 
$sc1$ is a $211$ bits multiprecision number such that :
\begin{prop}
  \label{chap3:exp:mp:prop0}
 $sc1=db.d = \frac{x}{512}$ exactly
\end{prop}
\\
\ref{exp:mp:3}& $iln2\_o512.d$ is a double precision floating-point number such that :
$iln2\_o512.d = \frac{512}{\ln{2}}(1 + \epsilon_{-54})$. This line put in $k$ the closest integer of $db.d \otimes \frac{512}{\ln{2}}$. We use the property of $DOUBLE2INT$ which convert a floating-point number in an integer with rounding to nearest.

Moreover $k$ satisfy the following property :
\begin{prop}
  \label{chap3:exp:mp:prop1}
  $$
\lfloor \frac{x}{\ln 2} \rfloor  \leq k \leq   \lceil \frac{x}{\ln 2} \rceil
   ~~~ \mbox{et} ~~~   -1075 \leq |k| \leq 1025
$$
\end{prop}
And $k$ is a $11$ bits integer.
\\
\ref{exp:mp:4}, \ref{exp:mp:4b}&
By construction we have :
$$red + red\_low = \frac{\ln 2}{512}(1 + \epsilon_{-450})$$
and $red$ is construct in order to make the multiplication of $red$ by $k$ exact if $|k| \leq 2^{11}$.

\\
\ref{exp:mp:5}&
At the end of the test on $k$ we have : 
$red + red\_low = k \otimes \frac{\ln 2}{512}(1 + \epsilon_{-450})$

with $|k| < 2^{11}$, then :
\begin{prop}
  \label{chap3:exp:mp:prop2}
$$red  + red\_low = k \times \frac{\ln 2}{512}(1 + \epsilon_{-411})$$
\end{prop}
\\
\ref{exp:mp:6},\ref{exp:mp:6b}&
By the properties \pref{chap3:exp:mp:prop0} and \pref{chap3:exp:mp:prop2} we have :
\begin{prop}
  \label{chap3:exp:mp:prop3}
$red = \frac{x}{512} \ominus \left( k \times \frac{\ln 2}{512} \right) (1 + \epsilon_{-411})$
\end{prop}

In addition we have seen in the quick phase that at most $58$ bits could be cancelled, during this subtraction.
\begin{prop}
  \label{chap3:exp:mp:prop4}
$|red| \leq \frac{\ln 2}{1024} \leq 2^{-10}$,

$red = \frac{x}{512} - k \times \frac{\ln 2}{512}  + \epsilon_{-211} $
\end{prop}
\\
\ref{exp:mp:7}-\ref{exp:mp:8}&
We now perform the polynomial evaluation where the coefficient have the following properties.
\begin{prop}
  \label{chap3:exp:mp:prop5}
$|constant\_poly\_ptr[0]=c_0| \leq 2^{-25}$, $|constant\_poly\_ptr[1]=c_1| \leq 2^{-21}$,
$|constant\_poly\_ptr[2]=c_2| \leq 2^{-18}$, $|constant\_poly\_ptr[3]=c_3| \leq 2^{-15}$,
$|constant\_poly\_ptr[4]=c_4| \leq 2^{-12}$, $|constant\_poly\_ptr[5]=c_5| \leq 2^{-9}$,
$|constant\_poly\_ptr[6]=c_6| \leq 2^{-6}$,  $|constant\_poly\_ptr[7]=c_7| \leq 2^{-4}$,
$|constant\_poly\_ptr[8]=c_8| \leq 2^{-2}$,  $|constant\_poly\_ptr[9]=c_9| \leq 2^{-1}$
\end{prop}


We have :
\begin{itemize}
\item $P_0=c_1 \boxplus (red \boxtimes c_0)$   thus $|P_0| \leq 2^{-20}$ et $P_0 = c_1 + (red \times c_0 ) + \epsilon_{-231}$
\item $P_1=c_2 \boxplus (red \boxtimes P_0)$   thus $|P_1| \leq 2^{-17}$ et $P_1 = c_2 + (red \times P_0 ) + \epsilon_{-228}$
\item $P_2=c_3 \boxplus (red \boxtimes P_1)$   thus $|P_2| \leq 2^{-14}$ et $P_2 = c_3 + (red \times P_1 ) + \epsilon_{-225}$
\item $P_3=c_4 \boxplus (red \boxtimes P_2)$   thus $|P_3| \leq 2^{-11}$ et $P_3 = c_4 + (red \times P_2 ) + \epsilon_{-222}$
\item $P_4=c_5 \boxplus (red \boxtimes P_3)$   thus $|P_4| \leq 2^{-8}$ et $P_4 = c_5 + (red \times P_3 ) + \epsilon_{-219}$
\item $P_5=c_6 \boxplus (red \boxtimes P_4)$   thus $|P_5| \leq 2^{-5}$ et $P_5 = c_6 + (red \times P_4 ) + \epsilon_{-216}$
\item $P_6=c_7 \boxplus (red \boxtimes P_5)$   thus $|P_6| \leq 2^{-3}$ et $P_6 = c_7 + (red \times P_5 ) + \epsilon_{-214}$
\item $P_7=c_8 \boxplus (red \boxtimes P_6)$   thus $|P_7| \leq 2^{-1}$ et $P_7 = c_8 + (red \times P_6 ) + \epsilon_{-212}$
\item $P_8=c_9 \boxplus (red \boxtimes P_7)$   thus $|P_8| \leq 1 + 2^{-10} $ et $P_8 = c_9 + (red \times P_7 ) + \epsilon_{-211}$
\item $P_9=1   \boxplus (red \boxtimes P_8)$   thus $|P_9| \leq 1 + 2^{-9}$ et $P_9 = 1   + (red \times P_8 ) + \epsilon_{-210}$
\item $P_{10}=1   \boxplus (red \boxtimes P_9)$ thus $|P_{10}| \leq 1 + 2^{-8}$ et $P_{10} = 1   + (red \times P_9) + \epsilon_{-210}$
\end{itemize}

Therefore :
$$
res\_scs = P_{10} + \epsilon_{-208}
$$ 

We build the polynom such that  
$$\exp(r) = (1 + r + c_9. r^2 + \cdots + c_0 . r^{11}).(1 + \epsilon_{-164})$$
Therefore

$|res\_scs| \leq 1 + 2^{-8}$
with

$\exp(red) = (res\_scs  + \epsilon_{-208}).(1 + \epsilon_{-164})$
\\

\ref{exp:mp:9}&
We perform a squaring of the result $9$ times, that correspond to powering the result to the power $512$. 
At each iteration we perform a rounding error equal to $\epsilon_{-211}$.

Finally :

$|res\_scs| \leq  2^{3}$
and

$\exp(x) = 2^k. (res\_scs  + \epsilon_{-199}).(1 + \epsilon_{-164})$
\\
\ref{exp:mp:10}-\ref{exp:mp:11}&
With these lines we perform the multiplication of $res\_scs$ by $2^k$. This multiplication is done by a shift on the index of $k/30$, where $30$ correspond to the number of bits used within a multiprecision number. This shift is exact. Then a multiplication of $res\_scs$ by $2$ to the power the rest of the euclidian division of $k$ by $30$ is done. At the end of these instructions we have :
$$\exp(x) = (res\_scs).(1 + \epsilon_{-163})$$

\\

\end{longtable}
\end{preuve}




%%%%%%%%%%%%%%%%%%%%%%%%%%%%%%%%%%%%%%%%%%%%%%%%%%%%%%%%%%%%%
\section{Analysis of the exponential}
\label{section:exp_results}

\subsection{Test conditions}

Table \ref{tbl:systems} lists the combinations of processor, OS and
default \texttt{libm} used for our tests.
\begin{table}[!htb]
\begin{center}
\renewcommand{\arraystretch}{1.2}
\begin{tabular}{|c||c|c|c|}
\hline
Processor        & OS & compiler & default \texttt{libm} \\
\hline
\hline
Pentium III      & Debian GNU/Linux & gcc-2.95          & \texttt{glibc}, derived from \texttt{fdlibm} \\ 
\hline
UltraSPARC IIi   & SunOS 5.8        & gcc-2.95          & \texttt{fdlibm} \\
\hline
Xeon (Pentium 4) & Debian GNU/Linux & gcc-2.95          & \texttt{glibc}, derived from \texttt{fdlibm}\\
\hline
PowerPC G4       & MacOS 10.2       & gcc-2.95          & Apple specific \\
\hline
Itanium          & Debian GNU/Linux & gcc-2.95, gcc-3.2 & Intel optimized \\
\hline
\end{tabular}
\end{center}
\caption{The systems tested
  \label{tbl:systems}}
\end{table}

The following presents tests
performed under such conditions as to suppress most of the impact of
the memory hierarchy: A small loops performs 10 identical calls to
the function, and the minimum timing is reported, ensuring that both
code and data have been loaded in the cache, and that interruptions by
the operating system do not alter the timings.

These timings are taken on random values between $-745$ and $+744$,
which is the practical range for the exponential. We also report the
timing for the worse case for the correct rounding in rounding to
nearest mode of the exponential, which is
$x=7.5417527749959590085206221e-10$.

Our libray was tuned to take into account the adequation of the evaluation scheme to the memory hierarchies of current processors (our program for the exponential evaluation uses $2.8$Kbytes of table for the four rounding modes, whereas the one from \emph{fdlibm} use $13$Kbytes). However we do not have tested the impact over performance of this concern and is part of our futur works. 

\subsection{Results}

Tables \ref{tbl:exp_abstime} gives a summary of the timings of the
various libraries. The \accurate\ phase is called about 2 times out of 100000.


\begin{table}[!htb]
\begin{center}
\renewcommand{\arraystretch}{1.2}
\begin{tabular}{|l|r|r|r|}
\hline
 \multicolumn{4}{|c|}{Pentium 4 Xeon / Linux Debian sarge / gcc 3.3}   \\ 
 \hline
 \hline
 \texttt{libm}           & 236          & 5528          &        365 \\ 
 \hline
  \texttt{mpfr}          & 14636        & 204736        &      23299 \\ 
 \hline
  \texttt{libultim}      & 44           & 3105632       &        210 \\ 
 \hline
 \texttt{crlibm}         & 316          & 41484         &        432 \\ 
 \hline
 \hline
  \multicolumn{4}{|c|}{PowerPC G4 / MacOS X / gcc2.95}   \\
 \hline
                         & min time      & max time      & avg time \\
 \hline
 \texttt{libm}           & 7            & 14            &         12 \\
 \hline
  \texttt{mpfr}          & 972          & 2819          &       1367 \\
 \hline
  \texttt{libultim}      & 8            & 169390        &         12 \\
 \hline
 \texttt{crlibm}         & 4            & 916           &         15 \\
 \hline
\end{tabular}
\end{center}
\caption{Absolute timings for the exponential (arbitrary units)
  \label{tbl:exp_abstime}}
\end{table}





\subsection{Analysis}

\subsubsection*{Processor-specific libraries}
Documentation\cite{HarKubStoTan99} from Intel labs claim to provide an
exponential in only $48$ cycles. This performance is possible through
the wide use of non portable tricks such as inverse approximation,
fused multiply and add and double extended precision.  However, our
tests show that the environmental cost (mainly the cost of a function
call) is about $80$ clock cycles! Our tests have also shown that there
exists a slower path that takes up to $2767$ clock cycles, which is $14$ times slower. This path
seems to be taken very often since the average cost is $1.5$ times
more expensive than the smallest  execution time.

The same conclusion can be done for the mathematical library used on
Ultra-SPARC IIi system, where there exists a path $13$ times slower
than a normal execution.

These two observations show that our two-step procedure, with a much
slower second step, could be viable in the commercial world.

The mathematical library used on the PowerPC G4 with gcc is the one
from Apple. This library do not provide correct rounding and is $1.1$
slower than the version provided with \emph{crlibm}. It is, however,
the most accurate of the tested libraries.


\subsubsection*{The cost of correct rounding}

The \emph{libutlim} library provides correct rounding for an average
cost between $0.9$ (on a Pentium III) and $1.8$ (on an Itanium) times
the cost of the standard library. Our exponential give a result for an
average cost between $0.91$ (on Power-PC) and $2.66$ (on Ultra-SPARC
IIi) times the cost of the standard library, which is reasonable. On
the other hand, MPFR provide correct rounding for an average cost
between $13.5$ (on Power-PC) and $153$ (on Itanium) compared to the
\emph{libm}. 

The main advantage of \emph{crlibm} over \emph{libutlim} is the upper
bound on the execution time. On our tests, this bound for
\emph{crlibm} is $147$ times the average \emph{libm} cost, whereas for
\emph{libtultim} this bound goes up to $14499$ times the average
\emph{libm} cost. Our two steps strategy fully benefits from knowing
bounds on correct rounding worst cases.

We notice that our second step is in average $3$ times faster than the
multiprecision library MPFR. It shows that our multiprecision
operators from \emph{scslib}, hand tuned for $200$ bits of precision
perfectly fulfill the performance requirement of the second step.


\subsubsection*{Relations between the two steps of \emph{crlibm}}

Our second phase is $30$ times slower than the first step and is
called only once over $2^{13}$. The cost of the second step over the
average cost is :

$$
\frac{1 \times (2^{13}-1) + 30 \times 1}{2^{13}} = 1.003540039
$$
which corresponds to a $0.35\%$ overhead. This small overhead in
average  means that a possible performance improvement
is to reduce the precision of the first step, and by the same way the
number of instructions, to increase to number of time that the second
step is called. It will also make the proof simpler.


%%%%%%%%%%%%%%%%%%%%%%%%%%%%%%%%%%%%%%%%%%%%%%%%%%%%%%%%%%%%%
\section{Conclusion and perspectives}


 It is obvious from our performance measurements that our first step is
too accurate and too slow for a balanced average time. We will take
this experience into consideration when writing first steps for other
functions. The IBM library seems to get a better balance although its
second and later steps are much slower. Its code, unfortunately, is
little documented and difficult to prove.

Writing the proofs is a very time-consuming task, which could be
partially automated for one step which is common to most function: The
accumulation of error terms in order to compute the final error.




 \chapter{The trigonometric functions \label{chap:trigo}}
  This chapter is contributed by C. Daramy-Loirat
and F. de~Dinechin.  

\section*{Introduction}
This chapter describes the implementations of sine, cosine and
tangent, as they share much of their code.


%%%%%%%%%%%%%%%%%%%%%%%%

\section{Argument reduction}
We only need to compute sine and cosine over the interval
$[0;\frac{\Pi}{2}]$ thanks to their periodicity and symmetry properties.

The argument reduction is obtained by an additive Cody \& Waite
algorithm, according to the following scheme:

$x = y + k*\frac{\Pi}{256}$, such that $ |y| \leq \frac{\Pi}{512}$ and
$k$ is an integer in $[0,255]$.


For every $k$ from 0 to 63, $\sin (k*\frac{\Pi}{256})$ and $\cos
(k*\frac{\Pi}{256})$ are tabulated in double-double precision. This is
enough to have all possible values of sine and cosine for every $k$ in
$[0;255]$, by using the following properties:
        
\begin{equation}        \sin(x) = \cos(x - \frac{\Pi}{2})\end{equation}

\begin{equation}        \sin(x) = -\sin(-x)\end{equation}

\begin{equation}        \cos(x) = \cos(-x)\end{equation}

Finally, we compute a table method by the way:

\begin{equation}        
  \sin(x) = \sin(k*\frac{\Pi}{256} + y) =  \cos(k*\frac{\Pi}{256})* \sin(y) +  \sin(k*\frac{\Pi}{256}) *\cos(y) 
\end{equation}

\begin{equation}
        \cos(x) = \cos(k*\frac{\Pi}{256} + y) = \cos(k*\frac{\Pi}{256})* \cos(y) -  \sin(k*\frac{\Pi}{256}) *\sin(y) \end{equation}\\\begin{equation} tan(x) = \frac{\sin(x)}{\cos(x)} = \frac{\cos(k*\frac{\Pi}{256})* \sin(y) +  \sin(k*\frac{\Pi}{256}) *\cos(y)}{\cos(k*\frac{\Pi}{256})* \cos(y) -  \sin(k*\frac{\Pi}{256}) *\sin(y)}
\end{equation}
$y$ is small enough so we compute $\sin(y)$ and $\cos(y)$ through a
polynomial evaluation. (See file trigo\_fast.c, code $do\_cos$ and
$do\_sin$).

\section{polynomial evaluation}
Assume $\epsilon \leq 1$ and $y < \frac{\Pi}{512}$ \\
$\cos(y)$ and $\sin(y)$ are computed in double precision.

\subsection{Sine}
If $y \leq 2^{-26}$, then $\sin(y)\sim y + \epsilon*2^{-54.5}$ 
        
If $y > 2^{-26}$, then $\sin(y) \sim y *(1+ ts) + \epsilon *
2^{-63.8}$, where $ts = y^2*(s3 + y^2*(s5 + y^2*s7)))$ with $s3$, $s5$ and
$s7$ the Taylor coefficients.

\subsection{Cosine}

If $y \leq 2^{-27}$, then $\cos(y)\sim y + \epsilon*2^{-55.0}$

If $y > 2^{-27}$, then $\cos(y)\sim 1+ tc + \epsilon * 2^{-71.8}$,
where $tc = y^2*(c2 + y^2*(c4 + y^2*c6))$ with $c2$, $c4$ and $c6$ the
Taylor coefficients.
        


\section{Reconstruction}
Assume $a = k*\frac{\Pi}{256}$, $sa = \sin(a)$, $ca = \cos(a)$ and
$"_h"$ and $"_l" $ denote "hi" and 'low" parts of a double-double
floating point number.

\subsection{Sine}
According to equation (4), we have to compute: 
 \begin{eqnarray*}
  \sin(a+y) &=& \sin(a) \cos(y)  + \cos(a)\sin(y)  \\
  & \approx& (sa_h+sa_l)(1+t_c) + (ca_h+ca_l)(y_h+y_l)(1+t_s)
\end{eqnarray*}


As shown on the following graph, we don't need to take into account the smallest argument $ca_l * y_l * (1+ts)$\\
And we compute a $2^{-64}$ double-double precision result with only
one Mul12 and two Add12.
\begin{center}
 \small
 \setlength{\unitlength}{3ex}
      \framebox{
        \begin{picture}(22,9)(-3,-4.2)
        \put(10,4){\line(0,-1){8}}  \put(9.5,4){$\epsilon$}
  
          \put(4,3.2){$sa_h$} \put(0.05,3){\framebox(7.9,0.7){}}
          \put(12,3.2){$sa_l$}  \put(8.05,3){\framebox(7.9,0.7){}}

          \put(6,2.2){$sa_ht_c$} \put(2.05,2){\framebox(7.9,0.7){}}
          \put(14,2.2){$sa_lt_c$}  \put(10.05,2){\framebox(7.9,0.7){}}

          \put(5,1.2){$ca_hy_h$} \put(1.05,1){\framebox(7.9,0.7){}}
          \put(13,1.2){$ca_hy_h$}  \put(9.05,1){\framebox(7.9,0.7){}}

          \put(12,0.2){$ca_hy_l $}  \put(8.05,0){\framebox(7.9,0.7){}}
          \put(12,-0.8){$ca_ly_h $}  \put(8.05,-1){\framebox(7.9,0.7){}}

         \put(7,-1.8){$ca_hy_ht_s$} \put(3.05,-2){\framebox(7.9,0.7){}}
         \put(14,-2.8){$ca_hy_lt_s $}  \put(10.05,-3){\framebox(7.9,0.7){}}
         \put(14,-3.8){$ca_ly_ht_s $}  \put(10.05,-4){\framebox(7.9,0.7){}}
 
        \end{picture}
      }
  \end{center}

\paragraph{Cancellation}


There may be cancellation when $k = 0$, that is to say when $\sin(a) =
0$, then the computation is simplified automatically. (See in
trigo\_fast.c, code do\_cos and do\_sin).  We just need to compute
$\cos(a) * \sin(y)$, wich is reached with the polynomial precision. 

 

\subsection{Cosine}
According to equation (5), we have to compute in double-double precision:
 \begin{eqnarray*}
  \cos(a+y) &=& \cos(a) \cos(y)  - \sin(a)\sin(y)  \\
  & \approx& (ca_h+ca_l)(1+t_c) - (sa_h+sa_l)(y_h+y_l)(1+t_s)
\end{eqnarray*}

As shown on the following graph, we don't need to take into account
the smallest argument $sa_l*y_l*t_s$.  We compute a $2^{-63}$
precision result with only one Mul12 and two Add12.
\begin{center}
 \small
 \setlength{\unitlength}{3ex}
      \framebox{
        \begin{picture}(22,9)(-3,-4.2)
        \put(10,4){\line(0,-1){8}}  \put(9.5,4){$\epsilon$}
  
          \put(4,3.2){$ca_h$} \put(0.05,3){\framebox(7.9,0.7){}}
          \put(12,3.2){$ca_l$}  \put(8.05,3){\framebox(7.9,0.7){}}

          \put(6,2.2){$ca_ht_c$} \put(2.05,2){\framebox(7.9,0.7){}}
          \put(14,2.2){$ca_lt_c$}  \put(10.05,2){\framebox(7.9,0.7){}}

          \put(5,1.2){$-sa_hy_h$} \put(1.05,1){\framebox(7.9,0.7){}}
          \put(13,1.2){$-sa_hy_h$}  \put(9.05,1){\framebox(7.9,0.7){}}

          \put(12,0.2){$-sa_hy_l $}  \put(8.05,0){\framebox(7.9,0.7){}}
          \put(12,-0.8){$-sa_ly_h $}  \put(8.05,-1){\framebox(7.9,0.7){}}

         \put(7,-1.8){$-sa_hy_ht_s$} \put(3.05,-2){\framebox(7.9,0.7){}}
         \put(14,-2.8){$-sa_hy_lt_s $}  \put(10.05,-3){\framebox(7.9,0.7){}}
         \put(14,-3.8){$-sa_ly_ht_s $}  \put(10.05,-4){\framebox(7.9,0.7){}}
 
        \end{picture}
      }
  \end{center}

\paragraph{Cancellation}

There may be cancellation when $k = 0$, that is to say when $\sin(a) =
0$, then the computation is simplified automatically. (See in
trigo\_fast.c, code do\_cos and do\_sin)

We just need to compute $\cos(a) * \cos(y)$, wich is reached with the
polynomial precision.


\subsection{Tangent}

The tangent is obtained by the division of the sine by the cosine.
The procedure "Div22" guarantees a $2^{-104}$ precision result, wich is surely enough to have a precise final result.

If $x < 2^{-26}$, x is returned.
In other cases, we will compute a $2^{-63}$ precision sine and a $2^{-71}$ precision cosine.


So we have a final relative error smaller than $2^{-9}$. This is the
reason why our rounding constant is equal to $1+2^{-9}$.


 \chapter{The arctangent \label{chap:atan}}
 \newcommand{\xred}{X_{\mathrm{red}}}
\newcommand{\xredhi}{X_{\mathrm{red hi}}}
\newcommand{\xredlo}{X_{\mathrm{red lo}}}

\section{Overview}

We compute atan in two steps : the first one gives us a precision around 64
bits. The second one compute about 130 bits of precision is order to have
correct rounding in all cases (the worst-case is ...).

\subsubsection{Definition interval and exceptional cases}

The inverse tangent is defined over all real number.

\begin{itemize}
\item If $x = Nan$ , then $\arctan(x)$ should return $NaN$
\item If $x = \pm\infty$ , then $\arctan(x)$ should return
$\pm\round(\pi/2)$. 
\end{itemize}
we choose to return $\pm\round(\pi/2)$ when $|x|>2^{54}$ since
$\pm\round(\pi/2) =\pm\rounddown(\pi/2) $. That could be a probem for
rounding up but we choose not to have a x with $\arctan(x) > \pi/2$.
\section{Quick phase}

We try to have about 64 bits of precision.

\subsection{Overview of the algorithm.}

There are two steps in the algorithm: an argument reduction and a polynomial
approximation with a degree 9 polynomial. We return $\arctan(x)$ when
$x>0$ and $-\arctan(-x)$ when $x<0$ in order to compute $\arctan(x)$ for positive
values only.

We compute $\arctan(x)$ as 
\begin{equation}
\arctan(x) = \arctan( b_i ) + \arctan(\frac{x-b_i}{1+x.b_i}) \label{eq:arctan_redu}
\end{equation}

The $b_i$ are exact double and $\arctan(b_i)$ are stored in
double-double.

We defined $\xred = \dfrac{x-b_i}{1+x.b_i}$ for the rest of this chapter.

We build 62 intervals $[a_i;a_{i+1}]$ and 62 $b_i$ in order that $ x \in
[a_i;a_{i+1}] \Rightarrow \dfrac{x-b_i}{1+x.b_i} < e$

We make a dichotomy in order to find $i$ such as $ x \in [a_i;a_{i+1}]
$. That's why we choose 62 $b_i$ and $e=2^{-6.3}$ (since 62 is close to
$2^6$ and a power of 2 is better for dichotomy).

Then we use a 9 degree polynomial for the approximation of $\arctan(\xred)$:
in order to have 66 bits of precision.

\begin{equation}
\begin{split} \arctan(x)& \approx x - \dfrac{1}{3} .x^3 + \frac{1}{5}.x^5
- \frac{1}{7}.x^7 + \frac{1}{9}.x^9 \\ \label{eq:poly_eval}
  & \approx x . ( 1 + Q(x^2))
\end{split}
\end{equation}
where 
Q is evaluated thanks to a Horner scheme:
$ Q(z) = z. (-\frac{1}{3} + z.(\frac{1}{5} + z.(-\frac{1}{7} +
z.\frac{1}{9}))) $
where each operation is computed in double.

At the end, the reconstruction implements equation and (\ref{eq:poly_eval})  
(\ref{eq:arctan_redu}) in double-double
arithmetic.
 to improve performances.

%\begin{equation} \arctan(x) = \arctan( b_i ) + \arctan(\frac{x-b_i}{1+x.b_i}).\label{eq:\arctan_redu}
%\end{equation}


\subsection{Details of computer program}
  
\subsubsection{Exceptional cases}
\begin{lstlisting}[caption={Exceptional cases},firstnumber=1]

  db_number x_db;
  x_db.d = x;
  unsigned int hx = x_db.i[HI_ENDIAN] & 0x7FFFFFFF; 

  /* Filter cases */
  if ( hx >= 0x43500000)           /* x >= 2^54 */
    {
      if ( ( (hx & 0x000fffff) | x_db.i[LO_ENDIAN] ) == 0)
        return x+x;                /* NaN */
      else
        return HALFPI.d;           /* \arctan(x) = Pi/2 */
    }
  else
    if ( hx < 0x3E400000 )
      {return x;                   /* x<2^-27 then \arctan(x) =~ x */}

\end{lstlisting}
\begin{tabular}{ll}
Lines 3 & Test if x is greatear than $2^{54}$, $\infty$ or $NaN$. \\
Line 5,6 & return $\arctan(NaN) = NaN$\\
Line 8 & \texttt{HALFPI} is the greatest double smaller than
&$\dfrac{\pi}{2}$ in order not to have $\arctan(x) > \dfrac{pi}{2}$.\\
Line 11 & When $x<2^{-27}$ : $x^2 < 2^{-54}$. Plus we know that $\arctan(x) = \displaystyle {\sum_{i=0}^{\infty}
\frac{x^{2i+1}}{2i+1}(-1)^i}$.
\end{tabular}
\begin{eqnarray}
\lefteqn{ 
     \Big| \frac{\arctan(x)-x}{x}  \Big|  = 
     \Bigg|\frac{ \displaystyle {\sum_{i=0}^{\infty}
     \Big( \frac{x^{2i+1}}{2i+1}(-1)^i} \Big) - x}{x} \Bigg|
                                \nonumber 
                                }\\
& & {} = \Big|\displaystyle {\sum_{i=1}^{\infty}}
     \frac{x^{2i}}{2i+1}(-1)^i\Big|\nonumber \\ 
& & {} < \frac{x^2}{3}\nonumber \\
& & {} < 2^{-54} \nonumber
\end{eqnarray}

\bigskip


\subsubsection{Argument reduction}
\begin{lstlisting}[caption={Reduction},firstnumber=1]

  if (x > my_e) /* test if reduction is necessary : */ 
  {
    double xmBIhi,xmBIlo;      

      if (x > value[61][B].d) {
        i=61;
        Add12( xmBIhi , xmBIlo , x , -value[61][B].d);
      }
      else {
        /* determine i so that a[i] < x < a[i+1] */
        i=31;
        if (x < value[i][A].d) i-= 16;
        else i+=16;
        if (x < value[i][A].d) i-= 8;
        else i+= 8;
        if (x < value[i][A].d) i-= 4;
        else i+= 4;
        if (x < value[i][A].d) i-= 2;
        else i+= 2;
        if (x < value[i][A].d) i-= 1;
        else i+= 1;
        if (x < value[i][A].d) i-= 1;
          
        xmBIhi = x-value[i][B].d;
        xmBIlo = 0.0;
      }
\end{lstlisting}

\begin{tabular}{ll}
Lines  1 & test if $x > 2^{-6.3}$ and so need to be reduced\\
Line 5 & test if $x>b[61]$ because when $i \in [0;60] : b_i/2 < x <
b_i$ (or $ x/2 < b_i < x$) and then \\&$x-b_i$ is computed exactly
value thanks to Sterbenz lemma.\\
Line 10...21 & compute $i$ so that $\frac{x-b_i}{1+x.b_i} < 2^{-6.3} $\\
Line 7 and 23 & compute $xmBIhi + xmBIlo = x - b_i$

\end{tabular}

\begin{lstlisting}[caption={Reduction : 2nd part},firstnumber=1]

      Mul12(&tmphi,&tmplo, x, value[i][B].d);

      if (x > 1)
        Add22(&x0hi,&x0lo,tmphi,tmplo, 1.0,0.0);
      else {Add22( &x0hi , &x0lo , 1.0,0.0,tmphi,tmplo);}

      DIV2( xmBihi , xmBilo , x0hi,x0lo, Xredhi,Xredlo);

\end{lstlisting}
\begin{tabular}{ll}
Line 1 & compute $x.b_i$\\
Line 3-5 & We need to have a Add22Comp but as we know that $x.b_i > 0$ (so
$tmphi>0$), We test if\\& $tmphi$ is greater than 1 in order to be
faster.\\
Line 7 & compute $\xred = \dfrac{x-b_i}{1+x.b_i}$
\end{tabular}
\bigskip
\subsubsection{Polynomial evaluation and reconstruction}

\begin{lstlisting}[caption={Polynomial Evaluation and recontruction},firstnumber=1]

      Xred2 = Xredhi*Xredhi;

      q = Xred2*(coef_poly[3]+Xred2*
                 (coef_poly[2]+Xred2*
                  (coef_poly[1]+Xred2*
                   coef_poly[0]))) ;

      /* reconstruction : atan(x) = atan(b[i]) + atan(x) */
      double testlo = Xredlo+ value[i][ATAN_BLO].d + Xredhi*q;
      double tmphi2, tmplo2;
      Add12( tmphi2, tmplo2, value[i][ATAN_BHI].d, Xredhi);
      Add12( atanhi, atanlo, tmphi2, (tmplo2+testlo));

\end{lstlisting}

\begin{tabular}{ll}
Line 1 & Computation of $(\xred)^2$\\
Line 3 & Computation of the polynomial evaluation\\
Line 5-8 & We use an approximation of the reconstruction to compute faster  \\
       & $\arctan(b_i))_{hi}+\arctan(b_i)_{lo} +
        (\xredhi+\xredlo).(1+Q)$\\
       & We compute this in tree steps $test_{lo}= \xredlo+ \arctan(b_i)_{lo}
        + \xredhi.q$.\\
       & then we add $\xredhi+\arctan(b_i)_{hi}$ at the end we add the two results.\\
\end{tabular}

We represent the different values on the next figures :

\label{fig:rec}
\begin{center}
 \small
 \setlength{\unitlength}{3ex}
      \framebox{
        \begin{picture}(22,3.5)(-3,-4.15)
         \put(9.5,-0.5){\line(0,-1){4}}  \put(9,-1){$\epsilon$}
  
          \put(4,-2){$\arctan(b_i)_{hi}$} \put(0.05,-2.15){\framebox(7.9,0.7){}}
          \put(12,-2){$\arctan(b_i)_{lo}$}  \put(8.05,-2.15){\framebox(7.9,0.7){}}

          \put(4,-3){$\xredhi$} \put(0.55,-3.15){\framebox(7.9,0.7){}}
          \put(12,-3){$\xredlo$}  \put(8.55,-3.15){\framebox(7.9,0.7){}}

          \put(5,-4){$\xredhi.Q$} \put(2.05,-4.15){\framebox(7.9,0.7){}}
          \put(13,-4){$\xredlo.Q$}  \put(10.05,-4.15){\framebox(7.9,0.7){}}

        \end{picture}
      }
  \end{center}
\label{fig:rec}
\begin{center}
 \small
 \setlength{\unitlength}{3ex}
      \framebox{
        \begin{picture}(22,3.5)(-3,-4.15)
         \put(9.5,-0.5){\line(0,-1){4}}  \put(9,-1){$\epsilon$}
  
          \put(4,-2){$\arctan(b_i)_{hi}$} \put(0.05,-2.15){\framebox(7.9,0.7){}}
          \put(12,-2){$\arctan(b_i)_{lo}$}  \put(8.05,-2.15){\framebox(7.9,0.7){}}

          \put(4,-3){$\xredhi$} \put(0.55,-3.15){\framebox(7.9,0.7){}}
          \put(12,-3){$\xredlo$}  \put(8.55,-3.15){\framebox(7.9,0.7){}}

          \put(5,-4){$\xredhi.Q$} \put(2.05,-4.15){\framebox(7.9,0.7){}}

        \end{picture}
      }
  \end{center}
\label{fig:rec}
\begin{center}
 \small
 \setlength{\unitlength}{3ex}
      \framebox{
        \begin{picture}(22,2.5)(-3,-4.15)
         \put(9.5,-0.5){\line(0,-1){4}}  \put(9,-1){$\epsilon$}
  
          \put(0.5,-3){$(\arctan(b_i)_{hi} + \xredhi)_{hi}$}
          \put(0.05,-3.15){\framebox(7.9,0.7){}}
          \put(8.5,-3){$(\arctan(b_i)_{hi} + \xredhi)_{lo}$} \put(8.05,-3.15){\framebox(7.9,0.7){}}

          \put(5,-4){$testlo$} \put(2.05,-4.15){\framebox(7.9,0.7){}}


        \end{picture}
      }
  \end{center}
\bigskip
\subsection{Error analysis}

We choose four rounding constant : two when there is a argument reduction, two in
the other case. For each case, we have make two constant on order to
improve performances. We will compute the error separately.

An accurate computation is done in \texttt{maple/arctan\_coef.mw} 

\subsubsubsection{Notes on $b_i$, $a_i$ and $\arctan(b_i)$}
The $b_i$ and $a_i$ are computed thanks to the \texttt{allbi} maple
procedure (see \texttt{maple/arctan\_coef.mw}). There is no approximation
error on $b_i$ since they are computed (in the maple procedure) as
double. $\arctan (b_i)$ are stored in double-double so there is an
approximation of $2^{-105}$ on them. We have a possible error about $a_i$
because they are real number stored as double but we have a margin on $a_i$
that prove us that $\xred$ is always smaller than $e$.

\\
\subsubsection{Error about argument reduction}
\begin{lstlisting}[caption={Reduction part 1},firstnumber=1]

      if (x > value[61][B].d) {
        i=61;
        Add12( xmBihi , xmBilo , x , -value[61][B].d);
      }
      else
      {
        ...
        /* determine i so that a[i] < x < a[i+1] */
        ...          
        xmBihi = x-value[i][B].d;
        xmBilo = 0.0;
      }
      
\end{lstlisting}

In the worst case, we have $\epsilon_{105}$ relative error when we compute
$x-b_i$.

\begin{lstlisting}[caption={Reduction part 2},firstnumber=1]

      Mul12(&tmphi,&tmplo, x, value[i][B].d);

      if (tmphi > 1)
        Add22(&x0hi,&x0lo,tmphi,tmplo, 1.0,0.0);
      else {Add22( &x0hi , &x0lo , 1.0,0.0,tmphi,tmplo);}
      
      DIV2( xmBihi , xmBilo , x0hi,x0lo, xhi,xlo);

\end{lstlisting}

\begin{tabular}{ll}
Line 1 & The error due to the Mul12 $< \epsilon_{105}$\\
Line 4-5 & Add22 : $\epsilon_{105}$\\
Line 6 & DIV2 makes $\epsilon_{104}$ (according to Ziv ... ref) error so we have :
\end{tabular}

\begin {equation}
\epsilon_{\xred} = (1+\epsilon_{105})(1+\epsilon_{105}+\epsilon_{105}+\epsilon_{104})-1 \cong
\epsilon_{102.6}
\end {equation}

\subsubsection{Error about polynomial evaluation}
\begin{lstlisting}[caption={Polynomial Evaluation},firstnumber=1]

      Xred2 = Xredhi*Xredhi;
      
      q = Xred2*(coef_poly[3]+Xred2*
                 (coef_poly[2]+Xred2*
                  (coef_poly[1]+Xred2*
                   coef_poly[0]))) ;

      /* reconstruction : atan(x) = atan(b[i]) + atan(x) */
      double testlo = Xredlo+ value[i][ATAN_BLO].d + Xredhi*q;
      double tmphi2, tmplo2;
      Add12( tmphi2, tmplo2, value[i][ATAN_BHI].d, Xredhi);
      Add12( atanhi, atanlo, tmphi2, (tmplo2+testlo));

\end{lstlisting}
\begin{tabular}{ll}
Line 1 & $\xred \times \xred$ makes an error of $\epsilon_{53}$ so $x2 =
      o((\xredhi)^2) = (\xredhi)^2 + \epsilon_{53} = x^2 + \epsilon_{105} +
      \epsilon_{53} + \epsilon_{53} $\\ 
      &the error about x2 is $\epsilon_{52}$ \\
Line 3 & Horner approximation with error on x2 :
      Maple compute an error around $\epsilon_{50.7}$\\ 
Line 5-9 & We need to add $\arctan(b_i)_{hi}+\arctan(b_i)_{lo} +
      (\xredhi+\xredlo).(1+Q)$\\
      &
      We compute this in tree steps $test_{lo}= \xredlo+ \arctan(b_i)_{lo}
      + \xredhi.q$.\\
      & then we add $\xredhi+\arctan(b_i))_{hi}$ at the end we add the two results.\\

      &We have two kinds of errors. A first because we forgot $\xredlo.Q$
      but $\xredlo < \xredhi.2^{-53}$ \\ &and $q<x^2<2^{-12.6}$ so $error =
      \epsilon_{65.6}$. \\
      & A second error due to the operations : the computation of
      $test_{lo}$ causes 3 errors :\\
      & $ \epsilon_{53}.(\xredlo+ \arctan(b_i)_{lo}) < \delta_{105}$\\
      & $\epsilon_{53}.\xredhi.q < \delta_{53+6.3+6.3*2} < \delta_{71.9}$\\
      & $ \epsilon_{53} . testlo < \delta_{65.5}$\\

      &If we add all these errors, we have 

      $testlo_{error} = < \epsilon_{52}.e^3 + \epsilon_{105}.e +
      \epsilon_{65.6}.e = \delta_{70.89}$\\

Line 8 & Add12 add an relative error of $\epsilon_{105}$ so the absolute error is
      $\epsilon_{105}.(\arctan(b_i)_{hi} + \xredhi) < \epsilon_{105}.\frac{\pi}{2}$\\
Line 9 & Add12 add an relative error of $\epsilon_{105}$ so the absolute error is 
       less than $\epsilon_{105}.\frac{\pi}{2}$
\end{tabular}
\bigskip

\subsubsection {Error due to the polynomial approximation}

The error due to the polynomial approximation is $\delta_{approx} =
\infnorm{ \arctan(x) - x.(1+Q)}= \delta_{72,38}$ 

\subsubsection {Final error and rounding constant}

We have to add all error : 
\begin{equation}
\delta_{error} = \delta_{72,38} (\text{due to the error on the polynomial
approximation}) + \epsilon_{105}.\frac{\pi}{2} +
\epsilon_{105}.\frac{\pi}{2} + \delta_{approx} = \delta_{70.45}
\end{equation}

So when $i < 10$, the relative error is $\epsilon_{64.15}$ that leads to a
rounding constant of $1.000876$.

And when $i > 10$ the relative error is $\epsilon_{70.27}$ that leads to a
rounding constant of $1.00000126$.
\end{tabular}

\subsubsection{Error when there is no reduction}
\begin{lstlisting}[caption={No reduction},firstnumber=1]

      x2 = x*x;
      q = x2*(coef_poly[3]+x2*
                 (coef_poly[2]+x2*
                  (coef_poly[1]+x2*
                   coef_poly[0]))) ;
      Add12(atanhi,atanlo, x , x*q);

\end{lstlisting}

The code is very simple so there is few error :
\begin{tabular}
Line 1 & $\epsilon_{53}$
Line 2 & The Maple procedure to compute Horner approximation gives $\epsilon_{51}$
Line 3 & $delta_{no\_reduction} = \epsilon_{105}.x + \epsilon_Q.x^3 +
\epsilon_{x.Q}.x^3 + |arctan(x) - x.(1+Q)| $
\end{tabular}

When $x>2^{-10}$ the relative error is $\epsilon_{62.9}$. The
constant is $1.0021$. 

When $x<2^{-10}$ the relative error is $\epsilon_{70.4}$. The
constant is $1.0000114$. 

\section{Accurate phase}
The accurate phase is the same as the quick phase, except that number are
scs and not double.

The intervals are the same as in quick phase. The only difference is that
$\arctan(b_i)$ as a third double to improve the precision of $\arctan(b_i)$ to
150 bits.

The polynomial degree is 19 in order to have 136 bits of precision.

\begin{equation} \arctan(x) \approx
x-\frac{1}{3}.x^3+\frac{1}{5}.x^5-\frac{1}{7}.x^7+\frac{1}{9}.x^9-\frac{1}{11}.x^{11}+\frac{1}{13}.x^{13}-\frac{1}{15}.x^{15}+\frac{1}{17}.x^{17}-\frac{1}{19}.x^{19}
\label{eq:arctan_scspoly}
\end{equation}

\section{Analysis of the performance}

\subsection{Speed}
Table \ref{tbl:arctan_abstime} (produced by the \texttt{crlibm\_testperf}
executable) gives absolute timings for a variety of processors and

\begin{table}[!htb]
\begin{center}
\renewcommand{\arraystretch}{1.2}
\begin{tabular}{|l|r|r|r|}
\hline
\hline

 \multicolumn{4}{|c|}{Pentium 4 Xeon / Linux Debian / gcc 2.95}   \\
 \hline
                         & min time      & max time      & avg time \\
 \hline
 \texttt{libm}           & 832          & 920           &        890 \\
 \hline
  \texttt{mpfr}          & 882740       & 2203288       &     933186 \\
 \hline
  \texttt{libultim}      & 656          & 2256          &        831 \\
 \hline
 \texttt{crlibm}         & 688          & 30908         &        964 \\
 \hline


\multicolumn{4}{|c|}{PowerPC G4 / Macos X.2 / gcc }   \\
 \hline
                         & min time      & max time      & avg time \\
 \hline
 \texttt{libm}           & 8            & 13            &          9 \\
 \hline
  \texttt{mpfr}          & 67196        & 894424        &     205574 \\
 \hline
  \texttt{libultim}      & 9            & 19            &          9 \\
 \hline
 \texttt{crlibm}         & 7            & 25            &          8 \\
 \hline

\end{tabular}
\end{center}
\caption{Absolute timings for the inverse tangent (arbitrary units)
  \label{tbl:arctan_abstime}}
\end{table}

\subsection{Memory requirements}
Table size is
\begin{itemize}
\item for the \quick\ phase,
  $62\times (1+1+2) \times8=1984$ bytes for the 62 $a_i$, $b_i$,
  $\arctan(b_i)$ (hi and lo), plus another $8$ bytes for the rounding
  constant, plus $4\times8$ for the polynomial and $8$ Bytes for
  $\frac{\pi}{2}$ or $2032$ Bytes in total.
  
\item for the \accurate\ phase, we just have $10$ SCS constants for the
  polynomial, and 62 other double for $\arctan(b_i)_{lo_{lo}}$.
  If we add all : $10*11*8 + 62*8 = 1376$
\end{itemize}
If we add the fast phase and the acurate one, we have a total of 3408
Bytes.
\section{Conclusion and perspectives}


 \chapter{The hyperbolic sine and cosine \label{chap:csh}}
This chapter is contributed by Matthieu Gallet under the supervision of
F. de~Dinechin.  


\section{Overview}

Like the algorithms for others elementary functions, we will compute
our hyperbolic cosine and sine in one or two steps.  The first one,
which is called 'fast step' is always executed and provides a
precision of about 63 bits.  This is a sufficient precision in most
cases, but sometimes more precision is needed, and then we enter the
second step (called `slow pass') using the SCS library.


\subsubsection*{Definition interval and exceptional cases}
The hyperbolic cosine and the hyperbolic sine are defined for all real
numbers, and then for all floating point numbers.  These functions are
divergent toward $+\infty$ and $-\infty$, so for $|x| > 710.47586007$,
$\cosh(x)$ and $\sinh(x)$ should return $+\infty$ or the greatest
representable number (depending ot the choosen rounding mode).

\begin{itemize}
\item If $x = NaN$ , then $\sin(x)$ and $\cosh(x)$ should return $NaN$
\item If $x = +\infty$ , then $\cosh(x)$ and $\sinh(x)$ should return $+\infty$. 
\item If $x = -\infty$ , then $\cosh(x)$ should return $+\infty$. 
\item If $x = -\infty$ , then $\sinh(x)$ should return $-\infty$. 
\end{itemize}

This is true in all rounding modes.

Concerning denormals, $\cosh(x) \geq 1$, so $\cosh(x)$ can't return
denormal numbers, even for denormal inputs.  For small inputs ($|x|
\leq 2^{-40}$), we have $\sinh(x) = x$ with 80 bits of precision, so
we can return a result without any computation on denormals.

\section{Quick phase}

\subsection{Overview of the algorithm}

The algorithm consists of two argument reduction using classical formulaes of hyperbolic trigonometry, followed by a polynomial evaluation using a Taylor polynom of degree $6$ (for $\cosh$) and $7$ (for $\sinh$).

These formulaes are:
\begin{itemize}
\item  $\sinh(x + y) = sinh(x) * cosh(y) + sinh(y) * cosh(x)$
\item  $\cosh(x + y) = cosh(x) * cosh(y) + sinh(x) * sinh(y)$
\item  $cosh(k*ln(2)) = 2^{k-1} + 2^{-k-1}$
\item   $sinh(k*ln(2)) = 2^{k-1} - 2^{-k-1}$
\end{itemize}



After having treated special cases ($NaN$, $+\infty$, $-\infty$), we do a first range reduction, to have an small input x (between $\frac{-ln(2)}{2}$ and $\frac{ln(2)}{2}$).
So, we write $x = k*\ln(2) + y$, where k is given by rounding to the nearest integer $x * \frac{1}{ln(2)}$.
Now, $\frac{-ln(2)}{2} \leq y  \leq \frac{ln(2)}{2}$, but it is even too large to have a sufficient precision during polynomial evaluation with small polynoms, and we do a second range reduction, by writing $y = a + b$, where $a = index * 2^{-8}$ (index is an integer)  and $|b| \leq 2^{-9}$.

Mathematically, we have: $$\sinh(x) = (2^{k-1} + 2^{-k-1})*\sinh(y) + (2^{k-1} - 2^{-k-1}) * \cosh(y)$$
and
 $$\cosh(x) = (2^{k-1} + 2^{-k-1})*\cosh(y) + (2^{k-1} - 2^{-k-1}) * \sinh(y)$$
The second range reduction allows to compute $sinh(y)$ and $cosh(y)$ as $sinh(y) = sinh(a) * cosh(b) + sinh(b) * cosh(a)$ and $cosh(y) = cosh(a) * cosh(b) + sinh(a) * sinh(b)$. In the C code, we have $ch\_hi + ch\_lo = cosh(y)$ and $sh\_hi + sh\_lo = sinh(y)$.

A quick computation shows that $-89 \leq index \leq 89$, and we can pre-compute so few values of $sinh(a)$ and $cosh(a)$ and store them in a table as double doubles.


The constants $2^{k-1}$ and $2^{-k-1}$ are constructed by working directly on their hexadecimal representation.


$cosh(b)$ and $sinh(b)$ are computed with Taylor polynoms. It's well-known that $$cosh(b) = \sum_{n \geq 0}{\frac{x^{2n}}{(2n)!}}$$ and $$sinh(b) = \sum_{n \geq 0}{\frac{x^{2n+1}}{(2n+1)!}}$$
For our needs, a degree $6$ polynom for cosh and a degree $7$ polynom for sinh give enough accuracy.

We write $\cosh(b) = 1 + tcb$ and $\sinh(b) = b *(1 + tsb)$, where
$$tcb = b^{2} * (\frac{1}{2} + b^{2} * (\frac{1}{24} + b^{2} * \frac{1}{720}))$$
$$tsb = b^{2} * (\frac{1}{6} + b^{2} * (\frac{1}{120} + b^{2} * \frac{1}{5040}))$$
We recognize the Horner scheme used for the evaluation of the polynoms, with all the coefficients being coded on doubles numbers.

If the input is too small (i.e. $|b| \leq 2^{-40}$), $tsb$ and $tcb$ are not calculated but directly set to $0$, to avoid any problem with denormalized numbers.


At this stage, we have computed all the needed sub-terms before the final reconstruction, which is done in two steps, corresponding to the two-step range-reduction. The reconstruction is computed in double-double arithmetic.
In the first reconstruction, some sub-terms can be ignored without any loss of precision, due to their very small relative values.
For this step, it exists a particular case, when $index = 0$, since it is the only case where $|sinh(a)| < 2^{-9}$ ($sinh(a) = 0$).
Now we have the definitive values of $cosh(y)$ and $sinh(y)$. 

In the second reconstruction, we begin by computing all needed products before adding their results (i.e. $2^{k-1}*cosh(y)$, $2^{k-1}*sinh(y)$,...). When $|k| \geq 35$, some terms can also be ignored, to increase speed. Computations are also done using double double arithmetics, with the Add22 function.

After this last step, we can do some tests to see whether the accurate phase using SCSLib is needed or not.


\subsection{Error analysis}

Many of the described computations can introduce a new term of error, so we must control them to obtain the final guaranteed precision.
\begin{itemize}
\item{First range reduction}
We have to consider two different cases:
\begin{itemize}
\item{$|x| \leq \frac{ln(2)}{2}$}

We have $k = 0$, and there is no reduction, and no term of error.
\item{$|x| > \frac{ln(2)}{2}$}

We have $k \neq 0$, and we must compute the term of error introduced by the range reduction.
Since $k$ is an integer, we can assume that there is no error on it. 
$ln(2)$ is a constant which is stored in the function in the double double format, and we have $ln(2) = ln2_{hi} + ln2_{lo} + \maxabserr{repr\_ln2}$, where $|\maxabserr{repr ln2}| \leq 1.94e-31$.
The total absolute error of this reduction is $\maxabserr{range\_reduc} = 3.437e-27$, so the maximum relative error is $\maxrelerr{range\_reduc} = 9.9e-27$ (we have $|x| \geq 0.36$), and that represents about $86.38$ bits of precision.

\end{itemize}

\item{Second range reduction}
This range reduction is exact (we only cut y in two parts, with no multiplication nor division), so no new term of error is introduced.

\item{Error in tabulation}
Since $cosh(a)$ and $sinh(a)$ are stored as double doubles, and since there are transcendental numbers (when $a \neq 0$), some error is done on their approximation.
A simple Maple procedure can compute this error, which about $\maxabserr{ca} = 6.08e-33$ for cosh and $\maxabserr{sa} = 1.47e-33$ for sinh. That is large overkill compared to precision on other values.
\item{Error in polynomial approximations}
We use the $errlist\_quickphase$ and $compute\_horner\_rounding\_error$ Maple procedures to compute thes errors on $tcb$ and $tsb$, which are $\maxabserr{rounding\_cosh} = 6.35e-22$ and $\maxabserr{rounding\_sinh} = 1.94e-22$. It exists an `mathematical'  error due to the polynomial approximation. The sum of theses errors gives $\maxabserr{tcb} = 6.35e-22$ and $\maxabserr{tsb} = 1.11e-21$.
\item{First reconstruction}
This reconstruction is done by adding all the pre-calculated terms ($tcb$, $tsb$, $ca = cosh(a)$, $sa = sinh(a)$), in an order which try to minimiza the total error.$\maxabserr{sh} = 2.10e-25$. Maple scripts are used to compute the error, since there are many terms.
There are 2 different cases:
\begin{itemize}
\item{$a = 0$}

$ch_{hi}+ch_{lo} = \widehat{cosh(\widehat{y})} + \abserr{cosh0}$, where $|\abserr{cosh0}| \leq \maxabserr{cosh0} = 6.35e-22$, and 
$sinh(y) = \widehat{sinh(\widehat{y})} + \abserr{sinh0}$, where $|\abserr{sinh0}| \leq  \maxabserr{sinh0} = 5.4e-20$.
\item{$a \neq 0$}

$ch_{hi}+ch_{lo} = \widehat{cosh(\widehat{y})} + \abserr{cosh1}$, where $|\abserr{cosh1}| \leq \maxabserr{cosh1} = 2.39e-20$, and 
$sinh(y) = \widehat{sinh(\widehat{y})} + \abserr{sinh1}$, where $|\abserr{sinh1}| \leq  \maxabserr{sinh1} = 1.09e-22$.
\end{itemize}

\item{Second reconstruction}
This reconstruction is based on multiplying the obtained results before adding them. The products are exact since each product has a factor which a power of 2. 
We have to leave absolute errors for relative errors, since the range of values returned by $cosh$ is to large. 
We will distinguish three different cases:
\begin{itemize}
\item{$|k| \leq 35$}
All terms must be computed. We have $ch_{hi} + ch_{lo} = \widehat{cosh(\widehat{x})}*(1+\relerr{ch})$, where $|\relerr{ch}| \leq \maxrelerr{ch} = 7.66e-19$
\item{$k > 35$}
In this case, some terms are negligible. We have $ch_{hi} + ch_{lo} = \widehat{cosh(\widehat{x})}*(1+\relerr{ch})$, where $|\relerr{ch}| \leq \maxrelerr{ch} = 7.69e-19$
\item{$k < -35$}
This case is symmetric to the previous one, we just have to remplace k by -k.
\end{itemize}
\end{itemize} 


\newpage
\subsection{Details of computer program}

The procedures \texttt{cosh\_quick} and \texttt{sinh\_quick} contain the computation respectively shared by the 
functions \texttt{cosh\_rn}, \texttt{cosh\_ru}, \texttt{cosh\_rd} and \texttt{cosh\_rz} in one hand, and by the 
functions \texttt{sinh\_rn}, \texttt{sinh\_ru}, \texttt{sinh\_rd} and \texttt{sinh\_rz} in the other hand.
The eight functions \texttt{cosh\_rX} and \texttt{sinh\_rX} call \texttt{cosh\_quick} or \texttt{sinh\_quick} with an integer which represent the choosen rounding mode.
We will begin to prove the cosh function, and then we will prove the sinh function. Since both functions share a lot a code, only the different part between cosh and sinh will be proven for the sinh. 

\subsubsection{Exceptional cases and argument reduction}

This part  is shown for \texttt{cosh\_rn}, but it is quite identical for the three other functions.

\begin{lstlisting}[caption={Exceptional cases},firstnumber=1]

 double cosh_rn(double x){ 
  db_number y;
  y.d = x;
  y.i[HI_ENDIAN] = y.i[HI_ENDIAN] & 0x7FFFFFFF;     /* to get the absolute value of the input */
  if (y.d > max_input_ch.d) { /* out of range */
    y.i[LO_ENDIAN] = 0; y.i[HI_ENDIAN] = 0x7FF00000; return (y.d);
  }
  if ((y.i[HI_ENDIAN] & 0x7FF00000) >= (0x7FF00000)) {    /*particular cases : QNaN, SNaN, +- oo*/
   return (y.d);
  }
  return(cosh_quick(x, RN));
}
\end{lstlisting}

\begin{tabular}{ll}
Lines  3 &  Initialize y\\
Line 4 & Get the absolute value of y by removing the first bit.\\
Line 5 & Test if $cosh(|x|) = cosh(x)$ is representable as a double.\\
Line 6 & If this test is true, we must return $\infty$.\\
Line 8 & Test if $|x|$ is a special case, like NaN or $\infty$\\
Line 9 & If this test is true, we must return $|x|$ \\
Line 11 & $x$ is a correct input, we can return cosh\_quick. \\
\end{tabular}



\subsubsection{Procedure cosh\_quick}

\begin{lstlisting}[caption={Procedure \texttt{cosh\_quick} - variables},firstnumber=1]

 double cosh_quick(double x, int rounding_mode){

  /*some variable declarations */
  int k;
  db_number y;
  double res_hi, res_lo;
  double ch_hi, ch_lo, sh_hi, sh_lo;/* cosh(x) = (ch_hi + ch_lo)*(cosh(k*ln(2)) + (sh_hi + sh_lo)*(sinh(k*ln(2))) */
  db_number  table_index_float;
  int table_index;
  double temp_hi, temp_lo, temp;/* some temporary variables */
  double b_hi, b_lo,b_ca_hi, b_ca_lo, b_sa_hi, b_sa_lo;
  double ca_hi, ca_lo, sa_hi, sa_lo; /*will be the tabulated values */
  double tcb_hi, tsb_hi; /*results of polynomial approximations*/
  double square_y_hi;
  double ch_2_pk_hi, ch_2_pk_lo, ch_2_mk_hi, ch_2_mk_lo;
  double sh_2_pk_hi, sh_2_pk_lo, sh_2_mk_hi, sh_2_mk_lo;
  db_number two_p_plus_k, two_p_minus_k; /* 2^(k-1) + 2^(-k-1) */
  db_number absyh, absyl, u53, u;

\end{lstlisting}

Here there are all the variables which will be used in the code.

\subsubsection{First range reduction}

\begin{lstlisting}[caption={Procedure \texttt{cosh\_quick} - first range reduction},firstnumber=19]

  /* Now we can do the first range reduction*/
  DOUBLE2INT(k, x * inv_ln_2.d)
    if (k != 0){ /* b_hi+b_lo =  x - (ln2_hi + ln2_lo) * k */
      temp_hi = x - ln2_hi.d * k;                                         
      temp_lo = -ln2_lo.d * k;                                          
      Add12Cond(b_hi, b_lo, temp_hi, temp_lo); 
    }
    else {                                                         
      b_hi = x;  b_lo = 0.;
    }                                                               
\end{lstlisting}

\begin{tabular}{ll}
Line 20 & Put in k the closest integer of x * inv\_ln\_2. \\
        & We use the property of DOUBLE2INT that convert a floating-point number in rouding to nearest mode. \\
        & By its definition, $k$ satisfies the following properties:\\
        &  $\lfloor x \times inv\_ln2 \rfloor \leq k \leq \lceil x \times inv\_ln2\rceil$ \\
        & $|k| \leq \frac{x}{2} \times inv\_ln2$ \\
        & since $|x| \leq 710.475...$, we have $|k| \leq 1025$, so $k$ is coded on at most 11 bits. \\
Line 21 & First case : $k \neq 0$ \\
        & We have by contruction : $ln2_{hi} + ln2_{lo} = \ln(2) + \abserr{repr\_ln2}$, where $|\abserr{repr\_ln2}| \leq \maxabserr{repr\_ln2} = 1.95e-31$.\\
        & the last 11 bits of $ln_{hi}$ are set to zero by its construction \\
Line 22 & the $ln2_{hi} * k$ product is exact since $k$ is coded on at most 11 bits and the last 11 bits of $ln2_{hi}$ are zeros \\
        & we have to use the properties verified by $k$: $x * inv\_ln2 - 1 \leq k \leq x * inv\_ln2 + 1$\\
        & if $x \geq 0$ \\
        & we have $ k \geq 1$ and then $x \geq \frac{\ln(2)}{2}$, so $(x * inv\_ln2 + 1)*ln2_{hi} \leq 2*x$\\
        & since $|k| \leq \frac{x}{2} \times inv\_ln2$, we have $\frac{x}{2} \leq (x * inv\_ln2 - 1)*ln2_{hi}$\\
        & and then we have $\frac{x}{2} \leq k * ln2_{hi} \leq 2*x$\\
        & we can apply the Sterbenz theorem to prove that the result of this line is exact\\
        & if $x \leq 0$\\
        & we can use the same reasoning and then apply the Sterbenz theorem\\
        & and this line of code is always exact. \\
Line 23 & this product is not exact, we can loose at most 11 bits of precision\\
        & there is an error of $\abserr{round}$ which satisfies $|\abserr{round}|\leq \maxabserr{round} = 3.15e-30$ on $ln2_{lo}$ \\
        & so a bound to the maximal absolute error is $k_{max} * \maxabserr{round}$\\
Line 24 & We do an Add12 to have well-aligned double doubles in $b_{hi}$ and $b_{lo}$\\
        & The conditionnal version is used since temp\_hi  can be zero if $x$ is very close to $k * ln(2)$.\\
        & The total absolute error is bounded by $\maxabserr{b} = 3.43e-27$ \\
Line 27 & We have $k = 0$. We needn't to do any reduction, so $b_{hi} + b_{lo} = x$ exactly.\\
\end{tabular}

At this stage, we have $b_{hi} + b_{lo} = \widehat{y} + \abserr{b}$, where $|\abserr{b}| \leq \maxabserr{b} = 3.43e-24$. Now we will write $y = a + b$, where $a = 2^{-8}* index$. 

\subsubsection{Second range reduction}

\begin{lstlisting}[caption={Procedure \texttt{cosh\_quick} - second range reduction},firstnumber=29]

  /*we'll construct 2 constants for the last reconstruction */
  two_p_plus_k.i[LO_ENDIAN] = 0;
  two_p_plus_k.i[HI_ENDIAN] = (k-1+1023) << 20;
  two_p_minus_k.i[LO_ENDIAN] = 0;
  two_p_minus_k.i[HI_ENDIAN] = (-k-1+1023) << 20;

  /* at this stage, we've done the first range reduction : we have b_hi + b_lo  between -ln(2)/2 and ln(2)/2 */
  /* now we can do the second range reduction */
  /* we'll get the 8 leading bits of b_hi */
  table_index_float.d = b_hi + two_43_44.d;
  /*this add do the float equivalent of a rotation to the right, since -0.5 <= b_hi <= 0.5*/
  table_index = LO(table_index_float.d);/* -89 <= table_index <= 89 */
  table_index_float.d -= two_43_44.d;
  table_index += bias; /* to have only positive values */
  b_hi -= table_index_float.d;/* to remove the 8 leading bits*/
  /* since b_hi was between -2^-1 and 2^1, we now have b_hi between -2^-9 and 2^-9 */  
\end{lstlisting}

\begin{tabular}{ll}
Line 30-33 & Put in \texttt{two\_p\_plus\_k} and \texttt{two\_p\_minus\_k} the exact values of $2^{k-1}$ and $2^{-k-1}$.\\
Line 38-44 & The goal of the second range reduction is to write $y$ as $y = index * 2^{-8} + b$ \\
           & We have $|y| \leq \frac{ln(2)}{2} \leq \frac{1}{2}$ \\
           & so $2^{44} \leq 2^{44} + 2^{43} + y \leq 2^{44} + 2^{43} + 2^{42}$ \\
           & since the mantissa counts 53 bits, only the part above $2^{-8}$ si kept in table\_index\_float\\
           & It is easy to show that we have $-89 \leq table\_index \leq 89$ \\
           & so we can add $bias = 89$ to $table\_index$ to have only positive values. \\
           & then we remove this bits of $y$ to obtain the final $b = b_{hi} + b_{lo}$ \\
           & all these operations are exact, so the final absolute error doesn't increase \\

\end{tabular}


\subsubsection{Polynomial evaluation - First reconstruction}

\begin{lstlisting}[caption={Procedure \texttt{cosh\_quick} - polynomial evaluation - first reconstruction},firstnumber=45]
  y.d = b_hi;
  /*   first, y�  */
  square_b_hi = b_hi * b_hi;
  /* effective computation of the polynomial approximation */
  
  if (((y.i[HI_ENDIAN])&(0x7FFFFFFF)) < (two_minus_30.i[HI_ENDIAN])) {
    tcb_hi = 0;
    tsb_hi = 0;
  }
  else {
    /*   second, cosh(b) = b� * (1/2 + b� * (1/24 + b� * 1/720)) */
    tcb_hi = (square_b_hi)* (c2.d + square_b_hi * (c4.d + square_b_hi * c6.d));
    tsb_hi = square_b_hi * (s3.d + square_b_hi * (s5.d + square_b_hi * s7.d));
  }
 

  if( table_index != bias) {
    /* we get the tabulated the tabulated values */
    ca_hi = cosh_sinh_table[table_index][0].d;
    ca_lo = cosh_sinh_table[table_index][1].d;
    sa_hi = cosh_sinh_table[table_index][2].d;
    sa_lo = cosh_sinh_table[table_index][3].d;
    
    /* first reconstruction of the cosh (corresponding to the second range reduction) */
    Mul12(&b_sa_hi,&b_sa_lo, sa_hi, b_hi);
    temp =  ((((((ca_lo + (b_hi * sa_lo)) + b_lo * sa_hi) + b_sa_lo) + (b_sa_hi * tsb_hi)) + ca_hi * tcb_hi) + b_sa_hi);
    Add12Cond(ch_hi, ch_lo, ca_hi, temp);
      /* first reconstruction for the sinh (corresponding to the second range reduction) */
  }
  else {
    Add12Cond(ch_hi, ch_lo, (double) 1, tcb_hi);
  }
  
  
\end{lstlisting}

\begin{tabular}{ll}
Line 45 & Put in $y$ the value of $b_{hi}$, so we can use its hexadecimal aspect \\
Line 47    & Put $b^2$ in $square\_b_{hi}$. We have $square\_b_{hi} = \widehat{b} + \abserr{square\_b}$, where $|\abserr{square\_b}| \leq \maxabserr{square\_b} = 4.23e-22$ \\
Line 50    & Match $b_{hi}$ and then $b$ with $2^{-40}$\\
Line 51-52 & If $|b| \leq 2^{-40}$, we will have $|tcb|,[tsb| \leq \maxabserr{square\_b}$, so we can directly set $tcb$ and $tsb$ to zero: \\
           & converning the mathematical values, we have $|\widehat{tcb}|, |\widehat{tsb}| \leq 2^{-24}$. \\
           & We can avoid by this way any problem with denormalized numbers. \\
Line 55-56 & Polynomial evaluation of $cosh(x)-1$ and $\frac{sinh(x)}{x}-1$, following the H�rner scheme. \\
           & A maple procedure is used to compute the error on this computations\\
           & There are 2 reasons for the total error :\\
           & the effective computations, since all operations are done with 53 bits of precision.\\
           & the mathematical approximation, since we use polynoms \\
           & finally, we have $tcb = \widehat{cosh(\widehat{b}-1)} + \abserr{tcb}$, where $|\abserr{tcb}| \leq \maxabserr{tcb} = 6.35e-22$, \\
           & and $tsb = \widehat{(\frac{sinh(\widehat{b})}{\widehat{b}}-1)} + \abserr{tsb}$, where $|\abserr{tsb}| \leq \maxabserr{tsb} = 1.11e-21$ \\
Line 60    & If $y$ is very close to $0$, we have the 8 bits of the second range reduction which are null \\
Line 62-65 & We get tabulated values for $cosh(a)$ and $sinh(a)$. They are tabulated as double doubles: \\
           & we have $ca_{hi} + ca_{lo} = \widehat{cosh(\widehat{a})} + \abserr{ca}$, where $|\abserr{ca}| \leq \maxabserr{ca} = 6.08e-33$, \\
           & and $sa_{hi} + sa_{lo} = \widehat{sinh(\widehat{a})} + \abserr{sa}$, where $|\abserr{sa}| \leq \maxabserr{sa} = 1.47e-33$, \\
Line 68    & $b\_sa_{hi} + b\_sa_{lo} = sa_{hi} * b_{hi}$. This product is exact. \\
Line 69-70 & it is the reconstruction : $cosh(y) = cosh(a)*(1+tcb) + sinh(a)*b*(1+tsb)$ \\
           & A maple procedure is used to compute the error done in this reconstruction. \\
           & We have $ch_{hi}+ch_{lo} = \widehat{cosh(\widehat{y})} + \abserr{cosh1}$, where $|\abserr{cosh1}| \leq \maxabserr{cosh1} = 2.39e-20$\\
Line 75    & If $y$ is very close to $0$, we have $a = 0$ and $cosh(y) = cosh(b) = 1 + tcb$. \\
           & This addition is exact, so no error is introduced. \\
           & We have $ch_{hi}+ch_{lo} = \widehat{cosh(\widehat{y})} + \abserr{cosh0}$, where $|\abserr{cosh0}| \leq \maxabserr{cosh0} = 6.35e-22$\\
\end{tabular}


\subsubsection{Second reconstruction}

\begin{lstlisting}[caption={Procedure \texttt{cosh\_quick} - reconstruction},firstnumber=77]
  if(k != 0) {
    if( table_index != bias) {
      /* first reconstruction for the sinh (corresponding to the second range reduction) */
      Mul12(&b_ca_hi , &b_ca_lo, ca_hi, b_hi);
      temp = (((((sa_lo + (b_lo * ca_hi)) + (b_hi * ca_lo)) + b_ca_lo) + (sa_hi*tcb_hi)) + (b_ca_hi * tsb_hi));
      Add12(temp_hi, temp_lo, b_ca_hi, temp);
      Add22Cond(&sh_hi, &sh_lo, sa_hi, (double) 0, temp_hi, temp_lo);
    }
    else {
      Add12Cond(sh_hi, sh_lo, b_hi, tsb_hi * b_hi + b_lo);
    }
    if((k < 35) && (k > -35) )
      {
        ch_2_pk_hi = ch_hi * two_p_plus_k.d;
        ch_2_pk_lo = ch_lo * two_p_plus_k.d;
        ch_2_mk_hi = ch_hi * two_p_minus_k.d;
        ch_2_mk_lo = ch_lo * two_p_minus_k.d;
        sh_2_pk_hi = sh_hi * two_p_plus_k.d;
        sh_2_pk_lo = sh_lo * two_p_plus_k.d;
        sh_2_mk_hi = -1 * sh_hi * two_p_minus_k.d;
        sh_2_mk_lo = -1 * sh_lo * two_p_minus_k.d;
        
        Add22Cond(&res_hi, &res_lo, ch_2_mk_hi, ch_2_mk_lo, sh_2_mk_hi, sh_2_mk_lo);
        Add22Cond(&ch_2_mk_hi, &ch_2_mk_lo , sh_2_pk_hi, sh_2_pk_lo, res_hi, res_lo);
        Add22Cond(&res_hi, &res_lo, ch_2_pk_hi, ch_2_pk_lo, ch_2_mk_hi, ch_2_mk_lo);
      } 
    else if (k >= 35) 
      {
        ch_2_pk_hi = ch_hi * two_p_plus_k.d;
        ch_2_pk_lo = ch_lo * two_p_plus_k.d;
        sh_2_pk_hi = sh_hi * two_p_plus_k.d;
        sh_2_pk_lo = sh_lo * two_p_plus_k.d;
        Add22Cond(&res_hi, &res_lo, ch_2_pk_hi, ch_2_pk_lo, sh_2_pk_hi, sh_2_pk_lo);
      }
    else /* if (k <= -35) */
      {
        ch_2_mk_hi = ch_hi * two_p_minus_k.d;
        ch_2_mk_lo = ch_lo * two_p_minus_k.d;
        sh_2_mk_hi = -1 * sh_hi * two_p_minus_k.d;
        sh_2_mk_lo = -1 * sh_lo * two_p_minus_k.d;
        Add22Cond(&res_hi, &res_lo, ch_2_mk_hi, ch_2_mk_lo, sh_2_mk_hi, sh_2_mk_lo);
      }
  }
  else {
    res_hi = ch_hi;
    res_lo = ch_lo;
  }

\end{lstlisting}

\begin{tabular}{ll}
Line 77    & Test if $k = 0$ or not \\
Line 78-87 & We have $k \neq 0$, so we must compute $sinh(y)$ \\
           & This computation is done like the computation of $cosh(h)$ \\
           & We can use an Add12 (instead of Add12Cond) since $b_{hi} * ca_{hi} \geq  temp$ \\
           & A maple script gives $sinh(y) = \widehat{sinh(\widehat{y})} + \abserr{sinh1}$, where $|\abserr{sinh1}| \leq  \maxabserr{sinh1} = 1.09e-22$ \\

           & and $|\abserr{sinh1}| \leq  \maxabserr{sinh0} = 5.4e-20$ (when $sinh(a) = 0$) \\
Line 89    & We have $k\neq 0$, and $|k| \leq 35$ \\
Line 91-98 & we multiply $sinh(y)$ and $cosh(y)$ by powers of 2, so these products are exact \\
Line 100-102 & A maple script is used to compute the error: \\
             & We have $ch_{hi} + ch_{lo} = \widehat{cosh(\widehat{x})}*(1+\relerr{ch})$, where $|\relerr{ch}| \leq \maxrelerr{ch} = 7.66e-19$ \\
Line 104   & $k \geq 35$ \\
Line 106-109 & we multiply $sinh(y)$ and $cosh(y)$ by powers of 2, so these products are exact \\
           & Some terms are not computed, since they are too little \\
Line 110   &  A maple script is used to compute the error: \\
             & We have $ch_{hi} + ch_{lo} = \widehat{cosh(\widehat{x})}*(1+\relerr{ch})$, where $|\relerr{ch}| \leq \maxrelerr{ch} = 7.69e-19$ \\
Line 112   & $k \leq -35$ \\
Line 114-118 & this case is symmetric to the previous one. \\
           & We also have $ch_{hi} + ch_{lo} = \widehat{cosh(\widehat{x})}*(1+\relerr{ch})$, where $|\relerr{ch}| \leq \maxrelerr{ch} = 7.69e-19$ \\
Line 121   & we now have $k = 0$ \\
           & Since we have $1 \leq cosh(x)$, we have  $\maxrelerr{ch} \leq max(\maxabserr{cosh0},\maxabserr{cosh1}) = 2.39e-20$ \\
\end{tabular}

At this, stage, we have $ch_{hi} + ch_{lo} = \widehat{cosh(\widehat{x})}*(1+\relerr{ch})$, where  $|\relerr{ch}| \leq \maxrelerr{ch} = 7.69e-19 = 2^{-60.17}$ .


\subsection{Rounding}

\subsubsection{Rounding to the nearest}

The code for rounding is strictly identical to that of
Theorem~\ref{th:roundingRN1}.  The condition to this theorem that
$\mathtt{res\_hi}\ge 2^{-1022+53}$ is ensured by the image domain of
the $\cosh$, since $\cosh(x) \geq 1$. The rounding constant

\subsection{Directed rounding}
Here again, the code is strictly identical to that of
Theorem~\ref{th:roundingDirected}, and the conditions to this theorem
are ensured by the image domain of the cosh.




%%%%%%%%%%%%%%%%%%%%%%%%%%%%%%%%%%%%%%%%%%%%%%%%%%%%%%%%%%%%%%%%%%%%%%%%%%%%%%%%%%%%%
\section{Accurate phase}
For the accurate phase, we use SCSLib, which provides 208 bits of precision. That is a large overkill, since worst cases don't ask  more than 158 bits. Since an exp function already exists in crlibm, we can use it, by considering the following mathematic formulae : $cosh(x) = e^x + e^{-x}$.
\subsubsection{Polynomial evaluation - First reconstruction}

\begin{lstlisting}[caption={Procedure \texttt{cosh\_quick} - polynomial evaluation - first reconstruction},firstnumber=177]
  if ((k > -80) && (k < 80)) {
    exp_SC(exp_scs, x);
    scs_inv(exp_minus_scs, exp_scs);
    scs_add(res_scs, exp_scs, exp_minus_scs);
    scs_div_2(res_scs);
  }
  else if (k >= 80) {
    exp_SC(res_scs, x);
    scs_div_2(res_scs);
  }
  else {
    exp_SC(res_scs, -x);
    scs_div_2(res_scs);
  }
\end{lstlisting}

\begin{tabular}{ll}
Line 177   & We will distinguish three cases, since if $|x|$ is large enough, we can ignore some computations. \\
Line 178-181 & The used formulae is $cosh(x) = exp(x) + \frac{1}{exp(x)}$. \\
           & SCSlib ensure 208 bits of precision, so the final result has at least 158 bits of precision,\\
           & and that is enough for a correct rounding \\
Line 182   & If $|k| \geq 80$, $e^{-x}$ is negligible, since we have $e^{-x} \leq 2^{-160} e^x$. \\
           & So we continue to have our 158 bits of precision \\
Line 186   & This case is symetric to the previous one. \\
\end{tabular}

%%%%%%%%%%%%%%%%%%%%%%%%%%%%%%%%%%%%%%%%%%%%%%%%%%%%%%%%%%%%%
\section{Analysis of cosh performance}
\label{section:cosh_results}
The input numbers for the performance tests given here are random
positive double-precision numbers with a normal distribution on the
exponents. More precisely, we take random 63-bit integers and cast
them into double-precision numbers.  


In average, the second step is taken in 0.13\% of the calls.


\subsection{Speed}
Table \ref{tbl:cosh_abstime} (produced by the \texttt{crlibm\_testperf}
executable) gives absolute timings for a variety of processors and
operating systems.  

\begin{table}[!htb]
\begin{center}
\renewcommand{\arraystretch}{1.2}
\begin{tabular}{|l|r|r|r|}
\hline
 \multicolumn{4}{|c|}{Processor / system / compiler}   \\ 
 \hline
                         & min time      & max time      & avg time \\ 
 \hline
 \texttt{libm}           & 784          & 1312          &       1034 \\ 
 \hline
  \texttt{mpfr}          & 53944        & 141460        &      63796 \\ 
 \hline
 \texttt{crlibm}         & 788          & 39720         &       1283 \\ 
 \hline
\hline
\end{tabular}
\end{center}
\caption{Absolute timings for the hyperbolic cosine
  \label{tbl:cosh_abstime}}
\end{table}

Contributions to this table for new processors/OS/compiler combinations are welcome.



%%%%%%%%%%%%%%%%%%%%%%%%%%%%%%%%%%%%%%%%%%%%%%%%%%%%%%%%%%%%%
\section{Conclusion and perspectives}

No cosh function exists in Ziv's libultim library, so the only quick hyperbolic cosine which was available for testing is in system libm. Our hyperbolic cosine is 30\% slower than this cosh, but this drawback is compensated by the fact that crlibm can ensure correct rounding, and 4 different rounding modes. System libm has only one rounding mode, and an average failure ratio of 42\%; most of the time only the last bit is false, but sometimes 11 or 12 bits can be false. 





\appendix



\bibliographystyle{plain} 
\bibliography{elem-fun}

\end{document}


