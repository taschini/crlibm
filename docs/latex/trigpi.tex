



\subsection{$\cos(\pi x)$ for small arguments}


  \begin{equation}
    \cos(\pi x) = 1-(\pi x)^2/2 + O(x^4)\label{eq:cospiTaylor}
  \end{equation}
 where $O(x^4)$ is positive.

  Therefore $\cos(\pi x)$ is rounded to $1$ in RN and RU mode if
  $\pi x<\sqrt{2^{-53}}$. In RD and RZ modes, we have $\cos(0)=1$ and
  $\cos(x)=1-2^{-53}$ for $|x|<2^{-26}$.




\subsection{$\sin(\pi x)$ for small arguments}
The Taylor expansion is
\begin{equation}
  \sin(\pi x) = \pi x - (\pi x)^3/6 + O(x^5) = \pi x(1-(\pi
  x)^2/6) + O(x^5)\label{eq:sinpiTaylor}
\end{equation}
  where $O(x^5)$ has the sign of $x$. 

The situation is therefore more complex than for the radian
trigonometrics. We chose to use a two-step approach even for the small
arguments.  We therefore want a bound on the error of approximating
$sin(\pi x)$ with $sin(\pi x)$ which is between $2^{-60}$ and
$2^{-64}$.  This bound is given by (\ref{eq:sinpiTaylor}): For
$x<2^{-31}$, we have $(\pi x)^2/6 <<2^{-61.28}$.  We may then use an
algorithm that efficiently computes an approximation to $\pi x$ with a
relative rounding error smaller than $2^{-74}$. The total relative
error will be smaller than $2^{-61}$.


\subsection{$\sin(\pi x)$ for small arguments}
The Taylor expansion is
\begin{equation}
  \sin(\pi x) = \pi x - (\pi x)^3/3 + O(x^5) = \pi x(1-(\pi
  x)^2/3) + O(x^5)\label{eq:tanpiTaylor}
\end{equation}
where $O(x^5)$ has the sign of $x$.

The situation is therefore similar to the previous case, and we will get an
overall relative error smaller than $2^{-60}$ for $x<2^{-31}$.


\subsection{Computing $\pi x$}

There exists an algorithm, due to Brisebarre and Muller, which
computes the correctly rounded value of $\pi x$, for any
double-precision number $x$, in two FMA operations.  Its proof is a
variation of the Kahan/Douglas algorithm mentionned in Chapter
\ref{chap:trigo}. Unfortunately, it is of little use here. A first
problem is that it requires an FMA, however an equivalent algorithm
using double-double arithmetic should be easy to derive. A more
important problem is that it is only relevant if one may prove that
the correctly rounded value of $\pi x$ is also the correctly rounded
value of $\sin(\pi x)$. This happens when the relative difference
between $\pi x$ and $\sin(\pi x)$ is smaller than the worst-case
critical accuracy, which is $2^{-100}$ (TODO mis au hasard) for
$x<2^{-31}$. We conclude, again from (\ref{eq:sinpiTaylor}) that this
algorithm is useful for $x<2^{-55}$ (TODO voir le TODO precedent :).
As we have a two-step approach anyway, the cost of the additional test
is difficult to justify. 

However, if an FMA is available, we will use the same sequence of two
FMAs to evaluate $\pi x$ using  a double-double
approximation to $\pi$.

In the general case, we will be contented with an approximation of
$\pi x$ accurate to anything much more than $2^{-60}$, as suggested
before. Let us start with the straightforward double-double multiplication:\\
\texttt{ Mul12(&rh,&rl, x,0, PIH, PIL);}\\
where $x$ is completed with a zero and \texttt{PIH} and \texttt{PIL}
form a double-double approximation of $\pi$. This would provide much
too much accuracy, so the algorithm is adapted to the specific case as
follows:
\begin{itemize}
\item In the previous algorithm, all the multiplications by zero are of course optimised out;
\item The previous algorithm first splits \texttt{x} into \texttt{xh}
  and \texttt{xl}, and does the same for \texttt{PIH}. An obvious
  optimisation is to pre-split \texttt{PIH} into \texttt{PIH} and
  \texttt{PIM}.
\item A last optimisation is to neglect the term \texttt{xl*PIL}.
\end{itemize}

The final algorithm is therefore :
\begin{lstlisting}[caption={Multiplication by $\pi$ \label{lst:trigpi:pix}},firstnumber=1]
  const double c  = 134217729.; /* 2^27 +1 */   
  double t, xh, xl;                           

  /* Splitting of x. Both xh and xl have at least 26 consecutive LSB zeroes */
  t = x*c;     
  xh = (x-t)+t;
  xl = x-xh;   

  Add12(rh,rl, xh*PIH, (xl*PIH + xh*PIM) + (xh*PIL + xl*PIM) );               
\end{lstlisting}

The splitting is exact (Dekker). In the Add12, all the multiplications
are exact except \texttt{xh*PIL}. The \texttt{Add12} itself is also
exact. The error is therefore purely due to the three additions, and
lead to a conservative majoration of the relative error of $2^{-53-22}
= 2^{-75}$. 

This bound could probably be refined if needed. In
particular, it might be that the two additions \texttt{(xl*PIH +
  xh*PIM)} and (xh*PIL + xl*PIM) are both exact if \texttt{PIH} and
\texttt{PIM} have at least 27 LSB zeroes.
